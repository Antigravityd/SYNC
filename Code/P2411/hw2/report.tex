\documentclass{article}
\usepackage[letterpaper]{geometry}

\title{2411 HW 2}
\author{Duncan Wilkie}
\date{17 September 2021}

\begin{document}

\maketitle

\section{}
The trapezoid rule is the only one that can be used here, since there is no way to calculate $f(\frac{a+b}{2})$.
The program appears in the Script Files section.
Gaussian quadrature is also unuseable here, since we cannot choose evaluation points.
\section{}
Finding a unique solution for $\alpha, \beta, \gamma$ is a proof of uniqueness of the parabola, since no two parabolas share the same equation.
\[f(0)=4.5\Leftrightarrow \alpha(0)^2+\beta(0)+\gamma = 4.5 \Leftrightarrow \gamma = 4.5\]
\[f(-1)=2\Leftrightarrow 4\alpha+2\beta + 4.5 = 2\Leftrightarrow 4\alpha+2\beta = -2.5\]
\[f(1)=0.9\Leftrightarrow \alpha+\beta + 4.5 = 0.9 \Leftrightarrow \alpha+\beta=-3.6\]
Subtracting twice the third resulting equation from the second yields
\[2\alpha = 4.7 \Leftrightarrow \alpha = 2.35\]
Plugging this in to the second equation,
\[2.35+\beta = -3.6\Leftrightarrow \beta = -5.95\]
Applying Simpson's rule to the integral yields
\[\frac{b-a}{2}\left(f(a)+4f\left(\frac{b+a}{2}\right)+f(b)\right)=\frac{1-(-1)}{2}\left(f(-1)+4f(0)+f(1)\right)=\left(2+4(4.5)+0.9\right)=11.4\]
\section{}
The program and its results appears in the Script Files section. The approximate and the exact computations agree to the 14th place.
\section{}
The Gauss points are the roots of these polynomials. Applying the quadratic formula in $y^2$ yields
\[y^2=\frac{30/8\pm\sqrt{(30/8)^2-4(35/8)(3/8)}}{2(35/8)}=\frac{3}{7}\pm\frac{2\sqrt{\frac{6}{5}}}{7}\]
\[\Rightarrow y = \pm\sqrt{\frac{3}{7}\pm\frac{2\sqrt{\frac{6}{5}}}{7}}=\pm0.33998, \pm0.86114\]
The points for the trapezoid rule on [-1,1] with 4 points are -0.6, -0.2, 0.2, 0.6. These are equally-spaced, as opposed to the variably-spaced Gauss points.
The trapezoid rule also requires evaluation of the endpoints, whereas Gaussian quadrature does not.

\section*{Script Files}

\end{document}
%%% Local Variables:
%%% mode: latex
%%% TeX-master: t
%%% End:
