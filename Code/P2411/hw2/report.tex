\documentclass{article}
\usepackage[letterpaper]{geometry}

\title{2411 HW 2}
\author{Duncan Wilkie}
\date{17 September 2021}

\begin{document}

\maketitle

\section{}
The trapezoid rule is the only one that can be used here, since there is no way to calculate $f(\frac{a+b}{2})$.
The program appears in the Script Files section.
Gaussian quadrature is also unuseable here, since we cannot choose evaluation points.
\section{}
Finding a unique solution for $\alpha, \beta, \gamma$ is a proof of uniqueness of the parabola, since no two parabolas share the same equation.
\[f(0)=4.5\Leftrightarrow \alpha(0)^2+\beta(0)+\gamma = 4.5 \Leftrightarrow \gamma = 4.5\]
\[f(-1)=2\Leftrightarrow 4\alpha+2\beta + 4.5 = 2\Leftrightarrow 4\alpha+2\beta = -2.5\]
\[f(1)=0.9\Leftrightarrow \alpha+\beta + 4.5 = 0.9 \Leftrightarrow \alpha+\beta=-3.6\]
Subtracting twice the third resulting equation from the second yields
\[2\alpha = 4.7 \Leftrightarrow \alpha = 2.35\]
Plugging this in to the second equation,
\[2.35+\beta = -3.6\Leftrightarrow \beta = -5.95\]
Applying Simpson's rule to the integral yields
\[\frac{b-a}{2}\left(f(a)+4f\left(\frac{b+a}{2}\right)+f(b)\right)=\frac{1-(-1)}{2}\left(f(-1)+4f(0)+f(1)\right)=\left(2+4(4.5)+0.9\right)=11.4\]
\section{}
The program and its results appears in the Script Files section. The approximate and the exact computations agree to the 14th place.
\section{}
The Gauss points are the roots of these polynomials. Applying the quadratic formula in $y^2$ yields
\[y^2=\frac{30/8\pm\sqrt{(30/8)^2-4(35/8)(3/8)}}{2(35/8)}=\frac{3}{7}\pm\frac{2\sqrt{\frac{6}{5}}}{7}\]
\[\Rightarrow y = \pm\sqrt{\frac{3}{7}\pm\frac{2\sqrt{\frac{6}{5}}}{7}}=\pm0.33998, \pm0.86114\]
The points for the trapezoid rule on [-1,1] with 4 points are -0.6, -0.2, 0.2, 0.6. These are equally-spaced, as opposed to the variably-spaced Gauss points.
The trapezoid rule also requires evaluation of the endpoints, whereas Gaussian quadrature does not.

\section*{Script Files}
\subsection*{Program 1}
\begin{verbatim}
Script started on Fri 17 Sep 2021 03:32:54 PM CDT
tput: unknown terminal "st-256color"
tcsh: No entry for terminal type "st-256color"
tcsh: using dumb terminal settings.
[dwilk14@tigers ~/HW2]$ cat dwilk14_hw2p1.cpp
#include <fstream>
#include <iostream>

using namespace std;

int main() {
  double x[9] = {-1., -0.75, -0.50, -0.25, 0, 0.25, 0.5, 0.75, 1.};
  double fx[9] = {-24.0000, -16.9063, -11.5000, -7.5938, -5.0000, -3.5313, -3.0000, \
    -3.2188, -4.0000};

  // we are unable to apply Simpson's rule in a satisfactory manner, as there is no way to 
  //calculate f((a+b)/2) for most of the points.

  double result = 0.;
  for (int i = 0; i < 8; i++) {
    double a = x[i];
    double b = x[i+1];

    result += (b-a) * 0.5 * (fx[i] + fx[i+1]);
  }

  cout << "Integral estimate: " << result << endl;

  return 0;

}
[dwilk14@tigers ~/HW2]$ g++ dwilk14_hw2p1.cpp -o dwilk14_hw2p1
[dwilk14@tigers ~/HW2]$ ./dwilk14_hw2p1
Integral estimate: -16.1876
[dwilk14@tigers ~/HW2]$ cp dwilk14_hw2p1.txt /home3/kristina/phys2411/.
[dwilk14@tigers ~/HW2]$ exit
exit

Script done on Fri 17 Sep 2021 03:35:01 PM CDT

\end{verbatim}
\subsection*{Program 2}
\begin{verbatim}
Script started on Fri 17 Sep 2021 03:35:09 PM CDT
tput: unknown terminal "st-256color"
tcsh: No entry for terminal type "st-256color"
tcsh: using dumb terminal settings.
[dwilk14@tigers ~/HW2]$ cat dwilk14_hw2p2.cpp
#include <iostream>
#include <cmath>

using namespace std;
double f(double x) {
  double k = 9.E9;
  double lambda = 2.E-10;
  double d = 0.1;

  return k * lambda / sqrt(pow(x,2) + pow(d,2));

}

int main() {
  double k = 9.E9;
  double lambda = 2.E-10;
  double d = 0.1;

  double L = 0.5;
  double step = L / 514;
  double a = 0.;

  double result;
  for (int i = 0; i < 514; i++) {
    double b = a + step;

    result += (b - a) / 6 * (f(a) + 4 * f((a + b) / 2) + f(b));

    a = b;
  }

  cout.precision(15);
  cout << "Integral estimate: " << result << " V" << endl;
  cout << "Exact: " << k * lambda * log((L + sqrt(pow(L, 2)+pow(d, 2))) / d) << " V" << endl;

  return 0;
}
[dwilk14@tigers ~/HW2]$ g++ dwilk14_hw2p2.cpp -o dwilk14_hw2p2
[dwilk14@tigers ~/HW2]$ ./dwilk14_hw2p2
Integral estimate: 4.16238901429091 V
Exact: 4.16238901429096 V
[dwilk14@tigers ~/HW2]$ cp dwilk14_hw2p2.txt /home3/kristina/phys2411/.
[dwilk14@tigers ~/HW2]$ exit
exit

Script done on Fri 17 Sep 2021 03:36:06 PM CDT

\end{verbatim}
\end{document}
%%% Local Variables:
%%% mode: latex
%%% TeX-master: t
%%% End:
