\documentclass{article}

\usepackage[letterpaper]{geometry}
\usepackage{amsmath}

\title{2411 HW 7}
\author{Duncan Wilkie}
\date{25 October 2021}

\begin{document}
\section{}
The corresponding program appears in the script files section.

\section{}
Gaussian elimination may be applied to solve the matrix inverse problem $\mathbf{A}\vec{x}=\vec{b}$. All linear systems of equations, where one has some linear combination of variables equal to a constant in each of the eqations, may be written as a problem of this form. Classify the use of Gaussian elimination to find that a system is over- or underdetermined as ``solving'' the system of equations (otherwise, one would need to actualy solve the system to determine if a well-defined solution were possible).
\subsection{}
This one can be reduced to a linear system by setting $x = \cos(\alpha)$ and $y = \tan^2(\phi)$. Gaussian elimination becomes directly applicable.
\subsection{}
Expanding $(u-2v)^2=u^2-4uv+4v^2$, it becomes evident we may linearize the system by setting $x = u^2$, $y=v^2$. Gaussian elimination becomes directly applicable.
\subsection{}
Gaussian elimination is applicable here without modification.
\subsection{}
Once again, Gaussian elimination is directly applicable.
\subsection{}
In this case, it is impossible to write $e^z$ as a linear function of $z$ (stated without proof---technically follows from a polynomial-ring-over-field proof of it being a trancendental function, which is highly nontrivial). Therefore, this cannot be reduced to a linear system, and Gaussian elimination is impossible to apply.

\section{}
The first print statement outputs
\[\begin{bmatrix}
    1 & 8 & 10\\
    2 & 1 & 11\\
    5 & -50 & -14
  \end{bmatrix}\]
The second print statement outputs the superposed version of the matrix from the in-place decomposition algorithm (the ``Hadamard sum,'' I suppose):
\[\begin{bmatrix}
    1 & 8 & 10 \\
    2 & -15 & -9 \\
    5 & 6 & -10
  \end{bmatrix}\]
\section*{Script Files}
\begin{verbatim}

\end{verbatim}
\end{document}
%%% Local Variables:
%%% mode: latex
%%% TeX-master: t
%%% End:
