\documentclass{article}
\usepackage[utf8]{inputenc}
\usepackage[letterpaper]{geometry}
\usepackage{amsmath}

\title{Math 3355 Quiz 1}
\author{Duncan Wilkie}
\date{11 September 2021}

\begin{document}

\maketitle

\section*{Problem 1}
Call the set of investors that invested in traditional annuities $A$, and the set of investors that invested in the stock market $B$.
By the inclusion-exclusion principle, $|A\cup B|=|A| + |B| - |A\cap B|$. Solving for $|A\cap B|$, the size of the set of investors who invested in both, yields \[|A\cap B| = |A|+|B|-|A\cup B| = 60\% + 35\% - 80\% = 15\%\]

\section*{Problem 2}
The binomial theorem says $(a+b)^n = \sum_{k=0}^n\binom{n}{k}a^{n-k}b^k$. In this case, $a=2x$ and $b=5y$, and further since the power of $x$ is 7 and that of $y$ is 3, this corresponds to $n=10$ and $k=3$.
Therefore, the full corresponding term is $\binom{10}{4}(2x)^7(5y)^3$.
Pulling the constant terms out of the exponentiations yields a coefficient of $\binom{10}{4}\cdot2^7\cdot5^3 = 3,360,000$.
\section*{Problem 3}
The probability that a random point from $(a,b)$ falls into a subinterval $(\alpha, \beta)$ is $\frac{\beta-\alpha}{b-a}$ by definition of random selection.
In this case, $(a,b) = (0,1)$ and $(\alpha, \beta) = (\frac{1}{3}, 1)$, so the probability the random point is drawn from the latter interval is \[\frac{\beta-\alpha}{b-a} = \frac{1-\frac{1}{3}}{1-0}=\frac{2}{3}\]
\end{document}
%%% Local Variables:
%%% mode: latex
%%% TeX-master: t
%%% End:
