k\documentclass{article}

\usepackage[letterpaper]{geometry}
\usepackage{amsmath}
\usepackage{amssymb}

\title{7380 HW Corrections}
\author{Duncan Wilkie}
\date{4 December 2021}

\begin{document}

\maketitle

\section*{5.20}
My confusion with this problem came down to the interpretation of ``with a piecewise continuous derivative,'' which I took to mean that a derivative is defined at every point of the domain of $f$, in concert with the usual analytical meaning of existence of a derivative. Under that definition, $f$ is vacuously piecewise continuous. I presume they mean instead ``has a piecewise continuous derivative on the set where $f$ is continuous,'' and proceed accordingly.

By definition,
\[\left\langle\frac{\partial f}{\partial x}, \phi\right\rangle=-\left\langle f, \frac{\partial \phi}{\partial x} \right\rangle=-\int_\mathbb{R} f\frac{\partial \phi}{\partial x}dx\]
It is sufficient for now to consider a function with a single jump discontinuity. Call the point of discontinuity $x_0$, and the value of the left- and right-hand limits $f^-(x_0)$ and $f^+(x_0)$, respectively. We have
\[\left\langle\frac{\partial f}{\partial x},\phi\right\rangle=-\int_{-\infty}^{a}f\frac{\partial\phi}{\partial x}dx\int_{b}^\infty f\frac{\partial\phi}{\partial x}dx - \int_a^bf\frac{\partial \phi}{\partial x}=-f\phi\bigg|_{-M}^{x_0}+\int_{-\infty}^{x_0}f'\phi dx-f\phi\bigg|_{x_0}^{M}+\int_{x_0}^\infty f'\phi dx\]\[=\int_{\mathbb{R}\setminus x_0}f'\phi dx-(f^+(x_0)-f^-(x_0))\phi(x_0)\]
Writing the integral in the form $f'\bigg|_{\mathbb{R}\setminus x_0}$ and noting that the constant term is the action of a shifted delta distribution, this may be identified for multiple discontinuities in a finite set $J$ with the distribution
\[f'\bigg|_{\mathbb{R}\setminus J}+\sum_{x_i\in J}\delta(x-x_0)(f^-(x_i)+f^+(x_i))\]

\section*{5.22}
Substituting $v=x+t$, $w=x-t$
\[(u, \phi_{tt})=\int_{\mathbb{R}^2}f(x+t)\phi_{tt}(x,t)=\int_{\mathbb{R}^2}f(v)\frac{\partial^2}{\partial t^2} \phi\left(\frac{v+w}{2},\frac{v-w}{2}\right)(-2)dvdw\]
\[=-2\int_{\mathbb{R}^2}f(v)\frac{\partial}{\partial t}\left( \frac{\partial \phi}{\partial w}\frac{\partial w}{\partial t}+\frac{\partial \phi}{\partial v}\frac{\partial v}{\partial t} \right)dvdw\]

\[=-2\int_{\mathbb{R}^2}f(v)\left[ \frac{\partial^2\phi}{\partial w^2}\left( \frac{\partial w}{\partial t} \right)^2-2\frac{\partial^2\phi}{\partial v\partial w}+\frac{\partial^2\phi}{\partial v}\left( \frac{\partial v}{\partial t} \right)^2 \right]dvdw=-2\int_{\mathbb{R}^2}f(v)\left[ \frac{\partial^2\phi}{\partial w^2}-2\frac{\partial^2\phi}{\partial v\partial w}+\frac{\partial^2\phi}{\partial v^2} \right]dvdw\]

Similarly for $u_{xx}$,
\[(u,\phi_{tt})=\int_{\mathbb{R}^2}f(x+t)\phi_{tt}(x,t)=\int_{\mathbb{R}^2}f(v)\frac{\partial^2}{\partial x^2}\phi\left(\frac{v+w}{2},\frac{v-w}{2}\right)(-2)dvdw\]
\[=-2\int_{\mathbb{R}^2}f(v)\frac{\partial}{\partial x}\left(\frac{\partial \phi}{\partial w}\frac{\partial w}{\partial x}+\frac{\partial \phi}{\partial v}\frac{\partial v}{\partial x}  \right)=-2\int_{\mathbb{R}^2}f(v)\left[ \frac{\partial^2\phi}{\partial w^2}\left( \frac{\partial w}{\partial x} \right)^2+2\frac{\partial^2\phi}{\partial w\partial v}+\frac{\partial^2 \phi}{\partial v^2}\left( \frac{\partial v}{\partial x} \right)^2 \right]\]
\[=-2\int_{\mathbb{R}^2}f(v)\left[  \frac{\partial^2\phi}{\partial w^2}+2\frac{\partial^2\phi}{\partial w\partial v}+\frac{\partial^2\phi}{\partial v^2}\right]\]

Taking $u_{xx}-u_{tt}$ and performing the integral with respect to $w$,
\[u_{xx}-u_{tt}=-2\int_{\mathbb{R}^2}f(v)\left[4\frac{\partial^2\phi}{\partial w\partial v}\right]dvdw=-8\int_{\mathbb{R}}f(v)\frac{\partial\phi}{\partial v}dv\]
\[=-8\int_\mathbb{R}f(x+t)\frac{\partial \phi}{\partial t}dt=8\int_\mathbb{R}f(x+t)\frac{\partial \phi}{\partial x}dx\]
The last two equalities come from the two choices of variable to write $v$ in terms of; $u_{xx}-u_{tt}$ is equal to them both. However, one is the negation of the other, since the difference between them is merely a change of symbol. In other words,
\[u_{xx}-u_{tt}=c=-c\]
which implies $c=0$, i.e. $u_{xx}=u_{tt}$.
\end{document}
%%% Local Variables:
%%% mode: latex
%%% TeX-master: t
%%% End:
