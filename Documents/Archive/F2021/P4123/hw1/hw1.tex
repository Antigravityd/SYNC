\documentclass{article}
\usepackage[letterpaper]{geometry}
\usepackage{amsmath}
\usepackage{amsfonts}


\title{4123 HW 1}
\author{Duncan Wilkie}
\date{16 September 2021}

\begin{document}

\maketitle

\section*{1a}
Since we would like to extremize with respect to time, we must find a functional that expresses the travel time in terms of the variables given.
We can write $dt = \frac{ds}{v(y)}=\frac{\eta(y)dx\sqrt{1+y'^2}}{c}$.
The integral of this expression for $dt$ is of course the total travel time,\[J[y]  = \int\frac{\eta(y)\sqrt{1+y'^2}}{c}dx\]
Since there is no direct dependence on $x$ in this expression, the first integral of the Euler-Lagrange equations must be satisfied.
We then have, letting $\kappa$ being an arbitrary constant,
\[f-y'\frac{\partial f}{\partial y'} = \kappa\Leftrightarrow \frac{\eta(y)\sqrt{1+y'^2}}{c}-y'\frac{\partial}{\partial y'}\left(\frac{\eta(y)\sqrt{1+y'^2}}{c}\right) = \kappa\]
\[\Leftrightarrow \frac{\eta(y)\sqrt{1+y'^2}}{c}-\frac{\eta(y)y'^2}{c}\frac{1}{\sqrt{1+y'^2}}=\kappa\Leftrightarrow\frac{\eta(y)}{c}\sqrt{1+y'^2}\left(1-\frac{y'^2}{1+y'^2}\right)=\kappa\]
\[\Leftrightarrow\frac{\eta(y)}{c}\sqrt{1+y'^2}\left(\frac{1}{1+y'^2}\right)=\kappa \Leftrightarrow\eta(y)=K{\sqrt{1+y'^2}}\]

\section*{1b}
The Euler-Lagrange equation applied to the functional above yields
\[\frac{\partial}{\partial y}\left(\frac{\eta(y)\sqrt{1+y'^2}}{c}\right)-\frac{d}{dx}\frac{\partial}{\partial y'}\left(\frac{\eta(y)\sqrt{1+y'^2}}{c}\right) = 0\]
\[\Leftrightarrow\frac{\sqrt{1+y'^2}}{c}\eta'(y)-\frac{d}{dx}\left(\frac{\eta(y)}{c}\frac{y'}{\sqrt{1+y'^2}}\right)=0\]
\[\Leftrightarrow\frac{\sqrt{1+y'^2}}{c}\eta'(y)-\frac{1}{c}\left(\frac{y'^2}{\sqrt{1+y'^2}}\eta'(y)+\eta(y)y''\left(\frac{-y'^2}{{(1+y'^2)}^{3/2}}+\frac{1}{\sqrt{1+y'^2}}\right)\right)=0\]
\[\Leftrightarrow \eta'(y)(1+y'^2)=y'^2\eta'(y)+\eta(y)y''\left(\frac{-y'^2}{1+y'^2}+1\right)\]
\[\Leftrightarrow 1=\frac{\eta(y)}{\eta'(y)}y''\left(\frac{-y^2}{1+y'^2}+1\right)\]
\[\Leftrightarrow 1 = \frac{\eta(y)}{\eta'(y)}y''\frac{1}{1+y'^2}\]
\[\Leftrightarrow y''=\frac{\eta'(y)(1+y'^2)}{\eta(y)}\]
If the refractive index decreases, $\eta'(y) < 0$. $\eta(y)$ is positive (except for certain exotic occasions, according to my MATH 7380 professor), and of course $1+y'^2$ is as well. Therefore $y'' < 0$, i.e.\ the path concaves downwards.
\section*{2}
We apply the Euler-Lagrange equation to get
\[1-2y''=0\]
Integrating twice,
\[\frac{x^2}{2}-2y=c_1x+c_2\textrm{, where }c_1,c_2\in\mathbb{R}\]
Applying the first boundary condition,
\[c_2 = 0\]
Applying the second,
\[\frac{-7}{2}=\frac{c_1}{2}+c_2\Rightarrow c_1 = -7\]
Substituting within the integrated form these values of $c_1,c_2$ yields
\[\frac{x^2}{2}-2y=-7x\Leftrightarrow y =\frac{x^2}{4}+\frac{7}{2}x\]
\section*{3}
I'll assume that we're trying to show that the desired equations hold when $I$ is extremized.
Applying the first integral,
\[y+\ln(y')-y'\left( \frac{1}{y'} \right)=c_1\Leftrightarrow  \ln(y')=c_2-y\]
where $c_1\in\mathbb{R}$ and $c_2=c_1+1$.
Exponentiating this,
\[y'=\tilde{c}e^{-y}\]
where $\tilde{c}=e^{c_2}$. This is the first equation.
Applying Euler-Lagrange to the integrand,
\[1-\frac{d}{dx}\frac{1}{y'}= 0\]
We can integrate this with respect to $x$ to obtain the second equation,
\[x-\frac{1}{y'}=0\Leftrightarrow y' = \frac{1}{x+c}\Leftrightarrow y = \ln(x+c)+c'\]
where $c,c'\in\mathbb{R}$. Applying the boundary conditions, $\ln(2) = \ln(1+c)+c'$ and $\ln(3)=\ln(2+c)+c'$.
Subtracting the first from the second, we obtain $\ln(3/2) = \ln(\frac{2+c}{1+c}) \Leftrightarrow 3+3c=4+2c\Leftrightarrow c=1$.
Applying this to the first condition, $c'=0$ and consequently $y=\ln(1+x)$.

\section*{4}
Given a sphere and two points, we may choose the spherical coordinate system such that one of the points is the north pole.
Assume without loss of generality that the second point has well-defined coordinates $(r,\theta,\phi)$. The north pole of course has coordinates $(R, 0, 0)$ presuming the sphere has radius $R$. For an arbitrary arc, the arc length in spherical coordinates (where $\theta$ is azimuthal)
is given by the functional
\[L=\int ds=\int\sqrt{dr^2+r^2d\phi^2+r^2\sin^2(\phi)d\theta^2}=\int\sqrt{\left( \frac{dr}{dt} \right)^2+r^2\left( \frac{d\phi}{dt} \right)^2+r^2\sin^2(\phi)\left( \frac{d\theta}{dt}\right)^2}dt\]
\[=\int r\sqrt{\theta'^2\sin^2(\phi)+\phi'^2}dt\]
where we have used that the radial coordinate is constant on the surface of a sphere.
From this, we obtain one of two Euler-Lagrange equations
\[\frac{d}{dt}\left( \frac{\theta'\sin^2(\phi)}{\sqrt{\theta'^2\sin^2(\phi)+\phi'^2}} \right) = 0\]
\[\Leftrightarrow \theta'^2\sin^4(\phi)=c^2\theta'^2\sin^2(\phi)+c^2\phi'^2\]
\[\Leftrightarrow \theta'=\frac{c\phi'}{\sqrt{\sin^4(\phi)-\sin^2(\phi)c^2}}\]
We integrate this for an expression of $\theta$:
\[\theta=c\int\frac{d\phi}{\sqrt{\sin^4(\phi)-\sin^2(\phi)c^2}}\]
Since the integrand is always positive, the integral is always positive over an interval of positive measure, and so the only way for $\theta=0$ is for $c=0$. However, $\theta = 0$ must occur, since the path starts at the north pole. Therefore, $c=0$ for all $t$, and the Euler-Lagrange equation becomes $\theta'\sin^2(\phi)=0$. $\sin^2(\phi)$ is zero only at the poles, and any path containing the poles with $\theta'\neq 0$ at those poles would have a discontinuous derivative---violating an implicit smoothness assumption made in the use of the Euler-Lagrange equations---so this implies $\theta'=0$ for all points on the path. This is a segment of a great circle: a path starting at the north pole and moving away with fixed $\theta$ defines a plane through the center of the sphere by any three distinct points on it.

\end{document}

%%% Local Variables:
%%% mode: latex
%%% TeX-master: t
%%% End:
