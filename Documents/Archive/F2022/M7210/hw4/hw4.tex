
\documentclass{article}

\usepackage[letterpaper]{geometry}
\usepackage{tgpagella}
\usepackage{amsmath}
\usepackage{amssymb}
\usepackage{amsthm}

\title{7210 HW 4}
\author{Duncan Wilkie}
\date{20 September 2022}

\newcommand{\nsub}{\trianglelefteq}
\newtheorem*{3.4.8}{3.4.8}
\newtheorem*{3.5.10}{3.5.10}
\newtheorem*{4.2.7}{4.2.7}
\newtheorem*{4.2.8}{4.2.8}
\newtheorem*{4.2.10}{4.2.10}
\newtheorem{lem}{Lemma}

\begin{document}

\maketitle

\begin{lem}
  If $H \nsub G$ is cyclic, then every $K \leq H$ has $K \nsub G$.
\end{lem}

\begin{proof}
  Write $H = \langle a \rangle$.
  Normality of $H$ in $G$ is $ga^{k}g^{-1} = a^{k'}$ for all $g \in G$
  Subgroups of cyclic groups are also cyclic and so $H = \langle (a^k)^{i} \rangle$; normality of $H$ is then
  \[
    ga^{ki}g^{-1} = (ga^{k}g^{-1})^{i} = a^{k'i} = (a^{i})^{k'}
  \]
  which is in $\langle a^{i} \rangle$ by closure.
\end{proof}

\begin{3.4.8}
  For a finite group $G$, the following are equivalent:
  \begin{enumerate}
  \item $G$ is solvable
  \item $G$ has a chain of subgroups $1 = H_{0} \nsub H_{1} \nsub \cdots \nsub H_{s} = G$ such that $H_{i+1}/H_{i}$ is always cyclic
  \item all composition factors of $G$ are of prime order
  \item $G$ has a chain of subgroups $1 = N_{0} \nsub N_{1} \nsub \cdots \nsub N_{t} = G$
    such that each $N_{i}$ is a normal subgroup of $G$ and $N_{i+1} / N_{i}$ is always Abelian.
  \end{enumerate}
\end{3.4.8}

\begin{proof}
  It's very evident that $3 \Rightarrow 2$, $3 \Rightarrow 4$, $3 \Rightarrow 1$, $2 \Rightarrow 1$, and $4 \Rightarrow 1$.
  The observations necessary for each are, in order:
  \begin{itemize}
  \item simple groups of prime order are cyclic,
  \item simple groups of prime order are cyclic, and every subgroup of a normal cyclic subgroup is normal in the parent group by Lemma 1.
  \item simple groups of prime order are cyclic and therefore Abelian,
  \item cyclic groups are Abelian,
  \item the only difference is an extra condition in 4.
  \end{itemize}

  It therefore suffices to prove $1 \Rightarrow 3$, as 3 implies every other statement, and every other statement implies 1.
  This certainly holds when $|G| = 1$, since $1 = G_{0} = G_{s} = G$ is a solution chain,
  and $1 / 1 = 1$ has no normal subgroups other than itself and the trivial group, so it is also a composition series.
  Presume for induction again that this holds for solvable groups with order less than $|G|$.
  By assumption, $G$ has a solution chain
  \[
    1 = G_{0} \nsub G_{1} \nsub \cdots \nsub G_{t}  = G
  \]
  such that $G_{i+1} / G_{i}$ is Abelian.
  $G_{t-1}$ has order less than $G$, so it has composition factors all of prime order.
  We may then write
  \[
    1 = H_{0} \nsub H_{1} \nsub \cdots \nsub H_{k} \nsub G_{t-1} \nsub G_{t} = G
  \]
  where $H_{i+1} / H_{i}$ is a simple group of prime order.
  Additionally, $G_{t} / G_{t-1}$ has order less than $G$, and so has a composition series
  \[
    1 = K_{0} \nsub K_{1} \nsub \cdots \nsub K_{m} = G_{t} / G_{t-1},
  \]
  where $K_{i+1} / K_{i}$ is a simple group of prime order, and therefore cyclic.
  By the lattice isomorphism theorem, the elements of this composition series correspond bijectively to subgroups of $G_{t}$ containing $G_{t-1}$;
  this bijection preserves normality by the text of the theorem, but it also preserves the cyclic nature of subgroups and their order,
  since the bijection sends a coset to the set of its representatives and a set of representatives to the coset they mutually represent,
  and the order of a coset is the order of any of its representatives.
  Call the images of $K_{i}$ under this bijection $K'_{i}$; we then have a chain
  \[
    1 = H_{0} \nsub H_{1} \nsub \cdots \nsub H_{k} \nsub G_{t-1} \nsub K'_{0} \nsub \cdots \nsub K'_{m} \nsub G_{t} = G
  \]
  such that $H_{i+1} / H_{i}$, $G_{t-1} / H_{k}$, $K'_{0} / G_{t-1}$, $K'_{i+1} / K'_{i}$, and $G_{t} / K'_{m}$ are simple of prime order.

\end{proof}

\begin{3.5.10}
  Write a composition series for $A_{4}$ and deduce that $A_{4}$ is solvable.
\end{3.5.10}

\begin{proof}
  Looking at the lattice of subgroups of $A_{4}$ in the text, we're inspired to consider \newline $G = \langle (1 \; 2)(3 \; 4), (1 \; 3)(2 \; 4) \rangle$,
  since it appears to have a unique position in the lattice.
  The generated subgroup is defined as all words of arbitrary powers of generator elements;
  since disjoint two-cycles commute, the powers of generator elements may be distributed into powers of all two-cycles appearing in the generators.
  These powers are the same two-cycles, which again commute, so the words of arbitrary powers of generator elements may be rearranged into the form
  $g = [(1 \; 2)(3 \; 4)]^{n}[(1 \; 3)(2 \; 4)]^{k}$ for some $n,k \in \mathbb{Z}$.
  For all $n,k \neq 0$, the power is the same as the underlying element, so we can restrict to $n, k \in \{0, 1\}$.
  The elements of $G$ are then
  \begin{itemize}
  \item  $1$
  \item $(1 \; 2)(3 \; 4)$
  \item $(1 \; 3)(2 \; 4)$
  \item $(1 \; 2)(3 \; 4)(1 \; 3)(2 \; 4) = (1 \; 4)(2 \; 3)$
  \end{itemize}
  Note now that conjugation preserves the decomposition signature.
  Consider $\mu = \tau\sigma\tau^{-1}$ and the cycle
  $a_{1}\overset{\sigma}{\mapsto}a_{2}\overset{\sigma}{\mapsto}\cdots\overset{\sigma}{\mapsto}a_{k}\overset{\sigma}{\mapsto}a_{1}$.
  Notice that just by expanding the definition
  \[
    \mu\tau(a_{i}) = \tau\sigma(a_{i}) = \tau(a_{i+1}).
  \]
  This implies that
  $\tau(a_{1})\overset{\mu}{\mapsto}\tau(a_{2})\overset{\mu}{\mapsto}  \cdots\overset{\mu}{\mapsto}\tau(a_{k})\overset{\mu}{\mapsto}\tau(a_{1})$,
  i.e. $k$-cycles of $\sigma$ bijectively correspond with $k$-cycles of $\tau\sigma\tau^{-1}$ by applying $\tau$ to each element of the cycle of $\sigma$.
  Of course, conjugation of a cycle decomposition is conjugation of each of the cycles:
  \[
    \tau\sigma\tau^{-1} = \tau\sigma_{1}\tau^{-1}\tau\sigma_{2}\tau^{-1}\cdots\tau\sigma_{n}\tau^{-1}.
  \]
  Since permutations are bijective, the disjointness of the initial cycle decomposition implies disjointness of the product of the conjugated cycles,
  so the ``decomposition signature'' of a permutation is preserved under conjugation, i.e. the number of cycles in the decomposition and their lengths.

  $G$ is exactly all twofold products of 2-cycles, and the above result implies conjugation yields another twofold product of 2-cycles, so $G \nsub A_{4}$.

  Further, by Lagrange's theorem, $|A_{4} / G| = |A_{4}| / |G| = 12 / 4 = 3$.
  There is only one group of order 3: a general such group has distinct elements $\{1, a, b\}$, and $ab = 1$, as $ab = a \Rightarrow b = 1$ and vice versa.
  Correspondingly, $a = b^{-1} \Leftrightarrow a^{-1} = b$.
  Further, since every finite group has an element of order $p$ for every prime $p$ dividing $|G|$, one of $a$ or $b$ cubes to 1.
  Suppose WLOG it's $a$; then $a^{2} = a^{3}a^{-1} = 1b = b$.
  This implies $a$ generates the group, since $a^{1}, a^{2} = b, a^{3} = 1$ enumerates every element.
  It is therefore isomorphic to $\mathbb{Z} / 3\mathbb{Z}$.

  This proves we have a composition series, since groups of prime order have no nontrivial subgroups
  and so no nontrivial normal subgroups by Lagrange's theorem.
  Fully, we've demonstrated
  \[
    1 = H_{0} \nsub G \nsub H_{1} = A_{4}
  \]
  such that $A_{4} / H_{1} \cong \mathbb{Z} / 3\mathbb{Z}$, which is simple.

  Additionally, the composition factor is Abelian, so the composition series serves to show $A_{4}$ is solvable.
\end{proof}

\begin{4.2.7}
  Let $Q_{8}$ be the quaternion group of order 8.
  $Q_{8}$ is isomorphic to a subgroup of $S_{8}$, but not to any subgroup of $S_{n}$ for $n \leq 7$.
\end{4.2.7}

\begin{proof}
  We can follow the outline given in the proof of Cayley's theorem.
  $Q_{8}$ as a list is
  \[
    1, -1, i, j, k, -i, -j, -k.
  \]
  This list represents a bijection $\phi$ between $Q_{8}$ and $9 = \{1, 2, 3, 4, 5, 6 , 7, 8\}$.
  Let $Q_{8}$ act on itself by left-multiplication (corresponding to $H = 1$ in Theorem 3, just like in Cauchy's theorem).
  The map $\phi \circ (g\cdot) \circ \phi^{-1}$ is an element of $S_{8}$ that takes a number between 1 and 8, represents it as an element of $Q_{8}$,
  does multiplication by $g \in Q_{8}$, and then uses the representation to find what number from 1 to 8 this corresponds to.
  The permutation representation of each element of $Q_{8}$ in 1-line notation on $Q_{8}$ and then in cycle notation on 9,
  in order according to the order given by the list, is
  \begin{itemize}
  \item $1, -1, i, j, k, -i, -j, -k \mapsto 1$,
  \item $-1, 1, -i, -j, -k, i, j, k \mapsto (1 \; 2)(3 \; 6)(4 \; 7)(5 \; 8)$,
  \item $i, -i, -1, k, j, 1, -k, -j \mapsto (1 \; 6 \; 2 \; 3)(4 \; 5)(7 \; 8)$,
  \item $j, -j, -k, -1, i, k, 1, -i \mapsto (1 \; 7 \; 2 \; 4)(3 \; 5 \; 6 \; 8)$,
  \item $k, -k, -j, -i, -1, j, i, 1 \mapsto (1 \; 8 \; 2 \; 5)(3 \; 7)(4 \; 6)$,
  \item $-i, i, 1, -k, -j, -1, k, j \mapsto (1 \; 3 \; 2 \; 6)(4 \; 8)(5 \; 7)$,
  \item $-j, j, k, 1, -i, -k, -1, i \mapsto (1 \; 4 \; 2 \; 7)(3 \; 8 \; 6 \; 5)$,
  \item $-k, k, j, i, 1, -j, -i, -1 \mapsto (1 \; 5 \; 2 \; 8)(3 \; 4)(6 \; 7)$.
  \end{itemize}

  If $Q_{8} \cong K \leq S_{7}$, then $K$ is a permutation representation of $Q_{8}$ acting on a 7 element set.
  Accordingly, the action of $K$ on the 8-element set $Q_{8}$ must fix some element of $Q_{8}$; this element is in the kernel of the action.
  However, the action of $Q_{8}$ on itself must be transitive, and if it fixes some element of $Q_{8}$, that is a separate orbit.
  Since isomorphisms preserve representation-independent properties of groups, such as transitivity of particular actions, this is a contradiction.
\end{proof}

\begin{4.2.8}
  If $H \leq G$ has finite index $n$ then there is a normal subgroup $K$ of $G$ with $K \leq H$ and $|G:K|\leq n!$
\end{4.2.8}

\begin{proof}
  Let $G$ act on the left cosets of $H$ by left-multiplication.
  There's an associated permutation representation induced by this action; by Theorem 3, the kernel of this action $K$ is a normal subgroup of $G$
  that is the largest such subgroup contained in $H$.
  Applying the first isomorphism theorem to the representation homomorphism, $G / K$ is isomorphic to a subgroup of $S_{|G : H|} = S_{n}$,
  since the image of the representation is the set on which $G$ acts.
  The cardinality of $S_{n}$ is $n!$, and so $|G : K| \leq n!$.
\end{proof}

\begin{4.2.10}
  Every non-Abelian group of order 6 has a non-normal subgroup of order 2. Classify groups of order 6.
\end{4.2.10}

\begin{proof}
  By Sylow's theorem, there exists subgroups of any group of order 6 with orders 2 and 3.
  There's only one group of order 2 and of order 3 up to isomorphism: $\mathbb{Z} / 2\mathbb{Z}$ and $\mathbb{Z} / 3\mathbb{Z}$ (for the latter cf. above).
  Let $a$ be the generator of the subgroup with order 2, and $b$ the generator of that with order 3.
  If $ab = ba$, then $G$ is cyclic, since distributivity of exponentiation over commuting products means $(ab)^{6} = (a^{2})^{3}(b^{3})^{2} = 1$;
  this implies $G$ is cyclic, generated by $ab$ (it has 6 elements and is a subgroup of $G$), but it must be non-Abelian.

  Therefore, $b\langle a \rangle = \{ba^{k} \mid k = 0, 1\} = \{b, ba\}$ is not equal to $\langle a \rangle b = \{a^{k}b \mid k = 0, 1\} = \{b, ab\}$,
  so the group of order 2 is not a normal subgroup.

  Every non-Abelian group of order 6 is isomorphic to $S_{3}$: the permutation representation of the action of $G$ by left-multiplication
  on the cosets of $\langle a \rangle$ has kernel contained in $\langle a \rangle$,
  since no elements of $\langle b \rangle$ commute with those of $\langle a \rangle$ as we've proven their generators don't commute and each is cyclic.
  The kernel is therefore trivial, since $\langle a \rangle$ has 2 elements and if the kernel included the nonidentity one it'd show
  $\langle a \rangle$ was be normal.
  Therefore, the representation homomorphism is an injective function between two groups of the same order, and so it's a bijection,
  showing $G \cong S_{3}$.
  All non-Abelian groups of order 6 were shown to be Abelian in the proof of the first part, so groups of order 6 have isomorphism class of either
  $S_{3}$ or $\mathbb{Z} / 6\mathbb{Z}$.
\end{proof}
\end{document}
