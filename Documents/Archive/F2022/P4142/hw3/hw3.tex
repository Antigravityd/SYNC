\documentclass{article}

\usepackage[letterpaper]{geometry}
\usepackage{tgpagella}
\usepackage{amsmath}
\usepackage{amsthm}
\usepackage{amssymb}
\usepackage{siunitx}
\usepackage{physics}
\usepackage{chemmacros}

\title{4142 HW 3}
\author{Duncan Wilkie}
\date{24 September 2022}

\newtheorem{prob}{Problem}
\newtheorem{lem}{Lemma}
\renewcommand*{\proofname}{Solution}

\begin{document}

\maketitle

\begin{prob}
  A particle of mass m is placed in a finite spherical well, $V (r) = -V_{0}$ for $r < a, = 0$ for $r > a$.
  Find the ground state by solving the radial equation for $\ell$ = 0.
  Show that there is a minimum value of $V_{0}a^{2}$ for a bound state to exist.
\end{prob}
\begin{proof}
  The radial equation is, employing the Heaviside step function $\Theta$,
  \[
    -\frac{\hbar^{2}}{2m}\dv[2]{u}{r} + \left[ -V_{0}\Theta(r - a) + \frac{\hbar^{2}}{2m}\frac{\ell(\ell + 1)}{r^{2}}\right]u = Eu
  \]
  Evaluating at $\ell = 0$,
  \[
    -\frac{\hbar^{2}}{2m}\dv[2]{u}{r} + \left[ -V_{0}\Theta(r - a)\right]u = Eu
  \]
  We can solve this as two different ODEs on different domains with boundary conditions set accordingly:
  \begin{align*}
    -\frac{\hbar^{2}}{2m}\dv[2]{u}{r} &= Eu   \Leftrightarrow  \dv[2]{u}{r} = -\frac{2mE}{\hbar^{2}}u & r &> a, \\
    -\frac{\hbar^{2}}{2m}\dv[2]{u}{r} &= (E+V_{0})u  \Leftrightarrow  \dv[2]{u}{r} = -\frac{2m(E+V_{0})}{\hbar^{2}}u & r &\leq a.
  \end{align*}
  These are second-order constant-coefficient ODEs which can be solved via the ansatz $e^{mr}$, leading to auxiliary equations of the form $m^{2}+c = 0$.
  Inside the well, $E+V_{0} > 0$,  so the analytic solutions are of the form $u = A\sin(kr + \delta)$,
  where $k$ is the coefficient of $u$ in the above.
  $u$ is related to the radial part of the wave function by $R(r) = u(r) / r$; $R(r)$ must be continuous at every point,
  and since $f(r) / r \to \infty$ if $f(r) \not\to 0$ as $r\to 0$, $\delta = 0$.
  For a bound state, $E < 0$, so outside the well has solutions of the form $u(r) = Be^{-k'r}+Ce^{k'r}$.
  The latter fails to be normalizable, so take $u(r) = Be^{-k'r}$.
  The ground state is then
  \[
    R(r) =
    \begin{cases}
      \frac{A}{r}\sin(kr) & r \leq a \\
      \frac{B}{r}e^{-k'r}  & r > a
    \end{cases}
  \]
  where $A$ and $B$ are constants determined by normalization and the fact that $R(r)\in C^{1}[0,\infty)$.

  The amplitude of a bound state is some continuous function of $V_{0}a^{2}$; the amplitude is bounded below by 0,
  corresponding to $a = 0$ which is indistinguishable from the free particle, where there's no bound state.
  Continuous functions that have a lower bound have global minima.
  The preimage of the global minima of this function is a set of values of $V_{0}a^{2}$ for which the amplitude is zero.
  Since there are some values for which the amplitude is nonzero, this set is bounded above; since $V_{0}a^{2}\geq 0$, it's bounded, period.
  Accordingly, it has a largest element.
  This largest element is the minimum value of $V_{0}a^{2}$ for a bound state to exist.

\end{proof}

\begin{prob}
  Verify that the radial wave functions of the $n = 1$, $\ell = 0$; $n = 2$, $\ell = 0$; and $n = 2$, $\ell = 1$ are normalized,
  and that the first two are mutually orthogonal. Use Griffiths Table 4.7 or equivalent.
\end{prob}

\begin{lem}
  The generalization of the gamma function $\int_{0}^{\infty}x^{n}e^{-ax}dx$ has value $\frac{n!}{a^{n+1}}$.
\end{lem}

\begin{proof}[Proof]
  Induct on $n$.
  The property clearly holds for $n = 0$.
  Suppose it holds for $n = m - 1$.
  Then
  \[
    \int_{0}^{\infty}x^{m}e^{-ax}dx = \eval{-\frac{x^{m}}{a}e^{-ax}}_{0}^{\infty} + \frac{m}{a}\int_{0}^{\infty}x^{m-1}e^{-ax}dx
    = \frac{m(m-1)!}{aa^{m}}
    = \frac{m!}{a^{m+1}},
  \]
  so the property holds for all $n$.
\end{proof}
\renewcommand*{\proofname}{Solution}
\begin{proof}
  $n = 1, \ell = 0$ corresponds to a radial wave function of $2a^{-3/2}e^{-r/a}$.
  Substituting into the norm,
  \[
    \braket{R_{10}}{R_{10}} = 4a^{-3}\int_{0}^{\infty}e^{-2r/a}r^{2}dr = 4a^{-3}\frac{2!}{(2 / a)^{3}} = 1
  \]
  Proceeding similarly, $n = 2, l = 0 \Rightarrow R(r) = \frac{1}{\sqrt{2a^{3}}}\qty(1-\frac{r}{2a})e^{-r/2a}$
  \[
    \Rightarrow \braket{R_{20}}{R_{20}} = \frac{1}{2a^{3}}\int_{0}^{\infty}\qty(1 - \frac{r}{2a})^{2}e^{-r/a}r^{2}dr
  \]
  \[
    = \frac{1}{2a^{3}}\qty(\int_{0}^{\infty}e^{-r/a}r^{2}dr - \frac{1}{a}\int_{0}^{\infty}e^{-r/a}r^{3}dr + \frac{1}{4a^{2}}\int_{0}^{\infty}e^{-r/a}r^{4}dr)
  \]
  \[
    = \frac{1}{2a^{3}}\qty(\frac{2!}{(1 / a)^{3}} - \frac{1}{a}\frac{3!}{(1 / a)^{4}} + \frac{1}{4a^{2}}\frac{4!}{(1 / a)^{5}})
  \]
  \[
    = \frac{1}{2a^{3}}\qty(2a^{3} - 6a^{3} + 6a^{3}) = 1
  \]
  Finally, $n=2, l = 1\Rightarrow R(r) = \frac{1}{2\sqrt{6a^{3}}}\qty(\frac{r}{a})e^{-r/2a}$
  \[
    \Rightarrow \braket{R_{21}}{R_{21}} = \frac{1}{24a^{3}}\int_{0}^{\infty}\frac{r^{2}}{a^{2}}e^{-r/a}r^{2}dr
  \]
  \[
    = \frac{1}{24a^{3}}\frac{1}{a^{2}}\frac{4!}{(1 / a)^{5}} = 1
  \]
\end{proof}

\begin{prob}
  Calculate the expectation values $\expval{r}$ and $\expval{r^{2}}$ for the ground state of the hydrogen atom.
  Use the Bohr radius to express your results.
  What is the expectation value of $\expval{x^{2}}$?
\end{prob}

\begin{proof}
  The ground state of the hydrogen atom has wave function
  \[
    \psi_{100}(r,\theta,\phi) = \frac{1}{\sqrt{\pi a^{3}}}e^{-r / a}
  \]
  We can then compute, using the lemma from above,
  \[
    \expval{r}{\psi_{100}} = \frac{1}{\pi a^{3}}\int_{0}^{\infty}re^{-2r/a}r^{2}dr\int_{0}^{\pi}\int_{0}^{2\pi}d\Omega = \frac{4}{a^{3}}\frac{3!}{(2 / a)^{4}}
    = \frac{3}{2}a
  \]
  \[
    \expval{r^{2}}{\psi_{100}} = \frac{1}{\pi a^{3}}\int_{0}^{\infty}r^{2}e^{-2r/a}r^{2}dr\int_{0}^{\pi}\int_{0}^{2\pi}d\Omega
    =\frac{4}{a^{3}}\frac{4!}{(2/a)^{5}} = \frac{3}{2}a^{2}
  \]
  \[
    \expval{x^{2}}{\psi_{100}} = \expval{r^{2}\cos^{2}\phi}{\psi_{100}} =\frac{1}{\pi a^{3}}\int_{0}^{\infty}r^{2}e^{-2r/a}r^{2}dr\int_{0}^{\pi}\sin\theta d\theta
    \int_{0}^{2\pi}\cos^{2}\phi d\phi
  \]
  \[
    = \frac{1}{\pi a^{3}}\frac{4!}{(2/a)^{5}}\cdot 2\cdot \pi = \frac{3}{2}a^{2}
  \]
  where for the last integral we've applied $\cos^{2}\phi = \frac{1}{2} + \frac{1}{2}\cos2\phi$.
\end{proof}

\begin{prob}
  A hydrogenic atom is also one with a single electron but nuclear charge $Ze$, e.g. \ch{_{92}^{238}U^{91+}}, the Uranium ion has Z = 92.
  Calculate the ground state (Bohr) radius and energy of such a system.
  Note no elaborate calculation is needed but inspection of how the result for hydrogen scales with Z
\end{prob}

\begin{proof}
  With a heavier nucleus, one must replace the charge term in the Coulomb potential according to $e^{2}\mapsto Ze^{2}$, which in this case is $92e^{2}$.
  In the calculation of the Bohr radius, this charge constant is quickly absorbed into constants to simplify expressions,
  so we can skip to the end of the calculation when these constants are unwrapped.
  The charge constant appears once and doesn't cancel with anything, so we can substitute in the result, obtaining
  \[
    a = \frac{4\pi\epsilon_{0}\hbar^{2}}{m_{e}Ze^{2}} = \frac{4\pi\epsilon_{0}\hbar^{2}}{m_{e}92e^{2}} = \SI{0.575e-12}{m}
  \]
  The ground state energy is, following similar reasoning,
  \[
    E_{1} = -\qty[\frac{m_{e}}{2\hbar^{2}}\qty(\frac{92e^{2}}{4\pi\epsilon_{0}})^{2}] = 92^{2}(-\SI{13.6}{eV}) = \SI{115}{keV}
  \]
\end{proof}

\begin{prob}
  Consider the Sun-Earth system as a quantum analog of the hydrogen atom but with gravitational instead of electromagnetic interaction.
  Estimate the Bohr radius of such a system.
  Taking the Earth’s orbit as circular, estimate the $n$ quantum number that would apply.
  Do so by equating the classical energy to the Bohr-Rydberg expression.
\end{prob}

\begin{proof}
  Once again, all that changes is the constant multiple of the Coulomb interaction term;
  $\frac{e^{2}}{4\pi\epsilon_{0}} \mapsto {GM_{\odot}M_{\oplus}}$.
  The preimage can be directly recognized in the formula for the Bohr radius.
  Substituting yields
  \[
    a = \frac{\hbar^{2}}{GM_{\odot}M^{2}_{\oplus}} = \SI{2.3e-138}{m}
  \]
  Equating the Bohr energy to the classical energy,
  \[
    -\qty[\frac{M_{\oplus}}{2\hbar^{2}}\qty({GM_{\odot}M_{\oplus}})^{2}]\frac{1}{n^{2}} = -\frac{GM_{\odot}M_{\oplus}}{r}
    \Leftrightarrow n = \sqrt{\frac{GM_{\odot}M^{2}_{\oplus}r}{2\hbar^{2}}}
  \]
  Evaluating at the measured values,
  \[
    = \sqrt{\frac{(\SI{6.67e-11}{N m^{2}/kg^{2}})(\SI{2e30}{kg})(\SI{6e24}{kg})^{2}(\SI{150e9}{m})}{2(\SI{1.05e-34}{J\cdot s})^{2}}}
    = \SI{1.8e74}{}
  \]
\end{proof}
\end{document}
