\documentclass{article}

\usepackage[letterpaper]{geometry}
\usepackage{tgpagella}
\usepackage{amsmath}
\usepackage{amssymb}
\usepackage{amsthm}
\usepackage{tikz}
\usepackage{minted}
\usepackage{physics}
\usepackage{siunitx}

\sisetup{detect-all}
\newtheorem{plm}{Problem}
\renewcommand*{\proofname}{Solution}

\title{4142 Test 2}
\author{Duncan Wilkie}
\date{22 November 2022}

\begin{document}

\maketitle

\begin{plm}
  How do the relativistic corrections to the energy levels in the iron atom ($Z = 26$) compare with those in the hydrogen atom?
\end{plm}

\begin{proof}
  The hydrogenic atom has $e^{2} \mapsto Ze^{2}$ in the potential and $E_{n} \mapsto Z^{2}E_{n}$; in general,
  the first-order relativistic correction is
  \[
    E_{r}^{1} = -\frac{1}{2mc^{2}}\qty[E^{2} - 2E\expval{V} + \expval{V^{2}}]
  \]
  which, according to the substitutions above, is
  \[
    E_{r}^{1} = -\frac{1}{2mc^{2}}\qty[Z^{4}E_{n}^{2} + 2Z^{2}E_{n}\qty(\frac{Ze^{2}}{4\pi\epsilon_{0}})\expval{\frac{1}{r}}
    + \qty(\frac{Ze^{2}}{4\pi\epsilon_{0}})^{2}\expval{\frac{1}{r^{2}}}]
  \]
  So, the corrections are larger by a factor greater than $Z^{2}$!
\end{proof}

\begin{plm}
  For a two-electronic configuration $2p3p$, what are the allowed values of total $S$, $L$, and $J$?
  What would it be for the configuration $2p^{2}$, upon taking Pauli exclusion into account?
  Express in the state notation ${}^{2S+1}L_{J}$.
\end{plm}

\begin{proof}
  The rule is that the possible total spins of the composite system $s_{1}, s_{2}$ are all the integer steps from $s_{1} + s_{2}$ to $s_{1} - s_{2}$.
  The two electrons have spin either $\frac{1}{2}$ or $-\frac{1}{2}$, so the total spin is either 1 or 0.
  The orbital angular momentum of each is, since they're both in $p$ states, $\ell = 1$, so the possible total orbital angular momenta
  are the integers from 2 to 0: 0, 1, 2.
  The possible total angular momenta $J$ are the possible values of $L + S$ for each possible value of $L$ and $S$: 0, 1, 2, 3.
\end{proof}

\begin{plm}
  Explain in a couple of lines why only the excited states of the hydrogen atom and no others exhibit the linear Stark effect, that is,
  show energy corrections linear in the field strength of an applied electric field.
\end{plm}

\begin{proof}
  The Stark effect potential is $e|E|z$, where one chooses $z$ to be the axis along which the uniform electric field points.
  The ground states are necessarily at minima of the energies of the unperturbed system, so when a perturbation is applied,
\end{proof}

\begin{plm}
  Which of the following would be reasonably trial functions for a Rayleigh-Ritz variational estimate of the ground state in a 1D potential:
  \[
    x + a, \; e^{-ax}, \; e^{-ax^{2}}, \; \frac{1}{x^{2} + a^{2}}
  \]
  where $a$ is a variational parameter?
\end{plm}

\begin{proof}
  $e^{-ax^{2}}$ is eminently reasonable, as it's the harmonic oscillator ground state,
  and since potentials that model ``more stable'' phenomena can be expected to have larger values of the even terms of their Taylor expansions
  (the odd terms having mismatched end behavior), these phenomena are well-approximated to second order,
  as the third-order coefficient bounds the error in the approximation, and such a coefficient is expected to be small already.

  $\frac{1}{x^{2} + a^{2}}$ also works, since it's normalized.
  The others, by contrast, are not normalized, so you aren't likely to even be able to compute the expectation necessary for variational methods.
\end{proof}

\begin{plm}
  The transition ${}^{2}P_{3/2} \to {}^{2}S_{1/2}$ is observed in a magnetic field of \SI{1}{T}.
  \begin{enumerate}
  \item Sketch the Zeeman splitting of both states.
  \item What are the Land\'e $g$ factors of the two states?
  \item Estimate the magnitude of the splitting between the levels.
  \end{enumerate}
\end{plm}

\begin{proof}
  The Zeeman splitting is roughly
  \begin{center}
    \begin{tikzpicture}
      \def\np{3}
      \def\ns{2}
      \def\h{3}
      \draw (0, 0) node[left] {${}^{2}P_{3/2}$} -- (1, 0);
      \foreach \i in {1, ..., \np}
      {
        \draw (1.05, 0) -- (2, \i * \h / \np - \h / \np / 2 - \h / 2);
      }
    \end{tikzpicture}
  \end{center}
\end{proof}

\begin{plm}
  An electron is in the spin state
  \[
    N
    \begin{pmatrix}
      1 - 2i \\
      2
    \end{pmatrix}
  \]
  \begin{enumerate}
  \item Determine the normalization constant $N$.
  \item If you measured $S_{z}$ on this state, what values would you get and with what probabilities?
    What is the expectation value?
  \item If you measured $S_{x}$ on this state, what values would you get and with what probabilities?
    What is the expectation value?
  \item If you measured $S_{y}$ on this state, what values would you get and with what probabilities?
    What is the expectation value?
  \end{enumerate}
\end{plm}
\end{document}
