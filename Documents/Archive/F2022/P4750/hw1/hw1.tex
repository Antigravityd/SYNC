\documentclass{article}

\usepackage[letterpaper]{geometry}
\usepackage{amsmath}
\usepackage{amsthm}
\usepackage{siunitx}

\newtheorem{prob}{Problem}

\title{4750 HW 1}
\author{Duncan Wilkie}
\date{31 August 2022}

\begin{document}

\maketitle

\begin{prob}
  Two events have Cartesian coordinates $(t,x,y,z)$ of $E_{1}=(1, 0, 1, 0)$ and $E_{2}=(2, 0, 1.5, 0)$ in SI units.
  Does there exist a reference frame in which the two events are simultaneous?
  Explain why, and find such reference frames.
\end{prob}

The spacetime interval between the two points is the norm induced by the Minkowski metric:
\[
  s^{2}= -(c\Delta t)^{2}+\Delta x^{2} + \Delta y^{2} + \Delta z^{2}
\]
\[
  =-(\SI{3e8}{m/s}[\SI{2}{s}-\SI{1}{s}])^{2}+(\SI{0}{m}-\SI{0}{m})^{2}+(\SI{1.5}{m}-\SI{1}{m})^{2}+(\SI{0}{m}-\SI{0}{m})^{2}
\]
\[
  =\SI{-9e16}{m^{2}} < 0
\]
The separation between them is therefore timelike, so there does not exist such a reference frame.

\begin{prob}
  Calculate the mass of objects with Swarzschild radii of: \SI{1}{mm}, \SI{1}{km}, \SI{1e3}{km}, \SI{1}{AU}.
  Are there black holes of these sizes?
  Could there be?

\end{prob}
The formula for the Swarzschild radius is
\[
  r_{s}=\frac{2GM}{c^{2}}\Leftrightarrow M = \frac{r_{s}c^{2}}{2G}
\]
Evaluating this at all of these values,
\[
  M = \frac{(\SI{1}{mm})(\SI{3e8}{m/s})^{2}}{2(\SI{6.67e-11}{N\cdot m^{2}/kg})} = \SI{6.75e23}{kg} = \SI{-6.47}{\log M_{\bigodot}}
\]
\[
  M = \frac{(\SI{1}{km})(\SI{3e8}{m/s})^{2}}{2(\SI{6.67e-11}{N\cdot m^{2}/kg})} = \SI{6.75e29}{kg} = \SI{-4.72}{\log M_{\bigodot}}
\]
\[
  M = \frac{(\SI{1e3}{km})(\SI{3e8}{m/s})^{2}}{2(\SI{6.67e-11}{N\cdot m^{2}/kg})} = \SI{6.75e32}{kg} = \SI{1.08}{\log M_{\bigodot}}
\]
\[
  M = \frac{(\SI{1.5e11}{m})(\SI{3e8}{m/s})^{2}}{2(\SI{6.67e-11}{N\cdot m^{2}/kg})} = \SI{1.01e38}{kg} = \SI{2.53}{\log M_{\bigodot}}
\]
A tabulation of black hole mass results (doi:10.1086/342878) has a range of $\SI{6.13}{M_{\bigodot}}$—$\SI{9.53}{\log M_{\bigodot}}$.
In consistent units, the first number is $\SI{0.79}{\log M_{\bigodot}}$, so only the last two numbers look plausible in terms of known black holes.
However, from a purely empirical perspective, this absence could be a selection bias:
the methodology for detecting black holes depends entirely on indirect observation, as they are ``black.''
Therefore, it's possible that black holes below a certain size don't sufficiently perturb the stars near them or don't generate X-ray bursts big enough.

\begin{prob}
  If you were to fall into the black hole at the center of the galaxy, what'd be the tidal force between your head and feet at the event horizon?
  Estimate the size of a black hole for which such force would begin to be uncomfortable.
\end{prob}

The spaghettification force may be found by summing the marginal gravitational force at the Swarzschild radius over the height of the person.
The mass on which the marginal gravitational force acts is the marginal amount of mass over the step in space;
we will assume a uniformly distributed human being (presuming total mass $m$, $m(r) = \frac{m\cdot(r-r_{s})}{l-r_{s}}$ for $r_{s}<r<r_{s}+l$, 0 otherwise).
\[
  F = \int_{r_{s}}^{r_{s}+l}-\frac{GMdm(r)}{r^{2}}
  = \int_{r_{s}}^{r_{s}+l}-\frac{GM}{r^{2}}\frac{m}{l-r_{s}}dr
  = \frac{GMm}{l-r_{s}}\left( \frac{1}{r_{s}+l}-\frac{1}{r_{s}} \right)
  = \frac{GMml}{r_{s}(r_{s}-l)(r_{s}+l)}
\]
We can eliminate $M$ by the Swarzschild radius formula equivalent $M = \frac{r_{s}c^{2}}{2G}$ to get
\[
  F = \frac{1}{2}\frac{mlc^{2}}{r_{s}^{2}-l^{2}}
\]
Estimating the height of a person to be $\SI{2}{m}$, the mass of a person to be $\SI{90}{kg}$,
and using the estimated Swarzschild radius of Sagittarius A$^{*}$ of \SI{1.22e10}{m},
\[
  F = \frac{1}{2}\frac{(\SI{90}{kg})(\SI{2}{m})(\SI{3e8}{m/s})^{2}}{(\SI{1.22e10}{m})^{2}-(\SI{2}{m})^{2}} = \SI{.054}{N}
\]

A healthy person can comfortably hang from a pull-up bar with around 40\% of their weight added,
so the estimated person above could handle roughly \SI{1300}{N} of tension without trouble.
Solving for $r_{s}$ in the formula we derived and entering this value, one would begin to feel discomfort at the event horizon of a black hole with size
\[
  r_{s} = \sqrt{\frac{mlc^{2}}{2F}+l^{2}}
  = \sqrt{\frac{(\SI{90}{kg})(\SI{2}{m})(\SI{3e8}{m/s})^{2}}{2(\SI{1300}{N})}+(\SI{2}{m})^{2}}
  = \SI{7.89e7}{m}
\]
\end{document}
