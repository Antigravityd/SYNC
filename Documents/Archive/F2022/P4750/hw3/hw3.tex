\documentclass{article}

\usepackage[letterpaper]{geometry}
\usepackage{tgpagella}
\usepackage{amsmath}
\usepackage{amssymb}
\usepackage{amsthm}
\usepackage{tikz}
\usepackage{graphicx}
\usepackage{minted}
\usepackage{physics}
\usepackage{siunitx}

\usemintedstyle{emacs}
\setminted{linenos=true}

\sisetup{detect-all}
\newtheorem{plm}{Problem}
\newtheorem{lem}{Lemma}
\renewcommand*{\proofname}{Solution}

\DeclareMathOperator{\sgn}{sgn}

\title{4750 HW 3}
\author{Duncan Wilkie}
\date{10 October 2022}

\begin{document}

\maketitle

\begin{plm}
  Consider a square wave of frequency $f_{0}$: $x(t) = \sgn\sin(2\pi f_{0}t)$.
  Calculate its Fourier transform and Fourier series.
  Consider a discrete sampling of the function with sampling frequencies $f_{0}/2$, $f_{0}/5$, and $f_{0}/1000$,
  and calculate the discrete Fourier transform in each case.
\end{plm}

\begin{proof}
  In the convention used in class, the Fourier transform of $f(x)$ is
  \[
    \hat{f}(\omega) = \int_{-\infty}^{\infty}f(t)e^{-i\omega t}dt
  \]
  The Fourier series of a periodic function $g(x)$ with period $T$ is
  \[
    g(x) = \sum_{n = -N}^{N}c_{n}e^{-inx / T}
  \]
  where
  \[
    c_{n} = \frac{1}{T}\int_{-T/2}^{T/2}g(t)e^{int/T}dt
  \]
  Our $x(t)$ has period $1 / f_{0}$, so the Fourier series coefficients are
  \[
    c_{n} = f_{0}\int_{-1/2f_{0}}^{1/2f_{0}}\sgn\sin(2\pi f_{0}t)e^{inf_{0}t}dt
  \]
  On $(0, 1/2f_{0})$, the argument to $\sgn$ ranges from 0 to 1, and on $(-1/2f_{0}, 0)$, it ranges from -1 to 0.
  Accordingly, on the two ranges, the value of $\sgn\sin$ is always $+1$ and always $-1$, respectively, so we can break the integral up as
  \[
    c_{n} = f_{0}\int_{0}^{1/2f_{0}}e^{inf_{0}t}dt - f_{0}\int_{-1/2 f_{0}}^{0}e^{in f_{0}t}dt
    = \frac{f_{0}}{inf_{0}}\qty(\eval{e^{inf_{0}t}}_{0}^{1/2f_{0}} - \eval{e^{inf_{0}t}}_{-1/2f_{0}}^{0})
  \]
  \[
    = \frac{-i}{n}\qty(e^{in/2} - e^{-in/2})
    = \frac{-i}{n}2i\sin(n / 2)
    = \frac{2}{n}\sin(n/2)
  \]

  The Fourier transform is, on grounds of just knowing what it means, something like $\delta(\omega - 2\pi f_{0})$, but let's do it mathematically.
  Using the Fourier series for $x$,
  \[
    \tilde{x}(\omega) = \int_{-\infty}^{\infty}x(t)e^{-i\omega t}dt
    = \int_{-\infty}^{\infty}\sum_{n = -\infty}^{\infty}\frac{2}{n}\sin(n/2)\sin(2\pi f_{0} t)e^{i\omega t}dt
  \]
  \[
    = \sum_{n=-\infty}^{\infty}\frac{2}{n}\sin(n/2)\int_{-\infty}^{\infty}\sin(2\pi f_{0} t)e^{i \omega t}dt
  \]
  We want to show that $\sin(2\pi f_{0}t) = \delta(\omega - 2\pi f_{0})$; the delta distribution is defined by its action as a linear functional.
  Noting that
  \[
    \tilde{\delta}(\omega) = \int_{-\infty}^{\infty}\delta(t - a)e^{-i\omega t}dt = e^{-i\omega a}
    \Rightarrow \delta(t-a) = \frac{1}{2\pi}\int_{-\infty}^{\infty}e^{-i\omega a}e^{i\omega t}d\omega = \frac{1}{2\pi}\int_{-\infty}^{\infty}e^{i\omega (t-a)}d\omega,
  \]
  we can write
  \[
    \int_{-\infty}^{\infty}\sin(2\pi f_{0}t)e^{i\omega t} = \int_{-\infty}^{\infty}\frac{e^{2\pi if_{0}t}-e^{-2\pi if_{0}t}}{2i}e^{i\omega t}dt
  \]
  \[
    = \frac{1}{2i}\int_{-\infty}^{\infty}e^{it(\omega + 2\pi f_{0})}dt - \frac{1}{2i}\int_{-\infty}^{\infty}e^{it(\omega - 2\pi f_{0})}dt
    = \pi i\qty[\delta(\omega - 2\pi f_{0}) - \delta(\omega + 2\pi f_{0})]
  \]
  Accordingly,
  \[
    \tilde{x}(\omega) = 2\pi i\qty[\delta(\omega - 2\pi f_{0}) - \delta(\omega + 2\pi f_{0})]\sum_{n = -\infty}^{\infty}\frac{2}{n}\sin(n/2)
  \]
  Taking $n = 0$ to have the value of the functional limit, 1, the sum appears to converge empirically, in fact to $2\pi$:
  \inputminted{scheme}{sum.scm}
  Symbolic tools do have an exact value for this, clearly by way of the complex plane; it'd be interesting to work out an exact solution,
  but I'm satisfied knowing my formula above is very likely well-defined.
\end{proof}

\newpage

\begin{plm}
  Find the paper from the first detection of gravitational waves, GW150914\_065416,
  and download 4096 seconds of data sampled at 4kHz for LIGO Livingston and Hanford detectors L1 and H1 (files are ~170 MB each).
  Using Matlab programs posted in Moodle for class (or if you prefer, adapting those to Python),
  calculate and plot the power spectral density of each time series using 2, 8 and 32 averages.
  Describe the differences between L1 and H1 PSDs, and the differences when using a different number of averages.
\end{plm}

\begin{proof}
  I implemented the power spectral density estimate in Haskell for fun.
  \inputminted[mathescape]{haskell}{GW150914.hs}
  \newpage
  The output can be fed into gnuplot.
  For the Livingston data, a very high-power low-frequency data point was skewing the observation, so it was suppressed to generate a readable plot.
  \inputminted{gnuplot}{p2.plt}
  Resulting in
  \begin{center}
    \includegraphics{h1s2.png}
  \end{center}
  \begin{center}
    \includegraphics{h1s8.png}
  \end{center}
  \begin{center}
    \includegraphics{h1s32.png}
  \end{center}
  \begin{center}
    \includegraphics{l1s2.png}
  \end{center}
  \begin{center}
    \includegraphics{l1s8.png}
  \end{center}
  \begin{center}
    \includegraphics{l1s32.png}
  \end{center}
  The Livingston data appears to have a much lower actual spectral density for a given frequency, but the contour is the same;
  fewer averages appears to ``smear out'' the spectrum, consistent with the uncertainty principle for Fourier series.
\end{proof}

\begin{plm}
  Using the Fluctuation-dissipation theorem, we proved that the power spectral density of
  thermal fluctuations of a harmonic oscillator with viscous damping coefficient $\gamma$ is given by
  \[
    x^{2}(\omega) = \frac{4k_{B}T\gamma}{k}\frac{1}{\qty(1 - (\omega / \omega_{0})^{2})^{2} + \gamma^{2}\omega^{2}}
  \]
  Integrate this expression over frequency to calculate the average of $x^{2}(t)$ (mean square, or $x^{2}_{rms}$),
  and write your result as the expression of the equipartition theorem.
\end{plm}

\begin{proof}
  We have
  \[
    \int_{-\infty}^{\infty}x^{2}(\omega)d\omega
    = \frac{4k_{B}T\gamma}{k}\int_{-\infty}^{\infty}\frac{d\omega}{\qty(1-(\omega/\omega_{0})^{2})^{2}+\gamma^{2}\omega^{2}}
  \]
  \[
    = \frac{4k_{B}T\gamma}{k}\int_{-\infty}^{\infty}\frac{\omega_{0}^{4}d\omega}{\omega_{0}^{4}
      - (\omega_{0}^{4}\gamma^{2}-2\omega_{0}^{2})\omega^{2} + \omega^{4}}
  \]
  The partial fraction decomposition of the integrand isn't hard:
  \[
    \frac{1}{a-bx+x^{2}} = \frac{1}{(x + r_{1})(x + r_{2})} = \frac{A}{x + r_{1}} + \frac{B}{x + r_{2}}
    \Leftrightarrow 1 = A(x+r_{2}) + B(x + r_{1}) \Rightarrow A+B = 0, r_{2}-r_{1} = \frac{1}{A}
  \]
  The roots of this polynomial are given by
  \[
    \omega^{2} = \frac{(\omega_{0}^{4}\gamma^{2} - 2\omega_{0}^{2})^{2} \pm \sqrt{(\omega_{0}^{4}\gamma^{2}-2\omega_{0}^{2})^{2}
        - 4\omega_{0}^{4}}}{2}
  \]
  and their difference is
  \[
    r_{2} - r_{1} =\sqrt{(2\omega_{0}^{2}-\omega_{0}^{4}\gamma^{2})^{2} - 4\omega_{0}^{4}} = \omega_{0}^{3}\sqrt{\gamma^{2}\omega_{0}^{2}-4\gamma}
  \]
  Accordingly, the integral is
  \[
    = \frac{4k_{B}T\gamma\omega_{0}^{4}}{k\omega_{0}^{3}\sqrt{\gamma^{2}\omega_{0}^{2}-4\gamma}}
    \qty(\int_{-\infty}^{\infty}\frac{d\omega}{\omega^{2} + r_{1}}
    - \int_{-\infty}^{\infty}\frac{d\omega}{\omega^{2} + r_{2}})
  \]
  \[
    = \frac{4k_{B}T\gamma\omega_{0}^{4}}{k\omega_{0}^{3}\sqrt{\gamma^{2}\omega_{0}^{2}-4\gamma}}
    \qty(\eval{\tan^{-1}\qty(\frac{\omega}{r_{1}})}_{-\infty}^{\infty} - \eval{\tan^{-1}\qty(\frac{\omega}{r_{2}})}_{-\infty}^{\infty})
  \]
  \[
    = \frac{4k_{B}T\gamma\omega_{0}^{4}}{k\omega_{0}^{3}\sqrt{\gamma^{2}\omega_{0}^{2}-4\gamma}}
    \qty(\frac{\pi\sgn(r_{1})}{2\sqrt{r_{1}}}+\frac{\pi\sgn(r_{1})}{2\sqrt{r_{1}}}
    - \frac{\pi\sgn(r_{2})}{2\sqrt{r_{2}}} - \frac{\pi\sgn(r_{2})}{2\sqrt{r_{2}}})
  \]
  \[
    = \frac{4\pi k_{B}T\gamma\omega_{0}}{k\sqrt{\gamma^{2}\omega_{0}^{2}-4\gamma}}
    (\sgn(r_{1}) - \sgn(r_{2}))
  \]
  In any case of the signs, the magnitude of the integral is either zero or
  \[
    = \frac{4\pi k_{B}T\gamma\omega_{0}}{k\sqrt{\gamma^{2}\omega_{0}^{2}-4\gamma}}
  \]
\end{proof}

\begin{plm}
  Consider a simple pendulum with its suspension point excited by seismic noise.
  Assume the resonance pendulum frequency is \SI{0.7}{Hz} with $Q=2$ (this is achieved using active damping),
  and the seismic noise is the amplitude spectral density measured at a LIGO Livingston seismometer on September 17, 2:00 UTC,
  available in Moodle for this homework.
  Using Matlab or your favorite program, plot the amplitude spectral density of the pendulum displacement,
  and calculate (numerically) the the rms motion between \SI{0.01}{Hz} and \SI{100}{Hz}
end{plm}

\begin{proof}
  The amplitude spectral density, the square root of the power spectral density, is a proxy for the Fourier transform.
  Accordingly, the amplitude spectral density of displacement, the inverse FT of the product of the FTs of the forcing signal and response signal,
  can be computed as
  \inputminted[mathescape]{haskell}{seismic.hs}
  \newpage
  Graphing this with gnuplot,
  \inputminted{gnuplot}{p4.plt}
  results in
  \begin{center}
    \includegraphics{seismic.png}
  \end{center}

\end{proof}
\end{document}
