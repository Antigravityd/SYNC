\documentclass{article}

\usepackage[letterpaper]{geometry}
\usepackage{amsmath}
\usepackage{amsfonts}
\usepackage{amssymb}

\title{3355 Exam 4}
\author{Duncan Wilkie}
\date{22 November 2021}

\DeclareMathOperator{\Cov}{Cov}
\begin{document}

\maketitle

\section*{1a}
The PDF is
\[f(x)=\begin{cases}\lambda e^{-\lambda x} & x\geq 0 \\ 0 & x < 0\end{cases}\]
\[=\begin{cases}\frac{e^{-x/4}}{4} & x\geq 0 \\ 0 & x < 0\end{cases}\]

\section*{1b}
The expectation is
\[E(X)=\int_\mathbb{R}xf(x)dx=\int_\mathbb{R}x\frac{e^{-x/4}}{4}dx
  =-x{e^{-x/4}}\bigg|_{0}^\infty+\int_\mathbb{R}e^{-x/4}dx = -4e^{-x/4}\bigg|_0^\infty=4\]

\section*{1c}
The CDF for the random variable is
\[F(x)=\begin{cases}
    1-e^{- x/4} & x\geq 0 \\
    0 & x < 0
  \end{cases}\]
The probability in question is
\[P(X > 2E) =1-F(2E)=1-F(8)=1-e^{-2}\approx0.865\]

\section*{2}
The PDF is the normal distribution with the given parameters:
\[f(x)=\frac{1}{\sigma\sqrt{2\pi}}\exp\left[ -\frac{(x-\mu)^2}{2\sigma^2} \right]=\frac{1}{100\sqrt{2\pi}}\exp\left[ -\frac{(x-82)^2}{20,000} \right]\]
The probabilities, since for continuous distributions the left and right limits are equal, are given in terms of the normal CDF as
\[P(X\geq 90)=1-F(90)\]
\[P(80\leq X< 90)=F(90)-F(80)\]
If we standardize these values, we obtain $Z$-scores $90\mapsto \frac{90-82}{100}=.08$ and $80\mapsto \frac{80-82}{100}=-.02$. Going to the standard normal table, we find $F(90)=0.5319$ and $F(80)=0.4920$. The desired probabilities become then
\[P(X\geq 90)=1-0.5319=0.4681=46.81\%\]
and
\[P(80\leq X<90)=0.5391-0.4920=0.0471=4.71\%\]

\section*{3a}
By definition, the integral over all variables must be one, so
\[\int_0^1\int_0^1c(x+2y)dxdy=\int_0^1\left(\frac{c}{2}+2cy \right)dy=\frac{c}{2}+c=\frac{3c}{2}=1\Rightarrow c=\frac{2}{3}\]

\section*{3b}
\[E(XY)=\int_0^1\int_0^1xy\frac{2}{3}\left(x+2y  \right)dxdy=\frac{2}{3}\int_0^1\left( \frac{y}{3}+y^2 \right)dy=\frac{2}{3}\left( \frac{1}{6}+\frac{1}{3} \right)=\frac{1}{3}\]

\section*{3c}
\[f_X(x)=\frac{2}{3}\int_0^1\left( x+2y \right)dy=\frac{2}{3}\left( x+1 \right)\]
and
\[f_Y(y)=\frac{2}{3}\int_0^1\left( x+2y \right)dx=\frac{2}{3}\left( \frac{1}{2}+2y \right)\]

\section*{3d}
We first calculate the expectations of each variable from the marginal densities above:
\[E(X)=\int_0^1xf_X(x)dx=\frac{2}{3}\int_0^1x\left( x+1 \right)dx=\frac{2}{9}+\frac{1}{3}=\frac{5}{9}\]
\[E(Y)=\int_0^1yf_Y(y)dy=\frac{2}{3}\int_0^1y\left( \frac{1}{2}+2y \right)dy=\frac{1}{6}+\frac{4}{9}=\frac{11}{18}\]
We now can compute
\[\Cov(X, Y)=E(XY)-E(X)E(Y)=\frac{1}{3}-\frac{5}{9}\frac{11}{18}=\frac{-1}{162}\]


\section*{4a}
We first need the marginal density for $y$:
\[f_Y(y)=\int_0^\infty f(x,y)dx=-3e^{-3x}\bigg|_0^\infty=3\]
The conditional PDF is by definition
\[f_{X|Y}(x|y)=\frac{f(x,y)}{f_Y(y)}=\begin{cases}3e^{-3x} & x,y>0 \\ 0 & \textrm{otherwise}\end{cases}\]

\section*{4b}
The expectation is that of the above PDF:
\[E(X|Y=y)=\int_0^\infty 3xe^{-3x}dx=-xe^{-3x}\bigg|_0^\infty+\int_0^\infty e^{-3x}dx=-\frac{e^{-3x}}{3}\bigg|_0^\infty= \frac{1}{3}\]
This is ordinarily a function of $y$, but there was no such dependence in the joint PDF.


\end{document}
%%% Local Variables:
%%% mode: latex
%%% TeX-master: t
%%% End:
