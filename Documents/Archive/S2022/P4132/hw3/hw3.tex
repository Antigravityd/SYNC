\documentclass{article}

\usepackage[letterpaper]{geometry}
\usepackage{siunitx}
\usepackage{amsmath}
\usepackage{amssymb}

\title{4132 HW 3}
\author{Duncan Wilkie}
\date{8 February 2022}

\begin{document}

\maketitle

\section*{7.28a}
The energy in the solenoid is
\[E=\frac{1}{2}LI^2=\frac{1}{2}\mu_0n^2l\pi R^2I^2\]
\section*{7.28b}
The vector potential of the solenoid was found in the book to be
\[\vec{A}=\begin{cases} \frac{1}{2}\mu_0nIs\hat{\phi} & s \leq R \\ \frac{1}{2}\mu_0nI\frac{R^2}{s}\hat{\phi} & s > R\end{cases}\]
By the formulae mentioned in the problem for work where the integral is over a loop at the surface of the solenoid,
\[W_0=\frac{1}{2}I\int\vec{A}\cdot d\vec{l}=\frac{1}{2}I\int\frac{\mu_0nI}{2}R\hat{\phi}\cdot d\vec{l}=\frac{\pi}{2}\mu_0nI^2R^2\]
is the work due to one turn. Over $l$, there are $nl$ turns, so the total work is
\[W=\frac{1}{2}\mu_0n^2l\pi R^2I^2\]

\section*{7.28c}
The magnetic field inside the solenoid is
\[\vec{B}=\mu_0nI\hat{z}\]
and zero outside, so the work (approximating the field to be zero outside the length of consideration) is
\[W=\frac{1}{2\mu_0}\int_0^{l}\int_0^{2\pi}\int_0^R(\mu_0nI)^2\hat{z}rdrd\theta dz=\frac{1}{2}\mu_0n^2l\pi R^2 I^2\]

\section*{7.28d}
The vector potential and magnetic fields are given by their formulae in the previous two problems, so
\[\vec{A}\times\vec{B}=\begin{cases}\frac{1}{2}\mu_0^2n^2I^2s\hat{r} & s \leq R \\ 0 & s > r\end{cases}\]
The energy stored is then
\[W=\frac{1}{2\mu_0}\left[ \int_0^l\int_0^{2\pi} \int_a^R \mu_0^2n^2I^2 rdrd\theta dz-\left( -\int_0^l\int_0^{2\pi}\frac{1}{2}\mu_0^2n^2I^2a (a d\theta dz)\right)\right]\]
\[=\frac{1}{2\mu_0}\left[ \pi l\mu_0^2n^2I^2(R^2-a^2)-\left( -\pi l\mu_0^2 n^2I^2a^2 \right) \right]=\frac{1}{2}\mu_0n^2l\pi R^2I^2\]
\section*{7.30}
The magnetic field inside the cable is by Amp\`ere's law
\[\int_C\vec{B}\cdot
  d\vec{l}=\mu_0I_{enc}\Leftrightarrow 2\pi sB=\mu_0\frac{\pi s^2}{\pi R^2}I\Leftrightarrow B=\frac{\mu_0Is}{2\pi R^2}\]
Outside the cable, the field is zero since the net enclosed current is zero.
The energy stored in the magnetic field of the cable over a length $l$ is then
\[W=\frac{1}{2\mu_0}\int_VB^2dV=\frac{1}{2\mu_0}\int_0^l\int_0^{2\pi}\int_0^R\frac{\mu_0^2I^2s^2}{4\pi^2 R^4} sdsd\theta dz=\frac{\mu_0lI^2}{4\pi R^4}\left( \frac{s^4}{4}\bigg|_0^R \right)=\frac{\mu_0lI^2}{16\pi}\]
This is equal to $\frac{1}{2}LI^2$, so the inductance is
\[L=\frac{\mu_0l}{8\pi}\]
The inductance per unit length is
\[L'=\frac{\mu_0}{8\pi}\]
\section*{7.34}
There is no current flowing through the gap, so any magnetic field must be due to changes in the electric field. The capacitance of a parallel-plate capacitor yields
\[C=\frac{q}{V}=\frac{\epsilon_0 A}{d}\Leftrightarrow V=\frac{wq}{\epsilon_0A}\Rightarrow \vec{E}=\frac{q}{\pi\epsilon_0a^2}\]
directed along $\vec{I}$. Differentiating, the displacement current is
\[\vec{J}=\epsilon_0\frac{\partial \vec{E}}{\partial t}=\frac{\vec{I}}{\pi a^2}\Rightarrow \vec{\nabla}\times \vec{B}=\frac{\mu_0\vec{I}}{\pi a^2}\]
This is the form of Amp\`ere's law, and so we may transition to an integral form where we integrate over a loop of radius $s$ concentric with and parallel to the plates of the capacitor. The magnetic field will be in the $\hat{\phi}$ direction and should be constant over the loop by rotational symmetry, so
\[\int_C\vec{B}\cdot d\vec{l}=\mu_0I_{denc}\Leftrightarrow 2\pi sB=\mu_0\frac{I}{a^2}s^2 \Leftrightarrow B=\frac{\mu_0I}{2\pi}\frac{s}{a^2}\]
directed along $\hat{\phi}$.

\section*{7.36a}
The displacement current for the given field is
\[\vec{J}_d=\epsilon_0\frac{\partial E}{\partial t}=\frac{\mu_0\epsilon_0I_0\omega^2}{2\pi}\cos(\omega t)\ln\left( \frac{a}{s} \right)\hat{z}\]

\section*{7.36b}
The total displacement current is
\[I_d=\int\vec{J}_dd\vec{a}=\int_0^{2\pi}\int_0^a\frac{\mu_0\epsilon_0I_0\omega^2}{2\pi}\cos(\omega t)\ln\left( \frac{a}{s} \right)sdsd\theta=\mu_0\epsilon_0I_0\omega^2\cos(\omega t)\left( \frac{s^2}{2}\ln\left( \frac{a}{s} \right)\bigg|_0^a+\int_0^a\frac{s}{2} \right)\]
\[=\mu_0\epsilon_0I_0\omega^2\cos(\omega t)\left( 0-\frac{s^2}{2}\ln(a)\bigg|_0+\lim_{s\to 0}\frac{s^2}{2/\ln(s)}+\frac{s^2}{4}\bigg|_0^a\right) =\mu_0\epsilon_0I_0\omega^2\cos(\omega t)\frac{a^2}{4}\]

\section*{7.36c}
The current is $I_0\cos(\omega t)$, so
\[\frac{I_d}{I}=\mu_0\epsilon_0\omega^2\frac{a^2}{4}\]
Solving for frequency and using the given values,
\[\Leftrightarrow \omega = \sqrt{\frac{4I_d}{\mu_0\epsilon_0 a^2I}}=\sqrt{\frac{4}{(\SI{1.26e-6}{H/m})(\SI{8.85e-12}{F/m})(\SI{2e-3}{m})}(0.01)}=\SI{1.34}{GHz}\]
The non-angular frequency is only a factor of $2\pi$ less. So, unless you're designing CPUs the displacement current is nearly negligible.

\section*{7.38}
For the symmetry to be maintained between ``electric things'' and ``magnetic things,'' we go to the Lorentz force law
\[\vec{F}=q\left( \vec{E}+\vec{v}\times\vec{B}\right)\]
and swap the two concepts:
\[\vec{F}=q_m\left( \vec{B}+\vec{v}\times\vec{E} \right)\]
However, notice that in the Maxwell equations everything that involves ``moving magnetic things'' (the magnetic current and the change in the magnetic field) has opposite sign compared to the corresponding electric things. Therefore, it would make sense to modify the sign of the second term of the above equation to get
\[\vec{F}=q_m\left( \vec{B}-\vec{v}\times \vec{E} \right)\]

\end{document}



%%% Local Variables:
%%% mode: latex
%%% TeX-master: t
%%% End:
