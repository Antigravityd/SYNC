\documentclass{article}

\usepackage[letterpaper]{geometry}
\usepackage{amsmath}
\usepackage{amssymb}
\usepackage{siunitx}
\usepackage{graphicx}

\title{4132 HW 9}
\author{Duncan Wilkie}
\date{14 April 2022}

\begin{document}

\maketitle

\section{}
If the light from the object at point $a$ originates at time $t_{0}$, it will arrive at Earth at time $t_{a}=t_{0}+d_{a}/c$;
the light from the object at point $b$, originating at time $t_{f}$ will arrive at time $t_{b}=t_{f}+d_{b}/c$.
The time interval is then
\[
  \Delta t=t_{b}-t_{a}=t_{f}-t_{0}+\frac{d_{b}-d_{a}}{c}
\]
The difference in distances to the Earth $d_{b}-d_{a}$ is just the negative length of the adjacent side to angle $\theta$:
\[
  d_{b}-d_{a}=v(t_{0}-t_{f})\cos\theta
\]
The apparent displacement is also easily computed trigonometrically:
\[
  \Delta s=v(t_{f}-t_{0})\sin\theta
\]
The apparent velocity is then
\[
  u=\frac{\Delta s}{\Delta t}=\frac{v(t_{f}-t_{0})\sin\theta}{t_{f}-t_{0}+\frac{v(t_{0}-t_{f})\cos\theta}{c}}
  =\frac{v\sin\theta}{1-\frac{v}{c}\cos\theta}
\]
This is maximized if
\[
  \frac{du}{d\theta}=0
  \Leftrightarrow \frac{\left( 1-\frac{v}{c}\cos\theta \right)v\cos\theta-v\sin\theta\left(\frac{v}{c}\sin\theta  \right)}
  {\left( 1-\frac{v}{c}\cos\theta \right)^{2}}=0
  \Leftrightarrow  \left( 1-\frac{v}{c}\cos\theta \right)v\cos\theta=v\sin\theta\left( \frac{v}{c}\sin\theta \right)
\]
\[
  \Leftrightarrow \cos\theta=\frac{v}{c}\sin^{2}\theta+\frac{v}{c}\cos^{2}\theta
  \Leftrightarrow \theta=\cos^{-1}\left( \frac{v}{c} \right)
\]
Evaluating $u$ at this maximum,
\[
  u=\frac{v\sin\left( \cos^{-1}\frac{v}{c} \right)}{1-\frac{v^{2}}{c^{2}}}
\]
By the definition of cosine, $\cos^{-1}(\frac{v}{c})$ is the angle of a triangle whose adjacent is $v$ and hypotenuse is $c$.
The opposite therefore has length $\sqrt{c^{2}-v^{2}}$, and the sine is the opposite over the hypotenuse, so
\[
  u=\frac{\frac{v}{c}\sqrt{c^{2}-v^{2}}}{1-\frac{v^{2}}{c^{2}}}
  =\frac{v}{\sqrt{1-\frac{v^{2}}{c^{2}}}}
\]
This becomes arbitrarily large as $v\to c$ (even through $v<c$), since the denominator diverges.

\section{}
From the length contraction formula, the problem is to find the free velocity such that
\[
  L_{VW}'=L_{VW}\sqrt{1-\frac{v_{VW}^{2}}{c^{2}}}
\]
is equal to
\[
  L_{Lincoln}'=L_{Lincoln}\sqrt{1-\frac{v_{Lincoln}^{2}}{c^{2}}}
\]
We are given $v_{VW}=.5c$; accordingly, $L_{VW}'=L_{VW}\sqrt{3}/2$.
We additionally know $L_{Lincoln}=2L_{VW}$, so we can set up
\[
  L_{VW}\sqrt{3}/2=2L_{VW}\sqrt{1-\frac{v_{Lincoln}^{2}}{c^{2}}}
  \Leftrightarrow \frac{3}{16}=1-\frac{v_{Lincoln}^{2}}{c^{2}}
  \Leftrightarrow v_{Lincoln}=\frac{\sqrt{13}}{4}c\approx 0.9c
\]

\section{}
Simultaneity of $A$ and $B$ in another reference frame $C$ at velocity $v$ is the statement that the zeroth coordinate of
\[
  \tilde{A}=\Lambda_{\nu}^{\mu}A^{nu}
\]
equals the zeroth coordinate of
\[
  \tilde{B}=\Lambda_{\nu}^{\mu}B^{\nu}
\]
From the matrix multiplication, these coordinates are
\[
  \tilde{A}^{0}=\gamma A^{0}-\gamma\beta A^{1}
  =\gamma ct_{A}-\gamma\beta x_{A}
\]
and
\[
  \tilde{B}^{0}=\gamma B^{0}-\gamma\beta B^{1}
  =\gamma ct_{B}-\gamma\beta x_{B}
\]
Equating these,
\[
  t_{A}\gamma c-x_{A}\gamma\beta
  =  \gamma ct_{B}-x_{B}\gamma\beta
\]
\[
  \Leftrightarrow t_{A}-t_{B}=(x_{A}-x_{B})\frac{\gamma\beta}{\gamma c}
  \Leftrightarrow v=c^{2}\frac{t_{A}-t_{B}}{x_{A}-x_{B}}
\]

\section{}
The proper velocity in a single dimension may be written in a form solvable for the ordinary velocity:
\[
  \eta=\frac{u}{\sqrt{1-u^{2}/c^{2}}}=\frac{1}{\sqrt{1/u^{2}-1/c^{2}}}
  \Leftrightarrow \frac{1}{u^{2}}=\frac{1}{\eta^{2}}+\frac{1}{c^{2}}
  \Leftrightarrow u=\frac{1}{\sqrt{\frac{1}{\eta^{2}}+\frac{1}{c^{2}}}}
  =\frac{\eta}{\sqrt{1+\frac{\eta^{2}}{c^{2}}}}
\]
\[
  =\frac{\SI{4e8}{m/s}}{\sqrt{1+\frac{(4c/3)^{2}}{c^{2}}}}
  =\SI{2.4e8}{m/s}
\]
This is under the ordinary velocity limit, so no law was violated.

\section{}
The relativistic kinetic energy of such a particle is
\[
  E_{kin}=E-mc^{2}=\frac{mc^{2}}{\sqrt{1-v^{2}/c^{2}}}-mc^{2}
\]
which, by assumption, equals $nmc^{2}$.
Solving for the ordinary velocity,
\[
  nmc^{2}=\frac{mc^{2}}{\sqrt{1-v^{2}/c^{2}}}-mc^{2}
  \Leftrightarrow n+1=\frac{1}{\sqrt{1-v^{2}/c^{2}}}
  \Leftrightarrow \frac{1}{(n+1)^{2}}=1-\frac{v^{2}}{c^{2}}
\]
\[
  \Leftrightarrow v^{2}=c^{2}\left( 1-\frac{1}{(n+1)^{2}} \right)
  \Leftrightarrow v=c\sqrt{1-\frac{1}{(n+1)^{2}}}
\]


\end{document}
%%% Local Variables:
%%% mode: latex
%%% TeX-master: t
%%% End:
