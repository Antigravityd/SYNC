\documentclass{article}

\usepackage[letterpaper]{geometry}
\usepackage{amsmath}
\usepackage{siunitx}
\usepackage{graphicx}

\DeclareMathOperator{\sech}{sech}

\title{4261 HW 8}
\author{Duncan Wilkie}
\date{?}

\begin{document}

\maketitle

\section*{19.1a}
Paramagnetic atoms tend to have net magnetic moment $J\neq 0$ which is able to align with an external magnetic field,
and diamagnetic atoms tend to have no magnetic moment, in which case Larmour diamagnetism occurs.

\section*{19.1b}
The electronic structure of each of these atoms is S:[Ne]$3s^23p^4$, V:[Ar]$4s^{2}3d^{3}$, Zr:[Kr]$5s^{2}4d^{2}$, Xe:[Xe],
Dy:[Xe]$6s^{2}4f^{10}$
Sulfur therefore has 4 electrons in a $p$-shell, and three possible $l_{z}$ states in the valence shell.
By Hund's first rule, the electrons fill with one spin-up in each of the $l_{z}=-1,0,1$ states,
and by Hund's second rule there is one spin-down in the $l_{z}=1$ state.
This yields $L=1$, $S=1$, and since the shell is more than half-filled, $J=L+S=2$.
Vanadium, with three electrons in a $d$-shell, has five possible $l_{z}$ states,
and they all fill spin-up by Hund's first rule as $l_{z}=2,1,0$.
The total is then $L=3$ and $S=\frac{3}{2}$, and since the shell is less than half-filled, the total angular momentum is $J=\frac{3}{2}$.
Zirconium has 2 electrons in a $d$ shell, which has five possible $l_{z}$ states, that by Hund's first rule fill spin-up as $l_{z}=2,1$.
This corresponds to $L=3$, $S=1$, which, since the shell is less than half full, yields a total angular momentum $J=2$
Since xenon is a noble gas, all the shells are filled, and $L=S=J=0$.
Dysprosium, with 10 electrons in an $f$ shell, fills each of the seven spin-up $l_{z}$ states and the three highest spin-down
as $l_{z}=3,2,1$.
This results in $L=6$, $S=2$ and, since the shell is more than half filled, the total angular momentum is $J=L+S=8$.

\section*{19.4a}
The electronic structure of Manganese is Mn:[Ar]$4s^{2}3d^{5}$.
The $d$ shell has five $l_{z}$ states, which are filled each spin-up according to Hund's first rule, implying $L=0$, $S=\frac{5}{2}$, and
$J=L+S=\frac{5}{2}$.
The paramagnetic contribution to the Curie susceptibility for a is
\[
  \chi=\frac{n\mu_{0}(\tilde{g}\mu_{B})^{2}}{3}\frac{J(J+1)}{k_{B}T}
\]
Evaluating this for $J=\frac{5}{2}$, $\tilde{g}=2$ (using the definition of $\tilde{g}$ and the fact that $L=0$ and $S=J$),
$T=\SI{2000}{K}$, and $n=\frac{N}{V}=\frac{P}{k_{B}T}$ (by the ideal gas law),
\[
  \chi=\frac{4\mu_{0}P\mu_{B}^{2}}{3}\frac{J(J+1)}{(k_{B}T)^{2}}
  =\frac{4(\SI{1.26e-6}{H/m})(\SI{e5}{Pa})(\SI{9.27e-24}{J/T})^{2}}{3}\frac{(5/2)(7/2)}{[(\SI{1.38e-23}{J/K})(\SI{2000}{K})]^{2}}
\]
\[
  =\SI{1.66e-7}{}
\]

\section*{19.4b}
The Larmour susceptibility for a noble gas is
\[
  \chi_{Larmour}=-\frac{Zne^{2}\mu_{0}\langle r^{2}\rangle}{6m}
\]
where $Z$ is the atomic number of the noble gas and $n$ is the density of the gas.
In the case of the filled orbitals of manganese, we can take $Z=20$, since manganese has a valence of five.
This yields, using the same values as above and presuming the radius to be an \r{A}ngstrom,
\[
  \chi_{Larmour}=-\frac{20(\SI{e5}{Pa})(\SI{1.6e-19}{C})^{2}(\SI{1.26e-6}{H/m})(\SI{e-10}{m})^{2}}
  {6(\SI{9.12e-26}{kg})(\SI{1.38e-23}{J/K})(\SI{2000}{K})}
  =\SI{-4.27e-14}{}
\]
This is insignificant in comparison to the paramagnetic contribution.

\section*{19.6a}
From the free spin 1/2 Hamiltonian, the partition function is
\[
  Z=e^{-\beta g\mu_{B}B/2}+e^{\beta g\mu_{B}B/2}=2\cosh\left( \beta g\mu_{B}B/2 \right)
\]
The Helmholtz free energy is then
\[
  F=-k_{B}T\log Z=-k_{B}T\log\left[ 2\cosh\left( \beta g\mu_{B}B/2 \right) \right]
\]
The magnetic moment is consequently
\[
  m=-\frac{\partial F}{\partial B}=-k_{B}T\frac{1}{2\cosh\left( \beta g\mu_{B}B/2 \right)}\cdot
  \left[-2\sinh\left( \beta g\mu_{B}B/2 \right)\cdot \beta g\mu_{B}/2\right]
  =\frac{g\mu_{B}}{2}\tanh(\beta g\mu_{B}B/2)
\]
The magnetization is this per unit volume (where $N$ is the number of spins in the volume):
\[
  M=mN/V
\]
The susceptibility is, using the fact $\tanh x\approx x$ as $x\to 0$ and taking a typically-permeable material so that $H\approx\mu_{0}B$,
\[
  \chi=\lim_{H\to 0}\frac{\partial M}{\partial H}=\frac{N\mu_{0}\mu_{B}^{2}}{k_{B}TV}
\]

\section*{19.6b}
We have
\[
  U=\frac{d}{d\beta}\log Z=\frac{d}{d\beta}\log\left[ 2\cosh\left( \beta g\mu_{B}B/2 \right) \right]
  =\frac{g\mu_{B}B}{2}\tanh(\beta g\mu_{B}B/2)
\]
and
\[
  C=\frac{\partial U}{\partial T}=\frac{g\mu_{B}B}{2}\sech^{2}(\beta g\mu_{B}B/2)\left( \frac{g\mu_{B}B/2}{k_{B}T^{2}} \right)
\]

\section*{20.1a}
Presuming the electrons to be all spin-up, the terms of the sum acting on such a state will be, using
$\frac{1}{2}(S_{i}^{+}S_{j}^{-}+S_{i}^{-}S_{j}^{+})+S_{i}^{z}S_{j}^{z}$,
\[
  -J\vec{S}_{i}\cdot\vec{S}_{j}=-J\vec{S}_{i}^{z}S_{j}^{z}
\]
since $S^{+}$ yields zero when acting on spin-up states.
The Hamiltonian acting on this spin-up state is therefore
\[
  \mathcal{H}=-\frac{1}{2}nzJS^{2}
\]
where $n$ is the number of electrons and $z$ is the number of nearest neighbors to any given electron.
Since this is a constant, this is an eigenstate of the Hamiltonian.

\section*{20.1b}
Using the same formula as above, if the spins are anti-aligned, some of the values of the first term will be non-zero when acting
on a state, which results in a non-eigenstate.

\section*{20.2a}
The Hamiltonian for such a system may be written as
\[
  \mathcal{H}=-\frac{1}{2}J(\vec{S}_{1}\cdot \vec{S}_{2}+\vec{S}_{2}\cdot \vec{S}_{3}+\vec{S_{1}}\cdot\vec{S}_{3})
\]
Noting that
\[
  (\vec{S}_{1}+\vec{S}_{2}+\vec{S}_{3})^{2}=\vec{S}_{1}\cdot\vec{S}_{1}+2\vec{S}_{1}\cdot\vec{S}_{2}+2\vec{S}_{1}\cdot\vec{S_{3}}
  +\vec{S}_{2}\cdot\vec{S}_{2}+2\vec{S}_{2}\cdot\vec{S}_{3}+\vec{S}_{3}\cdot\vec{S}_{3}
\]
Since $\vec{S}_{i}\cdot\vec{S}_{i}=S^{2}=\textrm{constant}$, this can be written as
\[
  \mathcal{H}=-\frac{1}{4}(\vec{S}_{1}+\vec{S}_{2}+\vec{S}_{3})^{2}+\textrm{constant}
\]
this is minimized whenever the term inside the square is minimized, i.e. when
\[
  \vec{S}_{1}+\vec{S}_{2}+\vec{S}_{3}=0
\]
This is true for three vectors separated by $120^{\circ}$.

\section*{20.2b}
An infinite triangular lattice's ground state would have this same result tiled infinitely.
This only works if the lattice cells are oriented such that each triangle has one of each orientation of atom at its vertex.

\section*{20.5a}
A much less convoluted way to do this is to consider a single atom in the chain and use the combination of partition functions
for distinguishable systems.
For the single atom, there are two possible states: one with energy $J$, and one with energy $-J$, corresponding to $S_{z}=+S$ and
$S_{z}=-S$.
The partition function for the atom is then
\[
  Z_{atom}=\sum_{s}e^{-\beta E(s)}=e^{\beta J}+e^{-\beta J}=2\cosh(\beta J)
\]
Combining the identical partition functions of the $N$ independent, distinguishable (due to their order) systems in the chain yields
\[
  Z=Z_{atom}^{N}=2^{N}\cosh^{N}(\beta J)
\]

The Helmholtz free energy is then
\[
  F=-k_{B}T\log Z=-k_{B}TN\log 2-k_{B}TN\log\cosh^{N}(\beta J)
\]
Since $\cosh$ is everywhere positive, this is a continuous function, implying no phase transition exists.

\section*{20.5b}
One may write the definition of the expectation of a statistical mechanical quantity in order to see the probability distribution,
\[
  \langle X\rangle = \frac{1}{Z}\sum_{s}X(s)e^{-\beta X(s)}=\sum_{s}X(s)P_{X}(s)
\]
it is clear the probability of a single atom being in one particular state is
\[
  \frac{e^{\beta J}}{e^{\beta J}+e^{-\beta J}}
\]
The probability that $M-1$ other atoms have the same state as this first one (which can be either state) is
\[
  \left(    \frac{e^{\beta J}}{e^{\beta J}+e^{-\beta J}}\right)^{M-1}
\]
The term in the parentheses is obviously less than one, and so the probability decays exponentially with $M$.
In the limit $T\to 0$ ($\beta\to \infty$), the term inside the parentheses approaches one, and so the atoms tend to align.
\end{document}
%%% Local Variables:
%%% mode: latex
%%% TeX-master: t
%%% End:
