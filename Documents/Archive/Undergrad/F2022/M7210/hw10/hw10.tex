\documentclass{article}

\usepackage[letterpaper]{geometry}
\usepackage{tgpagella}
\usepackage{amsmath}
\usepackage{amssymb}
\usepackage{amsthm}
\usepackage{tikz}
\usepackage{minted}
\usepackage{physics}
\usepackage{siunitx}

\sisetup{detect-all}
\newtheorem{plm}{Problem}
\DeclareMathOperator{\Tor}{Tor}
\DeclareMathOperator{\Hom}{Hom}

\title{7210 HW 10}
\author{Duncan Wilkie}
\date{15 November 2022}

\begin{document}

\maketitle

\begin{plm}[D\&F 10.1.8]
  An element $m$ of the $R$-module $M$ is called a \textbf{torsion element} if $rm = 0$ for some nonzero element $r \in R$.
  The set of torsion elements is denoted
  \[
    \Tor(M) = \{m \in M \mid rm = 0 \text{ for some nonzero } r \in R\}
  \]
  \begin{enumerate}
  \item Prove that if $R$ is an integral domain then $\Tor(M)$ is a submodule of $M$ (called the \textbf{torsion} submodule of $M$).
  \item Give an example of a ring $R$ and an $R$-module $M$ such that $\Tor(M)$ is not a submodule.
  \item Show that if $R$ has zero divisors then every nonzero $R$-module has torsion elements.
  \end{enumerate}
\end{plm}

\begin{proof}
  We must prove that $\Tor(M)$ is an additive subgroup closed under the action of ring elements: if $rm = 0$ and $sn = 0$ for $m, n \in M$
  and nonzero $r, s \in R$, then $rs \neq 0$ since this is an integral domain, and
  \[
    (rs)(m - n) = s(rm) - r(sn) = s(0) - r(0) = 0 \Rightarrow m - n \in \Tor(M).
  \]
  The additive identity is certainly an element of $\Tor(M)$.
  Further, if $rm = 0$ for some nonzero $r \in R$ then for all nonzero $s \in R$ one has that
  \[
    r(sm) = s(rm) = s(0) = 0 \Rightarrow sm \in \Tor(M).
  \]
  This proves $\Tor(M)$ is a submodule.

  Consider $\mathbb{Z} / 6\mathbb{Z}$ as a module over itself.
  The residue classes 2 and 3 are zero divisors: they multiply to the zero residue class.
  The difference $3 - 2 = 1$, however, is not a zero divisor as it is the multiplicative identity; accordingly,
  the torsion of this module isn't closed under addition (note that the failure of the general argument is because $rs = 0$,
  since this fails to be an integral domain).

  Suppose $rs = 0$ for nonzero $r, s \in R$ being zero divisors presumed to exist.
  Then for any element $m$ of a module $M$ over $R$ one has
  \[
    0 = 0m = (rs)m = r(sm)
  \]
  so $sm$ is a torsion element of $M$.
\end{proof}

\begin{plm}[D\&F 10.1.15]
  If $M$ is a finite Abelian group then $M$ is naturally a $\mathbb{Z}$-module.
  Can this action be extended to make $M$ into a $\mathbb{Q}$-module?
\end{plm}

\begin{proof}
  Consider a finite Abelian group $G$ with a $\mathbb{Q}$-module action extending that of $\mathbb{Z}$.
  Then for any nonzero $a$ the element $\frac{1}{|G|} \cdot a$ has that the $|G|$-fold sum $\frac{1}{|G|} \cdot a + \cdots + \frac{1}{|G|} \cdot a$
  equals 1, as $\frac{|G|}{|G|} \cdot a = 1 \cdot a = a$ to agree with the $\mathbb{Z}$-action.
  No partial sum can be zero, as this would imply that some multiple of $\frac{1}{|G|}$ would be zero
  However, this implies that the order of 1 is more than $|G|$, which cannot happen.
\end{proof}

\begin{plm}[D\&F 10.2.6]
  $\Hom_{\mathbb{Z}}(\mathbb{Z} / n\mathbb{Z}, \mathbb{Z} / m\mathbb{Z}) \cong \mathbb{Z} / (n, m)\mathbb{Z}$.
\end{plm}

\begin{proof}
  Denote an element of the left ring by $\phi: \mathbb{Z} / n\mathbb{Z} \to \mathbb{Z} / m\mathbb{Z}$.
  Let $\varphi$ be the map taking each $\phi$ to $\phi(1)$.
  This is a homomorphism out of the $\Hom$-ring:
  \[
    \varphi(\phi \circ \phi') = (\phi \circ \phi')(1) = \phi(\phi'(1)) = \phi(\phi'(1) \cdot 1) = \phi(1)\phi'(1) = \varphi(\phi)\varphi(\phi')
  \]
  \[
    \varphi(\phi + \phi') = (\phi + \phi')(1) = \phi(1) + \phi'(1) = \varphi(\phi) + \varphi(\phi').
  \]
  This is injective, since
  \[
    \varphi(\phi) = \varphi(\phi') \Rightarrow \phi(1) = \phi'(1) \Rightarrow \phi(a) = \phi(a \cdot 1) = a\phi(1) = a\phi'(1) = \phi'(a);
  \]
  concomitantly, its kernel is trivial, and the domain is isomorphic to the image of $\varphi$.
  This image is the whole of $\mathbb{Z} / (n, m)\mathbb{Z}$: each value of $\phi(1)$ specifies a distinct, valid homomorphism $\phi$.
\end{proof}

\begin{plm}[D\&F 10.3.2]
  Assume $R$ is commutative.
  Prove that $R^{n} \cong R^{m}$ iff $n = m$, i.e. two free $R$-modules of finite rank are isomorphic iff they have the same rank.
\end{plm}

\begin{proof}
  If $n = m$, then the two are exactly equal, and isomorphic by the identity map.
  Suppose that $R^{n} \cong R^{m}$ by $\varphi$, and take $I$ a maximal ideal in $R$.
  By exercise 12 in the previous section,
  \[
    R^{n} / IR^{n} = \underbrace{R / IR \times R / IR \times \cdots \times R / IR}_{n \text{ times}}
  \]
  and
  \[
    R^{m} / IR^{m} = \underbrace{R / IR \times R / IR \times \cdots \times R / IR}_{m \text{ times}}
  \]
  These two modules are isomorphic: $IR^{m} \cong IR^{n}$ by the map taking $\sum_{\text{finite}}a_{i}m_{i}$ to $\sum_{\text{finite}}a_{i}\varphi(m_{i})$
  where $\varphi$ is the isomorphism between $R^{m}$ and $R^{n}$ (which is clearly an invertible module homomorphism by multiplicativity
  and invertibility of $\varphi$), and if the numerators and denominators are isomorphic, certainly the quotients are.
  For commutative rings as modules over themselves, ideals and quotients coincide as module- and ring-theoretic notions.
  Submodules are additive subgroups closed under multiplication on the left by arbitrary ring elements,
  and left ideals are two-sided in commutative rings; the quotient of modules is a quotient of Abelian groups with scalar multiplication defined
  on coset representatives satisfying the module identities, which induces a ring product on the cosets and vice versa,
  since all scalars represent some coset.
  The module $R / IR$ considered as a ring is isomorphic to a field, and, accordingly, is module-isomorphic to a field acting on itself:
  the submodule $IR$ is also a subring of $R$; as it's elements are of the form $\sum_{\text{finite}}a_{i}r_{i}$, and $a_{i}$ are elements of $I$,
  a (two-sided) ideal, it equals $I$, and since $I$ is maximal, the ring quotient yields a field.
  This proves that each of the products above is module-isomorphic to $F^{n}$ and $F^{m}$.
  If these are module-isomorphic to each other, then $n = m$. % TODO: check
\end{proof}

\begin{plm}[D\&F 10.3.9]
  An $R$-module $M$ is called \textbf{irreducible} if $M \neq 0$ and if $0$ and $M$ are the only submodules of $M$.
  Show that $M$ is irreducible iff $M \neq 0$ and $M$ is a cyclic module with any nonzero element as a generator.
  Determine all the irreducible $\mathbb{Z}$-modules.
\end{plm}

\begin{proof}
  Suppose $M$ is irreducible.
  Then, it is generated by any element: if for some nonzero candidate $a \in M$ there is some element $m \in M$ such that $m \neq ra$,
  then $Ra$ is a proper submodule of $M$.

  Conversely, suppose $M$ is cyclic with any nonzero element as a generator.
  Then, for any nonzero submodule $N$ of $M$ any nonzero element $a$ of the submodule (which must exist!) generates $M$,
  so for all $m \in M$ one has $m = ra$ for some $r \in R$, and since $N$ must be closed under scalar multiplication, this means $N = M$.
  This proves that $M$ is irreducible.

  We're looking for Abelian groups such that any nonzero element has that all elements are obtained as multiples of any of its elements.
  These are of course cyclic, so $\mathbb{Z} / n\mathbb{Z}$ for some $n$, but if $n$ is composite,
  then its factors can't be obtained as multiples of each other, e.g. in $\mathbb{Z} / 6\mathbb{Z}$, the multiples of 2 are 0, 2, 4
  and the multiples of 3 are 0, 3.
  If $n$ is prime, on the other hand, then $\mathbb{Z} / n\mathbb{Z}$ is a field and $ka \equiv b\pmod n$ is solvable for any $b$ by inverting $k$.



\end{proof}

\begin{plm}[D\&F 12.1.2]
  Let $M$ be a module over the integral domain $R$.
  \begin{itemize}
  \item Suppose that $M$ has rank $n$ and that $x_{1}, x_{2}, \ldots, x_{n}$ is any maximal set of linearly independent elements of $M$.
    Let $N = Rx_{1} + \cdots Rx_{n}$ be the submodule generated by $x_{1}, x_{2}, \ldots, x_{n}$.
    Prove that $N$ is isomorphic to $R^{n}$ and that the quotient $M / N$ is a torsion $R$-module (equivalently, the elements $x_{1}, \ldots, x_{n}$
    are linearly independent and for any $y \in M$ there is a nonzero element $r \in R$ such that $ry$ can be written as a linear combination
    $r_{1}x_{1} + \cdots + r_{n}x_{n}$ of the $x_{i}$).
  \item Prove conversely that if $M$ contains a submodule $N$ that is free of rank $n$ (i.e., $N \cong R^{n}$) such that the quotient $M / N$
    is a torsion $R$-module then $M$ has rank $n$.
  \end{itemize}
\end{plm}

\begin{proof}
  For every choice of $r_{1}, r_{2}, \ldots, r_{n}$ where at least one $r_{i} \neq 0$, one has $r_{1}x_{1} + r_{2}x_{2} + \cdots + r_{n}x_{n} \neq 0$.
  Accordingly, the value of the above sum is distinct for every selection of the $r_{i}$'s, as if some other selection $r_{i}'$ has the same value,
  then $(r_{1} - r_{1}')x_{1} + (r_{2} - r_{2}')x_{2} + \cdots + (r_{n} - r_{n}')x_{n} = 0$.
  As such, the map taking $(r_{1}, r_{2}, \ldots, r_{n}) \in R^{n}$ to $r_{1}x_{1} + r_{2}x_{2} + \cdots + r_{n}x_{n} \in N$ is injective;
  it is also surjective, since by definition $N$ consists of all sums of this form.
  For any element $y \in M$, if there is no nonzero $r$ such that one can write $ry = r_{1}x_{1} + r_{2}x_{2} + \cdots + r_{n}x_{n}$,
  then $x_{1}, x_{2}, \ldots, x_{n}, y$ is linearly independent by subtracting $ry$ from both sides of the antecedent equation
  and noting existence of additive inverses.

  Conversely, if $M$ has a submodule $N$ that is free of rank $n$ such that $M / N$ is a torsion $R$-module,
  then for any $y_{1}, y_{2}, \ldots, y_{n+1} \in M$ the fact that $M / N$ is a torsion module means that any $R$-linear combination
  $r_{1}y_{1} + r_{2}y_{2} + \cdots + r_{n+1}y_{n+1}$ has that there exists some $r \neq 0$ such that
  $r(r_{1}y_{1} + r_2y_2 + \cdots + r_{n+1}y_{n+1}) \in N$.
  Since $N$ is free of rank $N$, then $rr_{1}y_{1} + rr_{2}y_{2} + \cdots + rr_{n+1}y_{n+1} = a_{1}x_{1} + a_{2}x_{2} + \cdots + a_{n}x_{n}$
  for $x_{1}, x_{2}, \ldots, x_{n}$ a basis for $N$ and $a_{i} \in R$.
  $M$ and $N$ are submodules of those over the field of fractions of $R$ over the same Abelian group.
  In these vector spaces, one of which is naturally a subspace of the other, we can divide the equation we left off with by $r$,
  getting $r_{1}y_{1} + r_{2}y_{2} + \cdots + r_{n+1}y_{n+1} = \frac{a_{1}}{r}x_{1} + \frac{a_{2}}{r}x_{2} + \cdots + \frac{a_{n}}{r}x_{n}$,
  showing that the left linear combination is an element of the subspace.
  Since it's a list of more than $n$ vectors, the terms of the sum are linearly dependent in the $n$-dimensional vector space.
  The coefficients giving this linear dependence can be each multiplied by the product of all their denominators to yield an $R$-linear dependence.
  Therefore, the terms of the sum and so also $y_{i}$ are linearly dependent in $M$.
\end{proof}
\end{document}
