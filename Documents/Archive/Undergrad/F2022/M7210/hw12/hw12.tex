\documentclass{article}

\usepackage[letterpaper]{geometry}
\usepackage{tgpagella}
\usepackage{amsmath}
\usepackage{amssymb}
\usepackage{amsthm}
\usepackage{tikz}
\usepackage{minted}
\usepackage{physics}
\usepackage{siunitx}

\sisetup{detect-all}
\newtheorem{plm}{Problem}

\title{7210 HW 12}
\author{Duncan Wilkie}
\date{29 November 2022}

\begin{document}

\maketitle

\begin{plm}[D\&F 13.1.1]
  Show that $p(x) = x^{3} + 9x + 6$ is irreducible in $\mathbb{Q}[x]$.
  Let $\theta$ be a root of $p(x)$.
  Find the inverse of $1 + \theta$ in $\mathbb{Q}(\theta)$.
\end{plm}

\begin{proof}
  The first part is immediate from Eisenstein's criterion over $\mathbb{Z}[x]$:
  $p$ is a monic polynomial with $3$ dividing all non-leading coefficients and $3^{2} = 9$ not dividing its constant coefficient,
  so it is irreducible in $\mathbb{Q}[x]$.
  In the field $\mathbb{Q}[x]$, which is of course a PID, we can find multipliers satisfying the B\'ezout
  relation
  \[
    a(x)(1 + x) + b(x)(x^{3} + 9x + 6) = 1
  \]
  since the gcd of the two fixed factors is 1 because $p$ is irreducible.
  Equivalently,
  \[
    a(x)(1 + x) = 1 - b(x)(x^{3} + 9x + 6),
  \]
  implying in $\mathbb{Q}(\theta) \cong \mathbb{Q}[x] / (x^{3} + 9x + 6)$ that $a(\theta)(1 + \theta) = 1$.
  We apply the Euclidean algorithm to find $a$:
  \[
    x^{3} + 9x + 6 = (x^{2} - x + 10)(1 + x) - 4
  \]
  \[
    1 + x = \qty(\frac{1}{4} + \frac{x}{4})4 + 0
  \]
  so $4 \sim 1$ is indeed the gcd.
  We equivalently have
  \[
    1 = \qty(-\frac{1}{4})(x^{3} + 9x + 6) + \qty(-\frac{x^{2}}{4} + \frac{x}{4} + \frac{5}{2})(1 + x)
  \]
  so $(1 + \theta)^{-1} = -\frac{1}{4}\qty(\theta^{2} - \theta + 10)$.
\end{proof}

\begin{plm}[D\&F 13.5.6]
  Prove that $x^{p^{n} - 1} - 1= \prod_{\alpha \in \mathbb{F}^{\times}_{p^{n}}} (x - \alpha)$.
  Conclude that $\prod_{\alpha \in \mathbb{F}^{\times}_{p^{n}}}\alpha = (-1)^{p^{n}}$ so the product of the nonzero elements of a finite field is $+1$
  if $p = 2$ and $-1$ if $p$ is odd.
  For $p$ odd and $n = 1$ derive \textbf{Wilson's Theorem}: $(p - 1)! \equiv -1 \pmod p$
\end{plm}

\begin{proof}
  Since $\mathbb{F}_{p^n}$'s multiplicative group is of order $p^n-1$, then certainly $\alpha^{p^n - 1} = 1 \Leftrightarrow \alpha^{p^n - 1} - 1 = 0$
  for all $\alpha \in \mathbb{F}_{p^{n}}^{\times}$.
  Accordingly, $x^{p^{n} - 1} - 1$ has a root at every unit in $\mathbb{F}_{p^{n}}$, meaning it has a linear factor for every such unit;
  since the product of such factors is a polynomial of degree ${p^{n} - 1}$, there can be no additional factors.

  Evaluating this at $x = 0$,
  \[
    \prod_{\alpha \in \mathbb{F}_{p^{n}}^{\times}} -\alpha = -1 \Leftrightarrow (-1)^{p^{n} - 1} \prod_{\alpha \in \mathbb{F}_{p^{n}}^{\times}} \alpha = -1
    \Leftrightarrow
    \prod_{\alpha \in \mathbb{F}_{p^{n}}^{\times}} \alpha = (-1)^{-p^{n}} = (-1)^{p^{n}}
  \]
  since integer parity is preserved under negation.

  For $n = 1$, the fields are of the form $\mathbb{Z} / p\mathbb{Z}$, and the elements are equivalence classes $\bar{0}, \bar{1}, \ldots,
  \overline{p - 1}$.
  Accordingly, applying this result yields the product of all field elements $\overline{p - 1} \cdot \overline{p - 2} \cdots \bar{1}$
  being equal to $(\overline{-1})^{p}$, which for odd primes, is $-1 \pmod p$, proving Wilson's theorem.
\end{proof}

\begin{plm}[D\&F 13.5.8]
  Prove that $f(x)^{p} = f(x^{p})$ for any polynomial $f(x) \in \mathbb{F}_{p}[x]$.
\end{plm}

\begin{proof}
  For finite fields, the Frobenius endomorphism is in fact a homomorphism, meaning it distributes over products and sums, so
  \[
    (a_{0} + a_{1}x + \cdots + a_{n}x^{n})^{p} = a_{0}^{p} + a_{1}^{p}x^{p} + \cdots + a_{n}^{p}\underbrace{x^{p} \cdot x^{p} \cdots x^p}_{n\text{ times}}.
  \]
  A coefficient is either zero, in which case the equality of the two forms is immediate, or an element of $\mathbb{F}_{p}^{\times}$,
  which has order $p - 1$, meaning $a_{i}^{p} = a_{i}a_{i}^{p - 1} = a_{i}$, so the polynomial is
  \[
    a_{0} + a_{1}x + \cdots + a_{n}(x^{p})^{n} = f(x^{p}).
  \]

\end{proof}

\begin{plm}[D\&F 13.5.9]
  Show that the binomial coefficient $\binom{pn}{pi}$ is the coefficient of $x^{pi}$ in the expansion of $(1 + x)^{pn}$.
  Working over $\mathbb{F}_{p}$ show that this is the coefficient of $(x^{p})^{i}$ in $(1 + x^{p})^{n}$ and hence prove that
  $\binom{pn}{pi} \equiv \binom{n}{i} \pmod p$.
\end{plm}

\begin{proof}
  By the binomial theorem (which was proven in a general context on a previous homework),
  \[
    (1 + x)^{pn} = \sum_{i = 0}^{pn} \binom{pn}{i}x^{i}
  \]
  so the coefficient of $x^{pi}$ is $\binom{pn}{pi}$.
  Using the Frobenius endomorphism on $\mathbb{F}_{p}$,
  \[
    (1 + x)^{pn} = (1^{p} + x^{p})^{n} = (1 + x^{p})^{n},
  \]
  which, together with the fact that $x^{pi} = (x^{p})^{i}$, demonstrates that $\binom{pn}{pi}$ is the coefficient of $(x^{p})^{i}$
  in $(1 + x^{p})^{n}$.
  This latter term is, by the binomial theorem again,
  \[
    (1 + x^{p})^{n} = \sum_{i = 0}^{n} \binom{n}{i}x^{pi}.
  \]
  Comparing the coefficients of $x^{pi}$,
  \[
    \binom{pn}{pi} = \binom{n}{i}
  \]
  in $\mathbb{F}_{p}$, i.e.
  \[
    \binom{pn}{pi} = \binom{n}{i} \pmod p.
  \]
\end{proof}

\end{document}
