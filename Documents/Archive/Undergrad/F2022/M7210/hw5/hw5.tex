\documentclass{article}

\usepackage[letterpaper]{geometry}
\usepackage{tgpagella}
\usepackage{amsmath}
\usepackage{amssymb}
\usepackage{amsthm}

\newtheorem*{4.2.14}{4.2.14}
\newtheorem*{4.3.5}{4.3.5}
\newtheorem*{4.3.13}{4.3.13}
\newtheorem*{4.3.17}{4.3.17}
\newtheorem*{4.3.29}{4.3.29}
\newtheorem*{4.5.13}{4.5.13}

\title{7510 HW 5}
\author{Duncan Wilkie}
\date{27 September 2022}

\begin{document}

\maketitle

\begin{4.2.14}
  No finite group $G$ of composite order $n$ with the property that $G$ has a subgroup of order $k$ for each positive integer $k$ dividing $n$
  is simple.
\end{4.2.14}

\begin{proof}
  The smallest prime $p$ in the prime factorization of $n$ is neither 1 nor $n$, since neither are prime.
  $G$ has a proper subgroup of order $p$ by definition; Corollary 5 implies this subgroup is normal.
  Therefore, $G$ has a nontrivial normal subgroup, and is not simple.
\end{proof}

\begin{4.3.5}
  If the center of $G$ is of index $n$, then every conjugacy class has at most $n$ elements.
\end{4.3.5}

\begin{proof}
  The centralizer of any element and the center are normal subgroups, and the latter is a subgroup of the former.
  So, the third isomorphism theorem applies, and
  \[
    (G / Z(G)) / (C_{G}(s) / Z(G)) \cong G / C_{G}(s)
  \]
  The order of the right side is the number of elements of the conjugacy class containing $s$.
  The numerator of the left side has order $n$ by assumption; by Lagrange's theorem, we then have $|G : C_{G}(s)| \mid n$,
  implying the size of each conjugacy class $|G: C_{G}(s)| \leq n$.
\end{proof}

\begin{4.3.13}
  Find all finite groups with exactly 2 conjugacy classes.
\end{4.3.13}

\begin{proof}
  Consider a group $G$ with the above property.
  Every element of the center corresponds to a singleton conjugacy class, and the identity is always in the center.
  If there are more than 2 elements in the center, then that's more than 2 conjugacy classes.
  If there are exactly 2 elements in the center, then there can be no other elements in the group,
  as they would belong to some other conjugacy classes, and the only group of order 2 is $\mathbb{Z} / 2\mathbb{Z}$.
  If the only element of the center is the identity, and the other conjugacy class has some non-central representative $s$,
  the class equation yields
  \[
    |G| = |Z(G)| + |G : C_{G}(s)| = 1 + |G : C_{G}(s)|
  \]
  By Lagrange's theorem,
  \[
    |G| = 1 + \frac{|G|}{|C_{G}(s)|}
    \Leftrightarrow |G| = \frac{1}{1 - \frac{1}{|C_{G}(s)|}}
  \]
  The left side is integral, and if the right side is to be also,
  \[
    |C_{G}(s)| - 1 \mid |C_{G}(s)| \Leftrightarrow -1 \equiv 1 \pmod {|C_{G}(s)|}.
  \]
  The only modulus for which this holds is 2, so $|C_{G}(s)| = 2$.
  Accordingly, $|G| = 2$, so $G \cong \mathbb{Z} / 2\mathbb{Z}$.
  Then $G$ is Abelian and has 2 elements in its center, a contradiction.

  The only group satisfying this property is then $\mathbb{Z} / 2\mathbb{Z}$.
\end{proof}

\begin{4.3.17}
  Let $A$ be a nonempty set and let $X$ be any subset of $S_{A}$.
  Let
  \[
    F(X) = \{a \in A \mid \sigma(a) = a \textrm{ for all } \sigma \in X\}
  \]
  be the set of elements fixed by $X$.
  Correspondingly, $M(X) = A - F(X)$ are the elements moved by $X$.
  Let $D = \{\sigma \in S_{A} \mid |M(\sigma)| < \infty\}$.
  Then $D$ is a normal subgroup of $S_{A}$.
\end{4.3.17}

\begin{proof}
  First, we show it's a subgroup.
  Suppose $\sigma, \tau \in S_{A}$ have $|M(\sigma)| = n$ and $|M(\tau)| = m$.
  Then $|M(\sigma\tau)| \leq n + m$, because if $\sigma(a) = a$ ($a \not\in M(\sigma)$) and $\tau(a) = a$ ($a \not\in M(\tau)$) then
  $\sigma\tau(a) = a$  ($a \not\in M(\sigma\tau)$);
  equivalently, if it's moved by the composition, it must be moved by one of permutations, so $M(\sigma\tau)\subseteq M(\sigma)\cup M(\tau)$.
  Accordingly, $\sigma\tau \in D$.
  Secondly, $\sigma^{-1} \in D$, since $\sigma(a) = a \Leftrightarrow \sigma^{-1}(a) = a$,
  implying no element fixed by $\sigma$ can be moved by $\sigma^{-1}$ or $|M(\sigma^{-1})| \leq |M(\sigma)|$.

  We now show it's closed under conjugation.
  Elements of $D$ have cycle decompositions, of finite (although varying) length, since only elements of $M(\sigma)$ are written.
  By the argument in the text, which doesn't depend on properties of $S_{n}$ other than that cycle decompositions are finite,
  $\tau\sigma\tau^{-1}$ has a cycle decomposition given by $\tau$ applied to each element of the cycle decomposition of $\sigma$.
  This implies that $\tau\sigma\tau^{-1}$ moves a finite number of elements: those appearing in its cycle decomposition.
  Therefore, $D$ is normal.
\end{proof}

\begin{4.3.29}
  Let $p$ be a prime and let $G$ be a group of order $p^{\alpha}$.
  Then $G$ has a subgroup of order $p^{\beta}$ for every $\beta$ with $0 \leq \beta \leq \alpha $.
\end{4.3.29}

\begin{proof}
  If $\alpha = 0$, then $\beta = 0$, and $G = \langle 1 \rangle$ has a subgroup of order 1.
  This provides a base case.
  Suppose for induction that the theorem holds for all prime-power-order groups with exponent less than $\alpha$.
  Then it is only necessary to show that $G$ has a subgroup of order $p^{\alpha - 1}$,
  since that subgroup has subgroups of order $p^{\beta}$ for $0 \leq \beta \leq \alpha - 1$ by the induction hypothesis,
  which are subgroups of $G$ themselves.

  By Theorem 8, the center of $G$ is nontrivial; it must have order $p^{\gamma}$ for some $\gamma$ with $1 \leq \gamma \leq \alpha - 1$
  by Lagrange's theorem.
  Since the center is a normal subgroup, we can consider $G / Z(G)$, which has order $p^{\alpha - \gamma} < p^{\alpha}$.
  By induction, this has subgroups of all orders $p^{\beta}$ for $0 \leq \beta \leq \alpha - \gamma$,
  and by the lattice isomorphism theorem, these are in bijective correspondence with subgroups of $G$ containing $Z(G)$.
  This bijection has the property that for any $A,B \leq G$ that if $A \leq B$ then $|B : A| = |G / B : G  / A|$,
  so taking $B = G / Z(G)$ and $A$ to be the subgroup of order $p^{\alpha - \gamma - 1}$,
  the corresponding subgroup of $G$ has index $p$.
  By Lagrange's theorem applied to this subgroup, call it $H$,
  \[
    p = \frac{|G|}{|H|} \Leftrightarrow |H| = \frac{|G|}{p} = p^{\alpha - 1}
  \]
  so there does indeed exist a subgroup of $G$ of order $p^{\alpha - 1}$.
\end{proof}

\begin{4.5.13}
  All groups of order 56 have a normal Sylow $p$-subgroup for some prime $p$ dividing their order.
\end{4.5.13}

\begin{proof}
  Let $G$ be a group of order 56.
  56 factors as $2^{3}\cdot 7$, so suppose no normal subgroups exist among the Sylow 2-subgroups and Sylow 7-subgroups.
  By the third part of Sylow's theorem, $n_{2} = 1 + k \cdot 2$, $n_{2} \mid 7$, $n_{7} = 1 + k' \cdot 7$, and $n_{7} \mid 8$.
  If $k, k' = 0$, the corresponding Sylow $p$-subgroup is normal.
  Therefore, we must take $k = 3 \Rightarrow n_{2} = 7$ and $k' = 1 \Rightarrow n_{7} \mid 8$,
  the former because 7 is prime and the latter from observing $n_{7} \leq 8$.
  The Sylow 7-subgroups are disjoint but for the identity, since the intersection of any two of them is a subgroup of both 7-subgroups,
  and by Lagrange's theorem, the order of the intersection must divide 7, i.e. be 1.
  Therefore, the 8 7-subgroups have $8 \cdot 6 = 48$ distinct elements.
  By the same logic, the 7-subgroups and each of the 2-subgroups are disjoint but for the identity.
  The 7 Sylow 2-subgroups need not be disjoint with each other, but they all have order 8,
  so in the worst case that they're all the same, there are at least 8 elements in addition to those from the 7-subgroups,
  bringing the element total to 56 distinct elements.
  However, it can't happen that all 2-subgroups are the same; by assumption, $n_{2} = 7 \neq 1$, so there's at least an additional element.
  This is more than the presumed order of the group, so at least one of the Sylow 2- or 7-subgroups of any group of order 56 is normal.
\end{proof}

\end{document}
