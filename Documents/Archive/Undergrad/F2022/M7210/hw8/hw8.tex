\documentclass{article}

\usepackage[letterpaper]{geometry}
\usepackage{tgpagella}
\usepackage{amsmath}
\usepackage{amssymb}
\usepackage{amsthm}
\usepackage{tikz}
\usepackage{minted}
\usepackage{physics}
\usepackage{siunitx}

\sisetup{detect-all}
\newtheorem{plm}{Problem}

\title{7510 HW 8}
\author{Duncan Wilkie}
\date{1 November 2022}

\begin{document}

\maketitle

\begin{plm}[D\&F 7.5.2]
  Let $R$ be an integral domain and let $D$ be a nonempty subset of $R$ that is closed under multiplication.
  Prove that the ring of fractions $D^{-1}R$ is isomorphic to a subring of the quotient field of $R$ (hence is also an integral domain).
\end{plm}

\begin{proof}
  Denote the quotient field of $R$ by $Q$ and consider the map $\iota: D^{-1}R \to Q$ that maps $\frac{r}{d}$,
  an equivalence class of elements of $R \times D$, to the equivalence class containing the pair $(r, d)$ in $Q$.
  This is a homomorphism:
  \[
    \iota\qty(\frac{r}{d} \cdot \frac{r'}{d'}) = \iota\qty(\frac{rr'}{dd'}) = \iota\qty(\frac{r}{d})\iota\qty(\frac{r'}{d'}),
  \]
  where multiplicative closure of $D$ is necessary for the middle term to be interpretable,
  and well-definedness is used to choose $(r, d)$ and $(r', d')$ as representatives for the computation of the product on the right.
  We can immediately notice that the kernel of $\iota$ is the additive identity of $D^{-1}R$: if $\iota(\frac{r}{d}) \in 0$,
  then $d0 = 0 = rq$ for $q \in Q \neq 0$, and since $R$ is an integral domain, this implies $r = 0$.
  By the first isomorphism theorem, then,
  \[
    D^{-1}R \cong \iota(D^{-1}R) \leq Q.
  \]
\end{proof}

\begin{plm}[D\&F 8.1.3]
  Let $R$ be a Euclidean domain.
  Let $m$ be the minimum integer in the set of norms of nonzero elements of $R$.
  Prove that every nonzero element of norm $m$ is a unit.
  Deduce that a nonzero element of norm zero (if such an element exists) is a unit.
\end{plm}

\begin{proof}
  Suppose $a \in R$ is nonzero and has $N(a) = m \leq N(a')$, for all $a' \in R$ and $N$ the norm on $R$.
  By the division algorithm, there exist $q,r \in R$ such that $1 = qa + r$ with $r  = 0$ or $N(r) < N(a)$;
  the second condition is never met by minimality of the norm of $a$, so $r = 0$, and $q$ is the inverse of $a$ (since Euclidean domains are commutative),
  making it a unit.
  Accordingly, since norms must have nonnegative codomain, any element of norm zero must have minimal norm, and therefore be a unit.
\end{proof}

\begin{plm}[D\&F 8.1.7]
  Find a generator of the ideal $(85, 1 + 13i)$ in $\mathbb{Z}[i]$, i.e. a greatest common divisor for 85 and $1 + 13i$, by the Euclidean algorithm.
  Do the same for the ideal $(47 - 13i, 53 + 56i)$.
\end{plm}

\begin{proof}
  In the Gaussian rationals, the division (taking the one with larger norm in the numerator) is
  \[
    \frac{85}{1 + 13i} = \frac{85(1 - 13i)}{170} = \frac{85}{170} - \frac{1105}{170}i
  \]
  The nearest Gaussian integer to this (entry-wise) is $-6i$.
  Accordingly,
  \[
    85 = (-6i)(1 + 13i) + (7 + 6i);
  \]
  indeed, $N(7 + 6i) = 85 < N(1 + 13i) = 170$.
  Continuing,
  \[
    \frac{1 + 13i}{7 + 6i} = \frac{(1 + 13i)(7 - 6i)}{(7 + 6i)(7 - 6i)} = \frac{85 + 85i}{85} = 1 + i,
  \]
  so
  \[
    1 + 13i = (1 + i)(7 + 6i) + 0
  \]
  Accordingly, the last nonzero remainder, $7 + 6i$, is the gcd of these two Gaussian integers.

  Same story:
  \[
    \frac{53 + 56i}{47 - 13i} = \frac{(53 + 56i)(47 + 13i)}{(47 - 13i)(47 + 13i)} = \frac{1763 + 3321i}{2378} \approx 1 + i
  \]
  \[
    \Rightarrow (53 + 56i) = (1 + i)(47 - 13i) + (-7 + 22i).
  \]
  \[
    \frac{47 - 13i}{-7 + 22i} = \frac{(47 - 13i)(-7 - 22i)}{(-7 + 22i)(-7 - 22i)} = \frac{-43 - 1125i}{533} \approx -1 - 2i
  \]
  \[
    \Rightarrow 47 - 13i = (-1 - 2i) \cdot (-7 + 22i) + (-4 - 5i).
  \]
  \[
    \frac{-7 + 22i}{-4 - 5i} = \frac{(-7 + 22i)(-4 + 5i)}{(-4 - 5i)(-4 + 5i)} = \frac{-82 - 123i}{41} = -2 - 3i
  \]
  \[
    \Rightarrow -7 + 22i = (-2 -3i)(-4 - 5i) + 0.
  \]
  The last nonzero remainder is $-4 - 5i$; this is then the gcd (up to a unit).
\end{proof}

\begin{plm}[D\&F 8.1.10]
  Prove that the quotient ring $\mathbb{Z}[i] / I$ is finite for any nonzero ideal $I$ of $\mathbb{Z}[i]$
\end{plm}

\begin{proof}
  Since $\mathbb{Z}[i]$ is a Euclidean domain, the ideal $I$ is principal.
  Call its generator $\alpha$.
  Take any representative $a$ of any nonidentity coset; apply the division algorithm with numerator $a + \alpha$ and denominator $\alpha$.
  Accordingly, for some $q,r \in R$ one has $a = q\alpha + r$ with either $r = 0$ or $N(r) < N(\alpha)$.
  However, if $r = 0$, then $a = q\alpha \Rightarrow a \in I$, but $a + I$ is assumed nonidentity.
  Accordingly, any nonidentity coset in $\mathbb{Z}[i] / I$ has a representative with norm less than $N(\alpha)$.

  The set of all possible representatives is therefore finite: the count of integers $a_{1}, a_{2}$ (the parts of $a = a_{1} + a_{2}i$)
  such that $a_{1}^{2} + a_{2}^{2} < N(\alpha)$ is certainly finite, and this is an upper bound on the number of cosets,
  since every coset must have a representative in this set, and if two cosets share a representative, they're not different cosets.
\end{proof}
\end{document}
