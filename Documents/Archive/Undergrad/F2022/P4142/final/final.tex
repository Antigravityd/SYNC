\documentclass{article}

\usepackage[letterpaper]{geometry}
\usepackage{tgpagella}
\usepackage{amsmath}
\usepackage{amssymb}
\usepackage{amsthm}
\usepackage{tikz}
\usepackage{minted}
\usepackage{physics}
\usepackage{siunitx}

\sisetup{detect-all}
\newtheorem{plm}{Problem}
\renewcommand*{\proofname}{Solution}

\title{4142 Final Exam}
\author{Duncan Wilkie}
\date{8 December 2022}

\begin{document}

\maketitle

\begin{plm}
  Which of the following are eigenstates of parity?
  For those that are, what is the eigenvalue?
  \[
    e^{-r^{2}}\sin\theta\cos\phi, \; Y_{0}^{0} + Y_{3}^{1}, \; z(3x^{2} - y^{4}), \; r + 4yz - r\sin\theta
  \]
\end{plm}

\begin{proof}
  The parity operator sends $\vb{r} \mapsto -\vb{r}$, and correspondingly, $\theta \mapsto \pi - \theta$ and $\phi \mapsto \pi + \phi$.
  The first state under parity is $e^{-r^{2}}\sin(\pi - \theta)\cos(\pi + \phi) = e^{-r^{2}}[\sin(\pi - \theta)][-\cos(\pi + \phi)]
  = -e^{-r^{2}}\sin\theta\cos\phi$, so this is an eigenstate of parity with eigenvalue $-1$.

  The second state, which, written out fully, is $\qty(\frac{1}{4\pi})^{1/2} - \qty(\frac{21}{64\pi})^{1/2}\sin(\theta)[5\cos^2\theta - 1]e^{-i\phi}$,
  is the sum of a constant and a non-eigenstate of parity, and so is not an eigenstate of parity.

  The third is an eigenstate of parity with eigenvalue $-1$, since $x \mapsto -x, \; y \mapsto -y, \; z \mapsto -z$ yields $-z(3x^{2} - y^{4})$.

  The last, since $r \mapsto r$, is an eigenstate of parity with eigenvalue $1$, because applying the operator yields $r + 4(-y)(-z)
  - r\sin(\pi - \theta) = r + 4yz - r\sin\theta$.
\end{proof}

\begin{plm}
  Give the column vector representation of $\psi = \frac{1}{\sqrt{10}}(Y_{2}^{2} - 2Y_{2}^{1} + 2Y_{2}^{-1} + Y_{2}^{-2})$.
\end{plm}

\begin{proof}
  With respect to the (ordered) basis $Y_{2}^{-2}, Y_{2}^{-1}, Y_{2}^{1}, Y_{2}^{2}$, this can be represented
  \[
    \psi = \frac{1}{\sqrt{10}}
    \begin{pmatrix}
      1 \\
      2 \\
      -2 \\
      1
    \end{pmatrix}
  \]
\end{proof}

\begin{plm}
  For two non-identical $d$ electrons, what are the possible ${}^{2S+1}L_{J}$ states?
\end{plm}

\begin{proof}
  $d$ electrons each have $\ell = 2$, so the possible total $L$ are the integers from 0 to 4.
  Each electron is either spin-up or spin-down, which yields possible total spins $S$ of $0$ and $1$.
  If $S = 0$, then the possible states are
  \[
    {}^{1}S_{0}, \; {}^{1}P_{1}, \; {}^{1}D_{2}, \; {}^{1}F_{3}, \; {}^{1}G_{4}
  \]
  and if $S = 1$, then the possible states are
  \[
    {}^{3}S_{1}, \; {}^{3}P_{0}, \; {}^{3}P_{1}, \; {}^{3}P_{2}, \; {}^{3}D_{1}, \; {}^{3}D_{2}, \; {}^{3}D_{3}, \; {}^{3}F_{2}, \; {}^{3}F_{3}, \;
    {}^{3}F_{4}, \; {}^{3}G_{3}, \; {}^{3}G_{4} \; {}^{3}G_{5}
  \]
\end{proof}

\begin{plm}
  Between which of the following pairs of hydrogen $n\ell m$ states is an electric dipole transition allowed?
  \[
    3p1 \to 2p0, \; 3d1 \to 5p1, \; 3d0 \to 2s0, \; 3p1 \to 2s0
  \]
\end{plm}

\begin{proof}
  Electric dipole transitions satisfy $\Delta \ell = \pm 1$ and $\Delta m_{\ell} = 0$, so the second transition is the only permitted one.
\end{proof}

\begin{plm}
  What is the wavelength of the $3p \to 2s$ Balmer-$\alpha$ line in hydrogen?
\end{plm}

\begin{proof}
  This is a change from $n = 3$ to $n = 2$, and correspondingly is an energy change of
  \[
    \Delta E = \frac{\SI{-13.6}{eV}}{3^{2}} - \frac{\SI{-13.6}{eV}}{2^{2}} = \SI{1.89}{eV}
  \]
  This corresponds to a photon wavelength
  \[
    E = hf = \frac{hc}{\lambda} \Rightarrow \lambda = \frac{hc}{E} = \frac{(\SI{4.14e-15}{eV \cdot s})(\SI{3e8}{m/s})}{\SI{1.89}{eV}}
    = \SI{658}{nm}
  \]
\end{proof}

\begin{plm}
  The bottom of a square well is perturbed as shown.
  Identify for each case the first non-trivial order of perturbation of energy levels, and also its sign for the ground state.
\end{plm}

\begin{proof}
  The first perturbation has a nontrivial first-order energy correction, as the unperturbed expectation value of $H' = V_{0}1_{(a, b)}$
  is nonzero (it is $V_{0}\int_{a}^{b}|\psi_{n}^{(0)}(x)|^{2}dx$).
  The sign is clearly positive, both from the mathematical form given above and semiclassical intuition.

  The second perturbation has instead a trivial (zero) first-order correction, because the perturbation is an odd function
  (translating the $y$-axis to lie on the discontinuity) and $|\psi_{n}^{(0)}|^{2}$ is even (it's parity in these coordinates tracks that of $n$).
  The second-order correction is nontrivial, however:
  the expectation value $\bra{\psi_{m}^{(0)}}H'\ket{\psi_{n}^{(0)}} = \frac{4}{a^{2}}\int_{0}^{a}\sin(\frac{m\pi}{a}x)H'(x)\sin(\frac{n\pi}{a}x)dx$
  isn't necessarily zero if the sines have different parity so as to make the overall integrand even.
  The energy correction will be (positively, since the ground state has the lowest energy)
  proportional to the sum of expectations of the above form with $m > 1$ and $n = 1$.
  These expectations will all be positive, because % TODO: finish
\end{proof}

\begin{plm}
  Identify the $n$ and $\ell$ quantum numbers for the radial wave function of the hydrogen atom pictured.
\end{plm}

\begin{proof}
  The number of nodes of the radial wave function of hydrogen is $n - \ell - 1$, which in this case is $2$, so $n - \ell = 3$.
  Since $0 \leq \ell \leq n - 1$, by exhaustion we either have $n = 2$, $\ell = 1$ or $n = 3$, $\ell = 0$.
  However, $R_{21} = \frac{1}{2\sqrt{6}}a^{-3/2}\qty(\frac{r}{a})e^{-r/2a}$ is never negative for any positive $r$, so this must be $R_{30}$.
\end{proof}

\begin{plm}
  Positronium is a bound state of an electron and positron analoguous to the hydrogen atom.
  What are its radius and ground state energy?
\end{plm}

\begin{proof}
  All that changes about the hydrogenic analysis is that the reduced mass of the system is substituted for the mass of the electron,
  because we must carry out calculations at the true center of mass to account for the much lighter mass of the positron compared to the proton.
  So,
  \[
    E_{1} = -\frac{\mu e^{4}}{32\hbar^{2}\pi^{2}\epsilon_{0}^{2}} = -\frac{m_{e}m_{p}e^{4}}{32\hbar^{2}\pi^{2}\epsilon^{2}(m_{e} + m_{p})}
    = -\frac{m_{e}e^{4}}{64\hbar^{2}\pi^{2}\epsilon_{0}^{2}}
    = -\frac{(\SI{9.11e-31}{kg})(\SI{1.6e-19}{C})^{4}}{64(\SI{1.05e-34}{J \cdot s})^{2}\pi^{2}(\SI{8.85e-12}{F / m})^{2}}
  \]
  \[
    = -\SI{1.09e-18}{J} = \SI{6.84}{eV}
  \]
\end{proof}

\begin{plm}
  Is $e^{-ax}$ a suitable trial wave function for variationally estimating the ground state energy of an attractive one-dimensional potential?
  Give reasons.
\end{plm}

\begin{proof}
  Well, it's not normalized, so no.
\end{proof}

\begin{plm}
  A $p$-electron experiences the Hamiltonian $H = A + B\vec{\ell}\cdot\vec{s} + C\vec{\ell}^{2} + D\vec{s}^{2}$, where $A, B, C, D$ are constants.
  What are its eigenvalues and eigenkets?
  Use Dirac notation and labels.
\end{plm}

\begin{proof}
  The notation here is confusing.
  My first interpretation, according to the vector arrows and dot product symbol, is that the Hamiltonian is just a sum of constants,
  and therefore every vector is an eigenvector with eigenvalue equal to the numeric value of $H$.
  % TODO: finish
\end{proof}

\begin{plm}
  Consider the two-dimensional oscillator with $V = \frac{1}{2}m\omega^{2}(x^{2} + y^{2})$.
  \begin{enumerate}
  \item Identify at least two coordinate systems in which the time-independent Schr\"odinger equation separates.
  \item In each case above, list the full set of mutually commuting operators and the labels they provide to identify the eigenkets.
  \item What are the eigenvalues and degeneracies of these states?
  \item If an additional perturbing potential $-\epsilon xy$ were imposed, what is the nature of the correction to the ground state energy?
  \end{enumerate}
\end{plm}

\begin{proof}
  It separates firstly in rectangular coordinates: letting $\psi = X(x)Y(y)$,
  \[
    -\frac{\hbar^{2}}{2m}\laplacian\psi + \frac{1}{2}m\omega^{2}x^{2}\psi + \frac{1}{2}m\omega^{2}y^{2}\psi = E\psi
  \]
  \[
    \Leftrightarrow - Y(y)\frac{\hbar^{2}}{2m}\pdv[2]{X(x)}{x} + \frac{1}{2}m\omega^{2}x^{2}X(x)Y(y) - X(x)\frac{\hbar^{2}}{2m}\pdv[2]{Y(y)}{y}
    + \frac{1}{2}m\omega^{2}y^{2}X(x)Y(y) = EX(x)Y(y)
  \]
  \[
    \Leftrightarrow \qty[-\frac{\hbar^{2}}{2mX(x)}\pdv[2]{X(x)}{x} + \frac{1}{2}m\omega^{2}x^{2}]
    + \qty[-\frac{\hbar^{2}}{2mY(y)}\pdv[2]{Y(y)}{y} + \frac{1}{2}m\omega^{2}y^{2}] = E
  \]
  It also separates in polar coordinates: letting $\psi = R(r)\Theta(\theta)$,
  \[
    -\frac{\hbar^{2}}{2m}\laplacian\psi + \frac{1}{2}m\omega^{2}r^{2}\psi = E\psi
  \]
  \[
    \Leftrightarrow -\Theta(\theta)\frac{\hbar^{2}}{2m}\pdv[2]{R(r)}{r} - \Theta(\theta)\frac{\hbar^{2}}{2mr}\pdv{R(r)}{r}
    - R(r)\frac{\hbar^{2}}{2mr^{2}}\pdv[2]{\Theta(\theta)}{\theta} + \frac{1}{2}m\omega^{2}r^{2}R(r)\Theta(\theta) = ER(r)\Theta(\theta)
  \]
  \[
    \Leftrightarrow \qty[-\frac{\hbar^{2}}{2mR(r)}\pdv[2]{R(r)}{r} - \frac{\hbar^{2}}{2mrR(r)}\pdv{R(r)}{r} + \frac{1}{2}m\omega^{2}r^{2}]
    - \frac{\hbar^{2}}{2mr^{2}\Theta(\theta)}\pdv[2]{\Theta}{\theta} = E
  \]
  \[
    \Leftrightarrow \qty[-\frac{\hbar^{2}r^{2}}{2mR(r)}\pdv[2]{R(r)}{r} - \frac{\hbar^{2}r}{2mR(r)}\pdv{R(r)}{r} + \frac{1}{2}m\omega^{2}r^{4}]
    - \frac{\hbar^{2}}{2m\Theta(\theta)}\pdv[2]{\Theta}{\theta} = Er^{2}
  \]
  \[
    \Leftrightarrow \qty[-\frac{\hbar^{2}r^{2}}{2mR(r)}\pdv[2]{R(r)}{r} - \frac{\hbar^{2}r}{2mR(r)}\pdv{R(r)}{r} + \frac{1}{2}m\omega^{2}r^{4}
    - Er^{2}]
    - \frac{\hbar^{2}}{2m\Theta(\theta)}\pdv[2]{\Theta}{\theta} = 0
  \]
  In each case, of course, we can apply the argument that since each top-level term depends on a single variable that they both must be constant,
  as there's no way for the other term to cancel out any variation in the other; this reduces the problem to independent ODEs.

  In two-dimensional Cartesian coordinates, $\hat{p}_{x}$ and $\hat{p}_{y}$ mutually commute among themselves and with the Hamiltonian,
  as do $p_{r} = p_{x}^{2} + p_{y}^{2}$ and $L = xp_{y} - yp_{x}$ in 2D polar coordinates.
  We can use the labels $n_{x}, n_{y}$ for the Cartesian eigenkets and $\ell, m$ for the polar.
  Everything commutes with the Hamiltonian because the potential is central.

  In the state $\ket{n_{x}n_{y}}$, % TODO: finish
\end{proof}

\begin{plm}
  The potential for a bouncing ball or an electron confined within a barrier and a constant electric field is $V(x) = \infty, x < 0$,
  $V(x) = cx, x > 0$, with $c$ a real, positive constant.
  \begin{enumerate}
  \item Use the Rayleigh-Ritz variational principle with a convenient single-parameter trial function to estimate the ground-state energy value.
  \item Apply the JWKB approximation to the above potential to find the eigenvalues.
    Compare with (1).
  \end{enumerate}
\end{plm}

\begin{proof}
  We'll choose a half-Gaussian $\psi = \sqrt[4]{\frac{4a}{\pi}}e^{-ax^{2}}$ for $x > 0$ and $\psi = 0$ for $x \leq 0$.
  Then,

  \[
    \expval{T} = -\frac{\hbar^{2}}{2m}\sqrt{\frac{4a}{\pi}}\int_{0}^{\infty}e^{-ax^{2}}\dv[2]{x}\qty(e^{-ax^{2}})dx =\frac{\hbar^{2}a}{m\sqrt{2}}
  \]
  and
  \[
    \expval{V} = c\sqrt{\frac{4a}{\pi}}\int_{0}^{\infty}xe^{-2ax^{2}}dx = \frac{c}{\sqrt{4\pi a}}.
  \]
  The total expectation of $H$ yields
  \[
    E_{1} \leq \expval{H}{\psi} = \expval{T} + \expval{V} = \frac{\hbar^{2}a}{m\sqrt{2}} + \frac{c}{\sqrt{4\pi a}}
  \]
  Minimizing this expression,
  \[
    \frac{\hbar^{2}a}{m\sqrt{2}} + \frac{c}{\sqrt{4\pi}}\frac{-1}{2a^{3/2}} = 0
    \Leftrightarrow a^{5/2} = \frac{cm}{2\hbar^{2}\sqrt{2\pi}}
    \Leftrightarrow a = \qty(\frac{cm}{2\hbar^{2}\sqrt{2\pi}})^{2/5}
  \]
  giving in turn an energy bound
  \[
    \frac{\hbar^{2}}{m\sqrt{2}}\qty(\frac{cm}{2\hbar^{2}\sqrt{2\pi}})^{2/5} + \frac{c}{\sqrt{4\pi}}\qty(\frac{cm}{2\hbar^{2}\sqrt{2\pi}})^{-1/5}
  \]
  The patched JWKB approximation for a potential well with one vertical wall gives
  \[
    \int_{0}^{x_{2}}p(x)dx = \qty(n - \frac{1}{4})\pi\hbar
  \]
  for $x_{2}$ a turning point.
  Taking $p(x) = \sqrt{2m[E - V(x)]} = \sqrt{2m[E - cx]}$ and $x_{2} = E/c$,
  \[
    \int_{0}^{E/c}\sqrt{2m[E - cx]}dx = \frac{2\sqrt{2m}}{3c}\qty[(E - c(E/c))^{3/2} - E^{3/2}] = -\frac{2\sqrt{2mE^{3}}}{3c}
    = \qty(n - \frac{1}{4})\pi\hbar
  \]
  \[
    \Leftrightarrow E_{n} = \frac{1}{\sqrt[3]{2m}}\qty(\frac{3c\pi\hbar}{2}\qty(n - \frac{1}{4}))^{2/3}
  \]
  The ground state energy is then
  \[
    \frac{1}{\sqrt[3]{2m}}\qty(\frac{3c\pi\hbar}{8})^{2/3}
  \]
  which asymptotically exceeds the variational bound as $c \to \infty$.
  This is consistent with how one would expect the JWKB to err, as it is based around $V(x)$ varying slowly with respect to
  $\frac{\hbar}{\sqrt{2m(V - E)}}$, which is a worse assumption when $V$ is a line with steeper slope,
  because as $c$ is increased the derivative of the expression approaches 0 while that of the potential rises without bound.
\end{proof}

\begin{plm}
  A spin-1/2 particle is in an eigenstate of $S_{y}$ with eigenvalue $\hbar/2$.
  A magnetic field $\vec{B}$ is applied along the $z$-direction for a time $T$ and the particle is allowed to process.
  At that point, the field is rapidly switched into the $x$-direction and the particle allowed to process for another interval $T$.
  What is the probability that an $S_{y}$ measurement now will show that the initial spin has flipped to $-\hbar/2$?
\end{plm}


\end{document}
