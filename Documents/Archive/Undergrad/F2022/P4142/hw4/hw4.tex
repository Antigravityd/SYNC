\documentclass{article}

\usepackage[letterpaper]{geometry}
\usepackage{tgpagella}
\usepackage{amsmath}
\usepackage{amssymb}
\usepackage{amsthm}
\usepackage{tikz}
\usepackage{minted}
\usepackage{physics}
\usepackage{siunitx}

\newtheorem{plm}{Problem}
\renewcommand*{\proofname}{Solution}

\title{4142 HW 4}
\author{Duncan Wilkie}
\date{10 October 2022}

\begin{document}

\maketitle

\begin{plm}
  A spin-1/2 particle is in the state
  \[
    \chi = \frac{1}{\sqrt{6}}
    \begin{pmatrix}
      1 - i \\
      2
    \end{pmatrix}
  \]
  What are the probabilities a measurement of $S_{z}$ yields $\hbar / 2$ and $-\hbar / 2$?
  What are the probabilities of a measurement of $S_{x}$?
\end{plm}

\begin{proof}
  The two measurement values are the up and down eigenstates of spin, which is the basis with respect to which the state is represented.
  The probabilities are then $|(1 - i)/\sqrt{6}|^{2} = \frac{1}{3}$ and $|2/\sqrt{6}|^{2} = \frac{2}{3}$.
  A general spinor can be represented with respect to the eigenstates of $S_{x}$ as $\chi = \frac{(a+b)}{\sqrt{2}}\chi_{+}^{(x)}
  + \frac{(a-b)}{\sqrt{2}}\chi_{-}^{(x)}$; accordingly, the probabilities that measurement of $S_{x}$ yields $\pm \hbar/2$ (the corresponding eigenvalues) are
  \[
    P(\chi_{+}^{(x)}) = |(a + b)/\sqrt{2}|^{2} = \frac{10}{12} = \frac{5}{6}
  \]
  and
  \[
    P(\chi_{-}^{(x)}) = |(a - b)/\sqrt{2}|^{2} = \frac{1}{6}
  \]
\end{proof}

\begin{plm}
  Find the eigenvalues and eigenstates of spin oriented in an arbitrary direction $(\theta, \phi)$.
\end{plm}

\begin{proof}
  The unit vector in spherical coordinates is
  \[
    \hat{r} = \sin\theta\cos\phi\hat{i} + \sin\theta\sin\phi\hat{j} + \cos\theta\hat{k}
  \]
  A linear combination of the spin eigenstates by the above coefficients yields a general state:
  \[
    S_{r} = \sin\theta\cos\phi\frac{\hbar}{2}
    \begin{pmatrix}
      0 & 1 \\
      1 & 0
    \end{pmatrix}
    + \sin\theta\sin\phi\frac{\hbar}{2}
    \begin{pmatrix}
      0 & -i \\
      i & 0
    \end{pmatrix}
    + \cos\theta\frac{\hbar}{2}
    \begin{pmatrix}
      1 & 0 \\
      0 & -1
    \end{pmatrix}
  \]
  \[
    = \frac{\hbar}{2}
    \begin{pmatrix}
      \cos\theta & \sin\theta\cos\phi - i\sin\theta\sin\phi \\
      \sin\theta\cos\phi + i\sin\theta\sin\phi & -\cos\theta
    \end{pmatrix}
  \]
  Computing the eigenvalues,
  \[
    \det|S_{r}-\lambda I| = \frac{\hbar}{2}\det
    \begin{pmatrix}
      \cos\theta - \frac{2}{\hbar}\lambda & \sin\theta\cos\phi - i\sin\theta\sin\phi \\
      \sin\theta\cos\phi + i\sin\theta\sin\phi & -\cos\theta-\frac{2}{\hbar}\lambda
    \end{pmatrix}
    = -\cos^{2}\theta + \frac{4}{\hbar^{2}}\lambda^{2}
    - \sin^{2}\theta
  \]
  \[
    = \frac{4\lambda^{2}}{\hbar^{2}} - 1
    \Rightarrow \lambda = \pm \hbar/2
  \]
  The eigenstates are
  \[
    \frac{\hbar}{2}
    \begin{pmatrix}
      \cos\theta & \sin\theta\cos\phi - i\sin\theta\sin\phi \\
      \sin\theta\cos\phi + i\sin\theta\sin\phi & -\cos\theta
    \end{pmatrix}
    \begin{pmatrix}
      a \\
      b
    \end{pmatrix}
    = \pm\frac{\hbar}{2}
    \begin{pmatrix}
      a \\
      b
    \end{pmatrix}
  \]
  \[
    \Leftrightarrow
    \begin{pmatrix}
      a\cos\theta + b\sin\theta\cos\phi - bi\sin\theta\sin\phi \\
      a\sin\theta\cos\phi + ai\sin\theta\sin\phi -b\cos\theta
    \end{pmatrix}
    = \pm
    \begin{pmatrix}
      a \\
      b
    \end{pmatrix}
  \]
  \[
    \pm a = a\cos\theta+b\sin\theta\cos\phi - bi\sin\theta\sin\phi,
  \]
  \[
    \pm b = a\sin\theta\cos\phi + ai\sin\theta\sin\phi - b\cos\theta
  \]
  \[
    \Rightarrow a = b[\sin\theta\cos\phi -i\sin\theta\sin\phi] / (\pm 1 - \cos\theta) = b\sin\theta e^{-i\phi} / (\pm 1 - \cos\theta),
  \]
  \[
    \Rightarrow b = ae^{i\phi}(\pm 1 - \cos\theta) / \sin\theta
  \]
  There are double-angle identities $1 + \cos\theta = 2\cos^{2}\frac{\theta}{2}$, $1 - \cos\theta = 2\sin^{2}\frac{\theta}{2}$,
  and $\sin\theta = 2\sin\frac{\theta}{2}\cos\frac{\theta}{2}$; accordingly, this becomes
  \[
    b = ae^{i\phi}(2\sin^{2}\frac{\theta}{2}) / (2\sin\frac{\theta}{2}\cos\frac{\theta}{2}) = ae^{i\phi}\tan\frac{\theta}{2}
  \]
  for the upper branch and
  \[
    b = ae^{i\phi}(-2\cos^{2}\frac{\theta}{2}) / (2\sin\frac{\theta}{2}\cos\frac{\theta}{2}) = -ae^{i\phi}\cot\frac{\theta}{2}
  \]
  for the lower.

  The normalization condition $|a|^{2} + |b|^{2} = 1$ implies
  \[
    |a|^{2} + |a|^{2}\tan^{2}\frac{\theta}{2} = 1 \Rightarrow |a| = \cos\frac{\theta}{2}
  \]
  for the upper branch and
  \[
    |a|^{2} + |a|^{2}\cot^{2}\frac{\theta}{2} = 1 \Rightarrow |a| = \sin\frac{\theta}{2}
  \]
  for the lower.
  Choosing the phase of the $a$'s to be zero,
  \[
    \chi^{+} =
    \begin{pmatrix}
      \cos\frac{\theta}{2} \\
      \cos\frac{\theta}{2}e^{i\phi}\tan\frac{\theta}{2}
    \end{pmatrix}
    =
    \begin{pmatrix}
      \cos\frac{\theta}{2} \\
      e^{i\phi}\sin\frac{\theta}{2}
    \end{pmatrix}
  \]
  and
  \[
    \chi^{-} =
    \begin{pmatrix}
      \sin\frac{\theta}{2} \\
      -\sin\frac{\theta}{2}e^{i\phi}\cot\frac{\theta}{2}
    \end{pmatrix}
    =
    \begin{pmatrix}
      \sin\frac{\theta}{2} \\
      -e^{i\phi}\cos\frac{\theta}{2}
    \end{pmatrix}
  \]
\end{proof}


\begin{plm}
  An electron is at rest in an oscillating magnetic field $\vec{B} = B_{0}\cos\omega t \hat{k}$.
  Construct the Hamiltonian for this system.
  Suppose the electron starts at $t = 0$ in the up-state with respect to $x$, that is,
  \[
    \chi = \frac{1}{\sqrt{2}}
    \begin{pmatrix}
      1 \\
      1
    \end{pmatrix}
  \]
  Solve the time-dependent Schr\"odinger equation to get the probability of finding the electron in the orthogonal ``down'' state at a later time $t$.
\end{plm}

\begin{proof}
  The Hamiltonian is
  \[
    H = - \gamma\vec{S} \cdot \vec{B} = -\gamma B_{0}\cos\omega t S_{z}
  \]
  The corresponding time-dependent Schr\"odinger equation is
  \[
    i\hbar\pdv{\Psi}{t} = H\Psi
    = -\gamma B_{0}\cos(\omega t)S_{z}\Psi
  \]
  This is two separable, first-order ODEs in $t$, easily solved:
  \[
    i\hbar\frac{\partial \Psi_{+}}{\partial t} = -\gamma B_{0}\cos(\omega t)\frac{\hbar}{2}\Psi_{+}
    \Leftrightarrow \frac{\partial \Psi_{+}}{\Psi_{+}} = i\gamma B_{0}\cos(\omega t)/2\partial t
    \Leftrightarrow \Psi_{+} = \Psi_{+}(0)e^{i\gamma B_{0}\sin(\omega t)/2}
  \]
  \[
    i\hbar\frac{\partial \Psi_{-}}{\partial t} = \gamma B_{0}\cos(\omega t)\frac{\hbar}{2}\Psi_{-}
    \Leftrightarrow \frac{\partial \Psi_{-}}{\Psi_{-}} = -i\gamma B_{0}\cos(\omega t)/2\partial t
    \Leftrightarrow \Psi_{-} = \Psi_{-}(0)e^{-i\gamma B_{0}\sin(\omega t)/2}
  \]
  The probability of a down-state result for $S_{x}$ is
  \[
    \qty|(a - b) / \sqrt{2}|^{2} = \frac{1}{4}\qty|e^{i\gamma B_{0}\sin(\omega t)/2} - e^{-i\gamma B_{0}\sin(\omega t)/2}|^{2}
  \]
  \[
    = \frac{1}{4}\qty|\cos(\gamma B_{0}\sin(\omega t) / 2) + i\sin(\gamma B_{0}\sin(\omega t) / 2) - \cos(-\gamma B_{0}\sin(\omega t) / 2)
    - i\sin(-\gamma B_{0}\sin(\omega t) / 2)|^{2}
  \]
  \[
    = \frac{1}{4}\qty|2i\sin(\gamma B_{0}\sin(\omega t) / 2)|^{2} = \sin^{2}(\gamma B_{0}\sin(\omega t) / 2)
  \]
\end{proof}

\begin{plm}
  Find the eigenvalues and eigenvectors of
  \[
    A =
    \begin{pmatrix}
      1 & 1 & 1 & 1 \\
      1 & 1 & 1 & 1 \\
      1 & 1 & 1 & 1 \\
      1 & 1 & 1 & 1
    \end{pmatrix}
  \]
\end{plm}

\begin{proof}
  It's a simple matter: for $v \neq 0$,
  \[
    Av = \lambda v \Leftrightarrow Av = \lambda Iv \Leftrightarrow (A-\lambda I) v = 0 \Leftrightarrow \det|A - \lambda I| = 0,
  \]
  with the last equality because invertibility is equivalent to the determinant being nonzero,
  and the proceeding statement says both $v$ and $0$ map to $0$, so $A - \lambda I$ isn't injective, and therefore isn't invertible.
  Accordingly, we take
  \[
    \det
    \begin{pmatrix}
      1 - \lambda & 1 & 1 & 1 \\
      1 & 1 - \lambda & 1 & 1 \\
      1 & 1 & 1 - \lambda & 1 \\
      1 & 1 & 1 & 1 - \lambda
    \end{pmatrix}
  \]
  \[
    = (1-\lambda)\det
    \begin{pmatrix}
      1 - \lambda & 1 & 1 \\
      1 & 1 - \lambda & 1 \\
      1 & 1 & 1 - \lambda
    \end{pmatrix}
    - \det
    \begin{pmatrix}
      1 & 1 & 1 \\
      1 & 1 - \lambda & 1 \\
      1 & 1 & 1 - \lambda
    \end{pmatrix}
    + \det
    \begin{pmatrix}
      1 & 1 - \lambda & 1 \\
      1 & 1 & 1 \\
      1 & 1 & 1 - \lambda
    \end{pmatrix}
  \]
  \[
    - \det
    \begin{pmatrix}
      1 & 1 - \lambda & 1 \\
      1 & 1 & 1 - \lambda \\
      1 & 1 & 1
    \end{pmatrix}
  \]
  Since permutation of the rows and columns changes the sign of the determinant,
  \[
    = (1-\lambda)\det
    \begin{pmatrix}
      1 - \lambda & 1 & 1 \\
      1 & 1 - \lambda & 1 \\
      1 & 1 & 1 - \lambda
    \end{pmatrix}
    - 3\det
    \begin{pmatrix}
      1 & 1 & 1 \\
      1 & 1 - \lambda & 1 \\
      1 & 1 & 1 - \lambda
    \end{pmatrix}
  \]
  \[
    =(1 - \lambda)\qty[(1 - \lambda)[(1 - \lambda)^{2} - 1] - (1-\lambda - 1) + (1 - [1 - \lambda])] \]
  \[- 3\qty[[(1-\lambda)^{2} - 1] - (1 - \lambda - 1) + ( 1 - (1 - \lambda))]
  \]
  \[
    = (1 - \lambda)\qty[(1 - \lambda)(\lambda^{2}-2\lambda) + 2\lambda] - 3\qty[ 2\lambda + \lambda^{2} + 2\lambda]
  \]
  \[
    = (1 - \lambda)\qty[-\lambda^{3} + 2\lambda^{2} + \lambda^{2} - 2\lambda + 2 \lambda] - 3\qty[\lambda^{2}]
    = \lambda^{4} - 3\lambda^{3} - \lambda^{3} + 3\lambda^{2}- 3\lambda^{2}
  \]
  \[
    = \lambda ^{4} - 4\lambda^{3}  =  \lambda^{3}(\lambda - 4)
  \]
  The eigenvalues are therefore 0 (with multiplicity 3) and 4.
  One can guess eigenvectors due to the simple form of the matrix:
  \[
    \begin{pmatrix}
      1 \\
      -1 \\
      0 \\
      0
    \end{pmatrix},
  \]
  \[
    \begin{pmatrix}
    1 \\
    0 \\
    -1 \\
    0
    \end{pmatrix},
  \]
  and
  \[
    \begin{pmatrix}
      1 \\
      0 \\
      0 \\
      -1
    \end{pmatrix}
  \]
  have eigenvalue 0, and
  \[
    \begin{pmatrix}
      1 \\
      1 \\
      1 \\
      1
    \end{pmatrix}
  \]
  has eigenvalue 4.
\end{proof}

\begin{plm}
  Instead of the $z$-representation of Pauli matrices with $\sigma_{z}$ diagonal,
  suppose you transformed to the $x$-representation by diagonalizing $\sigma_{x}$.
  Construct explicitly the similarity transformation that accomplishes this,
  and then examine what happens to the other two Pauli matrices in this new representation.
  Examine the result to see its geometrical interpretation.
\end{plm}

\begin{proof}
  \[
    \sigma_{x} =
    \begin{pmatrix}
      0 & 1 \\
      1 & 0
    \end{pmatrix}
  \]
  We want to find $P$ such that
  \[
    P^{-1}\sigma_{x}P = D
  \]
  where $D$ is some diagonal matrix with eigenvalue entries on the diagonal.
  Left-canceling $P^{-1}$,
  \[
    \sigma_{x}P = PD
    \Leftrightarrow  \sigma_{x} P_{i} = \lambda_{i}P_{i}
  \]
  where $P_{i}$ is the $i$th column vector of $P$ and $\lambda_{i}$ is the sole entry in the $i$th column (or row) of $D$.
  The problem then reduces to finding eigenvalues and eigenvectors of $\sigma_{x}$; by inspection,
  $
  \begin{pmatrix}
    1 \\
    1
  \end{pmatrix}
  $
  and
  $
  \begin{pmatrix}
    1 \\
    -1
  \end{pmatrix}
  $
  are eigenvectors, with eigenvalues 1 and $-1$, respectively.
  Accordingly,
  \[
    D =
    \begin{pmatrix}
      1 & 0 \\
      0 & -1
    \end{pmatrix}
  \]
  and
  \[
    P =
    \begin{pmatrix}
      1 & 1 \\
      1 & -1
    \end{pmatrix}
  \]
  The conjugacy action of $P$ on the spin matrices is
  \[
    \sigma_{x} \mapsto
    \begin{pmatrix}
      1 & 0 \\
      0 & -1
    \end{pmatrix}
    = \sigma_{z}
  \]
  \[
    \sigma_{y} \mapsto
    \begin{pmatrix}
      0 & i \\
      -i & 0
    \end{pmatrix}
    = -\sigma_{y}
  \]
  \[
    \sigma_{z} \mapsto
    \begin{pmatrix}
      0 & 1 \\
      1 & 0
    \end{pmatrix}
    = \sigma_{x}
  \]
  This is the qualitative result expected \textit{ab initio}: the choice of the $z$-axis to align spin along is arbitrary,
  and choosing to change it mid-problem should yield broadly consistent characterization of spin.
  Unexpected, however, is the change in sign of the $\sigma_{y}$ matrix.
  It appears this diagonalization does something like changing the chirality of the coordinate system when rotating
  the $x$-axis onto the $z$-axis.
\end{proof}
\end{document}
