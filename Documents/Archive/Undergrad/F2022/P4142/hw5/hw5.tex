\documentclass{article}

\usepackage[letterpaper]{geometry}
\usepackage{tgpagella}
\usepackage{amsmath}
\usepackage{amssymb}
\usepackage{amsthm}
\usepackage{tikz}
\usepackage{minted}
\usepackage{physics}
\usepackage{siunitx}

\sisetup{detect-all}
\newtheorem{plm}{Problem}
\renewcommand*{\proofname}{Solution}

\title{4142 HW 5}
\author{Duncan Wilkie}
\date{28 October 2022}

\begin{document}

\maketitle

\begin{plm}
  Consider the energy matrix
  \[
    \begin{pmatrix}
      $a$ & 0 & 0 & $-b$ \\
      0 & $c$ & $id$ & 0 \\
      0 & $-id$ & $-c$ & 0 \\
      $-b$ & 0 & 0 &  $-a$
    \end{pmatrix}
  \]
  Find the eigenvalues and eigenvectors of this Hamiltonian.
  It will help to not approach this as a direct diagonalization but regard the elements above as coupling between four states.
\end{plm}

\begin{proof}
  Looking at what will happen when the basis vectors, it's clear this is actually is two noninteracting problems:
  vectors in combinations of the first and last basis vectors get mapped to each other, and likewise for the second and third.
  The individual problems are
  \[
    \begin{pmatrix}
      a & -b \\
      -b & -a
    \end{pmatrix}
    \Rightarrow \det
    \begin{vmatrix}
      a - \lambda & -b \\
      -b & -a - \lambda
    \end{vmatrix} = 0
    \Leftrightarrow \lambda^{2} - a^{2} - b^{2} = 0 \Rightarrow \lambda = \pm \sqrt{a^{2} + b^{2}}
  \]
  and
  \[
    \begin{pmatrix}
      c & id \\
      -id & -c
    \end{pmatrix}
    \Rightarrow \det
    \begin{vmatrix}
      c - \lambda & id \\
      -id & -c - \lambda
    \end{vmatrix}
    = 0
    \Leftrightarrow \lambda^{2} - c^{2} - d^{2} \Rightarrow \lambda = \pm \sqrt{c^{2} + d^{2}}
  \]
  The corresponding eigenvectors are
  \[
    \begin{pmatrix}
      \frac{-a \pm \sqrt{a^{2} + b^{2}}}{b} \\
      1
    \end{pmatrix}
  \]
  and
  \[
    \begin{pmatrix}
      -\frac{id}{c} \\
      1
    \end{pmatrix},
    \begin{pmatrix}
      -1 \\
      1
    \end{pmatrix}
  \]
\end{proof}

\begin{plm}
  At time $t = 0$, a spin-1/2 particle is in the state $\ket{S_{z} = +}$.
  \begin{enumerate}
  \item If $S_{x}$ is measured at $t = 0$, what is the probability of getting a value $\hbar/2$?
  \item Instead, with no measurement at $t = 0$, suppose the system evolves in a magnetic field $\vec{B} = B_{0}\hat{e}_{y}$.
    Use the $S_{z}$ basis to calculate the state of the system at time $t$.
  \item Suppose you now measure $S_{x}$ at $t$.
    What is the probability of getting the value $\hbar/2$?
  \end{enumerate}
\end{plm}

\begin{proof}
  In general,
  \[
    \chi =
    \begin{pmatrix}
      a \\
      b
    \end{pmatrix}
    = \frac{a + b}{\sqrt{2}}\chi_{+}^{(x)} + \frac{a - b}{\sqrt{2}}\chi_{-}^{(x)},
  \]
  so in the state $S_{z} = + \Rightarrow a = 1, b = 0,$ so the probability that a measurement of $S_{x}$ yields $\hbar / 2$ is
  \[
    P(S_{x} = +) = \frac{1}{\sqrt{2}}.
  \]

  The Hamiltonian in the presence of the magnetic field is
  \[
    H = \gamma B_{0}S_{y} = -\frac{\gamma B_{0}\hbar}{2}
    \begin{pmatrix}
      0 & -i \\
      i & 0
    \end{pmatrix}
  \]
  The eigenstates of this are the usual eigenstates of $S_{y}$: in the $S_{z}$ basis,
  \[
    \chi_{+}^{(y)} =
    \begin{pmatrix}
      -i / \sqrt{2} \\
      1 / \sqrt{2}
    \end{pmatrix},\
    \chi_{-}^{(y)} =
    \begin{pmatrix}
      i / \sqrt{2} \\
      1 / \sqrt{2}
    \end{pmatrix}
  \]
  and accordingly
  \[
    \chi = \frac{b + ia}{\sqrt{2}}\chi_{+}^{(y)} + \frac{b - ia}{\sqrt{2}}\chi_{-}^{(y)}
  \]
  Since this is time-independent, the time-dependent solution expressed in the $S_{y}$ basis is
  \[
    \chi(t) =
    \begin{pmatrix}
      ae^{-iE_{+}t/\hbar} \\
      be^{-iE_{-}t/\hbar}
    \end{pmatrix}
  \]
  where $a, b$ are specified by the initial condition (in this case, $\chi(0) = \ket{S_{z} = +} \Rightarrow a = \frac{i}{\sqrt{2}},
  b = \frac{-i}{\sqrt{2}}$)
  and $E_{\pm}$ are the energies of the $\chi_{\pm}^{(y)}$ eigenstates, which are $\mp \frac{\gamma B_{0}\hbar}{2}$ by the eigenvalues of the states:
  \[
    \chi(t) =
    \begin{pmatrix}
      \frac{i}{\sqrt{2}}e^{i\gamma B_{0}t/2} \\
      \frac{-i}{\sqrt{2}}e^{-i\gamma B_{0}t/2}
    \end{pmatrix}.
  \]
  Changing back to the $S_{z}$ basis,
  \[
    \chi(t) = \qty(\frac{i}{\sqrt{2}}e^{i\gamma B_{0}t/2})\qty(\frac{-i}{\sqrt{2}}\chi_{+} + \frac{1}{\sqrt{2}}\chi_{-})
    + \qty(\frac{-i}{\sqrt{2}}e^{-i\gamma B_{0}t/2})\qty(\frac{i}{\sqrt{2}}\chi_{+} + \frac{1}{\sqrt{2}}\chi_{-})
  \]
  \[
    = \frac{e^{i\gamma B_{0}t/2}}{2}\chi_{+} + \frac{ie^{i\gamma B_{0}t/2}}{2}\chi_{-} + \frac{e^{-i\gamma B_{0}t/2}}{2}\chi_{+}
    - \frac{ie^{-i\gamma B_{0}t/2}}{2}\chi_{-}
  \]
  \[
    =
    \begin{pmatrix}
      \cos(\gamma B_{0}t/2) \\
      -\sin(\gamma B_{0}t/2)
    \end{pmatrix}
  \]
  This looks a lot like a clockwise parameterization of a circle.

  The change of basis from $S_{z}$ to $S_{x}$ is
  \[
    \chi = \frac{a + b}{\sqrt{2}}\chi_{+}^{(x)} + \frac{a - b}{\sqrt{2}}\chi_{-}^{(x)},
  \]
  so the coefficient of $\chi_{+}^{(x)}$ using the above expression for the $S_{z}$ basis is
  \[
    \frac{\cos(\gamma B_{0}t / 2) - \sin(\gamma B_{0}t/2)}{\sqrt{2}}
  \]
  The corresponding probability of a measurement resulting in $\chi_{+}^{(x)}$ (which has eigenvalue $\hbar/2$) is
  \[
    \qty(\frac{\cos(\gamma B_{0}t / 2) - \sin(\gamma B_{0}t/2)}{\sqrt{2}})^{2} = \frac{1}{2}\qty(1 - 2\sin(\gamma B_{0}t))
  \]
\end{proof}

\begin{plm}
  Three quarks, each of spin $1/2$, form a baryon.
  What are the allowed values of baryon spin?
\end{plm}

\begin{proof}
  Distinguish one pair of quarks.
  This combination has a spin of either 1 or 0, as those are the only integers between $\frac{1}{2} + \frac{1}{2}$ and $\frac{1}{2} - \frac{1}{2}$.
  The spin of the whole system is a combination of the spin of the pair with the spin of the remaining quark;
  it's either $1 + \frac{1}{2} \Rightarrow \frac{3}{2}, \frac{1}{2}$ or $0 + \frac{1}{2} \Rightarrow \frac{1}{2}, -\frac{1}{2}$.
  Accordingly, the only possibly baryon spins are $\frac{1}{2}$ and $\frac{3}{2}$.
\end{proof}

\begin{plm}
  Suppose an electron is in potentials $\vec{A} = B_{0}(x\hat{j} - y\hat{i})$, $\Phi = Kz^{2}$.
  \begin{enumerate}
  \item Find the electric and magnetic fields.
  \item Find the allowed energies of the electron.
  \end{enumerate}
\end{plm}

\begin{proof}
  Since $\Phi$ is time-independent, $\vec{B} = \curl{\vec{A}}$ and $\vec{E} = \grad{\Phi}$:
  \[
    \curl{\vec{A}} = \det
    \begin{vmatrix}
      \hat{i} & \hat{j} & \hat{k} \\
      \pdv{x} & \pdv{y} & \pdv{z} \\
      -B_{0}y & B_{0}x & 0
    \end{vmatrix}
    = 2B_{0}\hat{k},
  \]
  \[
    \grad{\Phi} = 2Kz\hat{k}.
  \]

  The minimal coupling Hamiltonian is
  \[
    H = \frac{1}{2m}(-i\hbar\grad - e\vec{A})^{2} + e\Phi
    = \frac{1}{2m}(-i\hbar\grad + eB_{0}(x\hat{j} - y\hat{i}))^{2} + eKz^{2}
  \]
  \[
    = \frac{1}{2m}(i\hbar\grad + eB_{0}(x\hat{j} - y\hat{i})) \cdot (i\hbar\grad + eB_{0}(x\hat{j} - y\hat{i})) + eKz^{2}
  \]
  \[
    = \frac{1}{2m}\qty[-\hbar^{2}\grad^{2} + i\hbar eB_{0}\qty(\div{(x\hat{j} - y\hat{i})} + (x\hat{j} - y\hat{i}) \cdot \grad)
    + e^{2}B_{0}^{2}(x^{2} + y^{2})] + eKz^{2}
  \]
  \[
    = \frac{1}{2m}\qty[-\hbar^{2}\laplacian + i\hbar eB_{0}\qty(x\pdv{y} - y\pdv{x}) + e^{2}B_{0}^{2}(x^{2} + y^{2})] + eKz^{2}
  \]
  \[
    = \frac{1}{2m}\qty[-\hbar^{2}\laplacian - eB_{0}L_{z} + e^{2}B_{0}^{2}r^{2}] + eKz^{2}
  \]
  \[
    = \frac{p^{2}}{2m} - \frac{eB_{0}}{2m}L_{z} + \frac{e^{2}B_{0}^{2}}{2m}r^{2} + eKz^{2}
  \]
  The $L_{z}$ operator commutes with this Hamiltonian: it commutes with central-potential Hamiltonians, with itself,
  and with the $z$ operator, so it must commute with a linear combination of these three.
  Therefore, it is simultaneously diagonalizable with the Hamiltonian; call its eigenvalue $\hbar m_{\ell}$ as usual.
  The right two potential terms are independent harmonic oscillator potentials with
  $k_{r} = \frac{e^{2}B_{0}^{2}}{m} \Rightarrow \omega_{r} = \frac{eB_{0}}{m}$
  and $k_{z} = 2eK \Rightarrow \omega_{z} = \sqrt{\frac{2eK}{m}}$.
  The energies of these two solutions are $\hbar\omega_{r}(n_{r} + \frac{1}{2})$ and $\hbar\omega_{z}(n_{z} + \frac{1}{2})$.
  When the Hamiltonian acts on a simultaneous eigenstate in the Schr\"odinger equation, the eigenvalues of these operators add
  to yield allowed energies
  \[
    E(n_{r}, n_{z}, m_{\ell}) = \hbar\omega_{r}\qty(n_{r} + \frac{1}{2}) + \hbar\omega_{z}\qty(n_{z} + \frac{1}{2}) + \hbar m_{\ell}
  \]
\end{proof}

\begin{plm}
  Consider the observables $A = x^{2}$ and $B = L_{z}$.
  \begin{enumerate}
  \item Find the uncertainty principle governing $A$ and $B$, that is, $\Delta A\Delta B$.
  \item Evaluate $\Delta B$ for the hydrogenic $\ket{n\ell m}$ state.
  \item Therefore, what can you conclude about $\expval{xy}$ in this state?
  \end{enumerate}
\end{plm}

\begin{proof}
  The generalized uncertainty principle states
  \[
    \Delta A\Delta B \geq \qty|\frac{1}{2i}\expval{\qty[{A}, {B}]}|.
  \]
  We first compute the commutator:
  \[
    \qty[A, B] = \qty[x^{2}, L_{z}] = x\qty[x, L_{z}] + \qty[x, L_{z}]x
    = x\qty[x, xp_{y} - yp_{x}] + \qty[x, xp_{y} - yp_{x}]x
  \]
  \[
    = x\qty([x, xp_{y}] - [x, yp_{x}]) + \qty([x, xp_{y}] - [x, yp_{x}])x
  \]
  \[
    = x\qty([x, x]p_{y} + x[x, p_{y}] - [x, y]p_{x} - y[x, p_{x}]) + \qty([x, x]p_{y} + x[x, p_{y}] - [x, y]p_{x} - y[x, p_{x}])x
  \]
  The commutator among any two position operators and that of a position and a momentum operator along different axes is zero; accordingly,
  \[
    = -xy[x, p_{x}] - y[x, p_{x}]x
    = -xyi\hbar - yxi\hbar
    = -i\hbar(xy + yx)
  \]
  \[
    = -2i\hbar xy
  \]
  The uncertainty principle is therefore
  \[
    \Delta A \Delta B \geq \hbar\qty|\expval{xy}|
  \]
  The variance of the $B$ operator is given by $\langle L_{z}^{2} \rangle - \langle L_{z} \rangle^{2}$;
  the former is by $L_{z}$ Hermitian $\bra{\psi}L_{z}^{2}\ket{\psi} = \bra{L_{z}\psi}\ket{L_{z}\psi}$.
  The $\ket{n\ell m_{\ell}}$ state of hydrogen is in an eigenstate of $L_{z}$ by assumption (corresponding to the eigenvalue $m_{\ell}$).
  In particular, $L_{z} \psi = \hbar m_{\ell}\psi \Rightarrow \expval{L_{z}^{2}} = \hbar^{2}m_{\ell}^{2}$,
  and $\expval{L_{z}}$ is the eigenvalue of $L_{z}$ corresponding to the eigenstate; of course, $\hbar m_{\ell}$.
  So, the variance of $B$ is $\hbar^{2}m_{\ell}^{2} - \hbar^{2}m_{\ell}^{2} = 0$.
  This necessarily means $\expval{xy} = 0$.
\end{proof}
\end{document}
