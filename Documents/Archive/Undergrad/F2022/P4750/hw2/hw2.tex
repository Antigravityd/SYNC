\documentclass{article}

\usepackage[letterpaper]{geometry}
\usepackage{tgpagella}
\usepackage{amsmath}
\usepackage{amsthm}
\usepackage{amssymb}
\usepackage{physics}
\usepackage[separate-uncertainty=true]{siunitx}

\title{4750 HW 2}
\author{Duncan Wilkie}
\date{21 September 2022}

\newtheorem{prob}{Problem}
\renewcommand*{\proofname}{Solution}

\begin{document}

\maketitle

\begin{prob}
  Find an expression for the amplitude of gravitational waves emitted by a binary system for an observer in the plane of the orbit
  (but far away from the system).
  Is that smaller or larger than the amplitude of gravitational waves emitted along an axis perpendicular to the orbit?
\end{prob}

\begin{proof}
  Far away, we may choose the transverse-traceless gauge.
  The corresponding $h_{ij}$ matrix differs from the perpendicular case due to the different axis; it is
  \[
    h_{ij}^{TT} =
    \begin{pmatrix}
      0 & 0 & 0 \\
      0 & h_{+} & h_{x} \\
      0 & h_{x} & -h_{+}
    \end{pmatrix}
  \]
  Following the argument in the perpendicular case, this yields
  \[
    h_{+} = \frac{1}{r}\frac{G}{c^{4}}(\ddot{Q}_{yy} - \ddot{Q}_{zz})\bigg|_{t - r/c}
  \]
  \[
    h_{\times} = \frac{2}{r}\frac{G}{c^{4}}\ddot{Q}_{yz}\bigg|_{t - r/c}
  \]
  Calculating the components,
  \[
    Q_{yy} = \int_{\mathbb{R}^{3}}\delta(z)\delta(x - R\cos\omega t)\delta(y - R\sin\omega t)(y^{2}-\frac{r^{2}}{3})d\vb{x}
    = R^{2}\sin^{2}\omega t - \frac{r^{2}}{3}
  \]
  \[
    Q_{zz} = \int_{\mathbb{R}^{3}}\delta(z)\delta(x - R\cos\omega t)\delta(y - R\sin\omega t)(z^{2}-\frac{r^{2}}{3})d\vb{x}
    = -\frac{r^{2}}{3}
  \]
  \[
    Q_{zz} = \int_{\mathbb{R}^{3}}\delta(z)\delta(x - R\cos\omega t)\delta(y - R\sin\omega t)(yz)d\vb{x}
    = 0
  \]
  Accordingly,
  \[
    h_{+} = \frac{2}{r}\frac{G}{c^{4}}R^{2}\cos 2\omega t
  \]
  \[
    h_{\times} = 0
  \]
  This is smaller than the perpendicular amplitude.
\end{proof}

\begin{prob}
  Consider a binary black hole system, with black hole mass \SI{30}{M_{\odot}} when their centers are 2.5 Schwarzschild radii away (that is, about to merge).
  If the system is 400 Mpc away, what is the approximate amplitude of the wave on Earth?
  Compare with Phys. Rev. Lett. 116, 061102 (2016).
  How close would this system would have to be if the strain on Earth is 1 ppm?
\end{prob}

\begin{proof}
  The amplitude from a binary black hole system is given by
  \[
    h \approx \frac{r_{s}^{2}}{Rr}
  \]
  where $r_{s}$ is the Schwarzschild radius, $R$ is the radius of orbit, and $r$ is the distance between observer and merger event.
  The Schwarzschild radius of the black holes is
  \[
    r_{s} = \frac{2GM}{c^{2}} = \frac{2(\SI{6.67e-11}{N\cdot m^{2}/kg^{2}})(30 \cdot \SI{2e30}{kg})}{(\SI{3e8}{m/s})^{2}}
    = \SI{89}{km}
  \]
  Accordingly,
  \[
    h \approx \frac{(\SI{89}{km})^{2}}{(2.5\cdot\SI{89}{km})(\SI{400}{Mpc} \cdot \SI{3.1e22}{m / Mpc})} = \SI{2.9e-21}{}
  \]
  This is 3 times the strain detected in the corresponding paper.
  We can rearrange the approximate formula for $h$ for $r$:
  \[
    r = \frac{r_{s}^{2}}{Rh}
  \]
  If the strain is \SI{1}{ppm}, or, equivalently, if $h = \SI{1e-6}{}$,
  \[
    r = \frac{r_{s}^{2}}{Rh} = \frac{(\SI{89}{km})^{2}}{(2.5\cdot \SI{89}{km})(\SI{1e-6}{})} = \SI{3.56e10}{m} = \SI{0.24}{AU}
  \]
\end{proof}

\begin{prob}
  The Crab pulsar has a rotation frequency of $\SI{30}{Hz}$ and is approximately 6,400 light years away.
  The period is observed to decrease about $\SI{40}{ns}$ per day, probably due to energy carried away by pulsar wind.
  If we assume all the energy is instead lost to gravitational waves,
  this provides allows a ``spin-down'' upper limit on the ellipticity $\epsilon$ of the neutron star at about a part in a thousand.
  Assume the radius of the Crab pulsar (a neutron star) is $\SI{10}{km}$ and its mass is \SI{1.4}{M_{\odot}}.
  Using the formula for power emitted by such systems $P = \frac{G}{5c^{5}}\langle \dddot{Q}_{ij}\dddot{Q}^{ij} \rangle$,
  derive the upper limit to $\epsilon$.
  What would be the amplitude of gravitational waves emitted by the Crab pulsar, assuming $\epsilon = \SI{e-3}{}$?
  Compare this estimate with results in Table 3 of Phys. Rev. D 105, 022002 (2022).
\end{prob}

\begin{proof}
  In the transverse-traceless gauge,
  \[
    \ddot{Q} = \ddot{I} = 2\epsilon I_{zz}\omega
    \begin{pmatrix}
      -\cos 2\omega t & \sin 2\omega t & 0 \\
      \sin 2\omega t & \cos 2\omega t & 0 \\
      0 & 0 & 0
    \end{pmatrix},
  \]
  so
  \[
    \dddot{Q} = 4\epsilon I_{zz}\omega^{2}
    \begin{pmatrix}
      \sin 2\omega t & \cos 2\omega t & 0 \\
      \cos 2\omega t & -\sin 2\omega t & 0 \\
      0 & 0 & 0
    \end{pmatrix}
  \]
  The expectation value is that of the sum of the squares of each entry:
  \[
    \langle \dddot{Q}_{ij}\dddot{Q}^{ij} \rangle
    = 16\epsilon^{2} I_{zz}^{2}\omega^{4}\langle \sin^{2}2\omega t + \cos^{2} 2\omega t + \cos^{2} 2\omega t + \sin^{2} 2\omega t \rangle
    = 16\epsilon^{2} I_{zz}^{2}\omega^{4}\langle 2 \rangle
    = 32\epsilon^{2} I_{zz}^{2}\omega^{4}
  \]
  We then get the expression for the gravitational wave radiated power as
  \[
    P = \frac{32G}{5c^{5}}\epsilon^{2}I_{zz}^{2}\omega^{4}
  \]
  The rotational kinetic energy of the star is
  \[
    E = \frac{1}{2}\vb*{\omega}\cdot {I}\vb*{\omega} = \frac{1}{2}I_{zz}\omega^{2}
  \]
  where $\vb*{\omega} = \omega\hat{z}$ is the angular velocity.
  The corresponding radiated power is
  \[
    P = \dv{E}{t} = \frac{1}{2}\qty(\dot{I}_{zz}\omega^{2} + 2I_{zz}\omega\dot{\omega})
  \]
  Equating the two expressions for power and solving for $\epsilon$,
  \[
    \epsilon^{2} = \frac{5c^{5}}{64GI_{zz}^{2}\omega^{4}}\qty(\dot{I}_{zz}\omega^{2} + 2I_{zz}\omega\dot{\omega})
  \]
  \[
    \Leftrightarrow \epsilon = \frac{c^{2}}{8I_{zz}\omega^{2}}\sqrt{\frac{5c}{G}\qty(\dot{I}_{zz}\omega^{2} + 2I_{zz}\omega\dot{\omega})}
  \]
  The moment of inertia changes negligibly, so
  \[
    \epsilon = \frac{c^{2}}{8I_{zz}\omega^{2}}\sqrt{\frac{10c}{G}I_{zz}\omega\dot{\omega}}
  \]
  The derivative of the angular velocity in terms of the derivative of period is computed by
  \[
    \omega = \frac{2\pi}{T} \Rightarrow \dot{\omega} = -\frac{2\pi\dot{T}}{T^{2}} = -2\pi f^{2}\frac{\Delta T}{\Delta t}
    = -2\pi(\SI{30}{Hz})^{2}\frac{\SI{-40e-9}{s}}{24\cdot 60 \cdot \SI{60}{s}}
    = \SI{2.6e-9}{Hz^{2}}
  \]
  The moment of inertia of a sphere about any axis through its center is $\frac{2}{5}MR^{2}$, and the star is very closely spherical,
  so an upper bound on the eccentricity is
  \[
    \epsilon = \frac{5c^{2}}{8MR\omega^{2}}\sqrt{\frac{Mc}{G}\omega\dot{\omega}}
  \]
  \[
    = \frac{5(\SI{3e8}{m/s})^{2}}{8(1.4\cdot\SI{2e30}{kg})(\SI{10}{km})(2\pi\cdot\SI{30}{Hz})^{2}}
    \sqrt{\frac{(1.4\cdot\SI{2e30}{kg})(\SI{3e8}{m/s})}{\SI{6.67e-11}{N\cdot m^{2} / kg^{2}}}(2\pi\cdot\SI{30}{Hz})(\SI{2.6e-9}{Hz^{2}})}
  \]
  \[
    = \SI{0.14}{}
  \]
  The amplitude of gravitational waves from a rotating ellipsoid is
  \[
    h = \frac{4G\epsilon I_{zz}\omega^{2}}{c^{4}r}.
  \]
  If $\epsilon = \SI{e-3}{}$, the amplitude is, using the spherical approximation to compute $I_{ zz}$,
  \[
    h = \frac{4(\SI{6.67e-11}{N\cdot m^{2}/kg^{2}})(\SI{e-3}{})\frac{2}{5}(1.4\cdot\SI{2e30}{kg})(\SI{10}{km})^{2}(2\pi\cdot\SI{30}{Hz})^{2}}
    {(\SI{3e8}{m/s})^{4}(\SI{400}{Mpc})(\SI{3.1e22}{m/Mpc})}
    = \SI{1.1e-29}{}
  \]
  The cited article lists likely detectable strains in a range from $\SI{5e-26}{}$ to $\SI{11e-26}{}$,
  so this is well below the detectable level, by a factor of ten thousand.
\end{proof}

\begin{prob}
  Estimate the wavelength and order of magnitude of gravitational waves on Earth emitted by:
  \begin{itemize}
  \item J2322+0509, a white dwarf system with a \SI{1200}{s} orbital period, \SI{0.76}{kpc} from the Sun,
    with masses \SI{0.27}{M_{\odot}} and \SI{0.24}{M_\odot} (Warren R. Brown et al 2020 ApJL 892 L35).
  \item the Hulse-Taylor binary pulsar, PSR B1913+16, with an orbital period of 7.75 hours, \SI{6.6}{kpc} away,
    and \SI{1.4}{M_\odot} masses for each neutron star.
    The orbit is eccentric, but use a circular orbit with 2 million km radius.
  \item an ice skater, rotating about a vertical axis (assume the observer is a wavelength away).
  \item Estimate the energy emitted in a single revolution and compare it with $\hbar\omega$,
    where $\omega$ is the gravitational wave frequency---what can you conclude from this?

  \end{itemize}
\end{prob}

\begin{proof}
  For the first,
  \[
    \lambda = \frac{c}{f} = cT = (\SI{3e8}{m/s})(\SI{1200}{s}) = \SI{3.6e11}{m}
  \]
  The radius of the orbit can be deduced from Kepler's third:
  \[
    R \approx \sqrt[3]{\frac{G(M_{1}+M_{2})T^{2}}{4\pi^{2}}} = \sqrt[3]{\frac{(\SI{6.67e-11}{N\cdot m^{2}/kg^{2}})
        [0.27\cdot\SI{2e30}{kg} + 0.24\cdot\SI{2e30}{kg}](\SI{1200}{s})^{2}}{4\pi^{2}}}
  \]
  \[
    = \SI{1.35e8}{m}
  \]
  \[
    h \propto \frac{1}{r}\frac{GMR^{2}\omega^{2}}{c^{4}}
    = \frac{1}{(\SI{6.6}{kpc})(\SI{3.1e19}{m / kpc})}\frac{(\SI{6.67e-11}{N\cdot m^{2}/kg^{2}})(0.25\cdot\SI{2e30}{kg})
      (\SI{1.35e8}{m})^{2}(2\pi/ (\SI{1200}{s}))^{2}}{(\SI{3e8}{m/s})^{4}}
  \]
  \[
% wrong computation    = \SI{2e-53}{}
  \]
  Proceeding similarly for the next,
  \[
    \lambda = (\SI{3e8}{m/s})(\SI{60}{s / hr}\cdot\SI{7.75}{hr}) = \SI{1.4e11}{m}
  \]
  \[
    h \propto \frac{1}{r}\frac{GMR^{2}\omega^{2}}{c^{4}}
    = \frac{1}{(\SI{6.6}{kpc})(\SI{3.1e19}{m/kpc})}
  \]
  \[
    \cdot\frac{(\SI{6.67e-11}{N \cdot m^{2}/kg^{2}})(1.4\cdot\SI{2e30}{kg})(\SI{2e9}{m})^{2}
      (2\pi / (\SI{60}{s / hr}\cdot\SI{7.55}{hr}))^{2}}{(\SI{3e8}{m/s})^{4}}
  \]
  \[
    = \SI{8.7e-20}{}
  \]
\end{proof}
\end{document}
