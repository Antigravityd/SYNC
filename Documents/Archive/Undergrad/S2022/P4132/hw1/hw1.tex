\documentclass{article}

\usepackage[letterpaper]{geometry}
\usepackage{amsmath}
\usepackage{amssymb}
\usepackage{siunitx}

\title{4132 HW 1}
\author{Duncan Wilkie}
\date{25 January 2021}

\begin{document}

\maketitle

\section*{7.2a}
The voltage drop across a capacitor is $ V = \frac{q}{C}$, and the voltage drop across the resistor is $V=IR=\frac{dq}{dt}R$. Tracking the voltage drops around the loop,
\[\frac{q}{C}+\frac{dq}{dt}R=0\Leftrightarrow -\frac{dt}{RC}=\frac{dq}{q}\Leftrightarrow \ln q=-\frac{t}{RC}+c_1\Leftrightarrow q(t)=c_2e^{-{t}/{RC}}\]
Applying the boundary condition $q(0)=CV_0$,
\[q(t)=CV_0e^{-t/RC}\]
The current is
\[I(t)=\frac{dq}{dt}=\frac{V_0}{R}e^{-t/RC}\]

\section*{7.2b}
The initial energy in the capacitor is $E=\frac{1}{2}CV_0^2$. Integrating, \[P=I^2R=\frac{V_0^2}{R}e^{-2t/RC}\]
\[\Rightarrow E=\int_0^\infty\frac{V_0^2}{R}e^{-2t/RC}dt=\frac{V_0^2}{R}\left( -\frac{RC}{2}e^{-2t/RC}\bigg|_0^\infty \right)=\frac{1}{2}CV_0^2\]

\section*{7.2c}
Similarly to above,
\[V_0-\frac{q}{C}-\frac{dq}{dt}R=0\]
Using non-homogeneous constant coefficients,
\[q(t)=c_1e^{-t/RC}+CV_0\]
Applying the boundary condition that $q(0)=0$,
\[q(t)=CV_0(1-e^{-t/RC})\]
Differentiating,
\[I(t)=\frac{V_0}{R}e^{-t/RC}\]

\section*{7.2d}
The total energy output of the battery is
\[E=\int_0^\infty V_0I(t)dt=\int_0^\infty \frac{V_0^2}{R}e^{-t/RC}dt=\frac{V_0^2}{R}\left( -RCe^{-t/RC}\bigg|_0^\infty \right)=C{V_0^2}\]
The energy dissipated by the resistor is found by
\[P=I^2R=\frac{V_0^2}{R}e^{-2t/RC}\Rightarrow E=\int_0^\infty \frac{V_0^2}{R}e^{-2t/RC}dt=\frac{1}{2}CV_0^2\]
The energy left on the capacitor at the end is the difference between the two:
\[E_o=E_R+E_C\Rightarrow E_C=CV_0^2-\frac{1}{2}CV_0^2=\frac{1}{2}CV_0^2\]
This half the total work done by the battery, and confirms what is obvious; as $t\to\infty$, the current becomes negligible, so the voltage across the capacitor is $V_0$, so the energy stored in it must be the above.

\section*{7.3a}
Consider imposing a constant voltage difference between the two spheres. This results in a charge difference between the two objects $q$ according to $V=\frac{q}{C}$. The electric field along a surface infinitesimally outside the boundary of one of the conductors is perpendicular to the surface at every point, and as such we may apply Gauss's law:
\[|\vec{E}|A=\frac{q}{\epsilon_0}\Rightarrow \vec{E}=\frac{q}{\epsilon_0 A}\hat{n}\] where $A$ is the surface area of the conductor and $\hat{n}$ is the outward unit normal. The current that flows may be found from the current density, which in turn may be found by Ohm's law over some closed surface $C$ just outside one of the conductors:
\[I=\int_C\vec{J}\cdot d\hat{a}=\sigma\int_C\vec{E}\cdot d\vec{a}=\frac{\sigma q}{\epsilon_0A}\int_C\hat{n}\cdot d\vec{a}=\frac{\sigma q}{\epsilon_0}\]
The resistance is therefore, using the definition of capacitance mentioned above,
\[R=\frac{V}{I}=\frac{\epsilon_0}{\sigma C}\]
as desired.

\section*{7.3b}
The arguments used to derive the above equation hold regardless of potential difference and charge on the conductors, so it applies to this situation as well. Setting up Ohm's law as a differential equation in charge,
\[\frac{q(t)}{C}+\frac{dq(t)}{dt}\frac{\epsilon_0}{\sigma C}=0\]
Separating variables and integrating,
\[\ln(q)=-\frac{\sigma}{\epsilon_0}t+c_0\Leftrightarrow q(t)=ce^{-\sigma t/\epsilon_0}\]
Differentiating this to get the current and multiplying by $R$ to get voltage only multiplies the constant term by a constant factor, so we may immediately apply the boundary condition $V(0)=V_0$ to get the desired equation for the voltage:
\[V(t)=V_0e^{-\sigma t/\epsilon_0}\]
This gives $\tau=\frac{\epsilon_0}{\sigma}$

\section*{7.7a}
The area of the loop is $A=lx(t)$, so the emf is $\mathcal{E}=-\frac{d\Phi}{dt}=-\frac{dAB}{dt}=-Blv$. The current in the resistor is $I=\frac{\mathcal{E}}{R}=\frac{Blv}{R}$. The current flows such that it produces a magnetic field in a direction opposing the magnetic field causing the emf, which by the right-hand rule is counterclockwise.

\section*{7.7b}
The force on the bar is
\[F=\int Id\vec{l}\times \vec{B}=-IlB\hat{i}=-\frac{B^2l^2v}{R}\hat{i}\]
where $\hat{i}$ is the rightwards unit vector, so the force is to the left.

\section*{7.7c}
Using Newton's second law,
\[m\frac{dv}{dt}=-\frac{B^2l^2v(t)}{R}\Leftrightarrow \ln v= -\frac{B^2l^2}{mR}t+c_0\Leftrightarrow v(t)=v_0\exp\left(  -\frac{B^2l^2}{mR}t\right)\]

\section*{7.7d}
Integrating the Joule heating law using the current and velocity expressions derived above,
\[P=I^2R=\frac{B^2l^2v_0^2\exp\left( -\frac{2B^2l^2}{mR}t \right)}{R}\]
\[\Rightarrow E=\int_0^\infty P(t)dt=\frac{B^2l^2v_0^2}{R}\int_0^\infty\exp\left( -\frac{2B^2l^2}{mR}t \right)dt=\frac{B^2l^2v_0^2}{R}\left( -\frac{mR}{2B^2l^2}\exp\left[ -\frac{2B^2l^2}{mR}t \right]\bigg|_0^\infty \right)\]
\[=\frac{1}{2}mv_0^2\]
as desired.

\section*{7.10}
The area of the square is $a^2$. The flux is of course
\[\Phi=\vec{B}\cdot{A}=BA\cos\theta=a^2B\cos\theta\]
where $\theta$ is the angle between the normal vector to the square and the magnetic field. The emf is then
\[\mathcal{E}=-\frac{d\Phi}{dt}=a^2B\omega\sin\theta\]
If we take $\phi$ to be the starting angle of the loop, we may write this as
\[\mathcal{E}(t)=a^2B\omega\sin(\omega t+\phi)\]

\section*{7.11}
Let the loop have side length $w$ and the region intersecting the field have height $h$. The flux is $whB$, and so the emf is
\[\mathcal{E}=-\frac{d\Phi}{dt}=wBv \]
This results in a current $I=\frac{wBv}{R}$ clockwise, on the horizontal leg of which there is a force
\[F=BIw=\frac{w^2B^2v}{R} \]
The vertical currents were ignored because the corresponding forces each point horizontally inward and cancel.
This force is directed upwards since the charges are moving right due to the current; it opposes the gravitational force:
\[m\frac{dv}{dt}=mg-\frac{w^2B^2v}{R}\]
The terminal velocity is when $\frac{dv}{dt}=0$, i.e.
\[v_t=\frac{mgR}{w^2B^2}\]
The mass of the loop is, in terms of the density of aluminum $\rho$ and cross-sectional area $A$,
\[m=4w\rho A\]
The resistance of the loop is, in terms of the conductivity of aluminum $\sigma$ and the cross-sectional area of the loop $A$,
\[R=\frac{4w}{\sigma A}\]
Plugging this in,
\[v_t=\frac{4w\rho Ag\left( 4w/\sigma A \right)}{w^2B^2}=\frac{16\rho g}{\sigma B^2}\]
The conductivity and density of pure aluminum are, according to a table I found online, \SI{3.538E7}{\mho/m} and \SI{2700}{kg/m^3} respectively. Plugging this in,
\[v_t=\frac{16(\SI{2700}{kg/m^3})(\SI{9.81}{m/s^2})}{(\SI{3.538E7}{\mho/m})(\SI{1}{T})}=\SI{0.012}{m/s}\]
Returning to the expression for Newton's second law for the forces, this is a differential equation for $v$ we may solve by non-homogeneous constant coefficients:
\[\frac{dv}{dt}+\frac{w^2B^2}{mR}v=g\]
The auxiliary equation for the homogeneous case is
\[x+\frac{w^2B^2}{mR}=0\Rightarrow v_h=\exp\left( -\frac{w^2B^2}{mR}t\right)\]
A particular solution is by inspection $v=\frac{mgR}{w^2B^2}$, so the general solution is
\[v(t)=\frac{mgR}{w^2B^2}+c_1\exp\left( -\frac{w^2B^2}{mR}t\right)\]
Taking the initial velocity to be zero, $c_1$ is equal to the first summand, so we may factor the constant out to get
\[v(t)=\frac{mgR}{w^2B^2}\left[ 1-\exp\left( -\frac{w^2B^2}{mR}t \right) \right]\]
This confirms the terminal velocity calculation above.
The time it takes to reach 90\% of the terminal velocity may be found by
\[0.9=1-\exp\left( -\frac{w^2B^2}{mR}t \right) \Leftrightarrow t=-\frac{mR}{w^2B^2}\ln(0.1)=\frac{v_t}{g}\ln(10)\]
\[=\frac{\SI{0.012}{m/s}}{\SI{9.81}{m/s^2}}\ln(10)={0.00281}\]
If the loop were to be cut, no current could flow, so the loop would fall freely with acceleration $g$.


\end{document}

%%% Local Variables:
%%% mode: latex
%%% TeX-master: t
%%% End:
