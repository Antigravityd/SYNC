\documentclass{article}

\usepackage[letterpaper]{geometry}
\usepackage{siunitx}
\usepackage{amsmath}
\usepackage{amssymb}

\title{4132 HW 5}
\author{Duncan Wilkie}
\date{27 February 2022}

\begin{document}

\maketitle

\section*{8.8}
The electric field is by Gauss's law
\[\vec{E}=
  \begin{cases}
    0 & r < a\\
    \frac{Q}{4\pi\epsilon_0r^2}\hat{r} & a < r < b\\
    0 & r > b
  \end{cases}
\]
The momentum density in the fields is
\[\vec{g}=\epsilon_0\vec{E}\times\vec{B}=\frac{QB_0}{4\pi r^2}\sin\varphi(\hat{r}\times\hat{z})\]
The angular momentum density is therefore
\[\vec{\ell}=\vec{r}\times\vec{g}=\frac{QB_0}{4\pi r}\sin\varphi(\hat{r}\times[\hat{r}\times\hat{z}])=\frac{QB_0}{4\pi r}\sin\varphi\hat{z}\]
The total angular momentum is
\[\vec{L}=\int_0^\pi\int_0^{2\pi}\int_a^b\frac{QB_0}{4\pi r}\sin\varphi\hat{z}(r^2\sin\varphi)drd\theta d\varphi=\frac{QB_0(b^2-a^2)}{2}\int_0^{\pi}\sin^2\varphi d\varphi\]
\[=\frac{QB_0(b^2-a^2)}{2}\left( {\varphi}\sin^2\varphi\bigg|_0^\pi-\int_0^\pi \varphi\sin(2\varphi)d\varphi \right)=\frac{QB_0(b^2-a^2)}{2}\left( \frac{\varphi\cos{2\varphi}}{2}\bigg|_0^\pi -\int_0^\pi\frac{\cos(2\varphi)}{2}d\varphi\right)\]
\[=\frac{QB_0(b^2-a^2)}{2}\left( \frac{\pi}{2}-\frac{\sin(2\varphi)}{4}\bigg|_0^\pi \right)=\frac{\pi QB_0(b^2-a^2)}{4}\]
The induced electric field from the changing magnetic field is by integrating a circular Amp\`erian loop centered on the $z$ axis
\[2\pi rE_{ind}=\pi r^2\frac{\partial B}{\partial t}\Leftrightarrow E_{ind}=\frac{r}{2}\frac{\partial B}{\partial t}\]
in the negative $\theta$ direction.
The resulting force on a patch of charge of area $dA$ on the inner sphere is
\[F=\sigma dA\vec{E}_{ind}\]
Taking the left cross product by $\vec{r}=a\hat{r}$ and integrating over the surface,
\[\tau_{in}=\frac{Q}{4\pi a^2}\int_0^\pi\int_0^{2\pi}a\frac{r}{2}\frac{\partial B}{\partial t}(r^2\sin\varphi)d\theta d\varphi=\]
The above argument holds even if $B$ is time-varying, so the torque is
\[\vec{\tau}=\frac{d\vec{\ell}}{dt}=\frac{\pi Q(b^2-a^2)}{4}\frac{\partial B}{\partial t}\]
The total angular momentum after the field is zero is by the fundamental theorem of calculus equal to the expression derived above for the total momentum in the fields; the integral of $\frac{\partial B}{\partial t}$ from whenever $B=B_0$ to whenever $B=0$ is $B_0$.

\section*{8.14a}
This is almost identical to the problem on a previous homework with concentric surface currents. The magnetic field between the cylinders is by integrating a concentric, circular Amp\`erian loop between the cylinders
\[2\pi rB=\mu_0\lambda v\Rightarrow \vec{B}=\frac{\mu_0\lambda v}{2\pi r}\hat{\phi}\]
The electric field is by Gauss's law
\[2\pi r LE=\frac{\lambda L}{\epsilon_0}\Rightarrow \vec{E}=\frac{\lambda}{2\pi\epsilon_0 r}\hat{r}\]
The energy per unit volume in between the cylinders is
\[u=\frac{1}{2}\left( \epsilon_0E^2+\frac{1}{\mu_0}B^2 \right)=\frac{1}{2}\left( \frac{\lambda^2}{4\pi^2\epsilon_0r^2}+ \mu_0\frac{\lambda^2v^2}{4\pi^2r^2}\right)=\frac{\lambda^2}{8\pi^2r^2}\left( \frac{1}{\epsilon_0}+\mu_0v^2 \right)\]
Integrating this over $r$ and $\phi$, the energy per unit length is
\[E=\frac{\lambda^2}{8\pi^2}\left( \frac{1}{\epsilon_0}+\mu_0v^2 \right)\int_0^{2\pi}\int_a^b\frac{1}{r^2}rdrd\phi=\frac{\lambda^2\ln\left( \frac{b}{a} \right)}{4\pi}\left( \frac{1}{\epsilon_0}+\mu_0v^2 \right)\]

\section*{8.14b}
The momentum per unit volume is
\[\vec{g}=\epsilon_0\left( \vec{E}\times\vec{B} \right)=\mu_0\frac{v\lambda^2}{4\pi^2r^2}\hat{z}\]
Integrating this over the same region as above, the momentum per unit length is
\[L=\frac{\mu_0v\lambda^2}{4\pi^2}\hat{z}\int_0^{2\pi}\int_a^b\frac{1}{r^2} rdrd\phi=\frac{\mu_0v\lambda^2}{2\pi}\ln\left( \frac{b}{a} \right)\hat{z}\]

\section*{8.14c}
The Poynting vector is
\[\vec{S}=\frac{1}{\mu_0}\left( \vec{E}\times\vec{B} \right)=\frac{v\lambda^2}{4\pi^2\epsilon_0r^2}\hat{z}\]
Integrating this over a cross-sectional plane, the energy per unit time crossing such a plane is
\[P=\frac{v\lambda^2}{4\pi^2\epsilon_0}
  \int_0^{2\pi}\int_a^b\frac{1}{r^2}rdrd\phi=\frac{v\lambda^2}{2\pi\epsilon_0}\ln\left( \frac{b}{a} \right)\]

\section*{9.2}
The wave equation is
\[\Delta f=\frac{1}{v^2}\frac{\partial^2 f}{\partial t^2} \Leftrightarrow -Ak^2\sin(kz)\cos(kvt)=\frac{1}{v^2}\left( -Ak^2v^2\sin(kz)\cos(kvt) \right)\Leftrightarrow 0=0\]
so this is a wave.
Using a product-to-sum formula,
\[f=\frac{A}{2}\left[ \sin(kz+kvt)+\sin(kz-kvt) \right]\]
which is of the correct form, taking $g=h=\frac{A}{2}\sin(kx)$.

\section*{9.8a}
The so-circularly-polarized wave is
\[f^\circ=\tilde{A}\left[ e^{i(kz-\omega t)}\hat{x}+e^{i(kz-\omega t+\pi/2)}\hat{y}\right]\]
The angle of the vector is
\[\theta=\tan^{-1}\left( \frac{\Re e^{i(kz-\omega t)}}{\Re e^{i(kz-\omega t+\pi/2)}} \right)=\tan^{-1}\left( \frac{\cos(kz-\omega t)}{-\sin(kz-\omega t)} \right)=-\tan^{-1}(\cot(kz-\omega t))\]
For fixed $z$, the argument is initially decreasing as time increases. The arctangent is monotonically increasing, and the cotangent is monotonically decreasing until a vertical asymptote is encountered since tangent is monotonically increasing in the same sense. Therefore, the composition is monotonically decreasing, and the negation is therefore monotonically increasing. Since the amplitude is fixed, this is motion in a circle, and by the argument in the previous sentence the motion is in the positive $\theta$ direction, i.e. counterclockwise.
If one wanted a wave to circle the other direction, all that would be needed is to flip the direction in which one of the functions in the expression for theta is monotonic. Shifting by $-\pi/2$ does it, since the denominator remains positive afterwards.

\section*{9.8b}
It's a positively-oriented helix. It's a lot of trouble to fire up gnuplot or TikZ with 3D support; I'd probably run out of time waiting for it to compile.

\section*{9.8c}
The string would have to be shaken like a sinusoid in the vertical direction and like a $90^\circ$ shifted sinusoid in the horizontal direction, i.e. in a circle, since that's precisely the parametric form of one.

\end{document}
%%% Local Variables:
%%% mode: latex
%%% TeX-master: t
%%% End:
