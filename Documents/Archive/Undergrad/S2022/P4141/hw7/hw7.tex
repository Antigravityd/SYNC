\documentclass{article}

\usepackage[letterpaper]{geometry}
\usepackage{amsmath}
\usepackage{amssymb}
\usepackage{siunitx}

\title{4141 HW 7}
\author{Duncan Wilkie}
\date{1 April 2022}

\begin{document}

\maketitle

\section{}
For a state $|\psi\rangle$, hermicity of $A$ is by definition
\[A=A^{\dagger}\Leftrightarrow\langle A\psi | \psi\rangle={\langle \psi | A\psi \rangle}\]
This implies
\[\langle A^{2} \rangle=\langle \psi| A^{2}|\psi \rangle=\langle \psi | A^{2}\psi \rangle=\langle A\psi | A\psi \rangle=||A\psi||^{2}\]
where the norm is not the complex absolute value but the norm induced by the $L^{2}$ inner product. Since it is a norm, it is positive-definite, which implies the final result, that $\langle A^{2} \rangle \geq 0$ (with $\langle A^{2} \rangle=0$ iff $A\psi=0$).

\section{}
For a wavefunction $\psi$,
\[\Pi \hat{p}\psi = \Pi\frac{\hbar}{i}\frac{\partial \psi}{\partial x}=\frac{\hbar}{i}\frac{\partial\psi}{\partial x}(-x)\]
\[\hat{p}\Pi\psi=\frac{\hbar}{i}\frac{\partial }{\partial x}\psi(-x)=-\frac{\hbar}{i}\frac{\partial \psi}{\partial x}(-x)\]
so these operators anticommute.
Therefore,
\[[\Pi,T]=\Pi\frac{\hat{p}^{2}}{2m}-\frac{\hat{p}^{2}}{2m}\Pi=\frac{1}{2m}\left( -\hat{p}\Pi\hat{p} + \hat{p}\Pi\hat{p}\right)=0\]

\section{}
We know that in general $[\hat{x},\hat{p}]=i\hbar\neq 0$, $\hat{H}=\hat{T}+\hat{V}=\frac{\hat{p}^{2}}{2m}+\hat{V}$, and $\Pi: x\mapsto -x$. For the case of the free particle, $\hat{V}=0$, we can then write down commutation relations
\[[\hat{x},\hat{p}]=i\hbar\neq 0\]
\[[\hat{x},\hat{H}]=\hat{x}\frac{\hat{p}^{2}}{2m}-\frac{\hat{p}^{2}}{2m}\hat{x}=\frac{\hat{p}}{2m}\left( \hat{x}\hat{p}-\hat{p}\hat{x} \right)-\frac{\hat{p}\hat{x}\hat{p}}{2m}+\frac{x\hat{p}^{2}}{2m}=\frac{1}{2m}\left( \hat{p}[\hat{x},\hat{p}] + [\hat{x},\hat{p}]\hat{p} \right)\]
\[=\frac{i\hbar}{2m}\left( \hat{p}+\hat{p} \right)=\frac{i\hbar}{m}\hat{p}\neq 0 \textrm{ if the particle is moving}\]
\[[\hat{x},\Pi]\psi=\hat{x}\Pi\psi-\Pi\hat{x}\psi=\int_{\mathbb{R}}\psi^{*}(-x)x\psi(-x)dx-\int_{\mathbb{R}}\psi^{*}(-x)(-x)\psi(-x)\neq 0\]
\[[\hat{p},\Pi]\neq 0 \textrm{ as shown above}\]
\[[\hat{p}, \hat{H}]=\hat{p}\frac{\hat{p}^{2}}{2m}-\frac{\hat{p}^{2}}{2m}\hat{p}=\frac{1}{2m}\left( \hat{p}^{3}-\hat{p}^{3} \right)=0\]
\[[\Pi,\hat{H}]=[\Pi,T]=0 \textrm{ as shown above}\]
Therefore, the subsets which are internally mutually commutative are
\[\{\hat{p},\hat{H}\}\]
and
\[\{\Pi,\hat{H}\}\]

\section{}
We must apply a change of basis to the original wavefunction so that one of its basis vectors is $|\psi_{f}\rangle$.
\[|\psi_{i}\rangle=|\psi_{f}\rangle\langle \psi_{f}|\psi_{i} \rangle+|\beta\rangle\langle \beta|\psi_{i} \rangle+|\gamma\rangle\langle \gamma|\psi_{i} \rangle=((i-1)/\sqrt{3}+\frac{1}{3})|\psi_{f}\rangle+\sqrt{2/3}|\beta\rangle\]
The probability is then the norm-squared of the coefficient of $|\psi_{f}\rangle$, so
\[\Rightarrow P(|\psi_{f}\rangle)=|i/3|^{2}=\frac{1}{3}\]
This expression of $|\psi_{i}\rangle$ retains normalization, which is always good to check.

\section{}
Normalization means $\langle \psi|\psi \rangle=1$. If this holds for $\psi$, then for some unitary operator $U$ we have
\[\langle U\psi|U\psi \rangle=\langle  \psi | U^{\dagger}U\psi \rangle=\langle \psi | I\psi \rangle=\langle  \psi|\psi \rangle=1\]
so unitary operators preserve normalization.
\end{document}
%%% Local Variables:
%%% mode: latex
%%% TeX-master: t
%%% End:
