\documentclass{article}

\usepackage[letterpaper]{geometry}
\usepackage{amsmath}
\usepackage{amssymb}
\usepackage{siunitx}
\usepackage{graphicx}

\title{4141 HW 8}
\author{Duncan Wilkie}
\date{8 April 2022}

\begin{document}

\maketitle

\section{}
Call the operator $G$.
By the definition of hermicity,
\[
  \langle Gf|g \rangle = \langle f | Gg \rangle
  \Leftrightarrow \int_{\mathbb{R}}\left( Gf(\phi) \right)^{*}g(\phi)d\phi=\int_{\mathbb{R}}f(\phi)^{*}Gg(\phi)d\phi
  \Leftrightarrow \frac{\hbar}{i}\int_{\mathbb{R}}f'(\phi)^{*}g(\phi)d\phi=\frac{\hbar}{i}\int_{\mathbb{R}}f(\phi)^{*}g'(\phi)d\phi
\]
\[
  \Leftrightarrow \frac{\hbar}{i}\left( f(\phi)^{*}g(\phi)\bigg|_{-\infty}^{\infty}-\int_{\mathbb{R}}f'(\phi)^{*}g'(\phi)d\phi\right)
    =\frac{\hbar}{i} \left( f(\phi)^{*}g(\phi)\bigg|_{-\infty}^{\infty}-\int_{\mathbb{R}}f'(\phi)^{*}g'(\phi)d\phi \right)
\]
Therefore, $G$ is Hermitian (periodicity implies everything converges---partitions may be chosen so the evaluation is zero on each interval).
Taking the inverse Fourier transform of $Gf$,
\[
  \mathcal{F}^{-1}(Gf)=\int_{\mathbb{R}}e^{2\pi ix\phi}\frac{\hbar}{i}f'(\phi)d\phi
  =\frac{\hbar}{i}f(\phi)e^{2\pi ix\phi}\bigg|_{-\infty}^{\infty}-\frac{\hbar}{i}\int_{\mathbb{R}}f(\phi)2\pi ixe^{2\pi ix\phi}d\phi
\]
Because $f$ is periodic, the first term is zero, since if one splits the evaluation into a sum of periodic sub-intervals, one obtains zero on each sub-interval.
This then becomes
\[
  =2\pi\hbar x\int_{\mathbb{R}}f(\phi)e^{2\pi i x \phi}d\phi = 2\pi\hbar x f(x)
\]
The expectation value of $G$ is therefore, in position space, the expectation value of the operator $4\pi^{2}\hbar^{2} x^{2}$, where the constants are likely due to a poor Fourier transform convention. The operator can then be simply identified with $x^{2}$.

\section{}
It's easy to represent $H$ as a matrix if one compares how it acts on $\psi=a|1\rangle+b|2\rangle$ to the action of a matrix on the vector $(a, b)$.
\[H=
  \epsilon\begin{pmatrix}
    \langle 1|1 \rangle + \langle 2|1 \rangle & \langle 1|2 \rangle+\langle 2|2 \rangle \\
    \langle 1|1 \rangle -\langle 2|1 \rangle& \langle 1|2 \rangle-\langle2|2  \rangle
  \end{pmatrix}
\]
By orthonormality, we may write this
\[
  =\epsilon
  \begin{pmatrix}
    1 & 1 \\
    1 & -1
  \end{pmatrix}
\]
This corresponds to an eigenvalue equation
\[
  \det(H-\lambda I)
  =\det
  \begin{vmatrix}
    1-\lambda & 1 \\
    1 & -1-\lambda
  \end{vmatrix}
  =\lambda^{2}-1-1=\lambda^{2}-2=0\Rightarrow \lambda = \pm\sqrt{2}
\]
with multiplicity 2. The eigenvectors may then be immediately computed as $(1-\sqrt{2})|1\rangle+|2\rangle$ and $(1+\sqrt{2})|1\rangle +|2\rangle$, which one may normalize by dividing by the norm if desired.

\section{}

\[
  \hat{O}=|\psi\rangle\langle \phi|
\]
\[
  \frac{\partial}{\partial x}f(x)=|h\rangle\langle h|g \rangle
\]

\section{}
By the generalized uncertainty principle,
\[
  \sigma_{E}^{2}\sigma_{x}^{2}\geq\left( \frac{1}{2i}\left\langle \left[ \hat{H},\hat{x} \right] \right\rangle \right)^{2}
  =\left( \frac{1}{2i}\left\langle \frac{\hat{p}^{2}}{2m}\hat{x}+\hat{V}\hat{x}-\hat{x}\frac{\hat{p}^{2}}{2m}
      -\hat{x}\hat{V} \right\rangle \right)^{2}
  =\left( \frac{1}{2i}\left\langle\frac{\hat{p}^{2}}{2m}\hat{x}-\hat{x}\frac{\hat{p}^{2}}{2m}\right\rangle \right)^{2}
\]
\[
  =\left(  \frac{1}{2i}\left\langle\frac{\hat{p}}{2m}\left[\hat{p}\hat{x}-\hat{x}\hat{p} \right] \right\rangle\right)^{2}
  =\left( \frac{1}{2i}\left\langle\frac{\hat{p}}{2m}i\hbar \right\rangle \right)^{2}
  =\left( \frac{\hbar}{4m}\langle \hat{p} \rangle\right)^{2}
\]
Taking the square root,
\[\sigma_{E}\sigma_{x}\geq \frac{\hbar|\langle p \rangle|}{4m}\]

\section{}
The definition of the ladder operators for the harmonic oscillator are
\[
  \hat{a}_{+}=\frac{1}{\sqrt{2\hbar m\omega}}\left( m\omega x-i\hat{p} \right)
\]
\[
  \hat{a}_{-}=\frac{1}{\sqrt{2\hbar m\omega}}\left( m\omega x+i\hat{p} \right)
\]
Adding them together, one eliminates $\hat{p}$ to obtain
\[
  \hat{a}_{+}+\hat{a}_{-}=\frac{2m\omega x}{\sqrt{2\hbar m\omega}}
  \Leftrightarrow x=(\hat{a}_{+}+\hat{a}_{-})\sqrt{\frac{\hbar}{2m\omega}}
\]
Subtracting them, one eliminates $\hat{x}$ to obtain
\[
  \hat{a}_{+}-\hat{a}_{-}=\frac{-2i\hat{p}}{\sqrt{2\hbar m\omega}}
  \Leftrightarrow \hat{p}=(\hat{a}_{+}-\hat{a}_{-})i\sqrt{\frac{\hbar m\omega}{2}}
\]
One can now self-compose these operators to obtain
\[
  \hat{x}^{2}=\frac{\hbar}{2m\omega}(\hat{a}_{+}^{2}+\hat{a}_{+}\hat{a}_{-}+\hat{a}_{-}\hat{a}_{+}+\hat{a}_{-}^{2})
\]
\[
  \hat{p}^{2}=\frac{\hbar m\omega}{2}(\hat{a}_{-}^{2}-\hat{a}_{-}\hat{a}_{+}-\hat{a}_{+}\hat{a}_{-}+\hat{a}_{+}^{2})
\]
The kinetic energy is then
\[
 {T}=\frac{p^{2}}{2m}=\frac{\hbar\omega}{4}(\hat{a}_{-}^{2}-\hat{a}_{-}\hat{a}_{+}-\hat{a}_{+}\hat{a}_{-}+\hat{a}_{+}^{2})
\]
and the potential is
\[V=\frac{1}{2}m\omega^{2}x^{2}=\frac{\hbar\omega}{4}(\hat{a}_{+}^{2}+\hat{a}_{+}\hat{a}_{-}+\hat{a}_{-}\hat{a}_{+}+\hat{a}_{-}^{2})\]
In calculating the expectation values for both these quantities, the squared terms drop out, because they result in something proportional to the state with respect to which the expectation is taken with $n$ plus or minus 2, which, since the eigenvector basis is orthonormal, results in zero when applied to any vector.
We then need only calculate, since the other term is the negation of this,
\[
  \langle \hat{a}_{-}\hat{a}_{+}+\hat{a}_{+}\hat{a}_{-}\rangle
  =\int_{\mathbb{R}}\psi_{n}^{*}(x)(\hat{a}_{-}\hat{a}_{+}+\hat{a}_{+}\hat{a}_{-})\psi_{n}(x)dx
  =\int_{\mathbb{R}}\psi_{n}^{*}(x)[(n+1)\psi_{n}(x)+n\psi_{n}(x)]dx
\]
By orthonormality,
\[
  =(n+1)\langle \psi_{n}|\psi_{n} \rangle+n\langle \psi_{n}|\psi_{n} \rangle=2n+1
\]
Therefore,
\[\langle T \rangle=\frac{\hbar\omega}{4}(2n+1)\]
\[\langle V \rangle=\frac{\hbar\omega}{4}(2n+1)\]
each of which is, indeed, half the total energy $E_{n}=\hbar\omega(n+\frac{1}{2})$, confirming the virial theorem.

\section{}
By the generalized Ehrenfest theorem,
\[
  \frac{d}{dt}\langle xp \rangle
  =\frac{i}{\hbar}\left\langle \left[ \hat{H},xp \right]\right\rangle+\left\langle \frac{\partial (xp)}{\partial t}\right\rangle
\]
The operator is time-independent, so we only need the commutator:
\[
  [\hat{H},xp]=[H,x]p+x[H,p]
\]
On the last homework, the first commutator was calculated to be $-\frac{i\hbar p}{m}$ ($x$ commutes with $V(x)$, so the fact that Hamiltonian was a free particle is irrelevant).
The second commutator may be computed as
\[[H,p]=(\frac{p^{2}}{2m}+V(x))p-p(\frac{p^{2}}{2m}+V(x))=V(x)p-pV(x)=V(x)\frac{\hbar}{i}\frac{\partial}{\partial x}-\frac{\hbar}{i}\frac{\partial V}{\partial x}\]
For a stationary state, the first will evaluate to zero.
We then have
\[
  \frac{d}{dt}\langle xp \rangle=\langle \frac{p}{m}p \rangle-\langle x\frac{\partial V}{\partial x} \rangle=2\langle T \rangle-\langle x\frac{\partial V}{\partial x} \rangle
\]
For a stationary state, the left side is certainly zero, since the wave function is time independent, so any time-independent operator acting on it will also be.
Therefore,
\[\langle 2T \rangle=\langle x\frac{\partial V}{\partial x} \rangle\Leftrightarrow \langle \frac{p^{2}}{m} \rangle=\langle x\frac{\partial V}{\partial x} \rangle\]
For a power-law potential, this becomes
\[\langle \frac{p^{2}}{m} \rangle=\langle 2T \rangle=\langle nkx^{n} \rangle=\langle nV  \rangle\]



\end{document}
%%% Local Variables:
%%% mode: latex
%%% TeX-master: t
%%% End:
