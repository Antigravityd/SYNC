\documentclass[10pt]{article}

\usepackage[letterpaper]{geometry}
\usepackage{amsmath}
\usepackage{amssymb}
\usepackage{siunitx}
\usepackage{graphicx}


\begin{document}
%\begin{tiny}


\section*{Crystal Structure}
A \textbf{lattice} is a vector space but with scalars in a ring instead of a field.
In our case, the ring is $\mathbb{Z}$. A \textbf{unit cell} is any subset of a lattice which may be tiled to recover the whole lattice.
A \textbf{primitive unit cell} is a unit cell that contains exactly one lattice point,
where lattice points are counted according to the angles the incoming and outgoing lattice vectors make.
The set of $\mathbb{R}^n$ points closest to a given lattice point is its \textbf{Wigner-Seitz cell},
constructed by drawing lines from one point to all its neighbors and forming a polygon from the perpendicular bisectors halfway along these lines.
Bases for lattices are constructed by choosing a reference point in the unit cell and labeling every lattice point in the unit cell as a linear combination of lattice vectors over $\mathbb{R}$.
The \textbf{primitive lattice vectors}, the basis for the primitive unit cell, for the elementary lattices considered in this class appear below.
\begin{center}
  \begin{tabular}{ |c|c|c|c| }
    \hline
    Lattice Type & $a_{1}$ & $a_{2}$ & $a_{3}$ \\
    \hline
    \hline
    simple cubic & [1,0,0] & [0,1,0] & [0,0,1] \\
    \hline
    bcc & [1,0,0] & [0,1,0] & $[\frac{1}{2},\frac{1}{2},\frac{1}{2}]$ \\
    \hline
    fcc & $[\frac{1}{2},\frac{1}{2},0]$ & $[\frac{1}{2},0,\frac{1}{2}]$ & $[0,\frac{1}{2},\frac{1}{2}]$ \\
    \hline
  \end{tabular}
\end{center}
The \textbf{reciprocal lattice points} $\vec{G}$ to a direct lattice $\vec{R}$ are defined by
$e^{i\vec{G}\cdot\vec{R}}$, and in one dimension if $\vec{R}$ has periodicity $a$ are spaced by $2\pi/a$.
The primitive lattice vectors of the reciprocal lattice are defined by $\vec{a}_{i}\cdot\vec{b}_{j}=2\pi\delta_{ij}$
and may be constructed as
$\vec{b}_{i}=\frac{2\pi \vec{a}_{i+1}\times\vec{a}_{i+2}}{\vec{a}_{1}\cdot(\vec{a}_{2}\times\vec{a}_{3})}$
where $a_{i}$ are the primitive lattice vectors of the direct lattice and the addition in the subscripts is modulo 3.
One may consider the reciprocal lattice to be Fourier-conjugate to the direct lattice by writing the lattice as a delta function
of the lattice points by $\rho(r)=\sum_{n}\delta(r-an)$ in one dimension,
easily generalizable to more by identification of $r$ with $\vec{k}$ and $an$ with $\vec{G}$.
The Fourier convention for this class is
$\mathcal{F}(f(x))(k)=\int_{\mathbb{R}^{n}}f(x)e^{ik\cdot x}dx$;
the inverse transform is correspondingly
$\mathcal{F}^{-1}(g(k))(x)=\frac{1}{(2\pi)^{n}}\int_{\mathbb{R}^{n}}g(k)e^{-ik\cdot x}$.
A \textbf{lattice plane} is a plane containing 3 non-collinear points of a lattice.
A \textbf{family of lattice planes} is an infinite set of equally-spaced parallel lattice planes which together contain
every lattice point.
Families of lattice planes bijectively correspond to directions of reciprocal lattice vectors,
so that the separation between planes is $d=2\pi/|\vec{G}_{min}|$ where $\vec{G}_{min}$ is the shortest reciprocal lattice vector
in that direction.
The \textbf{Miller indices} are simply an expression of the reciprocal lattice points inside any unit cell in terms of the reciprocal
space vectors conjugate to edge vectors chosen for the unit cell in direct space;
they can be negative, in which case they are conventionally denoted with an overline.
To construct them, one takes the definition of the reciprocal lattice vectors from above and extends it to non-primitive-lattice
vectors.
The Miller indices do not necessarily represent reciprocal lattice vectors unless the unit cell is primitive.
A \textbf{Brillouin zone} is a primitive unit cell of the reciprocal lattice.
The $n$th Brillouin zone may be constructed by taking the perpendicular bisector halfway between the origin and each of the
reciprocal lattice vectors in the unit cell.
The $n$th Brillouin zone is the region where it takes at least $n-1$ crossings of perpendicular bisectors to reach the origin,
where crossing at intersections of bisectors is disallowed.
The \textbf{Fermi surface} is a surface in $k$-space that divides filled from unfilled states at zero temperature.
It is halfway between the most energetic occupied and least energetic unoccupied states.
The \textbf{Fermi energy} is the chemical potential at zero temperature, and corresponds to the energy of states on the Fermi surface.
The \textbf{Fermi sea} is the collection of filled states at $T=0$.

\section*{Scattering}
Fermi's golden rule states the probability of transitioning from a state $|\vec{k}\rangle$ to $|\vec{k'}\rangle$ per unit time
under a small perturbation of the potential $V$ is
$\Gamma(\vec{k'},\vec{k})=\frac{2\pi}{\hbar}|\langle \vec{k'}|V|\vec{k} \rangle|^{2}\delta(E_{k'}-E_{k})$ where $E_{k'},E_{k}$
are the energies of the associated states and $|k\rangle,|k'\rangle$ are the free-particle wave functions of that which is scattering.
The matrix term may be written, by assuming everything to be periodic and integrating over the unit cell,
\[
  \langle \vec{k'}|V|\vec{k}\rangle=\left[ \frac{1}{L^{3}}\sum_{\vec{R}}e^{-i(\vec{k'}-\vec{k})\cdot\vec{R}} \right]
  \left[ \int_{\textrm{unit cell}}e^{-i(\vec{k'}-\vec{k})\cdot \vec{x}}V(\vec{x})d\vec{x} \right]
\]
The first term, and therefore the overall matrix term, is zero except for those vectors for which the \textbf{Laue condition}
$\vec{k'}-\vec{k}=\vec{G}$ holds.
Equivalent to the Laue condition is the \textbf{Bragg condition}, that $n\lambda = 2d\sin\theta$ where $\lambda$ is the wavelength
of incoming particles, $d$ is the spacing between layers of atoms in the crystal, $\theta$ is the angle of incidence,
and $n$ is any integer, corresponding to the $n$th diffraction level.
Returning to the equation for the matrix in Fermi's golden rule, the first term is a constant, nonzero only when the Laue condition
is satisfied. The second term is the \textbf{structure factor}, which one may write as a function of $\vec{G}$ alone via the Laue
condition: $S(\vec{G})=\int_{\textrm{unit cell}}e^{i\vec{G}\cdot\vec{k}}V(\vec{x})d\vec{x}$.
In terms of Miller indices, one may explicitly write $S_{(hkl)}=\sum_{\textrm{atom }j}f_{j}e^{2\pi i(hx_{j}+ky_{j}+lz_{j})}$ where
$[x_{j},y_{j},z_{j}]$ are the coordinates of the atom in the unit cell.
One may usually take the total potential to be a sum over the atoms in the unit cell of each of the atom's scattering potential,
as inter-atomic interactions are unlikely to have that large of an effect on the interaction between an incoming particle
and the atoms of the crystal.
In neutron scattering, these scattering potentials are nuclear and so approximately delta functions, and we may write
$V(\vec{x})=\sum_{\textrm{atoms} j}f_{j}\delta(x-x_{j})$ where $f_{j}$ is the atomic form factor, a number particular to the atom.
It is proportional to the nuclear scattering-length $b_{j}$ by a factor of $\frac{2\pi\hbar^{2}}{m}$.
X-rays scatter from electrons.
We may write the scattering potential as $V_{j}(\vec{x}-\vec{x}_{j})=Z_{j}g_{j}(\vec{x}-\vec{x}_{j})$ where $Z_{j}$ is the atomic
number of the atom and $g_{j}$ is a short-ranged function on the scale of Angstroms.
The form factor for X-rays is then $f_{j}(\vec{G})=Z_{j}g_{j}(\vec{x}-\vec{x}_{j})$, where the dependence on $\vec{G}$ is in the
functional form of $g_{j}$.
The $\vec{G}$ dependence may be approximated away if necessary.
The intensity of the scattered wave is written for a specific Miller index as $I_{(hkl)}\propto |S_{(hkl)}|^{2}$.
A significant implication of this is systematic absences of scattered particles for certain Miller indices;
in a simple cubic all indices are allowed, for bcc materials $h+k+l \equiv 0\mod 2$, and for fcc $h,k,$ and $l$ must have identical
parity.

Scattering experiments come in three common forms: Laue scattering, where the wavelength incident on a single crystal is varied
at constant angle, the rotating crystal method, where a constant wavelength is scattered off a single crystal at varying angle,
and powder diffraction, where multiple crystals with random orientations undergo scattering at constant wavelength and varying angle,
and the spacing of lattice planes is deduced from Bragg's law and the observed scattering intensity.
In powder diffraction, the scattering intensity is multiplied by a multiplicity factor corresponding to the fact that several
equivalent orientations of the crystal will scatter and constructively interfere at a given angle.

\section*{Electrons in Solids}
In the nearly free electron model, one considers a free particle Hamiltonian perturbed by a weak, periodic potential.
The matrix elements of this potential are the same as the scattering potential matrix sans the sum term inside the first bracket.
Far from Brillouin zone boundaries, the second-order perturbation series
$E(\vec{k})=E_{0}(\vec{k})+\sum_{\vec{k'}=\vec{k}+\vec{G}}\frac{|\langle \vec{k'}|V|\vec{k} \rangle|^{2}}{E_{0}(\vec{k})-E_{0}(\vec{k'})}$
where $\vec{k'}\neq \vec{k}$ works reasonably well.
However, near the Brillouin zone boundaries, the term in the denominator diverges, as at the zone boundary ${k'}=-k=\frac{n\pi}{a}$.
It's then necessary to apply degenerate perturbation theory near the boundary.
The effective Schr\"odinger equation is in this case
\[
  \begin{pmatrix}
    E_{0}(\vec{K}) & V_{\vec{G}}^{*} \\
    V_{\vec{G}} & E_{0}(\vec{k}+\vec{G})
  \end{pmatrix}
  \begin{pmatrix}
    a \\
    b
  \end{pmatrix}
  = E
  \begin{pmatrix}
    a \\
    b
  \end{pmatrix}
\]
where $V_{\vec{G}}=V_{k'-k}$ is the scattering potential.
When $k\equiv 0 \pmod{\vec{G}}$, the diagonal terms are equal, and the characteristic equation yields
$E_{\pm}=E_{0}(\vec{k})\pm |V_{\vec{G}}|$.
This is the band gap at the zone boundary.
If one is not quite on the zone boundary, a bunch of algebraic waterboarding will yield in one dimension
\[
  E_{\pm}=\frac{\hbar^{2}(n\pi/a)^{2}}{2m}\pm|V_{G}|+\frac{\hbar^{2}\delta^{2}}{2m}\left[  1\pm\frac{\hbar^{2}(n\pi/a)^{2}}{m}\frac{1}{|V_{G}|}\right]
\]
where $\delta$ is the perturbation away from the zone boundary.
This implies quadratic behavior of the energy near the zone boundary.
\textbf{Bloch's theorem} states that an electron in a periodic potential has \textbf{modified plane wave} eigenstates
$\Psi_{\vec{k}}^{\alpha}(\vec{r})=e^{i\vec{k}\cdot\vec{r}}u_{\vec{k}}^{\alpha}(\vec{r})$
where the \textbf{Bloch function} $u_{\vec{k}}^{\alpha}$ is periodic and $\vec{k}$ can be taken to be in the first Brillouin zone.
It follows from the Laue condition, as may be seen by decomposing the Bloch function over the reciprocal lattice vectors as
$u_{\vec{k}}^{\alpha}(\vec{r})=\sum_{\vec{G}}\tilde{u}_{\vec{G},\vec{k}}^{\alpha}e^{i\vec{G}\cdot\vec{r}}$.
This theorem is interpreted as stating that electrons traveling through a crystal behave almost as if they were propagating through
free space.
The number of $k$ states in one Brillouin zone is equal to the number of primitive unit cells in the system.
If each primitive unit cell has exactly one free electron, it appears that electron would fill the band.
However, this actually applies to spin-pairs of electrons, and so in such a case the band is only half full.
In such a case, when the band isn't full, an applied electric field will cause electrons from one band to move over to fill
the band in their neighbors, i.e. materials with partially filled bands are \textbf{metals}.
When there are two electrons per unit cell, it is possible to fill the band.
When this happens, the filled, low-energy band is called the valence band, and the high-energy band is called the
\textbf{conduction band}.
The energy difference between the two bands is called the \textbf{band gap}.
Such materials are called \textbf{band insulators}.
If the band gap is below about $\SI{4}{eV}$, thermal excitations are sufficient to move electrons into the conduction band.
These materials are \textbf{semiconductors}.
Some materials half-fill both the valence and conduction bands, despite having two electrons per unit cell.
These materials are metals.
The Fermi surface for a monovalent atom in the absence of a periodic potential is perfectly spherical, with volume exactly half
that of the Brillouin zone.
Including a periodic potential lowers the energy of the electron in regions close to the Brillouin zone boundary, pulling
the Fermi surface towards the zone boundary while keeping its area constant (so the number of electrons is fixed).
If, due to a very strong potential, the Fermi surface intersects the zone boundary, it must to so perpendicularly because
the energy is quadratic in small displacements from the zone boundary.
For bivalent atoms, the area of the Fermi surface equals that of the Brillouin zone.
For weak and intermediate potentials, the Fermi surface pokes out of the first Brillouin zone and spills over into the second.
In this case, some states are empty in the first zone and some states are filled in the second, so the material is a metal.
For sufficiently strong potentials, the Fermi surface is identical to the first Brillouin zone, and the whole band is filled.
Due to the band gap, these are insulators.
Insulators can't absorb photons with energies less than their band gap, because for that to happen the photon must excite the
electron from the valence to the conduction band.
If an insulator has a band gap greater than $\SI{3.6}{eV}$, it will be transparent, because all wavelengths of visible light are
lower-energy.
A \textbf{direct} band gap is one for which the value of $\vec{k}$ at the valence band maximum and the conduction band minimum are
the same.
Otherwise, it is called an \textbf{indirect} band gap.

%\end{tiny}
\end{document}
%%% Local Variables:
%%% mode: latex
%%% TeX-master: t
%%% End:
