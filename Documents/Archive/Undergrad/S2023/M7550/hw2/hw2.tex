\documentclass{article}

\usepackage[letterpaper]{geometry}
\usepackage{tgpagella}
\usepackage{amsmath}
\usepackage{amssymb}
\usepackage{amsthm}
\usepackage{tikz}
\usepackage{minted}
\usepackage{physics}
\usepackage{siunitx}

\sisetup{detect-all}
\newtheorem{plm}{Problem}
\newtheorem{lem}{Lemma}

\title{Math 7550 HW 2}
\author{Duncan Wilkie}
\date{10 February 2023}

\begin{document}

\maketitle

\begin{plm}
  If $\phi: \mathbb{R}^{m} \to \mathbb{R}^{n}$ is a linear map and we identify $T_{p}(\mathbb{R}^{k})$ with $\mathbb{R}^{k}$
  by identifying $\pdv{x_{i}}$ with the $i$th standard basis vector, show that $\phi_{*}$ is just $\phi$.

  In other words, using the $(d\phi)_{p}$ notation for $\phi_{*}$ at $p$, for any $p \in \mathbb{R}^{m}$, show that $(d\phi)_{p} = \phi$.
  (Note that you are implicitly showing that $\phi$ is differentiable)
\end{plm}

\begin{proof}
  The identification is given by considering paths through $p$ given by $\gamma_{v} = p + tv$, and noticing that $D_{\gamma_{v}} = v$.
  This restricted set of paths is sufficient to determine the rule of $\phi_{*}$,
  because the tangent vectors so-generated span a space of dimension $n$, and since they are all elements of $T_{p}M$,
  which is of dimension $n$, they must span $T_{p}M$.
  By definition,
  \[
    \phi_{*}(D_{\gamma_{v}}) = D_{\phi \circ \gamma_{v}} = \dv{t}\phi(p + tv)\bigg|_{t = 0} = \dv{t}\qty[\phi(p) + t\phi(v)]\bigg|_{t = 0}
    = \phi(v).
  \]
\end{proof}

\begin{plm}
  Generalize Problem 1: if $\phi: \mathbb{R}^{m} \times \mathbb{R}^{n} \to \mathbb{R}^{k}$ is a bilinear map, show that $\phi$ is differentiable,
  and that, for any $(p, q) \in \mathbb{R}^{m} \times \mathbb{R}^{n}$, we have $(d\phi)_{(p,q)}(x,  y) = \phi(p, y) + \phi(x, q)$.
\end{plm}

\begin{proof}
  Writing elements of the domain $(u, v)$, we can define a natural vector space structure from the isomorphic space $\mathbb{R}^{n+m}$:
  $a(u, v) + (u', v') = (au + u', av + v')$.
  With respect to this, we can define a path through $(p, q)$ analogous to that above by $\gamma_{(u, v)}(t) = (p, q) + t(u, v)$.
  Note that this vector space structure is used strictly to define $\gamma_{(u, v)}$; the result does not depend on it.
  For identical reasons as in the previous problem, such paths are sufficient to describe al germs of differential operators through $(p, q)$.
  Similarly, one can immediately see that $D_{\gamma_{(u, v)}} = (u, v)$ and
  \[
    \phi_{*}(D_{\gamma_{(u, v)}}) = D_{\phi \circ \gamma_{(u, v)}} = \dv{t}\phi[(p, q) + t(u, v)]\eval_{t = 0} = \dv{t}\phi(p + tu, q + tv)\eval_{t = 0}
  \]
  \[
    = \dv{t}\qty[\phi(p, q + tv) + \phi(tu, q + tv)]\eval_{t = 0}
    = \dv{t}\qty[\phi(p, q) + \phi(p, tv) + \phi(tu, q) + \phi(tu, tv)]\eval_{t = 0}
  \]
  \[
    = \dv{t}\qty[\phi(p, q) + t\phi(p, v) + t\phi(u, q) + t^{2}\phi(u, v)]\eval_{t = 0}
    = \phi(p, v) + \phi(u, q) + 2t\phi(u, v)\eval_{t = 0}
  \]
  \[
    = \phi(p, v) + \phi(u, q)
  \]
  which is of the desired form (despite my using different variables).
\end{proof}

\begin{plm}
  Let $M = M_{m,n}(\mathbb{R})$ be the set of $m \times n$ matrices with real entries.
  Fix $k \leq \min(m,  n)$, and let $U_{k} = \{A \in M \mid \rank(A) \geq k\}$.
  Describe a $C^{\infty}$ structure for $U_{k}$.
\end{plm}

\begin{proof}
  The topology on the set of matrices is not specified, so I'll presume it's the one borrowed from Euclidean space
  by considering matrices as vectors in $\mathbb{R}^{mn}$.
  The complement of $U_{k}$ is the set of matrices $A$ of rank less than $k$.
  A matrix has rank $< k$ iff for every $r \geq k$, the determinant of every $r \times r$ submatrix is zero.
  The number of such submatrices for a given $r$ is computed by choosing which rows and columns to include,
  so $\sum_{r = k}^{\min(n,m)}\binom{n}{r}\binom{m}{r}$; call this number $l$.
  Consider the linear map $f: M \to \mathbb{R}^{l}$ that, with respect to some arbitrary ordering of the submatrices counted by $l$,
  maps a matrix to the vector whose $i$th entry is the determinant of the $i$th submatrix.
  This is continuous, because the determinant is a linear functional,
  and the fact that maps into products are continuous iff all their projections are continuous.
  Then the complement of $U_{k}$ is the kernel of $f$; kernels of continuous maps are closed, and so $U_{k}$ is open in $M$.

  This allows us to immediately borrow the functional structure from $M$, identified with $\mathbb{R}^{mn}$,
  as $U_{k}$ is accordingly identified with an open subset of $\mathbb{R}^{mn}$, and as such has a chart.
\end{proof}

\begin{plm}
  The Grassman manifold or Grassmannian $G_{k,n}$: let $G_{k,n}$ be the set of all $k$-dimensional vector subspaces of $\mathbb{R}^{n}$.
  Show that $G_{k,n}$ is a smooth manifold of dimension $k(n - k)$.
\end{plm}

\begin{proof}
  There is an onto correspondence of lists of $k$ linearly-independent vectors in $\mathbb{R}^{n}$ and its $k$-dimensional vector subspaces,
  since such lists generate subspaces by spanning and every subspace can be so-generated since every subspace has a basis.
  Such lists, call them $F_{k,n}$, can be viewed as $k \times n$ matrices with real entries by just juxtaposing the column vectors expressed
  wrt. the standard basis; the condition that the list is linearly independent translates to these matrices having rank $k$, by definition of rank.
  Since $n \geq k$, and $k$ is one dimension of the matrix, the space of such lists is $U_k$ from the previous problem;
  we can borrow the $C^{\infty}$ structure defined there.
  Letting two elements of $F_{k,n}$ be equivalent via $\sim$ if the lists span the same vector subspace, one recognizes $G_{k,n} = F_{k,n} / \sim$.
  In a just world, we'd have a way of constructing quotient manifolds already.
  But we don't yet.

  The coordinate charts on $G_{k, n}$ may be given by taking some choice $J$ of $k$ elements of $\{1, \ldots, n\}$
  and producing an open subset $U_{k, n} \subseteq F_{k,n}$ given by the set of all $k \times k$ submatrices with column indices in $J$
  with nonzero determinant.
  All matrices in such an open subset have reduced-row echelon form (i.e. are in the same equivalence class)
  with the same row and column interchange operations;
  since there is a $k \times k$ submatrix with nonzero determinant in each one, the $k \times k$ submatrix at the extreme left
  of the reduced matrix is the identity matrix, and the leftover submatrix has dimensions $k \times (n - k)$.
  The charts are obtained by mapping each element of the open set to the vector in $\mathbb{R}^{k(n - k)}$ obtained
  by flattening this leftover submatrix.

  Certainly, every matrix in $F_{k,n}$ is in some $U_{k,n}$ by the definition of linear independence.
  The mapping is linear and invertible (as a linear map), since the reduced-row echelon form uniquely characterizes the column space.
  It is, accordingly, a local homeomorphism.
  Given two such mappings for two different open sets, the fact their inverses are linear maps imply the transition maps are linear,
  and accordingly smooth.
\end{proof}

\end{document}
