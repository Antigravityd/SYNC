\documentclass{article}

\usepackage[letterpaper]{geometry}
\usepackage{tgpagella}
\usepackage{amsmath}
\usepackage{amssymb}
\usepackage{amsthm}
\usepackage{tikz}
\usepackage{minted}
\usepackage{physics}
\usepackage{siunitx}

\sisetup{detect-all}
\newtheorem{plm}{Problem}

\title{7550 HW 6}
\author{Duncan Wilkie}
\date{5 May 2023}

\begin{document}

\maketitle

\begin{plm}[Volume form on a sphere]
  Let $S^{n}(r)$ be the sphere of radius $r$ in $\mathbb{R}^{n+1}$, given by $x_{1}^{2} + x_{2}^{2} + \cdots + x_{n+1}^{2} = r^{2}$,
  \[
    \omega = \frac{1}{r}\sum_{i = 1}^{n+1}(-1)^{i-1}x_{i}dx_{1} \wedge \cdots \wedge \widehat{dx_{i}} \wedge \cdots \wedge dx_{n+1}.
  \]
  \begin{enumerate}
  \item Compute the integral $\int_{S^{n}(1)}\omega$ and conclude that $\omega$ is not exact.
  \item Viewing $r$ as a function on $\mathbb{R}^{n+1} \setminus \{0\}$, show that $dr \wedge \omega = dx_{1} \wedge \cdots \wedge dx_{n+1}$.
  \end{enumerate}
\end{plm}

\begin{proof}
  By definition, since the sphere is compact and so $\omega$ is compactly supported,
  \[
    \int_{S^{n}(1)}\omega = \sum_{i = 1}^{n+1}(-1)^{i-1}\int_{S^{n}(1)}\frac{x_{i}}{r}dx_{1} \cdots \widehat{dx_{i}} \cdots dx_{n+1}.
  \]
  \[
    = \sum_{i = 1}^{n+1}(-1)^{i-1}\frac{x_{i}}{r}A_{S^{n}(1)},
  \]
  which, since the surface area of a nondegenerate hypersphere is nonzero, is nonzero, so $\omega$ is not exact.

  \[
    dr \wedge \omega = d(r \wedge \omega) + r \wedge d\omega
  \]

\end{proof}

\begin{plm}
  If $f: M \to N$ is a submersion of smooth manifolds, show that $f^{*}: \Omega^{*}(N) \to \Omega^{*}(M)$ is injective.
\end{plm}

\begin{proof}
  Submersion means $f_{*}: T_{p}M \to T_{f(p)}N$ is onto at all $p$.
  The dual is a contravariant functor, so the image of $f_{*}$ is the map $f^{*}: T_{f(p)}^{*}N \to T_{p}^{*}M$; epics in $C^{op}$
  are monic in $C$.
  Are monics preserved under dualization?
  Sufficient is right-adjointness; the internal hom (bi)functor is right-adjoint to the tensor product (bi)functor,
  so this is indeed satisfied (restricting the second argument to the bifunctors to $\mathbb{R}$), and $f^{*}$ is injective.

  Extending this map ``algebra-homomorphically'' will preserve injectivity, as it is such an extension by which $\Omega^{*}$ is defined.
\end{proof}

\begin{plm}
  Find the dimension of the vector space $H_{\Omega}^{1}(\mathbb{R}^{2} \setminus \{n \text{ points}\})$.
  Can you find differential forms representing basis elements?
  Can you describe the ring structure of $H_{\Omega}^{*}(\mathbb{R}^{2} \setminus \{n \text{ points}\})$?
\end{plm}

\begin{proof}
  First of all, the punctured plane is homotopic to a circle.
  The circle can be covered by intersecting arcs; these are homotopic to line segments $A$ and $B$, each of which (as it is contractible)
  has de Rahm cohomology $H_{\Omega}^{*} =
  \begin{cases}
    \mathbb{R}, & * = 0 \\
    0 & \text{else}
  \end{cases}
  $.
  The intersection is homotopic to the disjoint union of two line segments.
  This yields an initial segment of the Mayer-Vietoris sequence (surpressing the $_{\Omega}$ for visual clarity)
  \[
    0 \to H^{0}(A \cup B) \cong \mathbb{R} \to H^{0}(A) \oplus H^{0}(B) \cong \mathbb{R} \oplus \mathbb{R} \cong \mathbb{R}^{2}
    \to H^{0}(A \cap B) \cong \mathbb{R}^{2} \to H^{1}(A \cup B),
  \]
  from which we observe that the pentultimate map must have kernel isomorphic to $\mathbb{R}$---and, accordingly,
  its image is isomorphic to $\mathbb{R}^{2}/\mathbb{R} \cong \mathbb{R} \cong H^{1}(A \cup B)$.
  The next map must have kernel $\mathbb{R}$; the sequence is trivial beyond this point.

  We may take $A = \mathbb{R}^{2} \setminus \{p_{1}\}$ and $B = \mathbb{R}^{2} \setminus \{p_{2}, \ldots p_{n}\}$,
  so that their union is the plane and their intersection is the set of interest.
  Then, the Mayer-Vietoris sequence is, initially,
  \[
    0 \to H^{0}(A \cup B) \cong \mathbb{R} \to H^{0}(A) \oplus H^{0}(B) \cong \mathbb{R}^{2} \to H^{0}(A \cap B)
    \to H^{1}(A \cup B) \cong 0 \to H^{1}(A) \oplus H^{1}(B) \to H^{1}(A \cap B) \to H^{2}(A \cup B) \cong 0.
  \]

  At this point, we may suppose for induction that the $n$-punctured plane has $H^{1} \cong \mathbb{R}^{n}$.
  The once-punctured plane (circle) computation above forms a base case; evaluating the Mayer-Vietoris sequence above yields the segment
  \[
    0 \to H^{1}(A) \oplus H^{1}(B) \cong \mathbb{R} \oplus \mathbb{R}^{n} \cong \mathbb{R}^{n+1} \to H^{1}(A \cap B) \to 0.
  \]
  This expresses an isomorphism on the middle arrow; accordingly, the inductive hypothesis holds for all $n$;
  namely,
  \[
    H^{*}_{\Omega}(\mathbb{R}^{2} \setminus \{n \text{ points}\}) \cong
    \begin{cases}
      \mathbb{R} & * = 0 \\
      \mathbb{R}^{n} & * = 1 \\
      0 & \text{otherwise}
    \end{cases}.
  \]
  There are no higher terms because we've shown the $* = 2$ case and higher terms simply have no forms.
  So, $\dim H_{\Omega}^{1}(\mathbb{R}^{2} \setminus \{n \text{ points}\}) = n$.

  One can make several coordinate systems on the $n$-punctured plane by placing the origin at one of the punctures
  and describing all of the non-punctured points via polar coordinates.
  Letting the angular coordinate in each case be $\theta_{i}$, $d\theta_{i}$ are obviously closed.
  However, despite appearances, these $d\theta_{i}$ are \textit{not} exact: $\theta_{i}$ themselves are not smooth functions,
  as they're irreparably multi-valued
  (just like ordinary polar coordinates---their local smoothness makes them fine charts though).


\end{proof}

% \begin{plm}
%   Vector bundles with transition functions/``cocycles'' $\{g_{\alpha\beta}\}$ and $\{g_{\alpha\beta}'\}$ are equivalent if there are maps
%   $\tau_{\alpha}: U_{\alpha} \to GL(n, \mathbb{R})$ so that $g_{\alpha\beta} = \tau_{\alpha}g'_{\alpha\beta}\tau_{\beta}^{-1}$.
%   Vector bundles $E$ and $E'$ over $M$ are isomorphic if there is a bundle map $f: E \to E'$ with the restriction of $f$ to each fiber
%   a vector space isomorphism.
%   Show that two vector bundles are isomorphic iff their cocycles (for some open cover) are equivalent.
% \end{plm}

% \begin{proof} % TODO
%   Suppose that $f: E \to E'$ is a bundle map, and that the restriction to each fiber induces a vector space isomorphism,
%   let $g_{\alpha\beta}$ and $g'_{\alpha\beta}$ be transition functions for $E$ and $E'$, and $U_{\alpha}$ cover the base space.
%   If $x \in U_{\alpha} \cap U_{\beta}$, then


%   Suppose there is an open cover $U_{\alpha}$ such that $E$ and $E'$ have equivalent cocycles.
%   Then the
% \end{proof}


\Begin{plm}
  For the following subsets $X$ of $\mathbb{R}^{3}$, find the dimension of the vector space $H_{\Omega}^{1}(\mathbb{R}^{3} \setminus X)$.
  \begin{enumerate}
  \item $X = $ one point.
  \item $X = $ two distinct points.
  \item $X = $ one line.
  \item $X = $ two lines which do not intersect.
  \item $X = $ two lines which intersect in a point.
  \end{enumerate}
\end{plm}

\begin{proof} We omit the zeroth cohomology when possible, as it is identically $\mathbb{R}$ in each case (they're all connected spaces).
  \begin{enumerate}
  \item This scenario is cohomologous to that arising from a radial straight-line homotopy: a 2-sphere.
    Let $A$ and $B$ be open hemispheres whose boundaries lie in the interior of the other.
    The intersection is then a 1-sphere (cohomology computed above); the pieces are contractable.
    Mayer-Vietoris:
    \[
      0 \to H^{0}(A \cup B) \cong \mathbb{R} \to H^{0}(A) \oplus H^{0}(B) \cong \mathbb{R}^{2} \to H^{0}(A \cap B) \cong \mathbb{R}
    \]
    \[
      \to H^{1}(A \cup B) \to H^{1}(A) \oplus H^{1}(B) \cong 0
    \]
    The kernel of the map going into the cohomology we're looking for is isomorphic to $\mathbb{R}$, so the cohomology is trivial.
  \item This is homotopic to $S^{2} \wedge S^{2}$---let $A$ and $B$ be these spheres; their intersection is homeomorphic to a disc.
    Mayer-Vietoris, using the computation above for the last group:
    \[
      0 \to H^{0}(A \cup B) \cong \mathbb{R} \to H^{0}(A) \oplus H^{0}(B) \cong \mathbb{R}^{2} \to H^{0}(A \cap B) \cong \mathbb{R}
    \]
    \[
      \to H^{1}(A \cup B) \to H^{1}(A) \oplus H^{1}(B) \cong 0;
    \]
    the cohomology is trivial because it's the same sequence.
  \item This is homotopic to a torus, by a simultaneous stereographic projection of the line onto a circle
    and a straight-line radial homotopy.
    To compute this, do ``interpenetrating toroidal hemispheres'' (which flatten out to annuli, and ultimately circles):
    \[
      0 \to H^{0}(A \cup B) \cong \mathbb{R} \to H^{0}(A) \oplus H^{0}(B) \cong \mathbb{R}^{2} \to H^{0}(A \cap B) \cong \mathbb{R}^{2}
    \]
    \[
      \to H^{1}(A \cup B) \to H^{1}(A) \oplus H^{1}(B) \cong \mathbb{R}^{2} \to H^{1}(A \cap B) \cong \mathbb{R}^{2}
    \]
    The kernel going into the critical one is again isomorphic to $\mathbb{R}$, so $H^{1}(A \cup b) \cong \mathbb{R}^{2}/ \mathbb{R}.
    \cong \mathbb{R}$

  \item The space may be homotopically deformed until the lines are parallel; a cross-section with the lines out of the page
    can be reduced to a figure-8; and a stereographic-style projection made to produce a two-celled torus: $(S^{1} \wedge S^{1})
    \times S^{1}$.
    We can let $A$ and $B$ be toroids, and their intersection be a circle along which they're glued to form the above.
    \[
      0 \to H^{0}(A \cup B) \cong \mathbb{R} \to H^{0}(A) \oplus H^{0}(B) \cong \mathbb{R}^{2} \to H^{0}(A \cap B) \cong \mathbb{R}
    \]
    \[
      \to H^{1}(A \cup B) \to H^{1}(A) \oplus H^{1}(B) \cong \mathbb{R}^{2} \to H^{1}(A \cap B) \cong \mathbb{R}
    \]
    This will be trivial by the arguments made for the first two.
  \item Straight-line homotopies down the barrel of the lines followed by stereographic projection of the arms in opposite directions
    results in $T^{2}#T^{2}$, which can be divided into two punctured tori (homotopic to figure-8's) with intersection homotopic
    to a circle.
    \[
      0 \to H^{0}(A \cup B) \cong \mathbb{R} \to H^{0}(A) \oplus H^{0}(B) \cong \mathbb{R}^{2} \to H^{0}(A \cap B) \cong \mathbb{R}
    \]
    \[
      \to H^{1}(A \cup B) \to H^{1}(A) \oplus H^{1}(B) \cong \mathbb{R}^{4} \to H^{1}(A \cap B) \cong \mathbb{R}
    \]
    which is trivial by the arguments made for the first three. % TODO: I think by sequence chasing is dogwater
  \end{enumerate}
\end{proof}


\end{document}
