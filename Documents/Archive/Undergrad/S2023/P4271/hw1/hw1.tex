\documentclass{article}

\usepackage[letterpaper]{geometry}
\usepackage{tgpagella}
\usepackage{amsmath}
\usepackage{amssymb}
\usepackage{amsthm}
\usepackage{tikz}
\usepackage{minted}
\usepackage{physics}
\usepackage{siunitx}
\usepackage{mhchem}

\sisetup{detect-all}
\newtheorem{plm}{Problem}
\renewcommand*{\proofname}{Solution}


\title{4271 HW 1}
\author{Duncan Wilkie}
\date{1 February 2023}

\begin{document}

\maketitle

\begin{plm}
  Calculate the average binding energy for \ce{^{40}K}.
\end{plm}

\begin{proof}
  Applying the binding energy formula,
  \[
    B(19, 40) = \qty[ZM(\ce{^{1}H}) + (A - Z)M_{n} - M(A, Z)] \cdot c^{2}
  \]
  \[
    = \qty[(19)(\SI{938.783}{MeV/c^{2}}) + (40 - 19)(\SI{939.565}{MeV/c^{2}}) - (\SI{39.963998}{u})(\SI{931.49432}{MeV/c^{2}u})] \cdot c^{2}
  \]
  \[
    = \SI{341.5}{MeV}.
  \]
\end{proof}

\begin{plm}
  The mass of \ce{^{74}Ge} is \SI{73.921177}{u}.
  \begin{enumerate}
  \item Calculate its mass defect in \si{MeV/c^2}.
  \item Calculate its binding energy in \si{MeV}.
  \item What is the binding energy per nucleon?
  \end{enumerate}
\end{plm}

\begin{proof}
  The mass defect is
  \[
    \Delta = (M(A, Z) - A)c^{2} = (\SI{73.921177}{u} - \SI{74}{u})(\SI{931.49}{MeV/u}) = \SI{-73.423}{MeV}.
  \]
  The binding energy in terms of the mass defect is
  \[
    B(32, 74) = \qty[ZM(\ce{^{1}H}) + (A - Z)M_{n} - \qty(A + \frac{\Delta}{c^{2}})]c^{2}
  \]
  \[
    = 32(\SI{938.783}{MeV}) + (74 - 32)(\SI{939.565}{MeV}) - \qty(74(\SI{931.49}{MeV/u}) - \SI{73.423}{MeV})
    = \SI{645.95}{MeV}.
  \]
  There are 74 nucleons, so the binding energy per nucleon is \SI{8.729}{MeV}.
\end{proof}

\begin{plm}
  \;
  \begin{enumerate}
  \item Calculate the energy released in the beta decay of \ce{^{32}P}.
  \item If the resulting beta particle has energy \SI{650}{keV}, how much energy does the antineutrino have?
  \end{enumerate}
\end{plm}

\begin{proof}
  This is a $\beta^{-}$ decay, because the nucleus has a neutron excess; its equation is
  \[
    \ce{^{32}P -> ^{32}S + e^{-} + \bar{\nu}_{e}}.
  \]
  The energy released in the $\beta$-decay can be deduced from the difference in the atomic masses:
  \[
    Q = m(15, 17)c^{2} - m(16, 16)c^{2}
    = (\SI{31.973907}{u})(\SI{931.49}{MeV/u}) - (\SI{31.972071}{u})(\SI{931.49}{MeV/u})
  \]
  \[
    = \SI{1.71}{MeV}.
  \]
  Because the mass of the nucleus is large, the energy imparted by the recoil from the ejection is negligible.
  Accordingly, the released energy is approximately the sum of that in the two ejected particles; the neutrino energy can then be estimated to be
  \[
    E(\bar{\nu}_{e}) = Q - E(e^{-}) = \SI{1.71}{MeV} - \SI{0.650}{MeV} = \SI{1.06}{MeV}.
  \]
\end{proof}

\begin{plm}
  Use the semi-empirical mass formula to estimate the prompt energy released in spontaneous fission of \ce{^{239}Pu} as
  \begin{center}
    \ce{^{239}Pu -> ^{112}Pd + ^{124}Cd + 3n}
  \end{center}
\end{plm}

\begin{proof}
  Applying the semi-empirical mass formula to the initial case, which has $f_{5} = 0$ since the mass number is odd,
  \[
    M(94, 239) = ZM(\ce{^{1}H}) + (A - Z)(M_{n}) - a_{v}A + a_{s}A^{2/3} + a_{C}\frac{Z^{2}}{A^{1/3}} + a_{a}\frac{(Z - A/2)^{2}}{A}
  \]
  \[
     = 94(\SI{938.783}{MeV/c^{2}}) + 145(\SI{939.565}{MeV/c^{2}}) - 15.56(239) + 17.23(239)^{2/3} + 0.697\frac{(94)^{2}}{(239)^{1/3}}
   \]
   \[
     + 93.14\frac{(94 - 239/2)^{2}}{239}
   \]
   \[
     = \SI{222.673}{GeV/c^{2}}
   \]
   The associated mass defect is
   \[
     \Delta = [M(94, 239) - A]c^{2} = \SI{222.673}{GeV} - (\SI{239}{u})(\SI{931.49}{MeV/u}) = \SI{47.00}{MeV}
   \]
   To calculate the final energy, we must calculate the energy of each of the components.
   Using the formula above,
   \[
     M(46, 112) = 46(\SI{938.783}{MeV/c^{2}}) + 66(\SI{939.565}{MeV/c^{2}}) - 15.56(112) + 17.23(112)^{2/3} + 0.697\frac{(46)^{2}}{(112)^{1/3}}
   \]
   \[
     + 93.14\frac{(46 - 112/2)^{2}}{112} - \frac{11.2}{(112)^{1/2}}
   \]
   \[
     = \SI{104.241}{GeV/c^{2}}
     \Rightarrow \Delta = \SI{104.241}{GeV} - 112(\SI{931.49}{MeV/u}) = \SI{-85.88}{MeV}
   \]
   \[
     M(48, 124) = 48(\SI{938.783}{MeV/c^{2}}) + 76(\SI{939.565}{MeV/c^{2}}) - 15.56(124) + 17.23(124)^{2/3} + 0.697\frac{(48)^{2}}{(124)^{1/3}}
   \]
   \[
     + 93.14\frac{(48 - 124/2)^{2}}{124} - \frac{11.2}{(124)^{1/2}}
   \]
   \[
     = \SI{124.831}{GeV/c^{2}}
     \Rightarrow \Delta = \SI{115.436}{GeV} - 124(\SI{931.49}{MeV/u}) = \SI{-68.97}{MeV}
   \]
   \[
     3M_{n} = 3(\SI{939.565}{MeV/c^{2}}) = \SI{2.81}{GeV/c^{2}}
     \Rightarrow \Delta = \SI{2.81}{GeV/c^{2}} - 3(\SI{931.49}{MeV/u}) = \SI{15.53}{MeV}
   \]
   The total yield is then
   \[
     Q = \Delta(\ce{^{239}Pu}) - \Delta(\ce{^{112}Pd + ^{134}Cd + 3n})
     = \SI{47.00}{MeV} - (\SI{-85.88}{MeV} - \SI{68.97}{MeV} + \SI{15.53}{MeV})
   \]
   \[
     = \SI{186}{MeV}
   \]
\end{proof}

\begin{plm}
  Use \textit{accurate} masses from the 2021 atomic mass evaluation to calculate the actual prompt energy released in problem 4.
  Compare the answers.
  How different are they?
\end{plm}

\begin{proof}
  According to the tabulated mass defects,
  \[
     Q = \Delta(\ce{^{239}Pu}) - \Delta(\ce{^{112}Pd + ^{124}Cd + 3n})
     = \SI{48.59}{MeV} - (\SI{-86.32}{MeV} - \SI{76.70}{MeV} + \SI{15.53}{MeV})
   \]
   \[
     = \SI{196}{MeV}
   \]
   These are different by \SI{10}{MeV} or 5\%.
\end{proof}

\begin{plm}
  Following spontaneous fission from the last problem, a chain of beta decays converts the \ce{^{112}Pd} and \ce{^{124}Cd} to stable nuclei.
  \begin{enumerate}
  \item What are the final nuclei produced?
  \item What is the total beta-delayed energy released in the beta-decay chains?
    What is the ratio of beta-delayed to prompt energy released?
  \item (Bonus) About how long does it take (on average) for all the energy to be released?
  \end{enumerate}
\end{plm}

\begin{proof}
  The decay chains are
  \[
    \ce{^{112}Pd -> ^{112}Ag + e^{-} + \bar{\nu}_{e} ->  ^{112}Cd + e^{-} + \bar{\nu}_{e}}
  \]
  \[
    \ce{^{124}Cd -> ^{124}In + e^{-} + \bar{\nu}_{e} -> ^{124}Sn + e^{-} + \bar{\nu}_{e}}
  \]
  The beta-delayed energies of these chains are simply calculated from tabulated mass defects:
  \[
    Q = \Delta(\ce{^{112}Pd}) - \Delta(^{112}Cd) = \SI{-86.321}{MeV} - \SI{-90.574}{MeV} = \SI{4.25}{MeV}
  \]
  and
  \[
    Q = \Delta(\ce{^{124}Cd}) - \Delta(\ce{^{124}Sn}) = \SI{-76.699}{MeV} - \SI{-88.231}{MeV} = \SI{11.53}{MeV}
  \]
  The total beta-delayed energy is then \SI{15.79}{MeV}.
  This is 8.1\% of the prompt energy.
\end{proof}

\begin{plm}
  The isotope \ce{^{252}Cf} is commonly used to study the geology around underground wells due to the high number of neutrons emitted.
  Each spontaneous fission of \ce{^{252}Cf} produces an average of 3.77 neutrons.
  However, alpha decay dominates, with only 3.09\% of the decays being fission.
  \begin{enumerate}
  \item Calculate the activity (in Curies) of \ce{^{252}Cf} that would be requried to produce \SI{5e8}{neutrons/s}.
  \item If we make such a source, the neutron flux will decrease over time as the \ce{^{252}Cf} decays.
    How many \si{neutrons/s} will be produced 4 years later? ($t_{1/2} = \SI{2.645}{yr}$)
  \end{enumerate}
\end{plm}

\begin{proof}
  We want
  \[
    3.77 \cdot .0309   \cdot A = \SI{5e8}{}
    \Leftrightarrow A = \frac{\SI{5e8}{}}{3.77 \cdot .0309} = \SI{4.292e9}{Bq} = \SI{.116}{Ci}.
  \]
  The proportion of californium remaining is $e^{-t\ln(2)/t_{1/2}} = e^{-(4)\ln(2)/(2.645)} = 0.351$; accordingly, since everything else stays the same,
  the production is $0.351(\SI{5e8}{}) = \SI{175.2e6}{neutrons/s}$.
\end{proof}

\begin{plm}
  The oldest formerly living objects that can be measured by \ce{^{14}C} dating are about \SI{50000}{yo}.
  \begin{enumerate}
  \item What is the ratio of \ce{^{14}C/^{12}C} in such a sample?
  \item Assume the natural abundance ratio in the atmosphere is \SI{1.5e-12}{}.
    If you have to count 25 atoms of \ce{^{14}C} to make a measurement (20\% accuracy),
    how much mass of carbon would you need to date a \SI{50000}{yo} object?
  \end{enumerate}
\end{plm}

\begin{proof}
  Given the half-life of $\ce{^{14}C}$ being \SI{5700}{y}, the proportion remaining after \SI{50000}{y} is
  \[
    \frac{N}{N_{0}} = e^{-t\ln(2)/t_{1/2}} = e^{-(\SI{50000}{y})\ln(2)/(\SI{5700}{y})} = 0.00229.
  \]
  We want
  \[
    0.00229(\SI{1.5e-12}{})n_{C} = 25
    \Leftrightarrow n_{C} = \frac{25}{(0.00229)(\SI{1.5e-12}{})} = \SI{7.278e15}{atoms}
  \]
  \[
    = \SI{1}{g/mol}\cdot \frac{\SI{7.278e15}{atoms}}{\SI{6.02e23}{atoms/mol}} = \SI{12.1}{ng}
  \]
\end{proof}

\begin{plm}
  The Cassini spacecraft went into orbit around the planet Saturn in July 2004, after a nearly seven-year journey from Earth.
  On-board electrical systems were powered by heat from three radioisotope thermoelectric generators,
  which together utilized a total of \SI{32.7}{kg} of \ce{^{238}Pu}, encapsulated as \ce{PuO_2}.
  The isotope has a half-life of \SI{86.4}{y} and emits an alpha particle with an average energy of \SI{5.49}{MeV}.
  \begin{enumerate}
  \item How much total thermal power is generated in the spacecraft?
  \item To see why RTGs are so attractive, calculate the solar panel area needed to produce the same amount of wattage at the distance of Saturn.
    Assume your solar panel is 10\% efficient and that the Sun's luminosity is \SI{3.8e26}{W}.
  \end{enumerate}
\end{plm}

\begin{proof}
  The total number of atoms of $\ce{^{238}Pu}$ intially on the spacecraft is
  \[
    N = \frac{3 \cdot \SI{32.7}{kg}}{\SI{0.001}{kg / mol}} \cdot \SI{6.02e23}{atoms / mol} = \SI{59.1e27}{atoms},
  \]
  leading to an activity of
  \[
    A = \lambda N = \frac{\ln 2}{t_{1/2}}N = \frac{\ln 2}{\SI{86.4}{y} \cdot \SI{365}{d/y} \cdot \SI{24}{h/d} \cdot \SI{3600}{s/h}}
    \cdot \SI{59.1e27}{atoms}
    = \SI{8.622e17}{Bq}.
  \]
  Each decay liberates $\SI{5.49}{MeV}$, so the total thermal power is
  \[
    P = (\SI{5.49}{MeV})(\SI{8.622e17}{Bq}) = \SI{4.733e18}{MeV/s} = \SI{757.3}{kW}.
  \]

  Saturn's orbit has semi-major axis \SI{1.434e12}{m}, and, accordingly, the power density at such a distance is
  \[
    P = \frac{\SI{3.8e26}{W}}{4\pi(\SI{1.434e12}{m})^{2}} = \SI{14.71}{W/m^{2}}.
  \]
  Ten percent of this is $\SI{1.471}{W/m^{2}}$---roughly six orders of magnitude smaller than the RTGs.
\end{proof}

\end{document}
