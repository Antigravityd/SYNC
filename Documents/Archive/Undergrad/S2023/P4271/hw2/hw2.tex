\documentclass{article}

\usepackage[letterpaper]{geometry}
\usepackage{tgpagella}
\usepackage{amsmath}
\usepackage{amssymb}
\usepackage{amsthm}
\usepackage{tikz}
\usepackage{minted}
\usepackage{physics}
\usepackage{siunitx}
\usepackage{mhchem}

\sisetup{detect-all}
\newtheorem{thm}{Theorem}
\newtheorem{plm}{Problem}
\renewcommand*{\proofname}{Solution}

\title{Physics 4271 HW 2}
\author{Duncan Wilkie}
\date{10 February 2023}

\begin{document}

\maketitle

\begin{plm}
  Calculate the minimum energy required to be over the Coulomb barrier for:
  \begin{enumerate}
  \item $\ce{p + p}$,
  \item $\ce{p + ^{12}C}$,
  \item $\ce{^{4}He + ^{208}Pb}$.
  \end{enumerate}
\end{plm}

\begin{proof}
  The Coulomb potential is
  \[
    V(r) = \frac{1}{4\pi\epsilon_{0}}\frac{Z_{1}Z_{2}e^{2}}{r}.
  \]
  Whenever the reactants are close enough to make ``physical'' contact, the strong force kicks in;
  the Coulomb potential at this radius is roughly the peak of the Coulomb barrier.
  The radius of nuclei can be estimated as $r = \SI{1.2}{fm} \cdot A^{1/3}$.

  This gives us sufficient information to compute:
  \[
    r_{p + p} = 2r_{p} = 2(\SI{1.2}{fm})(1)^{1/3} = \SI{2.4}{fm}
    \Rightarrow E_{p + p} = \frac{1}{4\pi\epsilon_{0}}\frac{Z_{1}Z_{2}e^{2}}{r_{p + p}}
    = \frac{1}{4\pi(\SI{8.85e-12}{F/m})}\frac{1 \cdot 1 \cdot (\SI{1.6e-19}{C})^{2}}{\SI{2.4}{fm}}
  \]
  \[
    = \SI{9.6e-14}{J} = \SI{559}{keV}
  \]
  \[
    r_{\ce{p + {}^{12}C}} = r_{p} + r_{\ce{^{12}C}} = \SI{1.2}{fm} + \SI{1.2}{fm} \cdot (12)^{1/3} = \SI{3.95}{fm}
    \Rightarrow E_{\ce{p + ^{12}C}} = \frac{1}{4\pi\epsilon_{0}}\frac{Ze^{2}}{r_{\ce{p + ^{12}C}}}
  \]
  \[
    = \frac{1}{4\pi(\SI{8.85e-12}{F/m})}\frac{1 \cdot 6 \cdot (\SI{1.6e-19}{C})^{2}}{\SI{3.95}{fm}}
    = \SI{2.19}{MeV}
  \]
  \[
    r_{\ce{^{4}He + ^{208}Pb}} = r_{\ce{^{4}He}} + r_{\ce{^{208}Pb}} = \SI{1.2}{fm} \cdot 4^{1/3} + \SI{1.2}{fm} \cdot 208^{1/3}
    = \SI{9.01}{fm}
  \]
  \[
    \Rightarrow E_{\ce{^{4}He + ^{208}Pb}} = \frac{1}{4\pi\epsilon_{0}}\frac{Z_{1}Z_{2}e^{2}}{r}
    = \frac{1}{4\pi(\SI{8.85e-12}{F/m})}\frac{4 \cdot 208 \cdot (\SI{1.6e-19}{C})^{2}}{\SI{9.01}{fm}}
    = \SI{133}{GeV}
  \]
\end{proof}

\begin{plm}
  The cross section for charged-particle reactions is proportional to the probability of tunneling through the Coulomb barrier
  given by the Gamow factor, which has a convenient approximation:
  \[
    e^{-2\pi\eta} = e^{-2\pi Z_{1}Z_{2}e^{2}/\hbar\nu} = e^{-31.287Z_{1}Z_{2}\sqrt{\mu/E}}
  \]
  where $\mu$ is the reduced mass in \si{amu} and $E$ is the center-of-mass energy in \si{keV}.
  For the 3 cases you considered above, calculate the Gamow factor for an energy that is one-quarter the barrier energy you found in problem 1.
\end{plm}

\begin{proof}
  Plug-and-chug:
  \[
    \mu_{\ce{p + p}} = \frac{m_{p}m_{p}}{m_{p} + m_{p}} = \frac{m_{p}}{2} = \SI{0.5}{amu}
    \Rightarrow G_{\ce{p + p}} = \exp\qty(-31.287 \cdot 1 \cdot 1 \sqrt{\frac{\SI{0.5}{amu}}{\SI{559}{keV} / 4}})
    = \SI{0.94}{}
  \]
  \[
    \mu_{\ce{p + ^{12}C}} = \frac{m_{p}m_{\ce{^{12}C}}}{m_{p} + m_{\ce{^{12}C}}} = \frac{\SI{1}{amu}\cdot\SI{12}{amu}}{\SI{1}{amu}+\SI{12}{amu}}
    = \SI{0.92}{amu}
    \Rightarrow G_{\ce{p + ^{12}C}} = \exp\qty(-31.287 \cdot 1 \cdot 6 \sqrt{\frac{\SI{0.92}{amu}}{\SI{2.91}{MeV} / 4}})
  \]
  \[
    = \SI{0.81}{}
  \]
  \[
    \mu_{\ce{^{4}He + ^{208}Pb}} = frac{m_{\ce{^{4}He}}m_{\ce{^{208}{Pb}}}}{m_{\ce{^{4}He}} + m_{\ce{^{208}{Pb}}}}
    = \frac{\SI{4}{amu} \cdot \SI{208}{amu}}{\SI{4}{amu} + \SI{208}{amu}} = \SI{3.92}{amu}
  \]
  \[
    \Rightarrow G_{\ce{^{4}He + ^{208}Pb}} = \exp\qty(-31.287 \cdot 4 \cdot 208 \sqrt{\frac{\SI{3.92}{amu}}{\SI{133}{GeV} / 4}})
    = \SI{0.75}{}
  \]
\end{proof}

\begin{plm}
  A \SI{2}{MeV} beam of protons bombards a \ce{^{16}O} target and the differential cross section is measured to be \SI{0.094}{b/sr}
  at a lab angle of $167^{\circ}$.
  \begin{enumerate}
  \item What is the expected cross-section if you assume Rutherford scattering?
  \item What is the calculated Mott cross-section?
  \item How do your answers to (a) and (b) differ from the measured cross section and why might they be different?
  \end{enumerate}
\end{plm}

\begin{proof}
  Using the nonrelativistic Rutherford scattering formula (\SI{2}{MeV} is pretty slow),
  \[
    \qty(\dv{\sigma}{\Omega})_{\text{Rutherford}} = \frac{(zZe^{2})^{2}}{(4\pi\epsilon_{0})^{2} \cdot (4E_{kin})^{2} \sin^{4}\frac{\theta}{2}}
    = \frac{(1 \cdot 8 (\SI{1.6e-19}{C})^{2})^{2}}{(4\pi \cdot \SI{8.85e-12}{F/m})^{2} \cdot (4 \cdot \SI{2}{MeV})^{2}\sin^{4}\frac{167}{2}}
    = \SI{0.0212}{b/sr}.
  \]

  For the Mott cross-section, we first need the relativistic $\beta$:
  \[
    \beta = \frac{v}{c} \approx \frac{\sqrt{2E/m_{p}}}{c} = \SI{0.065}{}
  \]
  \[
    \qty(\dv{\sigma}{\Omega})_{\text{Mott}} = \qty(\dv{\sigma}{\Omega})_{\text{Rutherford}} \cdot \qty(1 - \beta^{2}\sin^{2}\frac{\theta}{2})
    = (\SI{0.0212}{b/sr}) \cdot \qty(1 - (\SI{0.065}{})^{2}\sin^{2}\frac{167}{2}) = \SI{0.0211}{b/sr}.
  \]
  This is within an order of magnitude, but barely.
  The discrepancy may be attributable to nuclear recoil (1/16 isn't \textit{too} small a ratio), or charge inhomogeneity in the nucleus.
\end{proof}

\begin{plm}
  Assume that $^{197}Au$ is made from a solid, uniform sphere of nuclear material with a radius of $R = \SI{1.2}{fm} \cdot A^{1/3}$.
  Calculate the form factor $F(q)$.
\end{plm}

\begin{proof}
  The charge distribution is
  \[
    \rho(r) =
    \begin{cases}
      \frac{Ze}{4\pi R^{3}/3} & r \leq R \\
      0 & r > R
    \end{cases}
  \]
  The nonzero density part can be computed to be $\frac{87 \cdot \SI{1.6e-19}{C}}{4\pi (\SI{1.2}{fm} \cdot 197^{1/3})/3} = \SI{4.76e-4}{C/m^{3}}$.
  Conditional on the Born approximation and presuming minimal nuclear recoil, the Fourier transform of this is the form factor.
  Going off of the book's statement of the form factor for homogeneous spheres
  (to avoid a hard-to-spot Bessel-function computation I'd expect you to spare us from), the form factor is
  \[
    F(\vb{q}) = Ze\frac{3\hbar^{3}}{|\vb{q}|^{3}R^{3}}\qty(\sin\frac{|\vb{q}|R}{\hbar} - \frac{|\vb{q}|R}{\hbar}\cos\frac{|\vb{q}|R}{\hbar})
  \]
  which computes to
  \[
    = \frac{\SI{7.71e-105}{}}{|\vb{q}|^{3}}\qty(\sin(\SI{6.65e19}{}|\vb{q}|) - \SI{6.65e19}{}|\vb{q}| \cos(\SI{6.65e19}{}|\vb{q}|))
\]
\end{proof}

\begin{plm}
  Show that the mean-square charge radius of a uniformly charged sphere is $\expval{r^{2}} = 3R^{2}/5$.
\end{plm}

\begin{proof}
  An exercise in probability theory and calculus, calculating the expectation of $r^{2}$ with a uniform volumetric pdf of radius $R$.
  \[
    \expval{r^{2}} = \int_{\mathbb{R}^{3}}r^{2}\frac{3}{4\pi R^{3}}\chi_{S}(r)dV
    = \frac{3}{4\pi R^{3}}\int_{0}^{\pi}\int_{0}^{2\pi}\int_{0}^{R}r^{4}\sin\theta drd\theta d\phi
    = \frac{3}{R^{3}}\qty(\frac{r^{5}}{5}\eval_{r = 0}^{r = R}) = \frac{3R^{2}}{5}
  \]
\end{proof}

\begin{plm}
  A nuclear charge distribution more realistic than the uniformly charged distribution is the Fermi distribution,
  $\rho(r) = \frac{\rho_{0}}{1 + \exp\qty[(r - c)/a]}$.
  Find the value of $a$ if $t = \SI{2.3}{fm}$
\end{plm}

\begin{proof}
  $t$ is the falloff parameter of the distribution, giving the separation between where $\rho$ is $10\%$ and $90\%$ of $\rho_{0}$;
  call these $r_{1}$ and $r_{2}$, respectively.
  We can then generate the equations
  \[
    \rho(r_{1}) = 0.1\rho_{0} = \frac{\rho_{0}}{1 + \exp\qty[(r_{1} - c) / a]}
  \]
  \[
    \rho(r_{2}) = 0.9\rho_{0} = \frac{\rho_{0}}{1 + \exp\qty[(r_{2} - c) / a]}
  \]
  \[
    t = \SI{2.3}{fm} = r_{2} - r_{1}
  \]
  Simplifying these,
  \[
    r_{1} = a\ln(9) + c
  \]
  \[
    r_{2} = c -  a\ln(9)
  \]
  \[
    \Rightarrow t = r_{2} - r_{1} = -a\ln(81) \Rightarrow a = \frac{t}{\ln(1/81)}
    = \frac{\SI{2.3}{fm}}{\ln(1/81)} = \SI{-0.523}{fm}
  \]

\end{proof}

\begin{plm}[Bonus]
  Evaluate $\expval{r^{2}}$ for the Fermi distribution in Problem 6.
\end{plm}

\begin{proof}
  Another probability exercise.
  \[
    \expval{r^{2}} = \int_{\mathbb{R}^{3}}r^{2}\rho(r)dV = \rho_{0}\int_{0}^{\pi}\int_{0}^{2\pi}\int_{0}^{\infty}\frac{r^{2}}{1 + \exp[(r - c) / a]}dV
    = 4\pi\rho_{0}a\int_{0}^{\infty}\frac{r^{4}}{1 + e^{(r - c) / a}}du
  \]
  Integrals of the form after the obvious $u$-sub are trancendental, in general, even with bounds from zero to infinity;
  with the non-zero lower bound that $u$-sub generates, it's no doubt worse.
  So, it doesn't pay to carry this further.
  This integral can be readily computed via any favorite numerical method.
  Here's mine, in Guile Scheme (complete overkill for the three digits of precision I tend to carry things to):
  \inputminted[mathescape]{scheme}{integrate.scm}
  I'm adjusting to a new computer setup, and can't actually \textit{execute} the forms above, which sucks.
  In any case, I'm unable to disambiguate the integrand, as I can't seem to find $c$.
\end{proof}

\begin{proof}[Addenum]
  I've found a new toy to play with.
  \begin{thm}[Ramanujan's Master Theorem]
    Holomorphic functions of $-z$ have simple Mellin transforms; namely, if $f(z): \mathbb{C} \to \mathbb{C}$ is differentiable with
    $f(z) = \sum_{k = 0}^{\infty} \frac{a_{k}}{k!}(-z)^{k}$, where $a_{k}$ is given by an invertible transition rule $\sigma: a_{k} \mapsto a_{k+1}$,
    then
    \[
      \int_{0}^{\infty}x^{t-1}f(x)dx = \Gamma(t)a_{-t}.
    \]
  \end{thm}
  There's a general formula, derivable through rearrangement of formal power series,
  \begin{thm}
    Letting $f(x) = \sum_{n = 0}^{\infty}b_{n}x^{n}$ and $\frac{1}{f(x)} = \sum_{n = 0}^{\infty}d_{n}x^{n}$, one has
    \[
      d_{n} =
      \begin{cases}
        \frac{1}{b_{0}} & n = 0 \\
        -\sum_{k = 0}^{n-1}\frac{d_{k}}{b_{0}}b_{n - k} & n \geq 1
      \end{cases}
    \]
    Or, eliminating the recursion via induction,
    \[
      d_{n} = \sum_{|\alpha| = n}(-1)^{l(\alpha)}\frac{b_{\alpha}}{b_{0}^{l(\alpha) - 1}} = (-1)^{n}\sum_{|\alpha| = n}(-1)^{l(\alpha) - n}
      \frac{b_{\alpha}}{(b_{0})^{l(\alpha) - 1}}
    \]
    where $\alpha$ is a natural multi-index of length $l(\alpha)$, $|\alpha|$ is the sum of all the terms of $\alpha$,
    and $b_{\alpha}$ is the product of $b_{\alpha_{i}}$.
  \end{thm}
  A natural extension of $d_{n}$ to negative $n$ is made by negating every entry of all multi-indices with $|\alpha| = -n$,
  so they can sum to negative values; it remains to be proven that this agrees with the state transition, but if it doesn't, that'd really suck.

  Accordingly, the function $f(x) = \frac{1}{1 + e^{x}}$ has Maclaurin expansion coefficients
  \[
    a_{n} = (-1)^{n}\sum_{|\alpha| = n}\frac{(-1)^{l(\alpha) - n}}{2^{l(\alpha) - 1}\prod_{\alpha}(\alpha_{i}!)}
  \]
  This allows us to compute
  \[
    \int_{0}^{\infty}\frac{x^{2}}{1 + e^{x}}dx = \Gamma(3) \cdot \qty[- \sum_{|\alpha| = - 3}\frac{(-1)^{l(\alpha) - 3}}{2^{l(\alpha) - 1}}
    \prod_{\alpha}(\alpha_{i}!)]
  \]
\end{proof}

\end{document}
