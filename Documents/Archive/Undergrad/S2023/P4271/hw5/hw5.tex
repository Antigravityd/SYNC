\documentclass{article}

\usepackage[letterpaper]{geometry}
\usepackage{tgpagella}
\usepackage{amsmath}
\usepackage{amssymb}
\usepackage{amsthm}
\usepackage{tikz}
\usepackage{tikz-feynman}
\usepackage{minted}
\usepackage{physics}
\usepackage{siunitx}
\usepackage{mhchem}

\sisetup{detect-all}
\newtheorem{plm}{Problem}
\renewcommand*{\proofname}{Solution}

\title{4271 HW 5}
\author{Duncan Wilkie}
\date{27 March 2023}

\begin{document}

\maketitle

\begin{plm}
  Which of the following reactions are allowed and which are forbidden by the conservation laws appropriate for weak interactions?
  \begin{itemize}
  \item $\nu_{\mu} + p \to \mu^{+} + n$
  \item $\nu_{e} + p \to n + e^{-} + \pi^{+}$
  \item $K^{+} \to \pi^{0} + \mu^{+} + \nu_{\mu}$
  \item $\nu_{e} + p \to e^{-} + \pi^{+} + p$
  \item $\tau^{+} \to \mu^{+} + \bar{\nu}_{\mu} + \nu_{\tau}$
  \end{itemize}
\end{plm}

\begin{proof}
  If a conserved quantity is not mentioned, that is because it is zero on both sides of the equation in question.
  The first is allowed, because baryon number and muon lepton number are conserved.
  The second is allowed, because baryon number and electron lepton number are all conserved.
  The third is disallowed, because strangeness goes from 1 to zero.
  The fourth is allowed, because baryon number and electron lepton number all conserved.
  The fifth is allowed, because tau lepton number and muon lepton number are conserved.
\end{proof}

\begin{plm}
  In the decay of $\ce{^{47}Ca}$ to $\ce{^{47}Sc}$, what kinetic energy is given to the neutrino when the electron
  has kinetic energy $\SI{0.8}{MeV}$.
\end{plm}

\begin{proof}
  Looking at the JAEA nuclear data tables, the difference in atomic masses of these two nucleides is, in energy terms, $\SI{1.992}{MeV}$.
  Accordingly, the kinetic energy of the neutrino is, taking a relativistic approximation so that the mass of both ejectates is negligible,
  \[
    K_{\nu} = BE - K_{e} = \SI{1.992}{MeV} - \SI{0.8}{MeV} = \SI{1.19}{MeV}
  \]
\end{proof}

\begin{plm}
  Classify the following decays by degree of forbiddenness:
  \begin{enumerate}
  \item $\ce{^{81}Ge}\qty(\frac{5}{2}^{-}) \to \ce{^{81}Ge}\qty(\frac{9}{2}^{+})$
  \item $\ce{^{93}Kr}\qty(\frac{1}{2}^{+}) \to \ce{^{93}Rb}\qty(\frac{5}{2}^{+})$
  \item $\ce{^{93}Kr}\qty(\frac{1}{2}^{+}) \to \ce{^{93}Rb}\qty(\frac{3}{2}^{+})$
  \item $\ce{^{178}Lu}\qty(1^{+}) \to \ce{^{178}Hf}\qty(3^{+})$
  \end{enumerate}
\end{plm}

\begin{proof}
  Looking at the table in the slides,
  \begin{enumerate}
  \item $\Delta J = 2$, parity changes $\Rightarrow$ first-forbidden
  \item $\Delta J = 2$, parity the same $\Rightarrow$ second-forbidden
  \item $\Delta J = 1$, parity the same $\Rightarrow$ allowed
  \item $\Delta J = 2$, parity the same $\Rightarrow$ second-forbidden
  \end{enumerate}
\end{proof}

\begin{plm}
  Draw the lowest-order Feynman diagrams for:
  \begin{enumerate}
  \item $\nu_{e}-\nu_{\mu}$ elastic scattering,
  \item $e^{+} + e^{-} \to e^{+} + e^{-}$,
  \item (Bonus) a fourth-order diagram for $\gamma + \gamma \to e^{-} + e^{+}$
  \end{enumerate}
\end{plm}

\begin{proof}
  The TikZ package for Feynman diagrams is unmaintained and doesn't work, so I caved and used a pen.
  \begin{center}
    \includegraphics[scale=0.08,angle=90]{feyn.jpg}
  \end{center}
\end{proof}

\begin{plm}
  Find the kinetic energy (in the rest frame of the original particle) of each product particle in the following two-body decays:
  \begin{itemize}
  \item $\pi^{+} \to \mu^{+} + \nu_{\mu}$
  \item $\Lambda^{0} \to p + \pi^{-}$
  \item $K^{+} \to \pi^{+} + \pi^{-}$
  \end{itemize}
\end{plm}

\begin{proof}
  The conservation of 4-momentum is, with $c = 1$,
  \[
    p_{0} = p_{a} + p_{b}
    \Rightarrow m_{b}^{2} = p_{b}^{2} = (p_{0} - p_{a})^{2} = p_{0}^{2} - 2p_{0} \cdot p_{1} + p_{1}^{2}
  \]
  \[
    = m_{0}^{2} - 2m_{0}E_{a} + m_{a}^{2} \Leftrightarrow E_{a} = \frac{m_{0}^{2} + m_{a}^{2} -  m_{b}^{2}}{2m_{0}}
  \]
  and, doing the same for $a$,
  \[
    E_{b} = \frac{m_{0}^{2} + m_{b}^{2} -  m_{a}^{2}}{2m_{0}}
  \]
  Evaluating these for the above situations, using the mass values tabulated online,
  \[
    E_{\mu} = \frac{m_{\pi^{+}}^{2} + m_{\nu_{\mu}}^{2} - m_{\mu^{+}}^{2}}{2m_{\pi^{+}}}
    = \frac{(\SI{140}{MeV})^{2} + (\SI{0.2}{MeV})^{2} - (\SI{106}{MeV})^{2}}{2(\SI{140}{MeV})}
    = \SI{29.9}{MeV}
  \]
  \[
    E_{\nu_{\mu}} = \frac{m_{\pi^{+}}^{2} + m_{{\mu}}^{2} - m_{\nu_{\mu}}^{2}}{2m_{\pi^{+}}}
    = \frac{(\SI{140}{MeV})^{2} + (\SI{106}{MeV})^{2} - (\SI{0.2}{MeV})^{2}}{2(\SI{140}{MeV})}
    = \SI{110}{MeV}
  \]
  \[
    E_{p} = \frac{m_{\Lambda^{0}}^{2} + m_{p}^{2} - m_{\pi}^{2}}{2m_{\Lambda^{0}}}
    = \frac{(\SI{1116}{MeV})^{2} + (\SI{938}{MeV})^{2} - (\SI{170}{MeV})^{2}}{2(\SI{1116}{MeV})}
    = \SI{939}{MeV}
  \]
  \[
    E_{\pi} = \frac{m_{\Lambda^{0}}^{2} + m_{\pi}^{2} - m_{p}^{2}}{2m_{\Lambda^{0}}}
    = \frac{(\SI{1116}{MeV})^{2} + (\SI{170}{MeV})^{2} - (\SI{938}{MeV})^{2}}{2(\SI{1116}{MeV})}
    = \SI{177}{MeV}
  \]
  For the last, the masses of the differently-charged pions is the same, so we can save some calculation and note
  \[
    E_{\pi^{+}} = E_{\pi^{-}} = \frac{m_{K^{+}}}{2} = \frac{\SI{494}{MeV}}{2} = \SI{247}{MeV}
  \]

\end{proof}

\section*{Topic Ranking}

\begin{enumerate}
\item Detectors for subatomic physics
\item Biological effects of radiation
\item Origin of the elements
\item Nuclear reactors
\item Fusion
\item The big bang
\item The atomic bomb
\item Nuclear disasters
\end{enumerate}

\end{document}
