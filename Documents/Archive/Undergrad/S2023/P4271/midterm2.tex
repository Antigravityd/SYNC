\documentclass{article}

\usepackage[letterpaper]{geometry}
\usepackage{tgpagella}
\usepackage{amsmath}
\usepackage{amssymb}
\usepackage{amsthm}
\usepackage{tikz}
\usepackage{minted}
\usepackage{physics}
\usepackage{siunitx}
\usepackage{revsymb}
\usepackage{mhchem}

\sisetup{detect-all}
\newtheorem{plm}{Problem}
\renewcommand*{\proofname}{Solution}

\title{4721 HW 6}
\author{Duncan Wilkie}
\date{6 April 2023}

\begin{document}

\maketitle

\begin{plm}
  For the following gamma transitions, give all permitted multipoles and indicate which might be the most intense:
  \begin{enumerate}
  \item $\frac{9}{2}^{-} \mapsto \frac{7}{2}^{+}$
  \item $\frac{1}{2}^{-} \mapsto \frac{7}{2}^{-}$
  \item ${1}^{-} \mapsto 2^{+}$
  \item $4^{+} \mapsto 2^{+}$
  \item $3^{+} \mapsto 3^{+}$
  \end{enumerate}
\end{plm}
\begin{plm}
  An even-$Z$, even-$N$ nucleus has the following sequence of levels: $0+$ (ground state), $2+$ (\SI{89}{keV}), $4+$ (\SI{288}{keV}),
  $6+$ (\SI{585}{keV}), $0+$ (\SI{1050}{keV}), $2+$ (\SI{1129}{keV}).
  Drawn an energy level diagram and show all reasonbably probable gamma-ray transitions and their dominant multipole assignments.
\end{plm}
\begin{plm}
  The excited states of $\ce{^{174}Hf}$ have two similar rotational bands, with energies (in \si{MeV}) given in the following table.
  Calculate the moments of inertia for these two bands and comment on the difference.
  \begin{table}[H]
    \centering
    \begin{tabular}{|c|c|c|c|c|c|c|c|}
      \hline
      & $E(0^{+})$ & $E(2^{+})$ & $E(4^{+})$ & $E(6^{+})$ & $E(8^{+})$ & $E(10^{+})$ & $E(12^{+})$ \\
      \hline
      Band 1 & 0 & 0.091 & 0.297 & 0.608 & 1.010 & 1.486 & 2.021 \\
      \hline
      Band 2 & 0.827 & 0.900 & 1.063 & 1.307 & 1.630 & 2.026 & 2.489 \\
      \hline
    \end{tabular}
  \end{table}
\end{plm}
\begin{plm}[Bonus]
  Show explicitly that a uniformly-charged ellipsoid at rest with a total charge of $Ze$ and semi-axes $a$ and $b$ has a quadrupole moment
  \[
    Q = \frac{2}{5}Z\qty(a^{2} - b^{2})
  \]
\end{plm}
\begin{plm}
  Use the answer to Problem 4 to determine the sizes of the semi-major and semi-minor axes of $\ce{^{165}Ho}$,
  which has a quadrupole moment of $Q = \SI{3.5}{b}$.
\end{plm}
\begin{plm}
  Find the angle between the angular momentum vector $\ell$ and the $z$-axis for all possible orientations when $\ell = 3$.
\end{plm}
\begin{plm}
  Calculate the binding energy and binding energy per nucleon of the deuteron.
\end{plm}
\begin{plm}
  At what energy in the laboratory system does a proton beam scattering off a proton target become inelastic---i.e.
  at what proton beam energy can pions be produced?
\end{plm}
\begin{plm}
  In the spherical shell model, what are the expected ground state spins and parities for:
  $\ce{^{11}B}$, $\ce{^{15}C}$, $\ce{^{17}F}$, $\ce{^{31}P}$, $\ce{^{141}Pr}$, and $\ce{^{207}Pb}$.
  Look up the experimental values.
  Do they agree?
\end{plm}
\begin{plm}
  The ground state of $\ce{^{17}F}$ has spin-parity of $\frac{5}{2}^{+}$ and the first excited state has a spin-parity of $\frac{1}{2}^{+}$.
  Using Povh Fig. 18.7, suggest two possible configurations for this excited state.
\end{plm}
\begin{plm}
  The ground state of a nucleus with an odd proton and an odd neutron (aka. an ``odd-odd'' nucleus)
  is determined from the angular momentum coupling of the odd proton and neutron: $\vec{J} = \vec{j}_{p} + \vec{j}_{n}$.
  Consider the following nuclei: $\ce{^{16}N}(2^{-})$, $\ce{^{12}B}(1^{+})$, $\ce{^{34}P}(1^{+})$, $\ce{^{28}Al}(3^{+})$.
  \begin{enumerate}
  \item Draw simple vector diagrams illustrating these couplings---i.e.  $\vec{J} = \vec{j}_{p} + \vec{j}_{n}$.
  \item Replace $\vec{j}_{p}$ and $\vec{j}_{n}$, respectively, by $\vec{\ell}_{p} + \vec{s}_{p}$ and $\vec{\ell}_{n} + \vec{s}_{n}$,
    illustrating the two vectors $\vec{\ell}$ and $\vec{s}$.
  \item Examine your four diagrams and deduce an empirical rule for the relative orientation
    of $\vec{s}_{p}$ and $\vec{s}_{n}$ in the ground state.
  \item Use this empirical rule to predict the $\vec{J}^{\pi}$ assignments of $\ce{^{26}Na}$ and $\ce{^{28}Na}$.
  \end{enumerate}
\end{plm}
\begin{plm}[Bonus]
  Let's suppose we can form $\ce{^{3}He}$ or $\ce{^{3}H}$ by adding a proton or a neutron (respectively) to $\ce{^{2}H}$,
  which has $\vec{J}^{\pi} = 1^{+}$.
  \begin{enumerate}
  \item What are the possile values of the total angular momentum for $\ce{^{3}He}$ and $\ce{^{3}H}$,
    given an orbital angular momentum $\ell$ for the added nucleon?
  \item Given that $\ce{^{3}He}$ and $\ce{^{3}H}$ have positive parity, which of these is still possible?
  \item What is the most likely value for the ground-state orbital angular momentum of $\ce{^{3}He}$ and $\ce{^{3}H}$.
  \end{enumerate}
\end{plm}
\begin{plm}
  Common forms assumed for the momentum distributions of valence quarks in the proton are:
  \[
    F_{u} = xu(x) = a(1 - x)^{3}, \; F_{d}(x) = xd(x) = b(1 - x)^{3}.
  \]
  If the valence quarks account for half of the proton's momentum---i.e.
  \[
    \int_{0}^{1}xu(x)dx + \int_{0}^{1}xd(x)dx = \frac{1}{2},
  \]
  find the values of $a$ and $b$.
  Hint: the $u$ quarks carry approximately twice as much momentum as the $d$ quarks in the proton.
\end{plm}
\begin{plm}
  What is the color wavefunction for mesons, in analogy to that for baryons of
  \[
    y_{baryon} = y_{space}y_{spin}(rgb + gbr + brg - rbg - bgr - grb)?
  \]
  Explain your answer.
\end{plm}
\begin{plm}
  The diagram below shows the internal gluon interactions in a proton.
  Complete the diagram by labelling the color of the quarks and gluons.
\end{plm}
\begin{plm}
  Which of the following processes are allowed?
  If not allowed, state why.
  If allowed, say whether the process is strong, weak, or electromagnetic.
  \begin{enumerate}
  \item $\nu_{e} + p \to e^{-} + \pi^{+} + p$
  \item $e^{+} + e^{-} \to \mu^{+} + \mu^{-}$
  \item $\Sigma^{-} \to n + \pi^{-}$
  \item $\bar{\nu}_{e} + p \to e^{-} + n$
  \item $e^{-} + p \to \nu_{e} + \pi^{0}$
  \end{enumerate}
\end{plm}
\begin{plm}[Double Points]
  The differential cross section for $e^{+} + e^{-} \to \mu^{+} + \mu^{-}$ is given by
  \[
    \dv{\sigma}{\Omega} = \frac{\alpha^{2}}{4s}(\hbar c)^{2}(1 + \cos^{2}\theta)
  \]
  in a collider experiment where $s = 4E_{e}$ and $E_{e}$ is the electron/positron energy.
  \begin{enumerate}
  \item Integrate over the solid angle to obtain an expression for the total cross section.
  \item If you use an electron beam energy of \SI{4}{GeV}, what rate of production of $\mu^{+}\mu^{-}$ would you expect at a luminosity of
    \SI{e33}{Hz/cm^{2}}?
  \item Calculate the ratio of the hadronic production cross section to that for $\mu^{+}\mu^{-}$ at $E_{e} = \SI{500}{GeV}$.
    If you use an electron beam energy of \SI{500}{GeV}, what must the luminosity be to measure the hadronic cross section within 24 hours
    with 10\% statistical uncertainty?
  \end{enumerate}
\end{plm}
\begin{plm}[Double Points]
  In an $e^{+}e^{-}$ collider experiment, a resonance $R$ is observed at $E_{cm} = \SI{10}{GeV}$ in both the $\mu^{+}\mu^{-}$
  and hadronic final states.
  The integrated cross sections are
  \[
    \int \sigma_{\mu \mu}(E)dE = \SI{10}{nb \cdot GeV}
  \]
  and
  \[
    \int \sigma_{h}(E)dE = \SI{300}{nb \cdot GeV}.
  \]
  Use a Breit-Wigner form for the resonance production to deduce the partial widths $\Gamma_{\mu\mu}$ and $\Gamma_{h}$ in \si{MeV}
  for the decays $R \to \mu^{+}\mu^{-}$ and $R \to \text{hadrons}$.
  Assume the integral
  \[
    \int_{resonance}\frac{dE}{(E - Mc^{2}) + \Gamma^{2}/4}dE \approx \frac{2\pi}{\Gamma}.
  \]
\end{plm}
\begin{plm}
  Find the threshold kinetic energy for each of the following reactions, assuming the first particle to be incident on the second particle
  at rest:
  \begin{enumerate}
  \item $K^{-} + p \to \Xi^{-} + K^{+}$
  \item $\bar{p} + p \to \Upsilon$
  \item $\pi^{-} + p \to \omega + n$
  \end{enumerate}
\end{plm}
\begin{plm}
  Which of the following reactions are allowed and which are forbidden by the conservation laws appropriate for weak interactions?
  \begin{itemize}
  \item $\nu_{\mu} + p \to \mu^{+} + n$
  \item $\nu_{e} + p \to n + e^{-} + \pi^{+}$
  \item $K^{+} \to \pi^{0} + \mu^{+} + \nu_{\mu}$
  \item $\nu_{e} + p \to e^{-} + \pi^{+} + p$
  \item $\tau^{+} \to \mu^{+} + \bar{\nu}_{\mu} + \nu_{\tau}$
  \end{itemize}
\end{plm}
\begin{plm}
  In the decay of $\ce{^{47}Ca}$ to $\ce{^{47}Sc}$, what kinetic energy is given to the neutrino when the electron
  has kinetic energy $\SI{0.8}{MeV}$.
\end{plm}
\begin{plm}
  Classify the following decays by degree of forbiddenness:
  \begin{enumerate}
  \item $\ce{^{81}Ge}\qty(\frac{5}{2}^{-}) \to \ce{^{81}Ge}\qty(\frac{9}{2}^{+})$
  \item $\ce{^{93}Kr}\qty(\frac{1}{2}^{+}) \to \ce{^{93}Rb}\qty(\frac{5}{2}^{+})$
  \item $\ce{^{93}Kr}\qty(\frac{1}{2}^{+}) \to \ce{^{93}Rb}\qty(\frac{3}{2}^{+})$
  \item $\ce{^{178}Lu}\qty(1^{+}) \to \ce{^{178}Hf}\qty(3^{+})$
  \end{enumerate}
\end{plm}
\begin{plm}
  Draw the lowest-order Feynman diagrams for:
  \begin{enumerate}
  \item $\nu_{e}-\nu_{\mu}$ elastic scattering,
  \item $e^{+} + e^{-} \to e^{+} + e^{-}$,
  \item (Bonus) a fourth-order diagram for $\gamma + \gamma \to e^{-} + e^{+}$
  \end{enumerate}
\end{plm}
\begin{plm}
  Find the kinetic energy (in the rest frame of the original particle) of each product particle in the following two-body decays:
  \begin{itemize}
  \item $\pi^{+} \to \mu^{+} + \nu_{\mu}$
  \item $\Lambda^{0} \to p + \pi^{-}$
  \item $K^{+} \to \pi^{+} + \pi^{-}$
  \end{itemize}
\end{plm}
\begin{plm}
  Calculate the average binding energy for \ce{^{40}K}.
\end{plm}
\begin{plm}
  The mass of \ce{^{74}Ge} is \SI{73.921177}{u}.
  \begin{enumerate}
  \item Calculate its mass defect in \si{MeV/c^2}.
  \item Calculate its binding energy in \si{MeV}.
  \item What is the binding energy per nucleon?
  \end{enumerate}
\end{plm}
\begin{plm}
  \;
  \begin{enumerate}
  \item Calculate the energy released in the beta decay of \ce{^{32}P}.
  \item If the resulting beta particle has energy \SI{650}{keV}, how much energy does the antineutrino have?
  \end{enumerate}
\end{plm}
\begin{plm}
  Use the semi-empirical mass formula to estimate the prompt energy released in spontaneous fission of \ce{^{239}Pu} as
  \begin{center}
    \ce{^{239}Pu -> ^{112}Pd + ^{124}Cd + 3n}
  \end{center}
\end{plm}
\begin{plm}
  Use \textit{accurate} masses from the 2021 atomic mass evaluation to calculate the actual prompt energy released in problem 4.
  Compare the answers.
  How different are they?
\end{plm}
\begin{plm}
  Following spontaneous fission from the last problem, a chain of beta decays converts the \ce{^{112}Pd} and \ce{^{124}Cd} to stable nuclei.
  \begin{enumerate}
  \item What are the final nuclei produced?
  \item What is the total beta-delayed energy released in the beta-decay chains?
    What is the ratio of beta-delayed to prompt energy released?
  \item (Bonus) About how long does it take (on average) for all the energy to be released?
  \end{enumerate}
\end{plm}
\begin{plm}
  The isotope \ce{^{252}Cf} is commonly used to study the geology around underground wells due to the high number of neutrons emitted.
  Each spontaneous fission of \ce{^{252}Cf} produces an average of 3.77 neutrons.
  However, alpha decay dominates, with only 3.09\% of the decays being fission.
  \begin{enumerate}
  \item Calculate the activity (in Curies) of \ce{^{252}Cf} that would be requried to produce \SI{5e8}{neutrons/s}.
  \item If we make such a source, the neutron flux will decrease over time as the \ce{^{252}Cf} decays.
    How many \si{neutrons/s} will be produced 4 years later? ($t_{1/2} = \SI{2.645}{yr}$)
  \end{enumerate}
\end{plm}
\begin{plm}
  The oldest formerly living objects that can be measured by \ce{^{14}C} dating are about \SI{50000}{yo}.
  \begin{enumerate}
  \item What is the ratio of \ce{^{14}C/^{12}C} in such a sample?
  \item Assume the natural abundance ratio in the atmosphere is \SI{1.5e-12}{}.
    If you have to count 25 atoms of \ce{^{14}C} to make a measurement (20\% accuracy),
    how much mass of carbon would you need to date a \SI{50000}{yo} object?
  \end{enumerate}
\end{plm}
\begin{plm}
  The Cassini spacecraft went into orbit around the planet Saturn in July 2004, after a nearly seven-year journey from Earth.
  On-board electrical systems were powered by heat from three radioisotope thermoelectric generators,
  which together utilized a total of \SI{32.7}{kg} of \ce{^{238}Pu}, encapsulated as \ce{PuO_2}.
  The isotope has a half-life of \SI{86.4}{y} and emits an alpha particle with an average energy of \SI{5.49}{MeV}.
  \begin{enumerate}
  \item How much total thermal power is generated in the spacecraft?
  \item To see why RTGs are so attractive, calculate the solar panel area needed to produce the same amount of wattage at the distance of Saturn.
    Assume your solar panel is 10\% efficient and that the Sun's luminosity is \SI{3.8e26}{W}.
  \end{enumerate}
\end{plm}
\begin{plm}
  Calculate the minimum energy required to be over the Coulomb barrier for:
  \begin{enumerate}
  \item $\ce{p + p}$,
  \item $\ce{p + ^{12}C}$,
  \item $\ce{^{4}He + ^{208}Pb}$.
  \end{enumerate}
\end{plm}
\begin{plm}
  The cross section for charged-particle reactions is proportional to the probability of tunneling through the Coulomb barrier
  given by the Gamow factor, which has a convenient approximation:
  \[
    e^{-2\pi\eta} = e^{-2\pi Z_{1}Z_{2}e^{2}/\hbar\nu} = e^{-31.287Z_{1}Z_{2}\sqrt{\mu/E}}
  \]
  where $\mu$ is the reduced mass in \si{amu} and $E$ is the center-of-mass energy in \si{keV}.
  For the 3 cases you considered above, calculate the Gamow factor for an energy that is one-quarter the barrier energy you found in problem 1.
\end{plm}
\begin{plm}
  A \SI{2}{MeV} beam of protons bombards a \ce{^{16}O} target and the differential cross section is measured to be \SI{0.094}{b/sr}
  at a lab angle of $167^{\circ}$.
  \begin{enumerate}
  \item What is the expected cross-section if you assume Rutherford scattering?
  \item What is the calculated Mott cross-section?
  \item How do your answers to (a) and (b) differ from the measured cross section and why might they be different?
  \end{enumerate}
\end{plm}
\begin{plm}
  Assume that $^{197}Au$ is made from a solid, uniform sphere of nuclear material with a radius of $R = \SI{1.2}{fm} \cdot A^{1/3}$.
  Calculate the form factor $F(q)$.
\end{plm}
\begin{plm}
  Show that the mean-square charge radius of a uniformly charged sphere is $\expval{r^{2}} = 3R^{2}/5$.
\end{plm}
\begin{plm}
  A nuclear charge distribution more realistic than the uniformly charged distribution is the Fermi distribution,
  $\rho(r) = \frac{\rho_{0}}{1 + \exp\qty[(r - c)/a]}$.
  Find the value of $a$ if $t = \SI{2.3}{fm}$
\end{plm}
\begin{plm}[Bonus]
  Evaluate $\expval{r^{2}}$ for the Fermi distribution in Problem 6.
\end{plm}

\end{document}
