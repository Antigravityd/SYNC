\documentclass{article}

\usepackage[letterpaper]{geometry}
\usepackage{tgpagella}
\usepackage{amsmath}
\usepackage{amssymb}
\usepackage{amsthm}

\title{7210 HW 3}
\author{Duncan Wilkie}
\date{13 September 2022}

\newtheorem*{3.1.25}{3.1.25}
\newtheorem*{3.1.36}{3.1.36}
\newtheorem*{3.1.41}{3.1.41}
\newtheorem*{3.2.16}{3.2.16}
\newtheorem*{3.2.19}{3.2.19}
\newtheorem*{3.2.21}{3.2.21}
\newtheorem*{3.3.3}{3.3.3}

\begin{document}

\maketitle

\begin{3.1.25}
  A subgroup $N$ of a group $G$ is normal iff $gNg^{-1}\subseteq N$ for all $g\in G$.
\end{3.1.25}

\begin{proof}
  The definition of $N \trianglelefteq G$ is that $gNg^{-1} = N$ for all $g \in G$.
  The forward equivalence is therefore trivial.
  Conversely, suppose $gNg^{-1} \subseteq N$ for all $g$.
  \[
    N = \{n \in N \mid n \in N\} = \{n \in N \mid gg^{-1}ngg^{-1} \in N\} = \{n \in N \mid g\left( g^{-1}n^{-1}g \right)^{-1}g^{-1} \in N\}
  \]
  Take any element $n \in N$;
  \[
    n = gg^{-1}ngg^{-1} = g(g^{-1}n^{-1}g)^{-1}g^{-1} = g(g'n'(g')^{-1})^{-1}g^{-1} = gn''g^{-1}
  \]
  where $g' = g^{-1}$, $n' = n^{-1} \in N$ by closure, and $n'' \in N$ is that which must then exist to satisfy $g'n'(g')^{-1} \subseteq N$.
  This demonstrates that every element of $N$ equals an element of $gNg^{-1}$, so that $N \subseteq gNg^{-1}$ and consequently $gNg^{-1} = N$.
\end{proof}

\begin{3.1.36}
  If $G$ is a group such that $G / Z(G)$ is cyclic, then $G$ is Abelian.
\end{3.1.36}

\begin{proof}
  If $G / Z(G)$ is cyclic, it has a single generator $xZ(G)$.
  This means any element of $G / Z(G)$ may be written $x^{a}Z(G)$ for some $a \in \mathbb{Z}$.

  Any element $g\in G$ generates a coset $gZ(G)$.
  Since the factor group is cyclic, this coset is $x^{a}Z(G)$ for some integer $a$.
  Symbolically,
  \[
    gZ(G) = x^{a}Z(G) \Leftrightarrow (x^{a}Z(G))^{-1}(gZ(G)) = 1 \cdot Z(G) \Leftrightarrow x^{-a}gZ(G) = Z(G)
    \Leftrightarrow x^{-a}g = z \in Z(G).
  \]
  Thus all $g\in G$ can be written $x^{a}z$ for some integer $a$, some central element $z$, and $x$ a representative of the generator coset.
  Therefore, since central $z$ commute with all elements of $G$,
  \[
    ab = x^{a_{1}}z_{1}x^{a_{2}}z_{2} = z_{2}x^{a_{1}}x^{a_{2}}z_{1} = z_{2}x^{a_{1}+a_{2}}z_{1} = z_{2}x^{a_{2}}x^{a_{1}}z_{1}
    = x^{a_{2}}z_{2}x^{a_{1}}z_{1} = ba
  \]
  for any $a,b \in G$.
\end{proof}

\begin{3.1.41}
  Let $G$ be a group.
  Then $N = \langle x^{-1}y^{-1}xy \mid x, y \in G \rangle$ is a normal subgroup of $G$ and $G / N$ is Abelian.
\end{3.1.41}

\begin{proof}
  Consider $g \in G$ and $n \in N$.
  \[
    gng^{-1} = gng^{-1}n^{-1}n = (gng^{-1}n^{-1})n
  \]
  The left factor is in $N$ by taking $x = g^{-1}$ and $y = n^{-1}$.
  The right factor is in $N$ by assumption, and $N$ is a (sub)group, and therefore closed under multiplication.
  This proves this subgroup is closed under conjugation, and is therefore normal.

  In $G / N$, consider the coset $ghN = \{ghn \mid n \in N \}$.
  The element of $ghN$ corresponding to the element of $N$ with $x = h$ and $y = g$ is $ghh^{-1}g^{-1}hg = hg$.
  This implies that $gh$ and $hg$ generate the same coset, as cosets by partition $G$.
  In other words,
  \[
    ghN = (gN)(hN) = hgN = (hN)(gN),
  \]
  therefore $G / N$ is Abelian.
\end{proof}

\begin{3.2.16}
  If $p$ is prime then $a^{p} \equiv a\pmod{p}$ for all $a\in\mathbb{Z}$.
\end{3.2.16}

\begin{proof}
  Consider $\frac{(\mathbb{Z} / p\mathbb{Z})^{\times}}{\langle a \rangle}$, where the generator is multiplicative, of course.
  The order of the numerator group is $p-1$; the order of the denominator group is $|a|$.
  Since the order of the factor group must be integral, $|a| \mid p-1$, implying $a^{p-1} \equiv 1 \pmod p \Leftrightarrow a^{p} = a \pmod p$.
\end{proof}

\begin{3.2.19}
  If $N$ is a normal subgroup of the finite group $G$ and $|N|$ and $|G : N|$ are relatively prime
  then $N$ is the unique subgroup of $G$ of order $|N|$.
\end{3.2.19}

\begin{proof}

  If there exists some other subgroup $H$ with $|H| = |N|$, $HN \leq G$ since $N$ is normal (Corollary 15).
  Then we can take $|G : HN|$; $|HN| = |H||N| / |H \cap N|$ (Proposition 13), so
  \[
    |G : HN| = \frac{|G|}{|HN|} = \frac{|G| \cdot |H \cap N|}{|H| \cdot|N|}
  \]
  By assumption, $|G| / |N|$ is coprime to $|N| = |H|$, so $|H|$ must divide $|H \cap N|$ if $|G : HN|$ is to remain integral
  (no factor of $|H|$ divides $|G| / |N|$, so they must \textit{all} divide $|H \cap N|$).

  However, since $H$ and $N$ are of the same size, $|H \cap N|$ is at most $|H|$, so it must equal $|H|$.
  This implies that $H$ and $H$ are identical, since each of their elements is in the intersection.
\end{proof}

\begin{3.2.21}
  $\mathbb{Q}$ has no proper subgroups of finite index, as does $\mathbb{Q} / \mathbb{Z}$.
\end{3.2.21}

\begin{proof}
  The left cosets by $N$ are of the form
  \[
    \frac{p}{q} + N = \left\{\frac{p}{q} + n \mid n \in N \right\}
  \]
  For a fixed $N$, each $\frac{p}{q}$ generates a different coset, since elementwise addition by $\frac{p}{q}$ is a nonidentity (since $N\neq \mathbb{Q}$,
  so there exists a ``hole'' that gets moved) set automorphism on $\mathcal{P}(\mathbb{Q})$ with inverse elementwise subtraction by $\frac{p}{q}$.
  Therefore the cosets are in bijection with $\mathbb{Q}$, i.e. $|\mathbb{Q} : N|$ is infinite.

  Using this, elements of $\mathbb{Q} / \mathbb{Z}$ are of the form $q + Z$, and each $q$ generates a distinct coset.
  The left cosets by another $N$ of this group are of the form
  \[
    \frac{p}{q} + N = \left\{ (\frac{p}{q} + \mathbb{Z}) + (n + \mathbb{Z}) \mid n + \mathbb{Z} \in N\right\}
    = \left\{ (\frac{p}{q} + n + \mathbb{Z})  \mid n + \mathbb{Z} \in N\right\}
  \]
  The representative is another element of $\mathbb{Q}$ that's distinct for every $\frac{p}{q}$, and since distinct such elements generate distinct cosets,
  the cosets are in bijection with $\mathbb{Q}$, i.e. $|\mathbb{Q} / \mathbb{Z} : N|$ is infinite.
\end{proof}

\begin{3.3.3}
  If $H$ is a normal subgroup of $G$ of prime index $p$ then for all $K \leq G$ either $K \leq H$ or $G = HK$ and $|K : K \cap H| = p$.
\end{3.3.3}

First of all, $HK \leq G$ since $H$ is normal; equivalently, $HK = KH$.
If $K \not\leq H$, then there's some element $k \in K$ that's not in $H$.
Then $kH$ generates $G / H$: by Lagrange's theorem, $|\langle kH \rangle|$ must divide $|G / H|$, but $|G / H| = |G : H|$ is prime,
and $\langle kH \rangle$ is nontrivial since $k\cdot 1 \not\in H \Rightarrow kH \neq H$.

Stating that another way, all cosets by $H$ in $G$ are of the form $k^{i}H$.
This implies $HK = KH = \{k'h \mid k'\in K, h \in H\} \supseteq \cup_{i} k^{i}H$, and since cosets of the quotient group partition $G$, the last equals $G$.
Since all elements are in $G$ by default, $G = HK = KH$.


We can now apply the second isomorphism theorem with the knowledge that $KH = G$.
Since $H$ is normal, $K \leq N_{G}(H)$, since $H$ normalizes to $G$.
This implies $K \cap H \trianglelefteq K$ and $KH / H = G / H \cong K / K \cap H$ by the second isomorphism theorem.
In particular,
\[
  |K : K \cap H| = |G : H| = p
\]


\end{document}
