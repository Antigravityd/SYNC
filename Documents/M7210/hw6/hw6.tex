\documentclass{article}

\usepackage[letterpaper]{geometry}
\usepackage{tgpagella}
\usepackage{amsmath}
\usepackage{amssymb}
\usepackage{amsthm}
\usepackage{physics}

\title{7510 HW 6}
\author{Duncan Wilkie}
\date{6 October 2022}

\newcommand{\nsub}{\trianglelefteq}
\newtheorem*{4.5.4}{4.5.4}
\newtheorem*{4.5.17}{4.5.17}
\newtheorem*{4.5.33}{4.5.33}
\newtheorem*{5.1.4}{5.1.4}
\newtheorem*{7.1.11}{7.1.11}
\newtheorem*{7.1.14}{7.1.14}

\newtheorem{problem}{Problem}
\begin{document}

\maketitle

\begin{problem}[D\&F 4.5.4]
  Exhibit all Sylow 2-subgroups and Sylow 3-subgroups of $D_{12}$ and $S_{3} \times S_{3}$.
\end{problem}

\begin{proof}[Solution]
  $|D_{12}| = 12 = 2^{2}\cdot 3$ and $|S_{3}\times S_{3}| = 3! \cdot 3! = 3^{2}\cdot 2^{2}$,
  so the Sylow 2-subgroups are of order 4 in each case, and the Sylow 3-subgroups of order 3 and 9, respectively.
  By Sylow's theorem, the number of Sylow $p$-subgroups must divide $m$, if $|G| = p^{\alpha}m$.
  Therefore, $n_{2}(D_{12}) \mid 3$, $n_{3}(D_{12}) \mid 4$, $n_{2}(S_{3} \times S_{3}) \mid 9$, and $n_{3}(S_{3} \times S_{3}) \mid 4$.
  Accordingly, $n_{2}(D_{12}) \in \{1, 3\}$,  $n_{3}(D_{12}) \in \{1, 2, 4\}$, $n_{2}(S_{3} \times S_{3}) \in \{1, 3, 9\}$,
  and $n_{3}(S_{3} \times S_{3}) \in \{1, 2, 4\}$.
  The condition $n_{p} \equiv 1 \pmod p$ eliminates $n_{3} = 2$ for both groups.
  The standard presentation for $D_{12}$ is $\braket{s, r}{s^{2}=r^{6}=1, sr=r^{-1}s}$.
  Some subgroups of order 4 are:
  \[
    \langle s, r ^{3} \rangle = \{1, s, r^{3}, sr^{3}\}
  \]
  \[
    \langle sr, r^{3} \rangle = \{1, sr, r^{3}, sr^{4}\}
  \]
  \[
    \langle sr^{5}, r^{3}\rangle = \{1, sr^{2}, r^{3}, sr^{5}\}
  \]
  By the above, this is exhaustive for $n_{2}(D_{12})$.
  All groups of order 3 are cyclic, and any cyclic group involving $s$ is of order 2, so the only possible groups come from the rotations.
  The only Sylow 3-subgroup one is therefore $\langle r^{2} \rangle$.

  Now for $S_{3} \times S_{3}$.
  Subgroups in each axis of order 2 will generate product subgroups of order 4.
  Subgroups of $S_{3}$ of order 2 are generated by transpositions: $(1 \; 2), (2 \; 3),$ and $(1 \; 3)$.
  The corresponding subgroups of order 4 are constructed by making two choices with replacement from this list,
  and taking the product of subgroups of $S_{3}$ so-generated.
  There are $3^{2} = 9$ ways to choose 2 items from a list of 3 elements with replacement, which is the largest possibility,
  so this is an exhaustive description of $Syl_{2}(S_{3} \times S_{3})$.
  The unique subgroup $S_{3}$ of order 3 consists of 3-cycles: $\{1, (1 \; 2 \; 3), (1 \; 3 \; 2)\}$.
  It is also normal, since it's of index 2; if $N, N'$ are normal, then $N \times N'$ is a normal subgroup as well:
  $(g, g')(n, n')(g, g')^{-1} = (gng^{-1}, g'n'g'^{-1})$, and each element is in $N$ and $N'$ by normality of the axis subgroups.
  Therefore, the product of the 3-cycle subgroups is a normal Sylow $p$-subgroup, and $n_{3}(S_{3} \times S_{3}) = 1$.
\end{proof}

\begin{problem}[D\&F 4.5.17]
  If $|G| = 105$, then $G$ has a normal Sylow 5-subgroup and a normal Sylow 7-subgroup.
\end{problem}

\begin{proof}
  $|G| = 3 \cdot 5 \cdot 7$; accordingly, by Sylow's theorem, $n_{5} \mid 21$ and $n_{7} \mid 15$.
  Additionally, $n_{5} \cong 1 \pmod 5$ and $n_{7} \cong 1 \pmod 7$, and since $3 \equiv 3 \pmod 5$, $7 \equiv 2 \pmod 5$,
  and all the factors of 15 are less than 7, the only admissible values are $n_{5} = n_{7} = 1$.
  By Corollary 20, then, the Sylow 5- and 7-subgroups of $G$ are normal.
\end{proof}

\begin{problem}[D\&F 4.5.33]
  If $P$ is a normal Sylow $p$-subgroup of $G$ and $H$ is any other subgroup of $G$, then $P \cap H$ is the unique Sylow $p$-subgroup of $H$.
\end{problem}

\begin{proof}
  Since $P \nsub G$, $N_{G}(P) = G$, so $H \leq N_{G}(P)$.
  Then by the diamond isomorphism theorem, $P\cap H \nsub H$, and $PH / P \cong H / P \cap H$.
  In particular,
  \[
    |PH| / |P| = |H| / |P \cap H| \Leftrightarrow |P \cap H| = \frac{|H||P|}{|PH|}
  \]
  Since $PH$ is a group, and trivially $H \leq PH$, by Lagrange's theorem $|H| \mid |PH|$, so letting $|PH| / |H| = k$,
  \[
    |P \cap H| = \frac{|P|}{k} = \frac{p^{\alpha}}{k}
  \]
  This must be an integer, which can only happen if $k = p^{\alpha'}$, meaning $|P \cap H| = p^{\alpha - \alpha'}$, i.e. $P \cap H$ is a $p$-subgroup.
  We now need to prove $p \nmid |H| / p^{\alpha - \alpha'} = |PH| / |P| = |PH| / p^{\alpha}$.
  Since $PH \leq G$, $|PH| \mid |G|$, so $|PH| / p^{\alpha} \mid |G| / p^{\alpha}$.
  The latter has no factor of $p$ by assumption that $P$ is a Sylow $p$-subgroup, so $P \cap H$ is in fact a Sylow $p$-subgroup.
  Since it's normal, it's unique.
\end{proof}

\begin{problem}[D\&F 5.1.4]
  If $A$ and $B$ are finite groups and $p$ is a prime, then any Sylow $p$-subgroup of $A \times B$ is of the form $P \times Q$,
  where $P \in Syl_{p}(A)$ and $Q \in Syl_{p}(B)$.
  Additionally, $n_{p}(A \times B) = n_{p}(A)n_{p}(B)$, and both results generalize to a direct product of arbitrary arity.
\end{problem}

\begin{proof}
  We may write $|A \times B| = |A| \cdot |B| = p^{\alpha}m$, where $p \nmid m$ and $1 \leq \alpha$.
  Take $A \times 1$ and $1 \times B$; these are subgroups of $A \times B$, and so have orders dividing $p^{\alpha}m$.
  Accordingly, their orders can be written $p^{\alpha'}m'$ and $p^{\alpha''}m''$, where $p \nmid m',m''$.
  Now, consider any of the Sylow $p$-subgroups $H$ of $A \times B$; it will be of order $p^{\alpha}$.
  The intersections of $H$ with $A \times 1$ and $1 \times B$ are in fact subgroups of both parents by trivial application of the subgroup criterion.
  Their orders must divide both $p^{\alpha}$ and $p^{\alpha'}m'$ or $p^{\alpha''}m'' $ by Lagrange's theorem; they then must be of the form $p^{\beta}$,
  and so their isomorphic image according to $A \times 1 \cong A$ and $1 \times B \cong B$ are Sylow $p$-subgroups $A$ and $B$,
  since $p \nmid m', m''$ and isomorphisms preserve orders in the subgroup lattice.

  We have by Sylow's theorem that if $|A \times B| = p^{\alpha}m$ where $p \nmid m$, then $n_{p}(A \times B) \mid m$ and $n_{p} \cong 1 \pmod p$.
  Additionally, from the property of the product group $|A \times B| = |A| \cdot |B|$, so $n_{p}(A \times B) \mid |A| \cdot |B| / p^{\alpha}$.
  $n_{p}(A)n_{p}(B) = \frac{|A|}{p^{\alpha}}\frac{|B|}{p^{\alpha'}} = \frac{|A \times B|}{p^{\alpha + \alpha'}}$.
  According to the above, $p^{\alpha}p^{\alpha'}$ is the order of Sylow $p$-subgroups, since they're the product of component $p$-subgroups.
  Therefore, the above is equal to $n_{p}(A \times B)$.

\end{proof}

\begin{problem}[D\&F 7.1.11]
  If $R$ is an integral domain and $x^{2} = 1$ for some $x \in R$ then $x = \pm 1$.
\end{problem}

\begin{proof}
  By un-distributing,
  \[
    0 = 1 - 1 = 1 - x^{2} = 1 - (x - x) - x^{2} = (1 - x) + x(1 - x) =  (1 + x)(1 - x),
  \]
  Since integral domains don't have zero divisors, for the last term to equal zero, then either one or both terms must equal zero.
  So, $1 + x = 0 \Leftrightarrow x = -1$, or $1 - x = 0 \Leftrightarrow x = 1$.
  It should be noted that the ``or'' in $x = \pm 1$ is not exclusive, as  $1 = -1 \Leftrightarrow 1 + 1 = 0$ holds in e.g. fields of characteristic 2,
  which are perfectly good integral domains.
\end{proof}

\begin{problem}[D\&F 7.1.14]
  If $x$ is a nilpotent element of a commutative ring $R$, then
  \begin{enumerate}
  \item $x$ is either zero or a zero-divisor,
  \item $rx$ is nilpotent for all $r \in R$,
  \item $1 + x$ is a unit in $R$, and
  \item the sum of a nilpotent element and a unit is a unit.
  \end{enumerate}
\end{problem}

\begin{proof}
  If there exists positive integral $m$ such that $x^{m} = 0$, then $x$ is nilpotent.
  $x$ can be zero or nonzero; if it's nonzero, then the set of positive integral $n$ for which $x^{n} = 0$ is nonempty (according to existence of $m$)
  and bounded below by 2, so there's a least element $m'$ (the smallest power to which $x$ may be taken yielding zero).
  Then $x^{m'} = x(x^{m'-1}) = 0$;  $x$ and $x^{m'-1}$ are nonzero elements of $R$ that multiply to zero, and so are zero divisors.

  Applying commutativity,
  \[
    (rx)^{m} = \underbrace{rx \cdot rx \cdots  rx}_{m\text{ times}}
    = \underbrace{r\cdot r \cdots r}_{m\text{ times}}\overbrace{x \cdot x \cdots x}^{m\text{ times}} = r^{m}x^{m} = r^{m} \cdot 0 = 0
  \]
  so $rx$ is nilpotent for all $r \in R$.

  Note that in
  \[
    (1 + x)(1 - x + x^{2} + \cdots \pm x^{m'-1}) = 1 - x + x^{2} + \cdots + x^{m - 1} + x - x^{2} + x^{3} + \cdots \pm x^{m'}
  \]
  all the inner terms additively cancel and $x^{m'} = 0$, yielding an overall result of 1.
  So, $(1 + x)$ is a unit with inverse equal to the alternating polynomial.

  Suppose $a \in R$ is a unit, i.e. there exists $a^{-1} \in R$ such that $aa^{-1} = 1$.
  Then $a + x = a(1 + a^{-1}x)$; $a^{-1}x$ is nilpotent by part 2, and by part 3, $1 + a^{-1}x$ is then a unit itself.
  So this is a product of two units, which has inverse $(1 + a^{-1}x)^{-1}a^{-1}$, and so the sum of a unit an a nilpotent element is a unit.
\end{proof}
\end{document}
