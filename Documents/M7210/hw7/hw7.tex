\documentclass{article}

\usepackage[letterpaper]{geometry}
\usepackage{tgpagella}
\usepackage{amsmath}
\usepackage{amssymb}
\usepackage{amsthm}
\usepackage{tikz}
\usepackage{minted}
\usepackage{physics}
\usepackage{siunitx}

\sisetup{detect-all}
\newtheorem{plm}{Problem}

\title{7210 HW 7}
\author{Duncan Wilkie}
\date{25 October 2022}

\begin{document}

\maketitle

\begin{plm}[D\&F 7.1.30]
  Let $A = \mathbb{Z} \times \mathbb{Z} \times \cdots$ be the direct product of copies of $\mathbb{Z}$ indexed by the positive integers
  (so $A$ is a ring under componentwise addition and multiplication) and let $R$ be the ring of all group homomorphisms from $A$ to itself
  with addition pointwise and multiplication defined as function composition.
  Let $\phi$ be the element of $R$ defined by $\phi(a_{1}, a_{1}, a_{3}, ...) = (a_{2}, a_{3}, ...)$.
  Let $\psi$ be the element of $R$ defined by $\psi(a_{1}, a_{2}, a_{3}, ...) = (0, a_{1}, a_{2}, a_{3},...)$
  \begin{enumerate}
  \item Prove that $\phi\psi$ is the identity of $R$ by $\psi\phi$ is not the identity of $R$ (i.e. $\psi$ is a right, but not a left, inverse for $\phi$).
  \item Exhibit infinitely many right inverses for $\phi$.
  \item Find a nonzero element $\pi$ in $R$ such that $\phi\pi = 0$ but $\pi\phi \neq 0$.
  \item Prove that there is no nonzero element $\lambda \in R$ such that $\lambda\phi = 0$ (i.e. $\phi$ is a left zero divisor but not a right zero divisor).
  \end{enumerate}
\end{plm}

\begin{plm}[D\&F 7.3.29]

\end{plm}

\begin{plm}[D\&F 7.3.33]

\end{plm}

\begin{plm}[D\&F 7.4.15]

\end{plm}

\begin{plm}[D\&F 7.4.27]

\end{plm}

\begin{plm}[D\&F 7.4.30]

\end{plm}

\begin{plm}[D\&F 7.4.37]

\end{plm}




\end{document}
