\documentclass{article}

\usepackage[letterpaper]{geometry}
\usepackage{tgpagella}
\usepackage{amsmath}
\usepackage{amssymb}
\usepackage{amsthm}
\usepackage{tikz}
\usepackage{minted}
\usepackage{physics}
\usepackage{siunitx}

\sisetup{detect-all}
\newtheorem{plm}{Problem}

\title{Math 7550 HW 1}
\author{Duncan Wilkie}
\date{}

\begin{document}

\maketitle

\begin{plm}[Bredon 2.2.1]
  Show that a second-countable Hausdorff space $X$ with functional structure $F$ is an $n$-manifold
  $\Leftrightarrow$ every point in $X$ has a neighborhood $U$ such that the functions $f_{1}, \ldots, f_{n} \in F(U)$
  such that: a real valued function $g$ on $U$ is in $F(U)$ iff there exists a smooth function $h(x_{1}, \ldots, x_{n})$
  of $n$ real variables satisfying $g(p) = f(h_{1}(p), \ldots, h_{n}(p))$ for all $p \in U$.
\end{plm}

\begin{plm}[Bredon 2.2.2]
  Complete the discussion of the two definitions of smooth manifolds by showing that if one goes from one of the descriptions to the other,
  as indicated, and then back, you end up with the same structure as at the start.
\end{plm}

\begin{plm}[Bredon 2.2.3]
  Show that a map $f: M \to N$ between smooth manifolds $M$ and $N$, with functional structures $F_{M}$ and $F_{N}$,
  is smooth in the sheaf-morphism sense iff it is smooth in the chart sense.
\end{plm}

\begin{plm}[Bredon 2.2.4]
  Let $X$ be the graph of the real-valued function $\theta(x) = |x|$ of a real variable $x$.
  Define a functional structure on $X$ by taking $f \in F(U)$ iff $f$ is the restriction to $U$ of a $C^{\infty}$ function
  on some open set $V$ in the plane with $U = V \cap X$.
  Show that $X$ with this functional structure is \textbf{not} diffeomorphic to the real line with the usual $C^{\infty}$ structure.
\end{plm}
\end{document}
