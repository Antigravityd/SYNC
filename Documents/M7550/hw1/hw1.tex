\documentclass{article}

\usepackage[letterpaper]{geometry}
\usepackage{tgpagella}
\usepackage{amsmath}
\usepackage{amssymb}
\usepackage{amsthm}
\usepackage{tikz}
% \usepackage{minted}
\usepackage{physics}
\usepackage{siunitx}

\sisetup{detect-all}
\newtheorem{plm}{Problem}
\newtheorem{lem}{Lemma}

\usetikzlibrary{cd}

\title{Math 7550 HW 1}
\author{Duncan Wilkie}
\date{}

\begin{document}

\maketitle

% \begin{plm}[Bredon 2.2.1]
%   Show that a second-countable Hausdorff space $X$ with functional structure $F$ is an $n$-manifold
%   $\Leftrightarrow$ every point in $X$ has a neighborhood $U$ such that there are functions $f_{1}, \ldots, f_{n} \in F(U)$
%   such that: a real valued function $g$ on $U$ is in $F(U)$ iff there exists a smooth function $h(x_{1}, \ldots, x_{n})$
%   of $n$ real variables satisfying $g(p) = h(f_{1}(p), \ldots, f_{n}(p))$ for all $p \in U$.
% \end{plm}

% \begin{proof}
%   First, the forward implication.
%   Suppose $X$ is a second-countable Hausdorff space with functional structure $F$ that makes it an $n$-manifold.
%   By definition, this means that each point has a neighborhood $U$ such that $(U, F_{U})$
%   is isomorphic to $(V, C^{\infty}(V)) \subseteq (\mathbb{R}^{n}, C^{\infty})$.
%   Such an isomorphism takes any map $F_{V} \ni f: V \to \mathbb{R}$ to $f \circ \phi: U \to \mathbb{R}$, which by definition is in $F_{U}$,
%   and vice versa ($\phi^{-1}(V) = U$ because isomorphism $\Rightarrow$ bijective in concrete categories).
%   Take the $f_{i}$ to be $\pi_{i} \circ \phi$, i.e. first map from $X$ into $\mathbb{R}^{n}$ by the sheaf isomorphism and project to the $i$th axis.
%   Suppose $g \in F(U)$.
%   Then $f \circ \phi^{-1}: V \to \mathbb{R}$ is a smooth function from $V \subseteq \mathbb{R}^{n}$ to $\mathbb{R}$,
%   and has that $h(f_{1}(p), \ldots, f_{n}(p)) = h(\pi_{1}(\phi(p)), \ldots, \pi_{n}(\phi(p)))$
%   % \[
%   %   \begin{tikzcd}
%   %     X \arrow[rd] \arrow[ld] \arrow[d] \\
%   %     A & A \times B & B
%   %   \end{tikzcd}
%   % \]
% \end{proof}


% \begin{plm}[Bredon 2.2.2]
%   Complete the discussion of the two definitions of smooth manifolds by showing that if one goes from one of the descriptions to the other,
%   as indicated, and then back, you end up with the same structure as at the start.
% \end{plm}

\begin{plm}[Bredon 2.2.3]
  Show that a map $f: M \to N$ between smooth manifolds $M$ and $N$, with functional structures $F_{M}$ and $F_{N}$,
  is smooth in the sheaf-morphism sense iff it is smooth in the atlas sense.
\end{plm}

\begin{proof}
  First, suppose the map $f: M \to N$ with functional structures $F_{M}$ and $F_{N}$ is smooth in the sense of inducing a morphism.
  In other words, for all $V \in \mathcal{O}(N)$ and $g \in F_{M}(V)$ one has $g \circ f \in F_{M}(f^{-1}(V))$.
  Since these are manifolds, there are isomorphisms of functional structure $\phi: M \to \mathbb{R}^{n}$ and $\psi: N \to \mathbb{R}^{n}$.
  By the equivalence of the definition of manifolds, these isomorphisms of functional structure correspond to charts
  when restricted to open sets in the manifolds.
  By the universal propery of the product, we can rewrite the map $\psi \circ f \circ \phi^{-1}$ in the form
  $([\pi_{1} \circ \psi] \circ f \circ \phi^{-1}, \ldots, [\pi_{n} \circ \psi]  \circ f \circ \phi^{-1})$.
  Projections are (linear and therefore) $C^{\infty}$, so since $\psi$ is a morphism the function in brackets is in $F_{N}$.
  Since $f$ is also a morphism, the function including one more composition is in $F_{M}$.
  Since $\phi^{-1}$ is yet again a morphism, each coordinate function is in the functional structure on $\mathbb{R}^{n}$, i.e. is $C^{\infty}$.
  A function is smooth on $\mathbb{R}^{n}$ iff each of its projections are smooth (i.e. all its partials exist)---so,
  $\psi \circ f \circ \phi$ is smooth in the atlas sense.

  Conversely, suppose $f : M \to N$ is smooth in the sense that for any charts $\phi, \psi$ of $U, V$ open in $M, N$, the map
  $\psi \circ f \circ \phi^{-1}: \phi(U) \to \psi(V)$ is $C^{\infty}$ where defined.
  Then $\phi$, $\psi$ induce isomorphisms $F_{M}(U)$ to $C^{\infty}(\phi(U))$ and $F_{N}(V)$ to $C^{\infty}(\psi(V))$
  by the equivalence of the two definitions of manifolds; if $g \in F_{N}(V)$, then by definition of the functional structure induced by an atlas
  $g = h \circ \psi$ for some $h \in C^{\infty}(\psi(V))$.
  This clearly implies that $g \circ f = h \circ \psi \circ f \in F_{M}(f^{-1}(V))$  (in turn, since $g$ was arbitrary, that $f$ is a morphism),
  because $h \circ \psi \circ f = a \circ \phi$ for some $a \in C^{\infty}(\phi(f^{-1}(V)))$
  can be written $h \circ [\psi \circ f \circ \phi^{-1}] \in C^{\infty}(\phi(f^{-1}(V)))$;
  this is immediate, since the bracketed quantity is smooth on all of $\phi(U)$ and $h$ is in $C^{\infty}(\psi(V))$.
\end{proof}

\begin{plm}[Bredon 2.2.4]
  Let $X$ be the graph of the real-valued function $\theta(x) = |x|$ of a real variable $x$.
  Define a functional structure on $X$ by taking $f \in F(U)$ iff $f$ is the restriction to $U$ of a $C^{\infty}$ function
  on some open set $V$ in the plane with $U = V \cap X$.
  Show that $X$ with this functional structure is \textbf{not} diffeomorphic to the real line with the usual $C^{\infty}$ structure.
\end{plm}

\begin{proof}
  Consider the open ball $V$ in the plane containing the origin, and the function $f$ on it mapping each point to its $y$-coordinate.
  This is clearly $C^{\infty}$---it's a projection.
  Therefore, it's restriction to $U = V \cap X$ is in $F(U)$ by definition.
  If some diffeomorphism $\phi: X \to \mathbb{R}$ existed, it must yield $f = g \circ \phi^{-1}$ for some $g \in C^{\infty}(\phi(U))$.
  However, both functions have signature $\mathbb{R} \to \mathbb{R}$---the first is not $C^{\infty}$,
  as it has a first derivative discontinuous at zero, so the other can't be $C^{\infty}$.
  Accordingly, no such diffeomorphism exists.
\end{proof}

\begin{plm}[Bredon 2.4.1]
  Consider the 3-sphere $S^{3}$ as the set of unit quaternions
  \[
    \{x + iy + jz + kw \mid x^{2} + y^{2} + z^{2} + w^{2} = 1\}.
  \]
  Let $\phi: U \to \mathbb{R}^{3}$  be $\phi(x + iy + jz + kw) = (y, z, w)$ where
  \[
    U = \{x + iy + jz + kw \in S^{3} \mid x  > 0\}.
  \]
  Consider $\phi$ as a chart.
  For each $q \in S^{3}$, define a chart $\psi_{q}$ by $\psi_{q}(p) = \phi(q^{-1}p)$.
  Show that this set of charts is an atlas for a smooth structure on $S^{3}$.
\end{plm}

\begin{proof}
  First, we must note that the topology on quaternions comes from the norm induced by the inner product $\langle p, q \rangle = pq^{*}$,
  where the conjugate is obtained by negating all three ``complex'' components of a quaternion.
  This agrees with the topology induced as a subspace of $\mathbb{R}^{4}$, because writing out the norm yields exactly that of Euclidean space;
  we may abusively regard sets and maps in the quaternions in either context, as convenient.
  Unit quaternion (left-)multiplication is then a homeomorphism: it's certainly an isometry,
  as for all $u \in S^{3}$ and $q \in \mathbb{H}$ one has
  \[
    \norm{uq} = \sqrt{uq(uq)^{*}} = \sqrt{uqq^{*}u^{*}} = \sqrt{u\norm{q}^{2}u^{*}} = \norm{u}\norm{q} = \norm{q}
  \]
  by the multiplicative property of conjugation and the fact that $S^{3}$ is defined by $\norm{u} = 1$;
  all isometries are homeomorphisms.
  $\phi$ is also a homeomorphism: it's the restriction of the projection map onto the $yzw$ hyperplane to $U$,
  and so continuous (even smooth, by linearity); invertibility comes from noting
  \[
    \phi^{-1}(y, z, w) = \sqrt{1 - y^{2} - z^{2} - w^{2}} + iy + jz + kw.
  \]
  This inverse is itself continuous (indeed, also smooth) when viewed as a map $\mathbb{R}^{3} \to \mathbb{R}^{4}$
  by the basic analyic theorems about those properties' stability under simple function-construction operations.
  Since unit quaternions are invertible, $\psi_{q}$ is the composition of homeomorphisms and so is a homeomorphism for every $q$.

  Clearly, every point $q$ is in the domain of the chart $\psi_{q}$; 1 is positive.

  Given two charts $\psi_{q}$, $\psi_{q'}$, the transation map is given by $T = \psi_{q} \circ \psi_{q'}^{-1}(x) = \phi(q^{-1}q'\phi^{-1}(x))$.
  Considering quaternions in terms of $\mathbb{R}^{4}$, unit quaternion multiplication fixes the origin, and so is an origin-preserving isometry.
  This implies it's linear, and, accordingly, smooth.
  The maps $\phi$ and $\phi^{-1}$ are as well, for the reasons noted above.
  Putting everything together, $T$ is the composition of a whole bunch of smooth functions, and so is smooth.
\end{proof}

% \begin{plm}[Bredon 2.4.2]
%   Let $X$ be a copy of the real line $\mathbb{R}$ and let $\phi: X \to \mathbb{R}$ be $\phi(x) = x^{3}$.
%   Taking $\phi$ as a chart, this defines a smooth structure on $X$.
%   Prove or disprove the following statements:
%   \begin{enumerate}
%   \item $X$ is diffeomorphic to $\mathbb{R}$;
%   \item the identity map $X \to \mathbb{R}$ is a diffeomorphism;
%   \item $\phi$ together with the identity map comprise an atlas;
%   \item on the one-point compactification $X^{+}$ of $X$,  $\phi$ and $\psi$ give an atlas, where $\psi(x) = \frac{1}{x}$,
%     for $x \neq 0, \infty$ and $\phi(\infty) = 0$.
%   \end{enumerate}
% \end{plm}


\end{document}
