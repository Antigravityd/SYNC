\documentclass{article}

\usepackage[letterpaper]{geometry}
\usepackage{tgpagella}
\usepackage{amsmath}
\usepackage{amssymb}
\usepackage{amsthm}
\usepackage{tikz}
\usepackage{minted}
\usepackage{physics}
\usepackage{siunitx}

\sisetup{detect-all}
\newtheorem{plm}{Problem}

\title{7550 HW 5}
\author{Duncan Wilkie}
\date{21 April 2023}

\begin{document}

\maketitle

\begin{plm}
  On an open set $U \subseteq \mathbb{R}^{n}$, show that the exterior derivative is the only operator
  $d: \Omega^{p}(U) \to \Omega^{p+1}(U)$ satisfying:
  \begin{enumerate}
  \item $d(\omega + \eta) = d\omega + d\eta$;
  \item $\omega \in \Omega^{p}(U), \eta \in \Omega^{q}(U) \Rightarrow d(\omega \wedge \eta) = d\omega \wedge \eta+(-1)^p\omega\wedge d\eta$;
  \item $f \in \Omega^{0}(U) \Rightarrow df(X) = X(f)$; and
  \item $f \in \Omega^{0}(U) \Rightarrow d(df) = 0$.
  \end{enumerate}
  Deduce that $d$ is independent of the coordinate system used to define it.
\end{plm}

\begin{proof}
  This smells a lot like it'd follow from uniqueness of some categorical construction, but I don't want to mess around with that.
  Accordingly, I'll have to do it the normal way.

  Suppose that  $d'$ satisfies each of the properties above.
  We merely must show that $d'$ satisfies the defining characteristic of $d$, namely, that if $X$ is a smooth vector field
  on $U$, that $d'f(X) = X(f)$, and, letting $\omega = fdx_{1} \wedge \cdots \wedge dx_{p}$,
  that $d'\omega = d'fdx_{1} \wedge \cdots \wedge x_{p}$ (which can be extended linearly to all forms).

  Let $X$ be a smooth vector field on $U$.
  In local coordinates, $X = \sum_{i = 1}^{n}a_{i}\pdv{x_{i}}$, where $a_{i}$ are smooth functions on $U$.
  Property 3 yields $d'f(X) = X(f)$, as zero-forms are exactly smooth functions.

  If $\omega = fdx_{1} \wedge \cdots \wedge dx_{p} = f \wedge dx_{1} \wedge \cdots \wedge dx_{p}$, then by property 2
  applied with $p = 0$ we can induct on $p$ (the index in the wedges).
  For $p = 0$, it's immediate that the pure zero-form $\omega = f$ has $d'\omega = d'f$.
  Suppose that the property holds for all pure wedges with $p - 1$ terms.
  Then
  \[
    d'\omega = d'f \wedge dx_{1} \wedge \cdots \wedge dx_{p} + (-1)^{0}f \wedge d'(dx_{1} \wedge \cdots \wedge dx_{p})
  \]
  \[
    = d'f \wedge dx_{1} \wedge \cdots \wedge dx_{p} + (-1)^{0}f \wedge \qty(\sum_{i}dx_{1} \wedge \cdots d'dx_{i} \wedge \cdots \wedge dx_{p})
  \]
  By property 2, which is $d'f = \sum_{i}\pdv{f}{x_{i}}dx_{i}$ in local coordinates, we can take $f = x_{i}$ (the coordinate function) to get
  \[
    d'x_{i} = \sum_{j}\pdv{x_{i}}{x_{j}}dx_{j} = dx_{i}.
  \]
  Accordingly, $dx_{i} = d'(x_{i})$, so we can apply property 4 to get $d'dx_{i} = 0$---all the terms on the right vanish,
  and we get what we want:
  \[
    d'\omega = d'f \wedge dx_{1} \wedge \cdots \wedge dx_{p}.
  \]

  By assumption in property 1, $d'$ is additive, but is it linear?
  Well, let $c \in \Omega^{0}(U)$ be a constant and $\eta$ be a pure form of the form of $\omega$ above.
  By property 2 and the definition of the wedge product and scalar product in the exterior algebra,
  \[
    d'(c\eta) = d'(c \wedge \eta) = d'c \wedge \eta + (-1)^{0}c \wedge d'\eta
  \]
  Trivially by the property proven above, $d'c = 0$---so $d'(c\eta) = cd'\eta$.
  This means that $d'$ is indeed linear on forms that look like $\omega$, and so can be honest-to-goodnes linearly extended to all forms.

  In any particular coordinate system, the text proves that the $d$ defined by that coordinate system has all four properties.
  Accordingly, since none of these properties (including the forms themselves) are coordinate-dependent,
  any two coordinate systems must define the same $d$ operator.
\end{proof}

Let $G$ be a Lie group, and $g \in G$.
Recall that left-translation by $g$, $L_{g}: G \to G$, is given by $L_{g}(h) = gh$,
and recall also the definition and importance of left-invariant vector fields.
A differential form $\omega$ on $G$ is left-invariant if $L_{g}^{*}\omega = \omega$ for each $g \in G$.
Let $E^{p}(G)$ denote the vector space of left-invariant $p$-forms on $G$, and $E^{*}(G) = \bigoplus_{p = 0}^{\dim G}E^{p}(G)$.
Here are some of their properties to establish.
Several of them are analogs of properties of left-invariant vector fields we've seen.

% \begin{plm}
%   Left-invariant forms are smooth.
% \end{plm}

% \begin{proof}
%   This is done by induction on arity of forms.
%   One-forms are elements of $(T_{q}G)^{*}$---
%   We must show that any left-invariant form, when expressed in local coordinates near $q$ as
%   \[
%     \omega_{q} = \sum f_{i_{1, \ldots, i_{p}}}dx_{i_{1}} \wedge \cdots \wedge dx_{i_{p}},
%   \]
%   has $f_{i_{1}, \ldots, i_{p}}$ smooth.

%   A general $p$-form $\omega$ presented as above acts on $p$ tangent vectors $X_{1}, \ldots, X_{p}$ at $q$---these
%   in turn act on smooth functions $f$ on $G$.
%   Left-invariance expresses
%   \[
%     \omega_{q}(X_{1}, \ldots, X_{p}) = L_{g}^{*}(\omega_{q}(X_{1}(f), \ldots, X_{p}(f)))
%     = \omega_{q}(X_{1}(f \circ L_{g}), \ldots, X_{p}(f \circ L_{g}))
%     = \omega_{q}.
%   \]
%   In local coordinates $x_{i}$ near $q$, the vector fields can be written $X_{i} = \sum_{j}a_{ij}\pdv{x_{j}}$,
%   and the leftmost expression above becomes
%   \[
%     \sum_{i,j}a_{ij}dx_{1} \wedge \cdots \wedge dx_{p}.
%   \]
%   This, by smoothness of Lie group multiplication, $f$, scalar multiplication, and pointwise addition, is a smooth function,
%   and so also the left side of the
%   If $g$ is chosen to be close to the identity (i.e. such that $gq$ is in the neighborhood where the local coordinates $x_{i}$ apply),
%   then we can express the above equality as
%   \[
%     \sum f_{i_{1, \ldots, i_{p}}}dx_{i_{1}} \wedge \cdots \wedge dx_{i_{p}}
%     = \sum f_{i_{1, \ldots, i_{p}}} \circ L_{g}dx_{i_{1}} \wedge \cdots \wedge dx_{i_{p}}.
%   \]
%   Since $dx_{i_{j}}$ is a basis for the exterior algebra(s near $q$),

%   However, $gq$ is a point in $G$---give it local coordinates $y_{1}, \ldots y_{n}$.
%   With respect to these,
%   \[
%     \omega_{q} = \omega_{gq} = \sum g_{i_{1}, \ldots, i_{p}} dy_{i_{1}} \wedge \cdots \wedge dy_{i_{p}};
%   \]

% \end{proof}

\begin{plm}
  $E^{*}(G)$ is a subalgebra of the algebra $\Omega^{*}(G)$ of all smooth differential forms on $G$.
  If $e$ denotes the identity element of $G$, the map $\omega \mapsto \omega_{e}$ is an algebra isomorphism of $E^{*}(G)$
  and the exterior algebra $\Lambda((T_{e}G)^{*})$.
  Note that this map gives an isomorphism of $E^{1}(G)$ with $(T_{e}G)^{*}$, that is, with the dual space of the Lie algebra $\mathfrak{g}$
  of $G$.
\end{plm}

\begin{proof}
  First, the subalgebra condition.
  Clearly, the zero-form is left-invariant.
  Let $\omega, \nu$ be left-invariant $p$-forms (smooth functions).
  Letting these act on a vector fields $X_{1}, \ldots, X_{p}$, which in turn act on a smooth function $f$,
  \[
    L_{g}^{*}(\omega(X_{1}(f), \ldots, X_{p}(f)) + \nu(X_{1}(f), \ldots, X_{p}(f)))
  \]
  \[
    = \omega(X_{1}(f \circ L_{g}), \ldots, X_{p}(f \circ L_{g})) + \nu(X_{1}(f \circ L_{g}), \ldots, X_{p}(f \circ L_{g}))
  \]
  \[
    = L_{g}^{*}(\omega(X_{1}(f), \ldots, X_{p}(f))) + L_{g}^{*}(\nu(X_{1}(f), \ldots, X_{p}(f)))
  \]
  \[
    = \omega(X_{1}(f), \ldots, X_{p}(f)) + \nu(X_{1}(f), \ldots, X_{p}(f)),
  \]
  and, if $k$ is a constant,
  \[
    L_{g}^{*}(k\omega(X_{1}(f), \ldots, X_{p}(f))
    = k\omega(X_{1}(f \circ L_{g}), \ldots, X_{p}(f \circ L_{g}))
    = k\qty(L_{g}^{*}(\omega(X_{1}(f), \ldots, X_{p}(f))));
  \]
  \[
    = k\omega(X_{1}(f), \ldots, X_{p}(f))
  \]
  This shows that $E^{*}(G)$ is a vector subspace.

  Further, if $\omega, \nu$ are as above,
  then
  \[
    L_{g}^{*}(\omega(X_{1}(f), \ldots, X_{p}(f)) \wedge \nu(X_{1}(f), \ldots, X_{p}(f)))
  \]
  \[
    = \omega(X_{1}(f \circ L_{g}), \ldots, X_{p}(f \circ L_{g})) \wedge \nu(X_{1}(f \circ L_{g}), \ldots, X_{p}(f \circ L_{g}))
  \]
  \[
    = L_{g}^{*}(\omega(X_{1}(f), \ldots, X_{p}(f))) \wedge L_{g}^{*}(\nu(X_{1}(f), \ldots, X_{p}(f)))
  \]
  \[
    = \omega(X_{1}(f), \ldots, X_{p}(f)) \wedge (X_{1}(f), \ldots, X_{p}(f)),
  \]
  so it is also a subalgebra.
  Note that all of these proofs are by way of showing that $L_{g}^{*}$ has a homomorphism property with respect to each operation checked,
  and accordingly is an algebra homomorphism.


  Since (the operations on) global forms are defined by gluing together forms at points,
  the map restricting to the identity is trivially a homomorphism.
  However, for all left-invariant $\omega$, the value $\omega_{g}$ is the same as $L_{g}^{*}\omega_{e}$,
  i.e. $\omega$ is determined solely by its value at $\omega_{e}$; this means that there exists a map from $\nu_{e}$ to $\nu$
  for all forms at the identity $\nu$.
  This is, in particular, given pointwise by $\nu_{g} = L_{g}^{*}\nu_{e}$; as $L_{g}^{*}$ has the homomorphism property,
  this inverse is a homomorphism also.

\end{proof}

\begin{plm}
  If $\omega$ is a left-invariant form and $X$ is a left-invariant vector field, then $\omega(X)$ is a constant function on $G$.
\end{plm}

\begin{proof}
  At a point $g$,
  \[
    \omega_{g}(X_{g}) = L_{g}^{*}\omega_{e}(L_{g}^{*}X_{e}) = \omega_{e}(X_{e}),
  \]
  i.e. for every $g$, $\omega(X)$ takes on the same value.
\end{proof}

\begin{plm}
  Let $\{X_{1}, \ldots, X_{n}\}$ and $\{\omega_{1}, \ldots, \omega_{n}\}$ be dual bases for $\mathfrak{g}$ and $E^{1}(G)$.
  Then there are constants $c_{ijk}$ so that $[X_{i}, X_{j}] = \sum_{k = 1}^{n}c_{ijk}X_{k}$.
  These \textit{structure constants} of $G$ with respect to the specified basis of $\mathfrak{g}$ satisfy $c_{ijk} + c_{jik} = 0$
  and $\sum_{r}(c_{ijr}c_{rks} + c_{jkr}c_{ris} + c_{kir}c_{rjs}) = 0$.
  Use the invariant formula for the exterior derivative to show that the exterior derivatives of the form $\omega_{i}$
  are given by the \textit{Maurer-Cartan} equations
  \[
    d\omega_{i} = \sum_{j < k}c_{jki}\omega_{k} \wedge \omega_{j}.
  \]
\end{plm}

\begin{proof} % TODO
  Applying that invariant form,
  \[
    (d\omega_{i})(X_{m}, X_{n}) = X_{m}(\omega_{i}(X_{n})) - X_{n}(\omega_{i}(X_{m})) - \omega_{i}([X_{m}, X_{n}]).
  \]
  \[
    = X_{m}(\omega_{i}(X_{n})) - X_{n}(\omega_{i}(X_{m})) - \omega_{i}\qty(\sum_{l = 1}^{n}c_{mnl}X_{l})
    = X_{m}(\omega_{i}(X_{n})) - X_{n}(\omega_{i}(X_{m})) - \sum_{l = 1}^{n}c_{mnl}\omega_{i}\qty(X_{l}).
  \]
  On the other hand,
  \[
    \qty(\sum_{j < k}c_{jki}\omega_{k} \wedge \omega_{j})(X_{m}, X_{n})
    = \sum_{j < k}c_{jki}\qty(\omega_{k}(X_{m})\omega_{j}(X_{n}) - \omega_{k}(X_{n})\omega_{j}(X_{m})))
    = \sum_{}
  \]
  Since $\omega_{i}$ is left-invariant, % TODO: finsih
\end{proof}

\begin{plm}
  Show that a Lie group $G$ is orientable. Hint: can you use (a basis for) $E^{1}(G)$ to produce a nowhere-vanishing $n$-form,
  where $n = \dim G$?
\end{plm}

\begin{proof}
  Let $\omega_{i}$ be a basis for $E^{1}(G)$.
  The isomorphism in Problem 3 yields, when restricted to the 1-forms, $E^{1}(G) \cong \Lambda^{1}(T_{e}^{*}G)$;
  the cotangent space $T_{e}^{*}G$ has the same dimension as $T_{e}G$, i.e. $n$---and
  the dimension of the $p$th grading of the exterior algebra is $\binom{n}{p}$, implying this basis has $\binom{n}{1} = n$ elements.
  We can choose the form given by $\nu = \omega_{1} \wedge \cdots \wedge \omega_{n}$.
  If there were some $g$ such that $\nu_{g} = 0$, then by left-invariance $\nu_{e} = 0$.
  However, this means that for all $X_{1}, \ldots, X_{n}$
  \[
    \omega_{1} \wedge \cdots \wedge \omega_{n}(X_{1}, \ldots, X_{n}) = \det[\omega_{i}(X_{j})] = 0;
  \]

  Choosing a basis for the tangent space, this expresses a linear dependence between $\omega_{i}$, contradicting that they're a basis,
  so this form must not vanish.
\end{proof}

\end{document}
