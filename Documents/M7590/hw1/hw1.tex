\documentclass{article}

\usepackage[letterpaper]{geometry}
\usepackage{amsmath}
\usepackage{amssymb}

\title{7590 HW 1}
\author{Duncan Wilkie}
\date{?}

\begin{document}

\maketitle

\section*{1a}
By the definition of the $L^2$ inner product and $A$, for any functions $f,g\in D(A)$ we have
\[\langle Af|g \rangle=\langle f|Ag \rangle\Leftrightarrow \int_0^1f''(x)g(x)dx=\int_0^1f(x)g''(x)dx \]
Integrating by parts,
\[f'g\bigg|_0^1-\int_0^1f'(x)g'(x)dx=\int_0^1f(x)g''(x)dx\]
\[\Leftrightarrow f'g\bigg|_0^1-fg'\bigg|_0^1+\int_0^1f(x)g''(x)dx=\int_0^1f(x)g''(x)dx\]
the evaluation terms  must both be zero at $0$ and $1$ since smooth compactly-supported functions on open sets vanish
in the limit to the boundary of their domains. % TODO: proof?
Therefore, this operator is symmetric.
However, not all elements of $D(A^{\dagger})$ are elements of $D(A)$: $g\in H$ is an element of $D(A^{\dagger})$ iff there exists
$h\in H$ such that $\forall f\in D(A)$
\[
  \int_{0}^{1}f''(x)g(x)dx=\int_{0}^{1}f(x)h(x)dx
\]
Applying the same integration-by-parts argument as above, we may equivalently write this as
\[\Leftrightarrow f'g\bigg|_0^1-fg'\bigg|_0^1+\int_0^1f(x)g''(x)dx=\int_0^1f(x)h(x)dx\]
Since $f$ is compactly supported, $f'$ is as well, so the evaluation terms are zero by the same argument given above.
Letting $g=x^{2}$, we then have
\[\int_{0}^{1}f(x)\cdot 2dx=\int_{0}^{1}f(x)h(x)dx\]
from which we can clearly see the $L^{2}([0,1])$ function $h=2$ is the element adjoint to $g$ with respect to $A$.
$g$ is therefore in $D(A^{\dagger})$.
It isn't in $D(A)$ though, since $x^{2}$ doesn't vanish at 1 and therefore isn't compactly supported on this interval.
This implies $D(A^{\dagger})\neq D(A)$, so $A\neq A^{\dagger}$, i.e. $A$ isn't self-adjoint.

\section*{1b}
Proceeding similarly,
\[
  \langle Af|g \rangle=\langle f|Ag \rangle
  \Leftrightarrow \int_{0}^{1}(if'(x))^{*}g(x)dx=\int_{0}^{1}(f(x))^{*}ig'(x)dx
\]
\[
  \Leftrightarrow -if^{*}g\bigg|_{0}^{1}+\int_{0}^{1}i(f(x))^{*}g'(x)dx=\int_{0}^{1}(f(x))^{*}ig'(x)dx
\]
By the same argument as above, the evaluation term is zero, in which case the equality follows immediately.
This operator is symmetric.
Once again, $x^{2}$ is in $D^{\dagger}(A)$ but not $D(A)$: from the formula derived for $\langle Af|g \rangle$ in the proof $A$ is symmetric,
the definition of membership in $D^{\dagger}(A)$ is
\[\int_{0}^{1}(f(x))^{*}2xdx=\int_{0}^{1}(f(x))^{*}h(x)dx\]
which, choosing $h=2x\in L^{2}([0,1])$, clearly holds.
$2x$ isn't compactly supported on $(0,1)$ since it doesn't vanish in the limit to 1, so $D(A^{\dagger})\neq D(A)$ and $A$ isn't self-adjoint.

\section*{1c}
One may write $\partial_{i}(a_{ij}(x)\partial_{j}f)=(\partial_{i}a_{ij})\partial_{j}f+a_{ij}\partial_{i}\partial_{j}f$.
Applying the definition,
\[
  \langle Af|g \rangle=\langle f|Ag \rangle
  \Leftrightarrow \int_{\Omega}[\partial_{i}a_{ij}(x)][\partial_{j}f(x)]g(x)dx+\int_{\Omega}a_{ij}(x)[\partial_{i}\partial_{j}f(x)]g(x)dx
  =\int_{\Omega}f(x)\partial
\]
% TODO: get better at vector calc

\section*{2a}

\end{document}
%%% Local Variables:
%%% mode: latex
%%% TeX-master: t
%%% End:
