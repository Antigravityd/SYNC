\documentclass{article}
\usepackage[letterpaper]{geometry}
\usepackage{amsmath}
\usepackage{amsfonts}

\title{2231 HW 4}
\author{Duncan Wilkie}
\date{1 October 2021}

\begin{document}

\maketitle

\section*{1a}
The field at the boundary of a Gaussian surface concentric with the cavity entirely contained within the conducting material must be zero.
Therefore, its integral over the whole surface is zero, and by Gauss's law the enclosed charge must be (net) zero, and subsequently $\sigma_a = -\frac{q_a}{4\pi a^2}$ and $\sigma_b=-\frac{q_b}{4\pi b^2}$.
Since the sphere is neutrally charged, the outer surface must have a charge equal to the negative of the total charge on the inner ones, so $\sigma_R=\frac{q_a+q_b}{4\pi R^2}$.

\section*{1b}
Gauss's law is again applicable: taking the surface to be a concentric sphere of radius $r$, the field is perpendicular and of constant magnitude, so
\[\int_SEdA=4\pi r^2|E|=\frac{q_{enc}}{\epsilon_0}\Leftrightarrow |E|=\frac{q_a+q_b}{4\pi\epsilon_0r^2}\]
pointing radially away from the sphere's center.

\section*{1c}
Within each cavity, the enclosed charge is simply the respective point charge, so by the same reasoning
\[|E|=\frac{q_a}{4\pi\epsilon_0r^2}\]
and
\[|E|=\frac{q_b}{4\pi\epsilon_0r^2}\]
The directions are of course radially outward from the centers of the cavities.

\section*{1d}
The point charges feel no force, since the electric fields within the cavities are identical to that produced if each were in free space.
Since there is no self-interaction of such a field in the case of free space, it is unreasonable to presume there would be such an interaction in this one.

\section*{1e}
The field outside the conductor would change, since the field would no longer be orthogonal to a concentric spherical Gaussian surface.
Additionally, the surface charge density on the outside of the sphere is no longer a constant function of position on the sphere, since that lack of orthogonality in the field at the conductor's boundary would induce lateral movement of the surface charges.
The arguments for the other answers depend only on the geometry of the conductor in relation to the point charges' position and the field inside conductors being zero; this doesn't change with the introduction of an external charge.

\section*{2}
The electric field due to an infinitely long cylinder is, using a concentric Gaussian cylinder, is $2\pi rLE=\frac{\lambda L}{\epsilon_0}\Leftrightarrow E=\frac{\lambda}{2\pi\epsilon_0r}$. In this case, $\lambda=\sigma 2\pi R$ where $R$ is the radius of the cylinder in question. The odd notation is due to me doing it wrong the first time, and not wanting to change a billion variables for some constants that cancel out in the end.
In the case of two concentric cylinders, the presence of the second doesn't affect the direction of the field or its being constant, so Gauss's law still holds, giving a compound field by superpostion of
\[E=\begin{cases}
    0, r < a \\
    \frac{\lambda_1}{2\pi\epsilon_0 r}, a<r<b \\
    \frac{\lambda_2+\lambda_1}{2\pi\epsilon_0 r}, r>b \\
  \end{cases}\]
Integrating this from $a$ to $b$ yields a plate voltage of
\[V=\int_a^b\frac{\lambda_1}{2\pi\epsilon_0 r}dr=\frac{\lambda_1}{2\pi\epsilon_0}\ln(\frac{b}{a})\]
The charge on the tubes being $q=\lambda_1 L$, the capacitance is
\[C=\frac{q}{V}=\frac{\lambda_1 L}{\lambda_1\ln(b/a)/(2\pi\epsilon_0)}=\frac{2\pi\epsilon_0 L}{\ln(b/a)}\]

\section*{3a}
The electric field for a line charge may be found via a method identical to the one above for cylindrical charges, but where $\lambda$ actually does represent a linear charge density.
Integrating the result from the distance to the grounded plate yields a potential
\[V=-\int_d^r\frac{\lambda}{2\pi\epsilon_0 r}dr=-\frac{\lambda}{2\pi\epsilon_0}\ln(\frac{r}{d})\]
for all points distance $R$ from the line charge.
\section*{3b}
Since the field is the same for any values of the $x$ variable due to its dependence on distance from the line charge alone, we have
\[E=\frac{\sigma}{\epsilon_0}\hat{n}\Leftrightarrow{\hat{n}}\cdot E=\frac{\sigma}{\epsilon_0}\Leftrightarrow \sigma = \frac{\lambda}{2\pi r}\hat{r}\cdot\hat{n}=\frac{\lambda}{2\pi \sqrt{d^2+y^2}}\hat{r}\cdot \hat{n}\]
Using that by simple geometric inspection $\hat{r}\cdot\hat{n}=\cos(\cos^{-1}(\frac{r}{d}))=\sqrt{d^2+y^2}/d$, the charge density is $\sigma=\frac{\lambda}{2\pi d}$.
\section*{4}
The part of the Laplace operator in spherical coordinates not made zero by derivatives with respect to angular variables is $\frac{1}{r}\frac{\partial^2}{\partial r^2}r$.
Applying this to a function and integrating,
\[\Delta f=0\Leftrightarrow \frac{\partial^2}{\partial r^2}(rf)=0\Leftrightarrow rf = c_1r+c_2\Leftrightarrow f=c_1+\frac{c_2}{r}\]
\end{document}
%%% Local Variables:
%%% mode: latex
%%% TeX-master: t
%%% End:
