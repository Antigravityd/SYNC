\documentclass{article}
\usepackage[letterpaper]{geometry}
\usepackage{amsmath}
\title{4123 HW 2}
\author{Duncan Wilkie}
\date{5 October 2021}

\begin{document}

\maketitle

\section*{1a}
From the reference frame of the attatchment point, we take $\theta$ to be the angle between the negative vertical axis and the first bob and $\phi$ to be the angle between the same axis and the second. The change of variables is $x_1=l_1\sin(\theta)$, $y_1=l_1-l_1\cos(\theta)$, $x_2=x_1+l_2\sin(\phi)=l_1\sin(\theta)+l_2\sin(\phi)$, and $y_2=y_1+l_2+l_2\cos(\phi)=l_1(1-\cos(\theta)+l_2(1-\cos(\phi))$. The potential as a function of these varaibles, measured of course with respect to the attatchment point, is therefore
\[U=-m_1gy_1-m_2g(y_1+y_2)=-(m_1+m_2)gl_1(1-\cos(\theta))-m_2gl_2(1-\cos(\phi))\]
and the kiniteic energy is \[T=\frac{1}{2}m_1v_1^2+\frac{1}{2}m_2v_2^2=\frac{1}{2}m_1(\dot{x_1}^2+\dot{y_1}^2)+\frac{1}{2}m_2(\dot{x_2}+\dot{y_2}^2)\]
\[=\frac{1}{2}(m_1+m_2)l_1^2\dot{\theta}^2+m_2l_1l_2\dot{\theta}\dot{\phi}\cos(\theta-\phi)+l_2^2\dot{\phi}^2\]
The overall Lagrangian is therefore
\[L=T-U=\frac{1}{2}(m_1+m_2)l_1^2\dot{\theta}^2+m_2l_1l_2\dot{\theta}\dot{\phi}\cos(\theta-\phi)+\frac{1}{2}m_2l_2^2\dot{\phi}^2+(m_1+m_2)gl_1(1-\cos(\theta))+m_2gl_2(1-\cos(\phi))\]
Applying the Euler-Lagrange equation,
\[\frac{\partial L}{\partial \theta}=\frac{d}{dt}\frac{\partial L}{\partial\dot{\theta}}\Leftrightarrow -m_2l_1l_2\dot{\phi}\dot{\theta}\sin(\theta-\phi)+(m_1+m_2)gl_1\cos(\theta)\]
\[=\frac{d}{dt}[(m_1+m_2)l_1^2\dot{\theta}+m_2l_1l_2\dot{\phi}\cos(\theta-\phi)]=(m_1+m_2)l_1^2\ddot{\theta}+m_2l_1l_2\dot{\phi}[\sin(\theta-\phi)\dot{\phi}-\sin(\theta-\phi)\dot{\theta}]+m_2l_1l_2\ddot{\phi}\cos(\theta-\phi)\]
\[\Leftrightarrow -m_2l_1l_2\dot{\theta}\dot{\phi}\sin(\theta-\phi)+(m_1+m_2)gl_1\cos(\theta)=(m_1+m_2)l_1^2\ddot{\theta}+m_2l_1l_2(\dot{\phi}[\sin(\theta-\phi)\dot{\phi}-\sin(\theta-\phi)\dot{\theta}]+\ddot{\phi}\cos(\theta-\phi))\]
and
\[\frac{\partial L}{\partial \phi}=\frac{d}{dt}\frac{\partial L}{\partial \dot{\phi}}\Leftrightarrow m_2l_1l_2\dot{\theta}\dot{\phi}\sin(\theta-\phi)+m_2gl_2\sin(\phi)\]
\[=\frac{d}{dt}[m_2l_1l_2\dot{\theta}\cos(\theta-\phi)+m_2l_2^2\dot{\phi}=m_2l_1l_2(\ddot{\theta}\cos(\theta-\phi)+\dot{\theta}[\dot{\phi}\sin(\theta-\phi)-\dot{\theta}\sin(\theta-\phi)])+m_2l_2^2\ddot{\phi}\]
\[\Leftrightarrow m_2l_1l_2\dot{\theta}\dot{\phi}\sin(\theta-\phi)+m_2gl_2\sin(\phi)=m_2l_1l_2(\ddot{\theta}\cos(\theta-\phi)+\dot{\theta}[\dot{\phi}\sin(\theta-\phi)-\dot{\theta}\sin(\theta-\phi)])+m_2l_2^2\ddot{\phi}\]
\section*{1b}
Applying the small-angle approximation to the Lagrangian, where $\cos(x)\approx 1-x^2/2$. We also approximate $\dot{\theta}\dot{\phi}\cos(\theta-\phi)$ as $\dot{\theta}\dot{\phi}$, since the amplitude factor should remain small when the oscillations are small, and so the lower-order approximation isn't terrible. This yields, in terms of the new constants,
\[L=ml^2\dot{\theta}^2+ml^2\dot{\theta}\dot{\phi}+ml^2\dot{\phi^2}-mgl\theta^2-mgl\phi^2/2\]
Applying the Euler-Lagrange equation,
\[\frac{\partial L}{\partial \theta}=\frac{d}{dt}\frac{\partial L}{\partial\dot{\theta}}\Leftrightarrow -2mgl\theta=\frac{d}{dt}[2ml^2\dot{\theta}+ml^2\dot{\phi}]=2ml^2\ddot{\theta}+ml^2\ddot{\phi}\]
\[\Leftrightarrow -\frac{2g\theta}{l}=2\ddot{\theta}+\ddot{\phi}\]
and
\[\frac{\partial L}{\partial \phi}=\frac{d}{dt}\frac{\partial L}{\partial \dot{\phi}}\Leftrightarrow -mgl\phi=\frac{d}{dt}[ml^2\dot{\theta}+2ml^2\dot{\phi}]=ml^2\ddot{\theta}+2ml^2\ddot{\phi}\]
\[\Leftrightarrow-\frac{g\phi}{l}=\ddot{\theta}+2\ddot{\phi}\]
As a matrix equation, this system can be written
\[A\ddot{x}=-Kx\]
where $A=\begin{bmatrix}
  2 & 1 \\
  1 & 2
\end{bmatrix}$, $K=\begin{bmatrix}
  \frac{2g}{l} & 0\\
  0 & \frac{g}{l}
\end{bmatrix}$, $\ddot{x}=\begin{bmatrix}
  \ddot{\theta} \\ \ddot{\phi}
\end{bmatrix}$, and $x=\begin{bmatrix}
  \theta \\ \phi
\end{bmatrix}$.
The oscillatory solutions for this system will have freqency $f$ and initial condtions $a$ satisfying \[(K-f^2A)a=0\Leftrightarrow 2a_1g/l-2a_1f^2=0, 2a_2g/l-2a_2f^2=0\Leftrightarrow f=\sqrt{\frac{g}{l}}\]
and all allowed frequencies will be multiples of this.

\section*{2}
With $L=\frac{1}{2}m(\dot{x}^2+\dot{y}^2+\dot{z}^2)-q\phi+q(A_x\dot{x}+A_y\dot{y}+A_z\dot{z})$,
\[\frac{\partial L}{\partial x}=\frac{d}{dt}\frac{\partial L}{\partial\dot{x}}\Leftrightarrow-q\frac{\partial \phi}{\partial x}+q\left( \frac{\partial A_x}{\partial x}\dot{x}+\frac{\partial A_y}{\partial x}\dot{y}+\frac{\partial A_z}{\partial x}\dot{z} \right)=\frac{d}{dt}\left[  m\dot{x}+qA_x\right]=m\ddot{x}+q\frac{d A_x}{d t}\]
\[=m\ddot{x}+q\left( \frac{\partial A_x}{\partial t}+\frac{\partial A_x}{\partial x}\dot{x}+\frac{\partial A_x}{\partial y}\dot{y}+\frac{\partial A_x}{\partial z}\dot{z} \right)\]
\[\Leftrightarrow m\ddot{x}=-q\frac{\partial\phi}{\partial x}-q\frac{\partial A_x}{\partial t}+q\left[ \dot{y}\left(  \frac{\partial A_y}{\partial x}-\frac{\partial A_x}{\partial y}\right)+\dot{z}\left( \frac{\partial A_z}{\partial x}-\frac{\partial A_x}{\partial z} \right) \right]\]
\[\frac{\partial L}{\partial y}=\frac{d}{dt}\frac{\partial L}{\partial \dot{y}}\Leftrightarrow-q\frac{\partial \phi}{\partial y}+q\left( \frac{\partial A_x}{\partial y}\dot{x}+\frac{\partial A_y}{\partial y}\dot{y}+\frac{\partial A_z}{\partial y}\dot{z} \right)=\frac{d}{dt}[m\dot{y}+qA_y]=m\ddot{y}+q\frac{d A_y}{d t}\]
\[=m\ddot{y}+q\left( \frac{\partial A_x}{\partial t}+\frac{\partial A_y}{\partial x}\dot{x}+\frac{\partial A_y}{\partial y}\dot{y}+\frac{\partial A_y}{\partial z}\dot{z} \right)\]
\[\Leftrightarrow m\ddot{y}=-q\frac{\partial\phi}{\partial y}-q\frac{\partial A_y}{\partial t} + q\left[\dot{x}\left( \frac{\partial A_x}{\partial y}-\frac{\partial A_y}{\partial x} \right)+\dot{z}\left( \frac{\partial A_z}{\partial y}-\frac{\partial A_y}{\partial z} \right)  \right]\]
\[\frac{\partial L}{\partial z}=\frac{d}{dt}\frac{\partial L}{\partial\dot{z}}\Leftrightarrow-q\frac{\partial\phi}{\partial z}+q\left( \frac{\partial A_x}{\partial z}\dot{x}+\frac{\partial A_y}{\partial z}\dot{y}+\frac{\partial A_z}{\partial z}\dot{z} \right)=\frac{d}{dt}[m\dot{z}+qA_z]=m\ddot{z}+q\frac{d A_z}{d t}\]
\[=m\ddot{z}+q\left( \frac{\partial A_z}{\partial t}+\frac{\partial A_z}{\partial x}\dot{x}+\frac{\partial A_z}{\partial y}\dot{y}+\frac{\partial A_z}{\partial z}\dot{z} \right)\]
\[\Leftrightarrow m\ddot{z}=-q\frac{\partial \phi}{\partial z}-q\frac{\partial A_z}{\partial t}+q\left[ \dot{x}\left( \frac{\partial A_x}{\partial z}-\frac{\partial A_z}{\partial x} \right)+\left( \frac{\partial A_y}{\partial z}-\frac{\partial A_z}{\partial y} \right) \right]\]
The Lorentz force is given by $\vec{F}=qE+qv\times B$. Writing the fields in terms of the corresponding potentials,
$F=-q\nabla\phi+q\frac{\partial A}{\partial t}+qv\times(\nabla\times A)$.
We write out
\[v\times(\nabla\times\vec{A})=v\times\left[\left(\frac{\partial A_z}{\partial y}-\frac{\partial A_y}{\partial z}\right)\hat{i}+\left(\frac{\partial A_x}{\partial z}-\frac{\partial A_z}{\partial x}\right)\hat{j}+\left(\frac{\partial A_y}{\partial x}-\frac{\partial A_x}{\partial y}\right)\hat{k}\right]\]
\[=\left[ v_y\left(\frac{\partial A_y}{\partial x}-\frac{\partial A_x}{\partial y} \right)-v_z\left( \frac{\partial A_x}{\partial z}-\frac{\partial A_z}{\partial x} \right)\right]\hat{i}-\left[ v_x\left( \frac{\partial A_y}{\partial x}-\frac{\partial A_x}{\partial y}\right)-v_z\left(\frac{\partial A_z}{\partial y}-\frac{\partial A_y}{\partial z}\right) \right]\hat{j}\]
\[+\left[ v_x\left(  \frac{\partial A_x}{\partial z}-\frac{\partial A_z}{\partial x}\right)-v_y\left(\frac{\partial A_z}{\partial y}-\frac{\partial A_y}{\partial z}  \right) \right]\hat{k}\]
Equating the components of this vector equation,
\[F_x=q\frac{\partial \phi}{\partial x}-q\frac{\partial A_x}{\partial t}+q\left[ v_y\left( \frac{\partial A_y}{\partial x}-\frac{\partial A_x}{\partial y} \right)-v_z\left( \frac{\partial A_x}{\partial z}-\frac{\partial A_Z}{\partial x} \right) \right]\]
\[F_y=q\frac{\partial \phi}{\partial y}-q\frac{\partial A_y}{\partial t}+q\left[ v_z\left( \frac{\partial A_z}{\partial y}-\frac{\partial A_y}{\partial z} \right)-v_x\left( \frac{\partial A_y}{\partial x}-\frac{\partial A_x}{\partial y} \right) \right]\]
\[F_z=q\frac{\partial \phi}{\partial z}-q\frac{\partial A_z}{\partial t}+q\left[ v_x\left( \frac{\partial A_x}{\partial z}-\frac{\partial A_z}{\partial x} \right)-v_y\left( \frac{\partial A_z}{\partial y}-\frac{\partial A_y}{\partial z} \right) \right]\]
These are exactly the values obtained for $F_i=m\ddot{x}_i$ in the application of the Euler-Lagrange equations to the target Lagrangian, it is the correct one.
\section*{3}
The Lagrangian is $L=T-V=\frac{1}{2}mv^2-mgy=\frac{1}{2}m(\dot{x}^2+\dot{y}^2)-mgy$.
The constraint for this system is $\vec{r}\cdot\vec{r}=R^2\Leftrightarrow x^2+y^2=R^2$.
We obtain two constrained Euler-Lagrange equations:
\[\frac{\partial L}{\partial y}+\lambda\frac{\partial f}{\partial y}=\frac{d}{dt}\frac{\partial L}{\partial \dot{y}}\Leftrightarrow -mg+2\lambda y=\frac{d}{dt}(m\dot{y})\Leftrightarrow2\lambda y-mg=m\ddot{y}\]
\[\frac{\partial L}{\partial x}+\lambda\frac{\partial f}{\partial x}=\frac{d}{dt}\frac{\partial L}{\partial \dot{x}}\Leftrightarrow  2\lambda x=\frac{d}{dt}(m\dot{x}
  )\Leftrightarrow2\lambda x=m\ddot{x}\]
Solving each of these equations of motion for the un-differentiated variable, substituting into the constraint, and solving for $\lambda$, we obtain
\[\left( \frac{m\ddot{x}}{2\lambda} \right)^2+\left( \frac{m\ddot{y}+mg}{2\lambda} \right)^2=R^2\]
\[\Leftrightarrow \lambda=\frac{m}{2R}\sqrt{\ddot{x}^2+(\ddot{y}+g)^2}\]
Since the aboce Euler-Lagrange equations have the form of Newton's second law, it follows the tension has components
\[F_{tx}=2\lambda x=\frac{mx}{R} \sqrt{\ddot{x}^2+(\ddot{y}+g)^2}\]
\[F_{ty}=2\lambda y=\frac{my}{R} \sqrt{\ddot{x}^2+(\ddot{y}+g)^2}\]


\section*{4}
For this problem, we have the same Lagrangian, but take spherical instead of polar coordinates:
\[L=\frac{1}{2}mv^2-mgy \Leftrightarrow L=\frac{1}{2}mv^2-mgr\sin(\phi)\sin(\theta)\]
Our constraint is $\vec{r}\cdot\vec{r}=l^2\Leftrightarrow r=l$, so this becomes
\[L=\frac{1}{2}ml^2(l^2\dot{\theta}^2+l^2\dot{\phi}^2\sin^2(\theta))-mgl\sin(\theta)\sin(\phi)\]
There are then two Euler-Lagrange equations:
\[\frac{\partial L}{\partial \phi}=\frac{d}{dt}\frac{\partial L}{\partial \dot{\phi}}\Leftrightarrow-mgl\sin(\theta)\cos(\phi)=\frac{d}{dt}\left[ ml^4\dot{\phi}\sin^2(\theta) \right]\]
\[\Leftrightarrow -mgl\sin(\theta)\cos(\theta)=ml^4\ddot{\phi}\sin^2(\theta)+ml^4\dot{\phi}\sin(2\theta)\dot{\theta}\]
and
\[\frac{\partial L}{\partial \theta}=\frac{d}{dt}\frac{\partial L}{\partial\dot{\theta}}\Leftrightarrow ml^2\dot{\phi}^2\sin(2\theta)-mgl\cos(\theta)\sin(\phi)=\frac{d}{dt}\left[ ml^4\dot{\theta} \right]\]
\[\Leftrightarrow ml^2\dot{\phi}^2\sin(2\theta)-mgl\cos(\theta)\sin(\phi)=ml^4\ddot{\theta}\]
For small angles, these become
\[-g(1-\theta/2)=l^3\ddot{\phi}\theta+2l^3\dot{\phi}\dot{\theta}\]
and
\[2l\dot{\phi}^2\theta-g(1-\theta^2/2)\phi=l^3\ddot{\theta}\]
\section*{5}
We have the same inital Lagrangian as before, but keep the problem in rectangular coordinates:
\[L=\frac{1}{2}mv^2-mgy\Leftrightarrow L=\frac{1}{2}m(\dot{x}^2+\dot{y}^2+\dot{z}^2)-mgy\]
with constraint $f(x,y,z)=c\Leftrightarrow\vec{r}\cdot\vec{r}=l^2\Leftrightarrow x^2+y^2+z^2=l^2$.
This yields three constrained Euler-Lagrange equations
\[\frac{\partial L}{\partial x}+\lambda\frac{\partial f}{\partial x}=\frac{d}{dt}\frac{\partial L}{\partial \dot{x}}\Leftrightarrow2\lambda x=m\ddot{x}\]
\[\frac{\partial L}{\partial y}+\lambda\frac{\partial f}{\partial y}=\frac{d}{dt}\frac{\partial L}{\partial \dot{y}}\Leftrightarrow2\lambda y-mg=m\ddot{y}\]
\[\frac{\partial L}{\partial y}+\lambda\frac{\partial f}{\partial z}=\frac{d}{dt}\frac{\partial L}{\partial \dot{z}}\Leftrightarrow2\lambda z=m\ddot{z}\]
Solving each of these for the un-differentiated variable, substituting into the constraint, and solving for $\lambda$, we obtain
\[\left( \frac{m\ddot{x}}{2\lambda} \right)^2+\left( \frac{m\ddot{y}+mg}{2\lambda} \right)^2+\left( \frac{m\ddot{z}}{2\lambda} \right)^2=l^2\]
\[\Leftrightarrow \lambda=\frac{m}{2l}\sqrt{\ddot{x}^2+(\ddot{y}+g)^2+\ddot{z}^2}\]
Since the Euler-Lagrange equations had the form of Newton's second law, we can read the components of the tension force from there as
\[F_{tx}=\frac{mx}{2l}\sqrt{\ddot{x}^2+(\ddot{y}+g)^2+\ddot{z}^2}\]
\[F_{ty}=\frac{my}{2l}\sqrt{\ddot{x}^2+(\ddot{y}+g)^2+\ddot{z}^2}\]
\[F{tz}=\frac{mz}{2l}\sqrt{\ddot{x}^2+(\ddot{y}+g)^2+\ddot{z}^2}\]
\end{document}
%%% Local Variables:
%%% mode: latex
%%% TeX-master: t
%%% End:
