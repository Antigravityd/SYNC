\documentclass[12pt]{article}

\usepackage[letterpaper]{geometry}

\title{4125 Self-Reflection}
\author{Duncan Wilkie}
\date{14 May 2022}

\begin{document}

\maketitle

\section*{}
I feel my engagement with the material in this course has been satisfactory.
While my effort and engagement, particularly regarding class attendance, exhibited room for improvement,
I don't think it has so negatively impacted my obtaining knowledge relevant to the course.
I have gotten what I had hoped to: broad familiarity with thermal physics, both in concept and practice.

In the preliminary questionnaire, I had written that my goal was to prepare for later appearance of the course material.
The other classes I had this semester have illuminated that I probably ought to have been a little more interested in the course for its own
merits: that I was just starting this subject and quantum mechanics made the introductory solid-state course exceptionally difficult in the
beginning, as Simon's \textit{Oxford Solid State Basics} immediately presumes familiarity with concepts from density of states
to degenerate perturbation theory; the math department's QFT course started immediately with the Atiyah-Segal axioms for topological
quantum field theory, a subject solely focused on the computation of partition functions---the relation of which to the standard concept
the Russian ex-physicist teaching the course failed to comprehensibly explain, with a vague allusion to the Atiyah-Singer index theorems.
The extent to which the simple concepts our text devotes about fifty pages to explicating are able to describe vastly heterogeneous
emergent phenomena of collective quantum systems is fascinating, and while I barely managed to stay afloat in 4261, thanks in no small part
to Shelton's (overly) lenient grading, the class certainly illuminated how absolutely critical statistical mechanics is.

In a couple aspects, I believe my effort and engagement was exemplary: the quality of my homework and take-home test responses were
near to the limits of my ability, and I read the textbook diligently, even going through the mixture thermodynamics material
we skipped because I found Schroeder's exposition made the material so interesting.
I endeavored to make the assignments interesting to myself by mixing in topics from my self-education in computer science, writing
Simpson integration and an optimization algorithm via bisection and the central difference in Scheme Lisp (that was surprisingly elegant and
fast) instead of appealing to an existing CAS on the problems on the last homework that required one.
I even packaged the {\LaTeX} syntax-highlighting macro collection \verb|minted| for the GNU Guix system, and partially packaged the
macros for gnuplot's TikZ output mode, as I wanted that typesetting functionality for my homework.
Several of the classmates I study with occasionally asked for my answers on problems they missed to study for exams.
However, in others it was lacking: I recall showing up to class only rarely, and while in those lectures I did attend (which, by the way,
were excellent compared to most other professors I've had in the department---and I'm not just saying that to help my pending grade)
I made my best effort to engage with the material and class discussion, that only partially makes up for the abysmal attendance rate.
This was in part due to a pretty serious ongoing struggle with sleep disregulation.
While I realize not being able to get out of bed is a somewhat silly excuse, it's only been through the last year or two it's been an issue.
I woke up at 5:45 for four years of high school with little problem; I am actually seeking medical help at this point.
In any case, it appears that this issue has had marginal effects on my uptake of the material, as I feel I attended more of the lectures on
the statistical mechanics chapters, and my exam scores were nearly perfect for the thermodynamics I learned almost entirely without attending
class.

I still feel slightly uncomfortable as to exactly what about Boltzmann statistics breaks for fermions and bosons; I can understand $Z_{1}\gg N$ itself well enough, and I can do the microstate combinatorics as well as anyone, but knowing what the consequences of that assumption
being violated are and exactly which results from the classical analysis are transferrable is something I'm shaky on.
For example, why does $\mathcal{P}(s)=e^{-\beta E(s)}$ hold in both cases \textit{\`a la} our discussion about the exam problem, but the
computation of the partition function break down?
Why can't one compute the energy expectation of a Gibbs statistics system in analogy with Boltzmann statistics?
Schroeder computes it in the Fermi gas chapter by deriving the density of states; if it is possible to do it the other way, it'd seem like
he wouldn't do it like this.
I probably couldn't reproduce the kinetic theory derivation of the ideal gas law unassisted, which seems like something I ought to be able
to do, given how simple the concept and how far-reaching the implications.
I think I have an extremely good grasp of how to derive all the consequent thermodynamic quantities from a couple initial state variables
using the differential identities for the various potentials and Maxwell's relations (to the point that if you gave experimental data I
probably could predict some additional properties for you).
Going from the quantum model of a system to those initial state variables is something I'm much less confident in, but I believe I can
rectify that with fairly minimal additional work.
I found Schroeder's explanation of these concepts downright unhelpful when contrasted with his stellar treatment of thermodynamics.
Consultation of another text that perhaps treats things slightly more mathematically may rectify this.

Overall, I'd rate my mastery of the material at an A.
The topics I'm uncertain in are easily enumerable, and even of the statistical mechanics portion of the class are a minority of subjects:
it's really only the first principles of chapter 7 as they relate to those of chapter 6, as I can do the classical statistics just fine and
felt the last homework dealing with the consequences of those principles in chapter 7 was pretty easy.

\end{document}
%%% Local Variables:
%%% mode: latex
%%% TeX-master: t
%%% End:
