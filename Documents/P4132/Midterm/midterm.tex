\documentclass{article}

\usepackage[letterpaper]{geometry}
\usepackage{amsmath}
\usepackage{amssymb}
\usepackage{siunitx}

\title{4132 Midterm}
\author{Duncan Wilkie}
\date{25 March 2022}

\begin{document}

\maketitle

\section*{1a}
The the magnetic field due to a straight wire is
\[\vec{B}=\frac{\mu_{0}I_{0}}{2\pi s}\hat{\phi}\]
The flux is therefore
\[\Phi=\int\vec{B}\cdot d\vec{a}=\int_{s}^{s+a}\frac{\mu_{0}I_{0}}{2\pi s} (ads)=\frac{\mu_{0}I_{0}a}{2\pi}\left( \ln(s+a)-\ln s \right)=\frac{\mu_{0}I_{0}}{2\pi}a\ln\left(1+\frac{a}{s}\right)\]
\section*{1b}
We may write the distance of the loop as a function of time as $s=vt$.
The resulting motional emf is then
\[\mathcal{E}=-\frac{d\Phi}{dt}=-\frac{\mu_{0}I_{0}}{2\pi}a\frac{d}{dt}\ln\left( 1+\frac{a}{vt} \right)=\frac{\mu_{0}I_{0}}{2\pi}\frac{a^{2}}{vt^{2}+{at}}\]
so the induced current is
\[I=\frac{\mathcal{E}}{R}=\frac{\mu_{0}I_{0}}{2\pi R}\frac{a^{2}}{vt^{2}+at}\]
The flux is decreasing, so the induced current will be in the direction such that the magnetic field it produces will be in the same direction as the magnetic field of the straight wire. Presuming the loop to lie above the straight wire when viewed in the plane of the loop, and the current in the stright wire to flow to the right, the induced current will be counterclockwise by the right-hand rule.

\section*{1c}
The motional emf is instead, plugging the given time-dependent current value in,
\[\mathcal{E}=-\frac{\mu_{0}a}{2\pi}\ln\left( 1+\frac{a}{s} \right)\frac{d}{dt}\left( (1-kt)I_{0} \right)=\frac{\mu_{0}akI_{0}}{2\pi}\ln\left( 1+\frac{a}{s} \right)\]
The current is therefore
\[I=\frac{\mathcal{E}}{R}=-\frac{\mu_{0} akI_{0}}{2\pi R}\ln\left( 1+\frac{a}{s} \right)\]
Since the emf has the same sign, the current ought to be in the same direction, but as a sanity check, the flux is decreasing as the magnetic field decreases once again, so the direction will remain counterclockiwise.

\section*{1d}
In the first case, the work is supplied by the force keeping the movement of the loop away from the wire at a constant speed instead of decelerating. In the second, the work is supplied by whatever force is decelerating the charges.

\section*{2a}
The electric field may be expected to be radial, so taking a Gaussian surface to be a cylinder coaxial with the others of radius $r$ and length $L$, the electric field is
\[\int_{S}\vec{E}\cdot d\vec{a}=\frac{Q_{enc}}{\epsilon_{0}}\Leftrightarrow 2\pi rL|\vec{E}|=\frac{Q_{enc}}{\epsilon_{0}}\]
The enclosed charge is zero except between the cylinders, in which case it is $2\pi a L \lambda$. The electric field then has magnitude
\[ |\vec{E}|=
  \begin{cases}
    \frac{a\lambda}{\epsilon_{0} r} & a < r < b \\
    0 & \textrm{otherwise}
  \end{cases}
\]
It will be directed away from the positive charge and towards the negative, i.e. radially outward.
The magnetic field may be expected to be circulating in a manner concentric with perpendicular, planar sections of the cylinders; choosing an Amperian loop of radius $r$ in the same plane in which the magnetic field circles, we may write
\[\int_{C}\vec{B}\cdot d\vec{l}=\mu_{0}I_{enc}\Leftrightarrow 2\pi r|\vec{B}|=\mu_{0}I_{enc}\]
The enclosed current is
\[I_{enc}=
  \begin{cases}
    \lambda v & a < r < b \\
    0 & \textrm{otherwise}
  \end{cases}
\]
so the magnetic field is
\[\vec{B}=\frac{\mu_{0}\lambda v}{2\pi r}\hat{\phi}\]
between the cylinders, where we have presumed the cylinders have axes coincedent with the $z$-axis and are moving in the positive direction, deducing the direction using the right-hand rule.
The volumetric energy density stored in the field may be immediately computed as
\[u=\frac{1}{2}\left( \epsilon_{0}|\vec{E}|^{2}+\frac{1}{\mu_{0}}|\vec{B}|^{2} \right)=\frac{1}{2}\left( \epsilon_{0}\frac{a^{2}\lambda^{2}}{\epsilon_{0}^{2}r^{2}} +\frac{1}{\mu_{0}}\frac{\mu_{0}^{2}\lambda^{2}v^{2}}{4\pi^{2}r^{2}}\right)=\frac{\lambda^{2}}{2r^{2}}\left( \frac{a^{2}}{\epsilon_{0}}+\frac{\mu_{0}v^{2}}{4\pi^{2}} \right)\]
This expression is valid between the cylinders; elsewhere, the energy density is zero. Integrating this expression over planes perpendicular to the $z$-axis, the energy density per unit length along the cylinders is
\[u'=\frac{\lambda^{2}}{2r^{2}}\int_{0}^{2\pi}\int_{a}^{b}\frac{a^{2}}{\epsilon_{0}}+\frac{\mu_{0}v^{2}}{4\pi^{2}}rdrd\theta=\frac{\lambda^{2}}{2r^{2}}\left( \frac{\pi a^{2}}{\epsilon_{0}}+\frac{\mu_{0 }v^{2}}{4\pi} \right)(b^{2}-a^{2})\]

\section*{2b}
The Poynting vector is
\[\vec{S}=\frac{1}{\mu_{0}}\left( \vec{E}\times \vec{B} \right)=\frac{av\lambda^{2}}{2\pi\epsilon_{0}r^{2}}\hat{z}\]
so the volumetric momentum density in the fields is
\[\vec{g}=\epsilon_{0}\mu_{0}\vec{S}=\frac{\mu_{0}av\lambda^{2}}{2\pi r^{2}}\hat{z}\]
Integrating over the same region as above, the momentum density per unit length along the $z$-axis is
\[\vec{g}'=\frac{\mu_{0}a v \lambda^{2}}{2\pi}\int_{0}^{2\pi}\int_{a}^{b}\frac{1}{r^{2}}rdrd\theta=\mu_{0}av\lambda^{2}\ln\frac{b}{a}\]
where it has been used, once again, that these quantities are only nonzero between the tubes.

\section*{2c}
The Poynting vector gives the energy flux density in the fields, so integrating the expression found above over such a plane yields a total power transported by the fields across the plane of
\[P=\int\vec{S}\cdot d\vec{a}=\int_{0}^{2\pi}\int_{a}^{b}\frac{av\lambda^{2}}{2\pi\epsilon_{0}r^{2}}rdrd\theta=\frac{av\lambda^{2}}{\epsilon_{0}}\left( \ln r\bigg|_{a}^{b} \right)=\frac{av\lambda^{2}}{\epsilon_{0}}\ln\frac{b}{a}\]
The bounds of integration aren't over the whole plane because the integrand is zero except between the tubes.

\section*{3}
If the TE$_{00}$ mode existed, then the solution to the wave guide equation for the magnetic field with rectangular boundary conditions
\[B_{z}=B_{0}\cos\left( \frac{m\pi x}{a} \right)\cos\left( \frac{n\pi y}{b} \right)\]
would imply that the magnetic field within the wave guide is constant across the whole cross-section and nondecreasing.\footnote{I didn't follow the roadmap in the hint for this problem given in the book because the deduction that TE$_{00}$ waves lead to $k=\omega/c$ and degeneracy of equation 9.180 requires presuming that the wave guide equation holds in order to use equation 9.183; it is no less logically sound to deduce the $z$-component is constant from the wave guide solution. If one is to do this problem without reliance on that assumption, it is necessary to state what one means by $m=n=0$ independent of the wave guide equation, which isn't possible due to their definition. Presumption of the wave guide solution's validity is implicit in the statement of the problem.}
Applying the integral form of Faraday's law on a cross-section of the wave guide with boundary infinitesimally inside the metal,
\[\int\vec{E}\cdot d\vec{l}=-\frac{d\Phi}{dt}\]
\[\Rightarrow 0 = -\frac{d}{dt}\left(abB_{z}e^{i(kz-\omega t)}  \right)=i\omega ab B_{z}e^{i(kz-\omega t)}\Rightarrow B_{z}=0\]
However, this would imply that the wave is not transverse-electric but transverse-magnetic.

\section*{4}
The reflection coefficient at the interface of a conducting material and a dielectric is
\[R=\left| \frac{1-\tilde{\beta}}{1+\tilde{\beta}}\right|^{2}=\frac{1-\tilde{\beta}}{1+\tilde{\beta}}\cdot\frac{1-\tilde{\beta}^{*}}{1+\tilde{\beta}^{*}}=\frac{1-\tilde{\beta}^{*}-\tilde{\beta}+|\tilde{\beta}|^{2}}{1+\tilde{\beta}^{*}+\tilde{\beta}+|\tilde{\beta}|^{2}}\]
\[\frac{1-2\Re{\tilde{\beta}}+|\tilde{\beta}|^{2}}{1+2\Re{\tilde{\beta}}+|\tilde{\beta}|^{2}}\]
where
\[\tilde{\beta}=\frac{\mu_{1}v_{1}}{\mu_{2}\omega}\tilde{k}_{2}=\frac{\mu_{1}v_{1}}{\mu_{2}\omega}(k+i\kappa)\approx \frac{v_{1}}{\omega}(k+i\kappa)\]
where we have used $\mu_{1}\approx\mu_{2}\approx\mu_{0}$.
The real part of $\tilde{k}$ is
\[k=\omega\sqrt{\frac{\epsilon_{2}\mu_{0}}{2}}\left[ 1+\sqrt{1+\left( \frac{\sigma}{\epsilon_{2}\omega} \right)^{2}} \right]^{1/2}\]
which, since this is a good conductor ($\sigma \ggg \epsilon_{2}\omega$), may be approximated via $\sqrt{1+x^{2}}\sim x$ when $x\to\infty$ as
\[k\approx\omega\sqrt{\frac{\epsilon_{2}\mu_{0}}{2}}\sqrt{\frac{\sigma}{\epsilon_{2}\omega}}=\sqrt{\frac{\mu_{0}\sigma\omega}{2}}\]
Therefore,
\[2\Re\tilde{\beta}=2\sqrt{\frac{\mu_{0}\sigma v_{1}^{2}}{2\omega}}\]
\[=2\sqrt{\frac{(\SI{1.25e-6}{H/m})(\SI{4e7}{\mho/m})(\SI{3e8}{m/s})^{2}}{2(\SI{4e15}{Hz})}}=\SI{47.43}{}\]
In addition, the real and imaginary parts of $\tilde{k}$ are approximately equal, so
\[|\tilde{k}|^{2}=k^{2}+\kappa^{2}\approx 2k^{2}\approx \mu_{0}\sigma\omega\]
\[\Rightarrow |\tilde{\beta}|^{2}=\mu_{0}\sigma v_{1}^{2}/\omega\]\[ =(\SI{1.25e-6}{H/m})(\SI{4e7}{\mho/m})(\SI{3e8}{m/s})^{2}/(\SI{4e15}{Hz})=\SI{1125}{}\]
The reflection coefficient is then
\[R=\frac{1-2\Re{\tilde{\beta}}+|\tilde{\beta}|^{2}}{1+2\Re{\tilde{\beta}}+|\tilde{\beta}|^{2}}=\frac{1-\SI{47.43}{}+\SI{1125}{}}{1+\SI{47.43}{}+\SI{1125}{}}
=\SI{0.919}{}\]


\end{document}
%%% Local Variables:
%%% mode: latex
%%% TeX-master: t
%%% End:
