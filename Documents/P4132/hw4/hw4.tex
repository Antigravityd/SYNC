\documentclass{article}

\usepackage[letterpaper]{geometry}
\usepackage{siunitx}
\usepackage{amsmath}
\usepackage{amssymb}

\title{4132 HW 4}
\author{Duncan Wilkie}
\date{15 February 2022}

\begin{document}

\maketitle

\section*{8.2a}
As in the other problem, the capacitance is
\[C=\frac{q(t)}{V}=\frac{\epsilon_0 A}{d}\Leftrightarrow V=\frac{wq(t)}{\epsilon_0A}\Rightarrow \vec{E}=\frac{q(t)}{\pi\epsilon_0a^2}\hat{z}=\frac{It}{\pi\epsilon_0 a^2}\hat{z}\]
presuming the current is along $\hat{z}$.
Differentiating,
\[\frac{\partial\vec{E}}{\partial t}=\frac{I}{\pi\epsilon_0 a^2}\hat{z}\]
By the Amp\`ere-Maxwell equation, observing that there's no current through the gap, we have
\[\nabla\times \vec{B}=\mu_0\epsilon_0\frac{\partial\vec{E}}{\partial t}=\frac{\mu_0I}{\pi a^2}\hat{z}\]
Integrating over a disc centered on the axis of the wire with radius $s \ll a$,
\[\int_C\vec{B}\cdot d\vec{\ell}=\int_0^{2\pi}\int_0^s \frac{\mu_0I}{\pi a^2}rdrd\theta \Rightarrow 2\pi s B=\mu_0 I \frac{s^2}{a^2} \Leftrightarrow  B = \frac{\mu_0 I}{2\pi a^2}s\]
directed in the $\hat{\phi}$ direction.

\section*{8.2b}
The energy density is
\[u_{em}=\frac{1}{2}\left( \epsilon_0E^2+\frac{1}{\mu_0}B^2 \right)=\frac{1}{2}\left( \frac{I^2t^2}{\epsilon_0\pi^2a^4}+\frac{\mu_0I^2s^2}{4\pi^2a^4} \right)=\frac{I^2}{2\pi^2a^4}\left( \frac{t^2}{\epsilon_0}+\frac{\mu_0 s^2}{4} \right)\]
The Poynting vector is
\[\vec{S}=\frac{1}{\mu_0}\left( \vec{E}\times\vec{B} \right)=-\frac{I^2ts}{2\epsilon_0\pi^2a^4}\hat{s}\]
Checking the continuity equation,
\[\frac{du_{em}}{dt}=\frac{I^{2}t}{\epsilon_{0}\pi^{2}a^{4}}\]
and
\[-\nabla\cdot \vec{S}=-\frac{1}{s}\frac{\partial}{\partial s}(sS_{s})=\frac{I^{2}t}{2\epsilon_{0}\pi^{2}a^{4}}+\frac{I^{2}t}{2\epsilon_{0}\pi^{2}a^{4}}=\frac{I^{2}t}{\epsilon_{0}\pi^{2}a^{4}}\]
so energy conservation holds.
\section*{8.2c}
The total energy in the gap is the integral of the energy density over the gap:
\[E=\int_Vu_{em}dV=\frac{I^2}{2\pi^2a^4}\int_0^w\int_0^{2\pi}\int_0^a\left( \frac{t^2}{\epsilon_0}+\frac{\mu_0 s^2}{4}\right)sdsd\phi dz=\frac{I^2}{2\pi^2a^4}\left( \pi w\frac{a^2t^2}{\epsilon_0} +\pi w\frac{\mu_0a^4}{8}\right)\]
\[=\frac{wI^2}{2\pi a^2}\left( \frac{t^2}{\epsilon_0}+\frac{\mu_0a^2}{8} \right)\]
The total power flowing into the gap is the integral of the Poynting vector over the surface of the gap. It's pointed radially outwards, so it's parallel to the caps and perpendicular to the sides of the cylinder, implying
\[P=\int_0^w\int_0^{2\pi}\frac{I^2ta}{2\epsilon_0\pi^2 a^4}ad\phi dz=\frac{wt I^2}{\epsilon_0\pi a^2}\]
Differentiating the total energy in the gap, we obtain
\[\frac{dE}{dt}=\frac{wtI^2}{\epsilon_0\pi a^2}=P\]

\section*{8.4a}
We may take the midpoint of the line connecting the two charges to be the origin, and the line to be the $z$ axis. The plane is then the $x$-$y$ plane. The vector from a general point $(r,\theta, z)$ to each of the charges in the spherical coordinate system using this origin is the vector difference $\vec{r}-\vec{s}$ where $r$ is the vector from the origin to $(r,\theta,z)$ and $\vec{s}$ is the vector from the origin to the charge in question. The electric field is then
\[\vec{E}=\frac{kq}{r^2+\theta^2+(z-a)^2}(\hat{r}-\hat{z})+\frac{kq}{r^2+\theta^2+(z+a)^2}(\hat{r}+\hat{z})\]
\[=\left( \frac{kq}{r^2+\theta^2+(z+a)^2}+\frac{kq}{r^2+\theta^2+(z-a)^2}\right)\hat{r}+\left( \frac{kq}{r^2+\theta^2+(z+a)^2}-\frac{kq}{r^2+\theta^2+(z-a)^2}\right)\hat{z}\]
The magnetic field is of course zero, as the charges are stationary. The Maxwell stress tensor is componentwise
\[T_{ij}=\epsilon_0\left( E_iE_j-\frac{1}{2}\delta_{ij}E^2\right)+\frac{1}\mu_0\left( B_iB_j-\frac{1}{2}\delta_{ij}B^2 \right)\]
\[\Rightarrow T_{11}=   \epsilon_{0} \left( \frac{kq}{r^2+\theta^2+(z+a)^2}+\frac{kq}{r^2+\theta^2+(z-a)^2}\right)^2\]\[-\frac{  \epsilon_{0}}{2}\left[ \left(  \frac{kq}{r^{2}+\theta^{2}+(z+a)^{2}}+ \frac{kq}{r^2+\theta^2+(z-a)^2}\right)^{2}+\left( \frac{kq}{r^2+\theta^2+(z+a)^2}- \frac{kq}{r^2+\theta^2+(z-a)^2} \right)^{2}\right]\]
\[T_{13}=  \epsilon_{0}\left( \frac{kq}{r^2+\theta^2+(z+a)^2}\right)^2-  \epsilon_{0}\left( \frac{kq}{r^2+\theta^2+(z-a)^2}\right)^2\]
\[T_{31}=  \epsilon_{0}\left( \frac{kq}{r^2+\theta^2+(z+a)^2}\right)^2-  \epsilon_{0}\left( \frac{kq}{r^2+\theta^2+(z-a)^2}\right)^2\]
\[T_{33}=
  \epsilon_{0}\left( \frac{kq}{r^2+\theta^2+(z+a)^2}-\frac{kq}{r^2+\theta^2+(z-a)^2}\right)^2\]\[-\frac{  \epsilon_{0}}{2}\left[ \left(  \frac{kq}{r^{2}+\theta^{2}+(z+a)^{2}}+ \frac{kq}{r^2+\theta^2+(z-a)^2}\right)^{2}+\left( \frac{kq}{r^2+\theta^2+(z+a)^2}- \frac{kq}{r^2+\theta^2+(z-a)^2} \right)^{2}\right]\]
All other components are zero.

We must now integrate this over the plane $z=0$. The differential area element is $rdrd\theta\hat{z}$; its left dot product by the stress tensor is
\[\overset{\leftrightarrow}{T}\cdot d\vec{a}=
  \begin{pmatrix}
    T_{13} \\
    0 \\
    T_{33}
  \end{pmatrix}rdrd\theta
\]
Evaluating this at $z=0$, the $r$ component is zero, but the $z$ component is
\[\left( \overset{\leftrightarrow}{T}\cdot d\vec{a}\right)_{z}=-\frac{q^{2}}{16\pi^{2}\epsilon_{0}}\left( \frac{1}{r^{2}+\theta^{2}+a^{2}} \right)^{2}rdrd\theta\]
Integrating this over all $r$ and $\theta$,
\[\vec{F}=-\frac{q^{2}}{16\pi^{2}\epsilon_{0}}\int_{0}^{2\pi}\int_{0}^{\infty}\frac{r}{(r^{2}+\theta^{2}+a^{2})^{2}}drd\theta=-\frac{q^{2}}{16\pi^{2}\epsilon_{0}}\int_{0}^{2\pi}\int_{\theta^{2}+a^{2}}^{\infty}\frac{\frac{1}{2}du}{u^{2}}d\theta\]
\[=-\frac{q^{2}}{32\pi^{2}\epsilon_{0}}\int_{0}^{2\pi}\frac{1}{\theta^{2}+a^{2}}d\theta=-\frac{q^{2}}{32\pi^{2}\epsilon_{0}}\left( \frac{1}{a}\tan^{-1}\left( \frac{\theta}{a} \right)\bigg|_{0}^{2\pi} \right)\]
Using $\tan^{-1}(x)\approx x$ for small $x$ from the first term of the Laurent series at $x=0$, this becomes
\[\vec{F}=-\frac{q^{2}}{16\pi\epsilon_{0} a^{2}}=-\frac{kq^{2}}{r^{2}}\bigg|_{r=2a}\]
which is valid for large $a$.
\section*{8.4b}
If the charges are opposite in sign, the electric field becomes (presuming the negative charge is the one below the $x$-$y$ plane)
\[\vec{E}=\frac{kq}{r^2+\theta^2+(z-a)^2}(\hat{r}-\hat{z})-\frac{kq}{r^2+\theta^2+(z+a)^2}(\hat{r}+\hat{z})\]
\[=\left( \frac{kq}{r^2+\theta^2+(z-a)^2}-\frac{kq}{r^2+\theta^2+(z+a)^2}\right)\hat{r}-\left( \frac{kq}{r^2+\theta^2+(z+a)^2}+\frac{kq}{r^2+\theta^2+(z-a)^2}\right)\hat{z}\]
The remainder of the argument proceeds identically, with stress-energy tensor components
\[T_{11}=   \epsilon_{0} \left( \frac{kq}{r^2+\theta^2+(z+a)^2}-\frac{kq}{r^2+\theta^2+(z-a)^2}\right)^2\]\[-\frac{  \epsilon_{0}}{2}\left[ \left(  \frac{kq}{r^{2}+\theta^{2}+(z+a)^{2}}+ \frac{kq}{r^2+\theta^2+(z-a)^2}\right)^{2}+\left( \frac{kq}{r^2+\theta^2+(z+a)^2}- \frac{kq}{r^2+\theta^2+(z-a)^2} \right)^{2}\right]\]
\[T_{13}=  \epsilon_{0}\left( \frac{kq}{r^2+\theta^2+(z+a)^2}\right)^2-  \epsilon_{0}\left( \frac{kq}{r^2+\theta^2+(z-a)^2}\right)^2\]
\[T_{31}=  \epsilon_{0}\left( \frac{kq}{r^2+\theta^2+(z+a)^2}\right)^2-  \epsilon_{0}\left( \frac{kq}{r^2+\theta^2+(z-a)^2}\right)^2\]
\[T_{33}=
  \epsilon_{0}\left( \frac{kq}{r^2+\theta^2+(z+a)^2}+\frac{kq}{r^2+\theta^2+(z-a)^2}\right)^2\]\[-\frac{  \epsilon_{0}}{2}\left[ \left(  \frac{kq}{r^{2}+\theta^{2}+(z+a)^{2}}+ \frac{kq}{r^2+\theta^2+(z-a)^2}\right)^{2}+\left( \frac{kq}{r^2+\theta^2+(z+a)^2}- \frac{kq}{r^2+\theta^2+(z-a)^2} \right)^{2}\right]\]
All that is different is then that $T_{33}$ will be positive this time, resulting in a force in the opposite direction as expected.

\section*{8.5a}
In betweeen the plates, the electric field is
\[\vec{E}=-\frac{\sigma}{\epsilon_{0}}\hat{z}\]
and the magnetic field is
\[\vec{B}=-\mu_{0}\sigma v\hat{x}\]
Both are zero elsewhere.
The electromagnetic momentum per unit area is then
\[\vec{p}=\int_{0}^{d}\vec{g}dz=\int_{0}^{d}\epsilon_{0}\vec{E}\times\vec{B}dz=\mu_{0}vd\sigma^{2}\hat{y}\]
implying the electromagnetic momentum in a region of area $A$ is
\[\vec{p}=\mu_{0}vd\sigma^{2}A\hat{y
  }\]
\section*{8.5b}
The magnetic force at the upper plate is
\[\vec{F}=\int_{A}\vec{K}\times \vec{B}d\vec{a}=\frac{1}{2}\mu_{0}\sigma^{2}vuA\hat{z}\]
where $K$ has magnitude $-u\hat{z}$ since the velocity of the charge in the $y$ direction results in a magnetic force in the $z$ direction which doesn't contribute to the change in momentum (since the velocity of the plate in the $z$ direction is fixed by assumption).
This is a constant force delivered over time $t=\frac{d}{u}$, so the impulse is
\[\Delta \vec{p}=\frac{d\sigma^{2}A}{u} \frac{\mu_{0}uv}{2}\hat{z}=\frac{\mu_{0}}{2}vd\sigma^{2}A\hat{z}\]
The magnetic field at time $t$ behaves like the Heaviside function at the position of the plate with ``jump'' or ``amplitude'' $-\mu_{0}\sigma v\hat{x}$, so its derivative in time behaves like a delta distribution whose locus travels down the $z$ axis with speed $u$. We have
\[\nabla \times \vec{E}=-\frac{\partial \vec{B}}{\partial t}=-\frac{\partial }{\partial t}\left( -\mu_{0}\sigma v \hat{x}\theta(d-ut) \right)=-\mu_{0}\sigma v u\delta(d-ut)\hat{x}\]
Using the fact that the curl of a discontinuous distributional vector field is equal to the jump in the cross product of the field with the vector normal to the surface along which it is discontinuous, $\vec{E}$ must have a jump of magnitude $\mu_{0}\sigma vu$. By the right-hand rule, the jump must be in the $\hat{y}$ direction. It also must be otherwise constantly oriented: if it starts to deviate from the $\hat{y}$ direction as the position varies, this would result in a nonzero curl of the electric field at a point with $z\neq ut$, contradicting the curl derived above. Considering an Amperian rectangle with normal parallel to $\hat{y}$ of length $l$, it is evident that
\[2Bl=\mu_{0}\sigma vu\Leftrightarrow |\vec{B}|=\frac{1}{2}\mu_{0}\sigma vu\]
and by the argument about the jump above it must be in the $-\hat{y}$ direction below the sheet.
The induced electric force on the bottom plate is therefore
\[\vec{F}=\frac{1}{2}\mu_{0}\sigma^{2} vuA\hat{y}\]
so the impulse is by an identical argument
\[\Delta \vec{p}=\frac{\mu_{0}}{2}vd\sigma^{2}A\hat{y}\]
The total impulse is the sum of these, which is exactly the $\mu_{0}vd\sigma^{2}A$ derived in part (a).


\end{document}
%%% Local Variables:
%%% mode: latex
%%% TeX-master: t
%%% End:
