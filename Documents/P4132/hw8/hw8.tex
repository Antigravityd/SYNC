\documentclass{article}

\usepackage[letterpaper]{geometry}
\usepackage{siunitx}
\usepackage{amsmath}
\usepackage{amssymb}
\usepackage{graphicx}

\title{4132 HW 8}
\author{Duncan Wilkie}
\date{19 April 2022}

\begin{document}

\maketitle

\section{}
The average power lost by the oscillating dipole is
\[\langle P \rangle=\frac{\mu_{0}p_{0}^{2}\omega^{4}}{12\pi c}\]
The current through the wire connecting the spheres is, by the definition of the charge on the spheres at time $t$,
\[I(t)=\dot{q}(t)=-q_{0}\omega\sin(\omega t)\]
The average power may be calculated from the RMS current, which is related to the peak current by
$I_{rms}=I_{0}/\sqrt{2}=\frac{q_{0}\omega}{\sqrt{2}}$
We then can calculate the average power using the DC equations using the RMS current:
\[\langle P \rangle=I_{rms}^{2}R=\frac{q_{0}^{2}\omega^{2}}{2}R\]
Equating the two expressions for average power and solving for $R$,
\[
  \frac{\mu_{0}p_{0}^{2}\omega^{4}}{12\pi c}=\frac{q_{0}^{2}\omega^{2}}{2}R
  \Leftrightarrow R=\frac{\mu_{0}p_{0}^{2}\omega^{2}}{6\pi c q_{0}^{2}}
\]
The initial dipole moment is by definition $p_{0}=q_{0}d\Leftrightarrow \frac{p_{0}}{q_{0}}=d$
and the angular frequency is related to the wavelength by
$\omega=2\pi f\Rightarrow \lambda = \frac{c}{f}=\frac{2\pi c}{\omega}\Leftrightarrow \omega=\frac{2\pi c}{\lambda}$
so we may write
\[
  R=\frac{\mu_{0}d^{2}\omega^{2}}{6\pi c}
  =\frac{2\pi\mu_{0}c}{3}\frac{d^{2}}{\lambda^{2}}
  =\frac{2\pi(4\pi\times 10^{-7})(\SI{3e8}{m/s})}{3}\frac{d^{2}}{\lambda^{2}}
  =80\pi^{2}\frac{d^{2}}{\lambda^{2}}\approx 790\frac{d^{2}}{\lambda^{2}}\si{\ohm}
\]
The FM broadcast band is at about $\SI{3}{m}$ in wavelength.
Taking the given wire length in an ordinary radio, the radiation resistance would evaluate to
\[
  R=790\frac{(\SI{5}{cm})^{2}}{(\SI{3}{m})^{2}}=\SI{0.22}{\ohm}
\]
Since the resistivity of copper is quite low, the resistance of even quite small-gauge copper wire at this length will be on the order of
\si{m\ohm}; the radiation resistance is the bulk of the power lost along the wire.

\section{}
Using the equation for the peak radiated power of an arbitrary time-dependent charge configuration found in the book,
\[P_{rad}(t_{0})=\frac{\mu_{0}}{6\pi c}\left[ \ddot{p}(t_{0}) \right]^{2}\]
where $p(t_{0})$ is the initial magnitude of the dipole moment of the charge configuration.
This initial dipole moment may be calculated by
\[
  \vec{p}=\int_{\textrm{charge}} \vec{r}\rho(\vec{r})dV
  =\int_{-\infty}^{\infty}\int_{0}^{2\pi}\int_{0}^{\infty}\delta(z)\delta(r-b)(\cos\phi\hat{x}+\sin\phi\hat{y})
  \lambda_{0}\sin\phi(rdrd\phi dz)
\]
\[
  =b\int_{0}^{2\pi}(b\cos\phi\hat{x}+b\sin\phi\hat{y})\lambda_{0}\sin\phi d\phi
  =b^{2} \lambda_{0}\hat{x}\int_{0}^{2\pi}\sin\phi\cos\phi d\phi+b^{2}\lambda_{0}\hat{y}\int_{0}^{2\pi}\sin^{2}\phi d\phi
\]
Using $2\sin(2\phi)=\sin\phi\cos\phi$, one notices the first is simply the integral of $\sin\phi$ over two periods which,
since $\sin$ has average value $0$ over one cycle, integrates to zero.
By the trig identities $\sin^{2}x=1-\cos^{2}x$ and $\cos(2x)=2\cos^{2}x-1\Leftrightarrow \cos^{2}x=\frac{1+\cos(2x)}{2}$, one may write
$\sin^{2}x=\frac{1-\cos(2x)}{2}$, and so the second integral becomes
\[
  \int_{0}^{2\pi}\sin^{2}\phi d\phi=\int_{0}^{2\pi}\left[ \frac{1}{2}-\frac{1}{2}\cos(2\phi)\right]d\phi
\]
The later integral is zero by the same reasoning for $\sin(2\theta)$ above, but the first is equal to $\pi$.
Therefore, the initial dipole moment is
\[\vec{p_{0}}=\pi\lambda_{0}b^{2}\hat{y}\]
As the ring rotates about its center, its dipole moment will be of the same magnitude with direction rotating at constant angular velocity.
We may therefore write
\[\vec{p}(t)=|p_{0}|(\hat{x}\cos(\omega t)+\hat{y}\sin(\omega t))\=\pi\lambda_{0} b^{2}(\hat{x}\cos(\omega t)+\hat{y}\sin(\omega t))\]
Differentiating twice with respect to time,
\[\ddot{\vec{p}}(t)=\pi\lambda_{0}b^{2}\omega^{2}(-\hat{x}\cos(\omega t)-\hat{y}\sin(\omega t)=-\omega^{2}\vec{p}(t)\]
which has magnitude $\pi\lambda_{0}b^{2}\omega^{2}$.
We may then write
\[
  P_{rad}(t_{0})=\frac{\mu_{0}}{6\pi c}[\pi\lambda_{0}b^{2}\omega^{2}]^{2}=\frac{\mu_{0}\pi\lambda_{0}^{2}b^{4}\omega^{4}}{6c}
\]


\section{}
Applying the Larmour formula with $a=g$, the power lost to radiation per second is
\[P_{rad}=\frac{\mu_{0}q^{2}g^{2}}{6\pi c}\]
The time it takes the electron to fall a distance $h$ is by
\[h=\frac{1}{2}gt^{2}\Leftrightarrow t=\sqrt{\frac{2h}{g}}\]
The lost potential energy is $mgh$, so the ratio of the lost potential energy to the radiated power over the time period is
\[
  \frac{P_{rad}t}{mgh}
  =\frac{(\mu_{0}q^{2}g^{2}/6\pi c)(\sqrt{2h/g})}{mgh}
  =\frac{\mu_{0}q^{2}}{6\pi mc}\sqrt{\frac{2g}{h}}
\]
\[
  =\frac{(\SI{1.26e-6}{N/A^{2}})(\SI{1.6e-19}{C})^{2}}{6\pi(\SI{9.11e-31}{kg})(\SI{3e8}{m/s})}\sqrt{\frac{2(\SI{9.81}{m/s^{2}})}{\SI{1}{cm}}}
  =\SI{2.77e-22}{}
\]

\section{}
The Coulomb attraction is equal to the centripetal acceleration, so
\[\frac{1}{4\pi\epsilon_{0}}\frac{q^{2}}{r^{2}}=m\frac{v^{2}}{r}
  \Leftrightarrow r=\frac{q^{2}}{4\pi\epsilon_{0}mv^{2}}
\]
This implies the velocity is only a reasonable fraction, say 10\%, of the speed of light whenever the radius is
\[r=\frac{(\SI{1.6e-19}{C})^{2}}{4\pi(\SI{8.85e-12}{F/m})(\SI{9.11e-31}{kg})(0.1)^{2}(\SI{3e8}{m/s})^{2}}=\SI{2.8e-13}{m}\]
which is $0.6\%$ of the initial radius, and at this point it's going at its fastest, so the proportion of the trip where the
electron is relativistic is vanishingly small, justifying the use of the Larmour formula.
Balancing the forces in the Bohr model, the magnitude of the centripetal force is equal to the magnitude of the Coulomb force.
We can then find an expression for the kinetic energy in terms of $r$ by
\[
  \frac{1}{4\pi\epsilon_{0}}\frac{q^{2}}{r^{2}}=m\frac{v^{2}}{r}
  \Rightarrow T=\frac{1}{2}mv^{2}=\frac{1}{8\pi\epsilon_{0}}\frac{q^{2}}{r}
\]
The electric potential between two point charges is of course
\[U=qV=-\frac{1}{4\pi\epsilon_{0}}\frac{q^{2}}{r}\]
In hindsight, the first may have been derived via the virial theorem from this potential.
The total energy is then
\[E=T+U=-\frac{1}{8\pi\epsilon_{0}}\frac{q^{2}}{r}\]
Differentiating this with respect to time, we obtain an expression for the power radiated as
\[P_{rad}=-\frac{dE}{dt}=-\frac{1}{8\pi\epsilon_{0}}\frac{q^{2}}{r^{2}}\frac{dr}{dt}\]
We may equate this with the Larmour formula, applying Newton's second law to obtain the acceleration,
\[P_{rad}=\frac{\mu_{0}q^{2}a^{2}}{6\pi c}=\frac{\mu_{0}q^{2}}{6\pi c}\left( \frac{1}{4\pi\epsilon_{0}}\frac{q^{2}}{mr^{2}} \right)^{2}\]
\[
  \Rightarrow -\frac{1}{8\pi\epsilon_{0}}\frac{q^{2}}{r^{2}}\frac{dr}{dt}=\frac{\mu_{0}q^{2}}{6\pi c}\frac{1}{16\pi^{2}\epsilon_{0}^{2}}
  \frac{q^{4}}{m^{2}r^{4}}
  \Leftrightarrow r^{2}dr=-\frac{\mu_{0}q^{4}}{12\pi^{2}\epsilon_{0}cm^{2}}dt
\]
Integrating,
\[
  \frac{r^{3}}{3}=-\sqrt{\frac{\mu_{0}^{3}}{\epsilon_{0}}}\frac{q^{4}}{12\pi^{2}m^{2}}t+C
\]
Applying the initial condition $r(t=0)=\SI{5e-11}{m}$, $C=(\SI{5e-11}{m})^{3}/3=\SI{4.17e-32}{m}$.
Solving for $t$,
\[
  t=\left(C-\frac{r^{3}}{3} \right)\frac{12\pi^{2}m^{2}}{q^{4}}\sqrt{\frac{\epsilon_{0}}{\mu_{0}^{3}}}
\]
Evaluating at $t=0$,
\[
  t=C\frac{12\pi^{2} m^{2}}{q^{4}}\sqrt{\frac{\epsilon_{0}}{\mu_{0}^{3}}}
  =(\SI{4.17e-32}{m})\frac{12\pi^{2}(\SI{9.11e-31}{kg})^{2}}{(\SI{1.6e-19}{C})^{4}}\sqrt{\frac{\SI{8.85e-12}{F/m}}{(\SI{1.26e-6}{N/A^{2}})^{3}}}
\]
\[
  =\SI{1.32e-11}{s}
\]
which is inconsistent with the existence of atomic matter.
\end{document}

%%% Local Variables:
%%% mode: latex
%%% TeX-master: t
%%% End:
