\documentclass{article}

\usepackage[letterpaper]{geometry}
\usepackage{amsmath}
\usepackage{amssymb}
\usepackage{siunitx}

\title{4141 HW 4}
\author{Duncan Wilkie}
\date{21 February 2022}

\begin{document}

\maketitle

\section*{1a}
The stationary states of an electron in an infinite square well of width $a$ are
\[\psi_{n}=\sqrt{\frac{2}{a}}\sin\left( \frac{n\pi}{a}x \right)\]
The energy is related to the wavenumber of the above functions by $k=\frac{\sqrt{2mE_{n}}}{\hbar}\Leftrightarrow E_{n}=\frac{k^{2}\hbar^{2}}{2m}$ using the definition of $k$ from the time-independent Schr\"odinger equation. Therefore,
\[E_{n}=\frac{n^{2}\pi^{2}\hbar^{2}}{2ma^{2}}\]
\[\Rightarrow E_{1}=\frac{\pi^{2}\hbar^{2}}{2ma^{2}}=\frac{\pi^{2}(\SI{1.05e-34}{J\cdot s})^{2}}{2(\SI{9.11e-31}{kg})(\SI{e-10}{m})^{2}}=\SI{6.05e-19}{J}\]
\[E_{2}=\frac{2\pi^{2}\hbar^{2}}{ma^{2}}=4E_{1}=\SI{2.42e-18}{J}\]
\[E_{3}=\frac{9\pi^{2}\hbar^{2}}{2ma^{2}}=9E_{1}=\SI{5.45e-18}{J}\]

\section*{1b}
The energy difference between the ground and second excited states are
\[E_{3}-E_{1}=\SI{4.84e-18}{J}\]
This corresponds to a photon frequency of
\[f=\frac{E}{h}=\frac{\SI{4.84e-18}{J}}{\SI{6.67e-34}{J\cdot s}}=\SI{7.26e15}{Hz}\]
needed to excite the electron.
As the electron de-excites, it will enter the first excited state, and then the ground state from there. The corresponding energies are
\[E_{3 2}=E_{3}-E_{2}=\SI{3.03e-18}{J}\]
and
\[E_{21}=E_{2}-E_{1}=\SI{1.82e-18}{J}\]
which correspond to photon frequencies
\[f_{32}=\frac{E_{32}}{h}=\SI{4.54e15}{Hz}\]
\[f_{21}=\frac{E_{21}}{h}=\SI{2.73e15}{Hz}\]

\section*{2}
The time-dependent wave function is
\[\Psi(x,t)=Ae^{ikx-i\hbar k^{2}/2m}+Be^{-ikx-i\hbar k^{2}/2m}\]
The flux is defined as
\[J=\frac{i\hbar}{2m}\left( \Psi\frac{\partial \Psi^{*}}{\partial x}-\Psi^{*}\frac{\partial\Psi}{\partial x}\right)=\]\[\frac{i\hbar}{2m}\bigg( \left[  Ae^{ikx-i\hbar k^{2}/2m}+Be^{-ikx-i\hbar k^{2}/2m}\right]\cdot\left[ -ikAe^{-ikx+i\hbar k^{2}/2m}+ikBe^{ikx+i\hbar k^{2}/2m} \right]\]\[ -\left[ Ae^{-ikx+i\hbar k^{2}/2m}+Be^{ikx+i\hbar k^{2}/2m} \right]\cdot\left[ ikAe^{ikx-i\hbar k^{2}/2m}-ikBe^{-ikx-i\hbar k^{2}/2m} \right]\bigg)\]
\[=\frac{i\hbar}{2m}\left( 2ik(B^{2}-A^{2}) \right)=\frac{k\hbar}{2m}(B^{2}-A^{2})\]

\section*{3}
If $\psi$ is real, the expectation value of $x\frac{dV}{dx}$ is
\[\int_{-\infty}^{\infty}\psi(x)x\frac{\partial V}{\partial x}\psi(x)dx=x\psi^{2}V\bigg|_{-\infty}^{\infty}-\int_{-\infty}^{\infty}(2x\psi\psi'+\psi^{2})Vdx\]
Since the wave function must be normalized, it must limit to zero at both infinities, so the first term is zero. Noting that by definiton
\[\langle V \rangle=\int_{-\infty}^{\infty}\psi V \psi dx\]
this is therefore
\[=-\langle V \rangle-2\int_{-\infty}^{\infty}\frac{\partial\psi}{\partial x}xV\psi dx\]
By the energy eigenvalue equation, this is
\[=-\langle {V} \rangle+\left( E+\frac{\hbar^{2}}{2m}\int_{-\infty}^{\infty}\left( \frac{\partial \psi}{\partial x} \right)^{2}dx \right)\]
Since $E=T+V$, we may we may rewrite this as
\[=\langle T \rangle+\frac{\hbar^{2}}{2m}\left( \psi\frac{\partial \psi}{\partial x}\bigg|_{-\infty}^{\infty}-\int_{-\infty}^{\infty}\psi\frac{\partial ^{2 }\psi}{\partial x} dx\right)=\langle T \rangle+\langle  T \rangle\]
Applying the Ehrenfest theorem achieves the desired result:
\[\left\langle \frac{p^{2}}{2m} \right\rangle=\frac{1}{2}\left\langle x\frac{\partial V}{\partial x} \right\rangle\]
Since we use the Ehrenfest theorem anyway, it's no less valid a proof to simply note that the virial theorem holds in classical mechanics, so it must hold for expectation values in quantum mechanics.
\section*{4}
Differentiating the eigenvalue equation
\[F(x)-\lim_{x'\to -\infty}F(x')=\lambda\psi(x)\]
where $F$ is the antiderivative of $x\psi(x)$,
\[x\psi=\lambda\frac{\partial \psi}{\partial x} \]
By separation of variables,
\[\psi(x)=ce^{x^{2}/2\lambda}\]
Therefore, all nonzero real nubmers are eigenvalues of this operator. Since going from the integral to the the differential equation is only a forward implication and not an equivalence, we know all eigenvalues of the differential equation are eigenvalues of the integral equation but not the converse; we must therefore check if zero is an eigenvalue of the original operator. If it were, there would be a nonzero function such that for all $x$
\[\int_{-\infty}^{x}xf(x)dx=0\]
which is clearly false.
The square of $\psi$ is
\[c^{2}e^{x^{2}/\lambda}\]
If $\lambda$ is negative, the integral of this square is
\[c^{2}e^{1/|\lambda|}\int_{-\infty}^{\infty}e^{-x^{2}}dx=\sqrt{\pi}c^{2}e^{1/|\lambda|}\]
which exists; if $\lambda$ is positive, the integrand would be proportional to $e^{x^{2}}$ which diverges and so isn't integrable.

\section*{5a}
Representing the sine as an exponential,
\[\psi(x,0)=N\left( \frac{e^{i\pi x/a}-e^{-i\pi x/a}}{2i} \right)^{5}\]
The total probability is then
\[1=\int_{-\infty}^{\infty}N^{2}\left( \frac{e^{i\pi x/a}-e^{-i\pi x/a}}{2i} \right)^{10}dx\]
Applying the binomial theorem,

\[=-\frac{N^{2}}{1024}\int_{0}^{a}\sum_{k=0}^{10}(-1)^{k}\binom{10}{k}e^{(10-k)i\pi x/a}e^{-ki\pi x/a}dx\]
\[=-\frac{N^{2}}{1024}\sum_{k=0}^{10}(-1)^{k}\binom{10}{k}\int_{0}^{a}e^{(10-2k)i\pi x/a}dx=-\frac{N^{2}}{1024}\sum_{k=0}^{10}(-1)^{k}\binom{10}{k}\left( \frac{a}{(10-2k) i\pi}e^{(10-2k)i\pi x/a}\bigg|_{0}^{a} \right)\]
This is zero for all $k$ but $5$, in which case we must return to the pentultimate expression and note that the integrand becomes one, so this becomes
\[=\frac{N^{2}}{1024}\binom{10}{5}a=aN^{2}\frac{252}{1024}=\frac{63}{256}aN^{2}\]
implying
\[N=\frac{16}{3\sqrt{7a}}\]
\section*{5b}
The stationary states of the infinite square well are
\[\psi_{n}=\sqrt{\frac{2}{a}}\sin\left( \frac{n\pi}{a}x \right)=\sqrt{\frac{2}{a}}\left( \frac{e^{in\pi x/a}-e^{-in\pi x/a}}{2i} \right)\]
The initial wave function, expressed using the binomial theorem once again, is
\[\psi(x,0)=\frac{N}{32i}\sum_{k=0}^{5}(-1)^{k}\binom{5}{k}e^{(5-2k)i\pi x/a}\]
Applying Fourier's trick to obtain the initial wave function as a linear combination of stationary states,
\[c_{n}=\int_{-\infty}^{\infty}\psi_{n}^{*}(x)\psi(x,0)dx=-\frac{N}{64}\sqrt{\frac{2}{a}}\sum_{k=0}^{5}(-1)^{k}\binom{5}{k}\int_{0}^{a}e^{i\pi (5-2k+n)x/a}-e^{i\pi(5-2k-n)x/a}dx\]
\[=-\frac{N}{64}\sqrt{\frac{2}{a}}\sum_{k=0}^{5}(-1)^{k}\binom{5}{k}\left( \frac{a}{i\pi(5-2k+n)}e^{i\pi(5-2k+n)x/a}\bigg|_{0}^{a}-\frac{a}{i\pi(5-2k-n)}e^{i\pi(5-2k-n) x/a}\bigg|_{0}^{a} \right)\]
The exponents are both integer multiples of $i\pi$ when evaluated at $a$ and zero when evaluated at zero. When $n$ is even, both exponents are odd multiples of $i\pi$, and vice versa. Therefore, the only $n$ that survive are the even ones, and the antiderivative evaluations both evaluate to $e^{i(2j+1)\pi}-1=-2$.
This then becomes
\[=\frac{ N\sqrt{{2}{a}}}{32\pi i}\sum_{k=0}^{5}(-1)^{k}\binom{5}{k}\left( \frac{1}{(5-2k+n)}-\frac{1}{(5-2k-n)} \right)\]
This is equal to zero for all $n$ where it is defined for every $k$, as can be seen by writing out two terms of the sum where $\binom{5}{k}$ would give the same number and noticing that rightmost factor is the same, but the signs are opposite.
However, this is not a justified manpulation when $5=2k-n$ or $5=2k+n$; equivalently, whenever $n=2k-5$ or $n=5-2k$. To determine the value when this happens, we need only go back to the original integral.

Since $k=0,1,2,3,4,5$ and $n\geq 1$ , the only $(n,k)$ pairs for which this happens are $(1,3)$, $(3,4)$, and $(5,5)$ for the first term being one and $(5,0)$, $(3,1)$, and $(1,2)$ for the second term being one. Notice that the other exponential and all terms from other values of $k$ always integrate to zero since the $n$ are all odd.
The nonzero coefficients in the expansion are therefore
\[c_{1}=-\frac{N}{64}\sqrt{\frac{2}{a}}\left( (-1)^{3}\binom{5}{3}a-(-1)^{2}\binom{5}{2}a \right)=20\frac{N\sqrt{2a}}{64}\]
\[c_{3}=-\frac{N}{64}\sqrt{\frac{2}{a}}\left( (-1)^{4}\binom{5}{4}a-(-1)^{1}\binom{5}{1}a \right)=-10\frac{N\sqrt{2a}}{64}\]
\[c_{5}=-\frac{N}{64}\sqrt{\frac{2}{a}}\left( (-1)^{5}\binom{5}{5}a-(-1)^{0}\binom{5}{0}a \right)=2\frac{N\sqrt{2a}}{64}\]
These are properly normalized, so likely correct.
The time evolution of the state is then the expansion in terms of stationary states with each term multiplied by $e^{-iE_{n}t/\hbar}$, i.e.
\[\Psi(x,t)=\frac{\sqrt{2/7}}{12 }\left( 20e^{-iE_{1}t/\hbar}-10e^{-iE_{2}t/\hbar}+2e^{-iE_{3}t/\hbar} \right)\]
where $E_{n}=\frac{n^{2}\pi^{2}\hbar^{2}}{2ma^{2}}$

\section*{5c}
The probability a particle is in energy state $E_{3}$ is the modulus-squared of the coefficient $c_{3}$:
\[P=|c_{3}|^{2}=\left| -10\frac{\sqrt{2/7}}{12} \right|^{2}=.1984\]
\end{document}
%%% Local Variables:
%%% mode: latex
%%% TeX-master: t
%%% End:
