\documentclass{article}

\usepackage[letterpaper]{geometry}
\usepackage{siunitx}
\usepackage{amsmath}
\usepackage{amssymb}

\title{4141 HW 6}
\author{Duncan Wilkie}
\date{11 March 2022}

\begin{document}

\maketitle

\section{}
For an ordinary square obstacle,
\[|T|^{2}=e^{-4a\sqrt{2m/\hbar^{2}(V_{0}-{E})}}\]
where $a$ is the width of the barrier and $V_{0}$ is the height of the barrier. The second barrier doesn't affect this calculation at all, since we may reset the point with respect to which the potentials are referenced to be such that $V_{1}=0$; then $E$ and $V_{0}$ become $E-V_{1}$ and $V_{1}-V_{0}$, which cancel when subtracted in the above formula

\section{}
The transmission coefficient for a finite square well is
\[T=e^{-2ika}\frac{2kk'}{2kk'\cos{2k'a}-i(k^{2}+k'^{2})\sin{2k'a}}\]
where
\[k=\sqrt{\frac{2mE}{\hbar^{2}}}\]
and
\[k'=\sqrt{\frac{2m(E+V_{0})}{\hbar^{2}}}\]
The initial energy of the car, taking the reference potential at the top of the bank of the river, is
\[E=\frac{1}{2}mv^{2}+mgh=\frac{1}{2}(\SI{1000}{kg})(\SI{25}{m/s})^{2}=\SI{313}{J}\]
The depth of the potential well is
\[V_{0}=(\SI{1000}{kg})(\SI{9.8}{m/s^{2}})(\SI{5}{m})=\SI{49}{kJ}\]
so the coefficients are
\[k=\sqrt{\frac{2(\SI{1000}{kg})(\SI{313}{J})}{(\SI{1.05e-34}{Js})^{2}}}=\SI{7.54e36}{}\]
and
\[k'=\sqrt{\frac{2(\SI{1000}{kg})(\SI{313}{J}+\SI{49000}{J})}{(\SI{1.05e-34}{Js})^{2}}}=\SI{9.46e37}{}\]
$a$ is of course $\SI{5}{m}$, so the transmission coefficient is then
\[T=e^{-2i(\SI{7.54e36}{})(\SI{5}{m})}\]\[\cdot\frac{2(\SI{7.54e36}{})(\SI{9.46e37}{})}{2(\SI{7.54e36}{})(\SI{9.46e37}{})\cos[2(\SI{7.54e36}{})(\SI{5}{m})]-i((\SI{7.54e36}{})^{2}+(\SI{9.46e37}{})^{2})\sin[2(\SI{5}{m})(\SI{9.46e37}{})]}\]
\[=e^{i\cdot\SI{7.54e37}{}}\frac{\SI{1.43e75}{}}{(\SI{1.43e75})(0.88)-i(\SI{9e75})(-0.96)}\]
which has magnitude-squared
\[|T|^{2}=\frac{(\SI{1.43e75}{})^{2}}{(\SI{1.26e75}{}+i[\SI{8.64e75}{}])(\SI{1.26e75}-i[\SI{8.64e75}{}])}=0.027\]

\section{}
For this problem, there is some transmitted wave after the potential barrier and some reflected wave; the transmitted wave is some incident wave on the square well. The square well produces some reflected wave, some trapped wave, and some transmitted wave. Writing this down, for a particle $e^{ikx}$ we have
\[\psi=
  \begin{cases}
    e^{ikx}+R_{1}e^{-ikx} & x < -b \\
    A_{1}e^{ik'x}+B_{1}e^{-ik'x} & -b < x < -a\\
    T_{1}e^{ikx}+R_{2}e^{-ikx} & -a < x < c \\
    A_{2}e^{ik'x} + B_{2}e^{-ik'x} & c < x < d\\
    T_{2}e^{ikx} & x > d
  \end{cases}\]
This is a system of 5 equations and 8 unknowns; we may obtain extra constraints by the restriction that the probability density is $C^{1}$ in time and therefore the value and time derivative of $\psi\psi*$ at the boundary points common to pairs of cases above must agree.
\section{}
The Schr\"odinger equation for this potential is
\[
  \begin{cases}
    -\frac{\hbar^{2}}{2m}\frac{\partial^{2}\psi}{\partial x^{2}}+V_{0}\psi=E\psi & |x| < \frac{a}{2}\\
    -\frac{\hbar^{2}}{2m}\frac{\partial^{2}\psi}{\partial x^{2}}=E\psi & \frac{a}{2} < |x| < b+\frac{a}{2} \\
  \end{cases}
\]
where $\psi$ is constrained to the ranges defined above. There is even symmetry to the Hamiltonian, which constrains the solutions to being even.
This may be written as a single equation by
\[-\frac{\hbar^{2}}{2m}\frac{\partial^{2} \psi}{\partial x^{2}}+[\theta(x+\frac{a}{2})-\theta(x-\frac{a}{2})]V_{0}\psi=E\psi\]
The weak form of this differential equation is
\[-\frac{\hbar^{2}}{2m}\int_{-b-\frac{a}{2}}^{b+\frac{a}{2}}\frac{\partial^{2}\psi}{\partial x^{2}}dx+V_{0}\int_{-\frac{a}{2}}^{\frac{a}{2}}\psi dx=E\int_{-b-\frac{a}{2}}^{b+\frac{a}{2}}\psi dx\]

\end{document}
%%% Local Variables:
%%% mode: latex
%%% TeX-master: t
%%% End:
