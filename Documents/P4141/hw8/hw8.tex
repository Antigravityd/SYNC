\documentclass{article}

\usepackage[letterpaper]{geometry}
\usepackage{amsmath}
\usepackage{amssymb}
\usepackage{siunitx}
\usepackage{graphicx}

\title{4141 HW 8}
\author{Duncan Wilkie}
\date{8 April 2022}

\begin{document}

\maketitle

\section{}
Call the operator $G$.
By the definition of hermicity,
\[
  \langle Gf|g \rangle = \langle f | Gg \rangle
  \Leftrightarrow \int_{\mathbb{R}}\left( Gf(\phi) \right)^{*}g(\phi)d\phi=\int_{\mathbb{R}}f(\phi)^{*}Gg(\phi)d\phi
  \Leftrightarrow \frac{\hbar}{i}\int_{\mathbb{R}}f'(\phi)^{*}g(\phi)d\phi=\frac{\hbar}{i}\int_{\mathbb{R}}f(\phi)^{*}g'(\phi)d\phi
\]
\[
  \Leftrightarrow \frac{\hbar}{i}\left( f(\phi)^{*}g(\phi)\bigg|_{-\infty}^{\infty}-\int_{\mathbb{R}}f'(\phi)^{*}g'(\phi)d\phi\right)
    =\frac{\hbar}{i} \left( f(\phi)^{*}g(\phi)\bigg|_{-\infty}^{\infty}-\int_{\mathbb{R}}f'(\phi)^{*}g'(\phi)d\phi \right)
\]
Therefore, $G$ is Hermitian.
Taking the inverse Fourier transform of $Gf$,
\[
  \mathcal{F}^{-1}(Gf)=\int_{\mathbb{R}}e^{2\pi ix\phi}\frac{\hbar}{i}f'(\phi)d\phi
  =\frac{\hbar}{i}f(\phi)e^{2\pi ix\phi}\bigg|_{-\infty}^{\infty}-\frac{\hbar}{i}\int_{\mathbb{R}}f(\phi)2\pi ixe^{2\pi ix\phi}d\phi
\]
Because $f$ is periodic, the first term is zero, since if one splits the evaluation into a sum of periodic sub-intervals, one obtains zero on each sub-interval.
This then becomes
\[
  =2\pi\hbar x\int_{\mathbb{R}}f(\phi)e^{2\pi i x \phi}d\phi = 2\pi\hbar x f(x)
\]
The expectation value of $G$ is therefore, in position space, the expectation value of the position operator $4\pi^{2}\hbar^{2} x^{2}$, so I presumably used a heterogeneous Fourier transform definition and the operator is identified with $x^{2}$.

\section{}
It's easy to represent $H$ as a matrix if one compares how it acts on $\psi=a|1\rangle+b|2\rangle$ to the action of a matrix on the vector $(a, b)$.
\[H=
  \epsilon\begin{pmatrix}
    \langle 1|1 \rangle + \langle 2|1 \rangle & \langle 1|2 \rangle+\langle 2|2 \rangle \\
    \langle 1|1 \rangle -\langle 2|1 \rangle& \langle 1|2 \rangle-\langle2|2  \rangle
  \end{pmatrix}
\]
By orthonormality, we may write this
\[
  =\epsilon
  \begin{pmatrix}
    1 & 1 \\
    1 & -1
  \end{pmatrix}
\]
This corresponds to an eigenvalue equation
\[
  \det(H-\lambda I)
  =\det
  \begin{vmatrix}
    1-\lambda & 1 \\
    1 & -1-\lambda
  \end{vmatrix}
  =\lambda^{2}-1-1=\lambda^{2}-2=0\Rightarrow \lambda = \pm\sqrt{2}
\]
with multiplicity 2. The eigenvectors may then be immediately computed as $(1-\sqrt{2})|1\rangle+|2\rangle$ and $(1+\sqrt{2})|1\rangle +|2\rangle$, which one may normalize by dividing by the norm if desired.

\section{}

\[
  \hat{O}=|\psi\rangle\langle \phi|
\]
\[
  \frac{\partial}{\partial x}f(x)=|h\rangle\langle h|g \rangle
\]

\section{}
The variance of a probability distribution $P$ is $\Delta P=\langle  P\rangle^{2}-\langle P^{2} \rangle$.
The -

\end{document}
%%% Local Variables:
%%% mode: latex
%%% TeX-master: t
%%% End:
