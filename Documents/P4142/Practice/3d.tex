\documentclass{article}

\usepackage[letterpaper]{geometry}
\usepackage{amsmath}
\usepackage{siunitx}

\title{Deriving The 3D QM Solutions From Scratch}
\author{Duncan Wilkie}
\date{31 August 2022}

\begin{document}

\maketitle

The 3D TISE is
\[
  -\frac{\hbar^{2}}{2m}\nabla^{2}\psi+V\psi = E\psi
\]
The spherical Laplacian is
\[
  \nabla^{2} = \frac{1}{r^{2}}\frac{\partial}{\partial r}\left( r^{2}\frac{\partial}{\partial r} \right)
  + \frac{1}{r^{2}\sin\theta}\frac{\partial}{\partial\theta}\left( \sin\theta\frac{\partial}{\partial\theta} \right)
  + \frac{1}{r^{2}\sin^{2}\theta}\left( \frac{\partial^{2}}{\partial\phi^{2}} \right)
\]

Presuming $\psi$ separates in spherical coordinates as $\psi=R(r)Y(\theta,\phi)$, the TISE is
\[
  -\frac{\hbar^{2}}{2m}\left[ \frac{Y}{r^{2}}\frac{\partial}{\partial r}\left( r^{2}\frac{\partial R}{\partial r}  \right)
    + \frac{R}{r^{2}\sin\theta}\frac{\partial}{\partial\theta}\left( \sin\theta\frac{\partial Y}{\partial\theta} \right)
    + \frac{R}{r^{2}\sin^{2}\theta}\left( \frac{\partial^{2}Y}{\partial\phi^{2}} \right)\right] + VRY = ERY
\]
\[
  \Leftrightarrow \left\{  \frac{1}{R}\frac{\partial}{\partial r}\left( r^{2}\frac{\partial R}{\partial r}  \right) +\frac{2mr^{2}}{\hbar^{2}}[E-V]\right\}
  + \frac{1}{Y}\left\{\frac{1}{\sin\theta}\frac{\partial}{\partial\theta}\left( \sin\theta\frac{\partial Y}{\partial\theta} \right)
  + \frac{1}{\sin^{2}\theta}\frac{\partial^{2}Y}{\partial\phi^{2}}\right\} = 0
\]
This separates the equation; the left side depends solely on $r$, and the right only on $\theta,\phi$, implying each term is constant,
as if the left term varies in $r$, the right term cannot vary to compensate and keep the whole equation equal to zero.

We therefore reduce this to the system
\[
  \frac{1}{R}\frac{\partial}{\partial r}\left( r^{2}\frac{\partial R}{\partial r} \right) + \frac{2mr}{\hbar^{2}}[E-V] = \ell(\ell+1),
\]

\[
  \frac{1}{Y}\left\{\frac{1}{\sin\theta}\frac{\partial}{\partial\theta}\left( \sin\theta\frac{\partial Y}{\partial\theta} \right)
  + \frac{1}{\sin^{2}\theta}\frac{\partial^{2}Y}{\partial\phi^{2}}\right\} = -\ell(\ell+1)
\]

First, we consider the angular equation: expanding,
\[
  \sin\theta\left( \cos\theta \frac{\partial Y}{\partial\theta} + \sin\theta\frac{\partial^{2}Y}{\partial\theta^{2}}\right)
  +\frac{\partial^{2}Y}{\partial\phi^{2}} = -\ell(\ell+1)Y\sin^{2}\theta
\]
We separate again.
Presuming $Y(\theta,\phi) = \Theta(\theta)\Phi(\phi)$,
\[
  \sin\theta\Phi(\phi)\left( \cos\theta \frac{\partial \Theta}{\partial\theta} + \sin\theta\frac{\partial^{2}\Theta}{\partial\theta^{2}}\right)
  +\Theta\frac{\partial^{2}\Phi}{\partial \phi^{2}}
  = -\ell(\ell+1)\Theta\Phi\sin^{2}\theta
\]
Dividing by $Y$ and moving the term on the right over,
\[
  \left\{\frac{\sin\theta}{\Theta}\left( \cos\theta \frac{\partial \Theta}{\partial\theta} + \sin\theta\frac{\partial^{2}\Theta}{\partial\theta^{2}}\right)
    +\ell(\ell+1)\sin^{2}\theta\right\}
  + \frac{1}{\Phi}\frac{\partial^{2}\Phi}{\partial\phi}
  =  0
\]
This is clearly separated, so the angular equation reduces to two equations
\[
  \frac{\sin\theta}{\Theta}\left( \cos\theta\frac{\partial\Theta}{\partial\theta}+\sin\theta\frac{\partial^{2}\Theta}{\partial\theta^{2}} \right)
  +\ell(\ell+1) = m^{2},
\]
\[
  \frac{\partial^{2}\Phi}{\partial\phi} = -m^{2}\Phi
\]
The second is trivial: the ansatz $e^{k\phi}$ yields auxiliary equation $k^{2}+m^{2}=0\Leftrightarrow k = \pm im$.
Allowing $m<0$, this is just $k=im$, so $\Phi(\phi) =  e^{im\phi}$.

The first equation is a standard result (I guess),
\[
  \Theta(\theta) = AP_{\ell}^{m}(\cos\theta),
\]
where
\[
  P_{\ell}^{m} - (-1)^{m}\left( 1-x^{2} \right)^{m/2}\left( \frac{d}{dx} \right)^{m}P_{\ell}(x),
\]
and
\[
  P_{\ell}(x) = \frac{1}{2^{\ell}\ell!}\left( \frac{d}{dx} \right)^{\ell}\left( x^{2}-1 \right)^{\ell}
\]
are the Legendre polynomials.
\end{document}
