\documentclass{article}

\usepackage[letterpaper]{geometry}
\usepackage{tgpagella}
\usepackage{amsmath}
\usepackage{amssymb}
\usepackage{amsthm}
\usepackage{siunitx}

\title{4142 HW 2}
\author{Duncan Wilkie}
\date{9 September 2022}

\newtheorem{prob}{Problem}

\begin{document}

\maketitle

\begin{prob}
  Given $Y_{2}^{1}(\theta, \phi) = -\sqrt{\frac{15}{8\pi}} \sin\theta\cos\theta e^{i\phi}$, use the raising operator $L_{+}$ to find $Y_{2}^{2}(\theta, \phi)$.
  What happens if $L_{+}$ is applied again?
\end{prob}

The raising operator has the form
\[
  L_{+} = \hbar e^{i\phi}\left( \frac{\partial}{\partial\theta} + i\cot\theta\frac{\partial}{\partial\phi} \right)
\]
Applying this to the given spherical harmonic,
\[
  L_{+}Y_{2}^{1} = \hbar e^{i\phi}\left( \frac{\partial}{\partial\theta} + i\cot\theta\frac{\partial}{\partial\phi} \right)
  \left( -\sqrt{\frac{15}{8\pi}} \sin\theta\cos\theta e^{i\phi} \right)
\]
\[
  = -\hbar e^{i\phi}\sqrt{\frac{15}{8\pi}}\left[ e^{i\phi}\frac{\partial}{\partial\theta}\frac{1}{2}\sin 2\theta
    +i\cot\theta\sin\theta\cos\theta\frac{\partial}{\partial\phi}e^{i\phi}\right]
\]
\[
  = -\hbar e^{2i\phi}\sqrt{\frac{15}{8\pi}}\left( \cos 2\theta - \cos^{2}\theta \right)
\]
\[
  = \hbar\sqrt{\frac{15}{8\pi}} \sin^{2}\theta e^{2i\phi}
\]
The raising operator satisfies $L_{+}f_{\ell}^{m} = \hbar\sqrt{(\ell-m)(\ell+m+1)}f_{\ell}^{m+1}$; dividing by $\hbar\sqrt{(2 - 1)(2 + 1 + 1)} = 2\hbar$,
\[
  Y_{2}^{2} = \sqrt{\frac{15}{32\pi}} \sin^{2}\theta e^{2i\phi}
\]
Since $m = \ell$, there are no more states with larger $m$, since $-\ell \leq m \leq \ell$.
$L_{+}$ then maps $Y_{2}^{2}$ to zero.

\begin{prob}
  A particle is in an angular momentum state of $\ell = 2$, $m = 1$.
  What is the probability of finding it at the position $\theta = \pi / 4,\ \phi = \pi / 4$ to within $d\theta = d\phi = \SI{0.01}{rad}$?
\end{prob}

We integrate the probability amplitude over this region:
\[
  P = \int_{\pi / 4 - 0.005}^{\pi / 4 + 0.005}\int_{\pi / 4 - 0.005}^{\pi / 4 + 0.005}Y_{2}^{1}Y_{2}^{1*}d\phi d\theta
  = \int_{\pi / 4 - 0.005}^{\pi / 4 + 0.005}\int_{\pi / 4 - 0.005}^{\pi / 4 + 0.005}\frac{15}{8\pi} \frac{1}{4}\sin^{2}2\theta e^{i\phi-i\phi}d\phi d\theta
\]
\[
  = 0.01\frac{15}{32\pi}\int_{\pi / 4 - 0.005}^{\pi / 4 + 0.005} \sin^{2}2\theta d\theta
  = 0.01\frac{15}{32\pi}\int_{\pi / 4 - 0.005}^{\pi / 4 + 0.005} \frac{1}{2}\left( 1 - \cos 4\theta \right) d\theta
\]
\[
  = 0.01\frac{15}{64\pi}\left( \theta\bigg|_{\pi / 4 - 0.005}^{\pi / 4 + 0.005} - \frac{1}{4}\sin 4\theta\bigg|_{\pi / 4 - 0.005}^{\pi / 4 + 0.005}\right)
\]
\[
  = 0.01\frac{15}{64\pi}\left(  0.01 - \frac{1}{4}\sin(\pi + 0.02) + \frac{1}{4}\sin(\pi - 0.02)\right)
  = \SI{7.33e-6}{}
\]

\begin{prob}
  A particle moving in a potential is described by the wave function
  \[
    \psi(x,y,z) = (xy+yz+zx)e^{-\alpha(x^{2}+y^{2}+z^{2})}.
  \]
  What is the probability that a measurement of $L^{2}$ and $L_{z}$ gives $6\hbar^{2}$ and $h$, respectively?
\end{prob}

In spherical coordinates, the wave function is
\[
  \psi(r, \phi, \theta) = r^{2}(\cos\phi\sin\phi + \sin\phi\cos\theta + \cos\theta\cos\phi)e^{-\alpha r^{2}}
\]

The coefficients in the expansion of $\psi$ in terms of the simultaneous eigenstates of $L^{2}$ and $L_{z}$ are given by,
denoting the angular part of $\psi$ by  $\psi_{ang} = \cos\phi\sin\phi + \sin\phi\cos\theta + \cos\theta\cos\phi$,
\[
  c_{\ell m} = \langle Y_{\ell}^{m} |  \psi_{ang} \rangle = \int_{0}^{\pi}\int_{0}^{2\pi} Y_{l}^{m*} \psi_{ang} d\phi d\theta
\]
\[
  = \int_{0}^{\pi}\int_{0}^{2\pi}\sqrt{\frac{(2\ell + 1)}{4\pi}\frac{(\ell-m)!}{(\ell+m)}} e^{-im\phi}P_{\ell}^{m}(\cos\theta)
  (\cos\phi\sin\phi + \sin\phi\cos\theta + \cos\theta\cos\phi)d\phi d\theta
\]
\[
  = \sqrt{\frac{(2\ell + 1)}{4\pi}\frac{(\ell-m)!}{(\ell+m)}} \bigg(
  \int_{0}^{\pi}\cos\phi\sin\phi e^{-im\phi}d\phi\int_{0}^{2\pi}P_{\ell}^{m}(\cos\theta)d\theta
\]
\[
  + \int_{0}^{\pi}\sin\phi e^{-im\phi}d\phi\int_{0}^{2\pi}\cos\theta P_{\ell}^{m}(\cos\theta)d\theta
\]
\[
  + \int_{0}^{\pi}\cos\phi e^{-im\phi}d\phi\int_{0}^{2\pi}\cos\theta P_{\ell}^{m}(\cos\theta)d\theta \bigg)
\]
$6\hbar^{2}$ corresponds to $\ell = 2$; $\hbar$ corresponds to $m = 1$, so we're looking for the coefficient of the $Y_{2}^{1}$ component.
The associated Legendre function $P_{2}^{1}(\cos\theta) = -3\sin\theta\cos\theta$ is odd, and it's multiplied by an even function in the second factor
in the last two terms, so each of the second factors is the integral of an odd periodic function over its period, which is zero in general
(any two integrals of a periodic function over two period-long intervals are equal, so use the interval centered on 0).

The coefficient is therefore zero, i.e. the probability of getting $6\hbar^{2}$ and $\hbar$ is zero.

\begin{prob}
  An asymmetric rotor with two moments of inertia is described by the Hamiltonian $H = (L_{x}^{2} + L_{y}^{2}) / 2I_{1} + L_{z}^{2}/2I_{2}$ with $I_{1}>I_{2}$.
  By identifying two suitable commuting operators, describe the energy eigenstates and their eigenvalues.
\end{prob}

\[
  H =
\]


For this problem alone, denote $L_{x}^{2} + L_{y}^{2} + L_{z}^{2} = L^{2}$.
This observable is compatible with $L_{z}$:
\[
  [L^{2}, L_{z}] = [L_{x}^{2}, L_{z}] + [L_{y}^{2}, L_{z}]
  = L_{x}[L_{x}, L_{z}] + [L_{x}, L_{z}]L_{x} + L_{y}[L_{y}, L_{z}] + [L_{y}, L_{z}]L_{y}
\]
\[
  = -i\hbar L_{x}L_{y} - i\hbar L_{y}L_{x} + i\hbar L_{y}L_{x} + i\hbar L_{x}L_{y}
  = 0
\]
Since these commute, they are simultaneously diagonalizable.
The same $L_{\pm}$ commutes with $L^{2}$:
\[
  [L_{x}\pm iL_{y}, L^{2}] = [L_{x}, L^{2}] \pm i[L_{y}, L^{2}] = 0 \pm 0i = 0
\]
Take a simultaneous eigenstate of $L^{2}$ and $L_{z}$:
\[
  L^{2}f = \lambda f
\]
\[
  L_{z}f = \mu f
\]

\end{document}
