\documentclass{article}

\usepackage[letterpaper]{geometry}
\usepackage{tgpagella}
\usepackage{amsmath}
\usepackage{amssymb}
\usepackage{amsthm}
\usepackage{tikz}
\usepackage{minted}
\usepackage{physics}
\usepackage{siunitx}

\sisetup{detect-all}
\newtheorem{plm}{Problem}
\renewcommand*{\proofname}{Solution}


\title{4142 HW 6}
\author{Duncan Wilkie}
\date{14 November 2022}

\begin{document}

\maketitle

\begin{plm}
  Consider the energy matrix
  \[
    H = V_{0}
    \begin{pmatrix}
      1 - \epsilon & 0 & 0 \\
      0 & 1 & \epsilon \\
      0 & \epsilon & 2
    \end{pmatrix}
  \]
  where $V_{0}$ is constant and $\epsilon$ is a small number.
  \begin{itemize}
  \item Find the exact eigenvalues and eigenvectors of this Hamiltonian.
  \item Use first- and second-order non-degenerate perturbation theory to derive the energy corrections to the unperturbed problem with
    $\epsilon = 0$.
  \item Use first order degenerate perturbation theory to derive the energy corrections.
    Compare the results.
  \end{itemize}
\end{plm}

\begin{proof}
  Clearly, one eigenvector is the first basis state with eigenvalue $V_{0}(1 - \epsilon)$.
  We can diagonalize the $2 \times 2$ matrix pretty easily:
  \[
    \det
    \begin{vmatrix}
      1 - \lambda & \epsilon \\
      \epsilon & 2 - \lambda
    \end{vmatrix}
    = 2 - 3\lambda + \lambda^{2} - \epsilon^{2}
    \Rightarrow \lambda = \frac{3 \pm \sqrt{1 + 4\epsilon^{2}}}{2}
  \]
  The corresponding eigenvectors are
  \[
    \begin{pmatrix}
      \frac{-1 \pm \sqrt{1 + 4\epsilon^{2}}}{2\epsilon} \\
      1
    \end{pmatrix}
  \]
  The eigenvalues and eigenvectors for the overall problem are then
  \[
    V_{0}(1 - \epsilon) \to
    \begin{pmatrix}
      1 \\
      0 \\
      0
    \end{pmatrix}
  \]
  \[
    V_{0}\frac{3 - \sqrt{1 + 4\epsilon^{2}}}{2} \to
    \begin{pmatrix}
      0 \\
      \frac{-1 - \sqrt{1 + 4\epsilon^{2}}}{2\epsilon} \\
      1
    \end{pmatrix}
  \]
  \[
    V_{0}\frac{3 + \sqrt{1 + 4\epsilon^{2}}}{2} \to
    \begin{pmatrix}
      0 \\
      \frac{-1 + \sqrt{1 + 4\epsilon^{2}}}{2\epsilon} \\
      1
    \end{pmatrix}
  \]

  We write the Hamiltonian as
  \[
    H = H_{0} + H' = V_{0}\qty[
    \begin{pmatrix}
      1 & 0 & 0 \\
      0 & 1 & 0 \\
      0 & 0 & 2
    \end{pmatrix}
    +
    \epsilon
    \begin{pmatrix}
      -1 & 0 & 0 \\
      0 & 0 & 1 \\
      0 & 1 & 0
    \end{pmatrix}
    ]
  \]
  We must first compute the 0th-order terms: the solution to the $H_{0}$ problem.
  This is trivial, as the problem is diagonalized: there are two eigenstates with energies $V_{0}$ and one with eigenstate $2V_{0}$
  corresponding exactly to the basis states with respect to which the problem is presented.
  First-order non-degenerate theory gives us
  \[
    E_{n}^{1} = \bra{\psi_{n}^{0}}H'\ket{\psi_{n}^{0}}.
  \]
  Sticking with the eigenbasis for $H_{0}$,
  \[
    E_{1}^{1} = \bra{\psi_{1}^{0}}H'\ket{\psi_{1}^{0}} = -\epsilon V_{0}
  \]
  \[
    E_{2}^{1} = \bra{\psi_{2}^{0}}H'\ket{\psi_{2}^{0}} = 0
  \]
  \[
    E_{3}^{1} = \bra{\psi_{3}^{0}}H'\ket{\psi_{3}^{0}} = 0
  \]
  The second-order non-degenerate theory yields
  \[
    E_{1}^{2} = \frac{\qty|\bra{\psi_{2}^{0}}H'\ket{\psi_{1}^{0}}|^{2}}{E_{1}^{0} - E_{2}^{0}}
    + \frac{\qty|\bra{\psi_{3}^{0}}H'\ket{\psi_{1}^{0}}|^{2}}{E_{1}^{0} - E_{3}^{0}}
    = 0
  \]
  \[
    E_{2}^{2} = \frac{\qty|\bra{\psi_{1}^{0}}H'\ket{\psi_{2}^{0}}|^{2}}{E_{2}^{0} - E_{1}^{0}}
    + \frac{\qty|\bra{\psi_{3}^{0}}H'\ket{\psi_{2}^{0}}|^{2}}{E_{2}^{0} - E_{3}^{0}}
    = \frac{|\epsilon V_{0}|^{2}}{V_{0} - 2V_{0}}
    = -{\epsilon^{2} V_{0}}
  \]
  \[
    E_{3}^{2} = \frac{\qty|\bra{\psi_{1}^{0}}H'\ket{\psi_{3}^{0}}|^{2}}{E_{3}^{0} - E_{1}^{0}}
    + \frac{\qty|\bra{\psi_{2}^{0}}H'\ket{\psi_{3}^{0}}|^{2}}{E_{3}^{0} - E_{2}^{0}}
    = \frac{|\epsilon V_{0}|^{2}}{2V_{0} - V_{0}}
    = \epsilon^{2}V_{0}
  \]

  For the more suitable first-order degenerate correction, we must first find an operator mutually commuting with $H_{0}$ and $H'$.
  Using the fact that if the product of symmetric matrices is symmetric, then they commute, any diagonal matrix will commute with both.
  Choose
  \[
    A =
    \begin{pmatrix}
      1 & 0 & 0 \\
      0 & 2 & 0 \\
      0 & 0 & 3
    \end{pmatrix}.
  \]
  Accordingly, the first and second basis eigenstates are suitable to remove their degeneracy in the unperturbed state.
  The matrix whose eigenvalues are the first-order degenerate corrections is given by
  \[
    W =
    \epsilon V_{0}
    \begin{pmatrix}
      -1 & 0 \\
      0 & 0
    \end{pmatrix}
  \]
  The eigenvalues are given trivially as $E_{\pm}^{1} = 0, -\epsilon V_{0}$
  These are exactly the non-degenerate first-order corrections to $E_{1}^{1}$ and $E_{2}^{1}$.
\end{proof}

\begin{plm}
  For the harmonic oscillator with $V(x) = kx^{2}/2$, the allowed energies are $E_{n} = \qty(n + \frac{1}{2})\hbar\omega$
  with $\hbar\omega = \sqrt{k / m}$.
  Suppose the spring is cooled so that the spring constant rises slightly to $k(1 + \epsilon)$.
  \begin{enumerate}
  \item Find the exact new energies.
    Expand in $\epsilon$ to second order.
  \item Instead, treat by perturbation theory to calculate the first order correction to the energy.
    Compare the results.
  \end{enumerate}
\end{plm}

\begin{proof}
  The new energies are given by $k \mapsto k(\epsilon + 1)$, so $E_{n}' = \qty(n + \frac{1}{2})\hbar\sqrt{\frac{k}{m}}\sqrt{1 + \epsilon}$.
  Using the Maclaurin series of $\sqrt{1 + x}$, this is to second order
  \[
    E_{n}^{2} = \qty(n + \frac{1}{2})\hbar\sqrt{\frac{k}{m}}\qty[1 + \frac{\epsilon}{2} - \frac{\epsilon^{2}}{8}].
  \]

  Perturbatavely, the perturbation to the Hamiltonian is $H' = \epsilon H_{0}$, so
  \[
    E_{n}^{1} = \bra{\psi_{n}^{0}} H' \ket{\psi_{n}^{0}} = \epsilon\qty(\frac{1}{2}\hbar\sqrt{\frac{k}{m}})
  \]
  This is precisely the first-order term in the series expansion of the exact solution.
\end{proof}

\begin{plm}
  Estimate the energy correction to the ground state of the hydrogen atom arising from the finite size of the proton (approx \SI{e-13}{cm}).
  Assume all the charge of the proton is distributed uniformly on
  \begin{enumerate}
  \item the surface of the proton,
  \item the volume of the proton.
  \end{enumerate}
\end{plm}

\begin{proof}
  With a spherical shell of charge, the electric field inside is zero by Gauss's law,
  so the potential inside is the potential of a point charge at the boundary: $-\frac{e^{2}}{4\pi \epsilon_{0}r_{p}}$,
  where $r_{p}$ is the radius of the proton.
  Similarly, for a solid sphere of charge, by Gauss's law the electric field is as a point charge at the center,
  but the size of that charge is proportional to the volume enclosed by a concentric Gaussian sphere:
  \[
    |E| = \frac{e^{2}}{4\pi\epsilon_{0}r^{2}}\frac{r^{3}}{r_{p}^{3}} = \frac{e^{2}}{4\pi\epsilon_{0}}\frac{r}{r_{p}^{3}}
  \]
  Integrating from $r_{p}$ (whereupon $V = -\frac{e^{2}}{4\pi\epsilon_{0}r_{p}}$ as before) to $r$, the potential is
  \[
    V = -\frac{e^{2}}{8\pi\epsilon_{0}r_{p}^{3}}(r^{2} - r_{p}^{2}) - \frac{e^{2}}{4\pi\epsilon_{0}r_{p}}
    = -\frac{e^{2}r^{2}}{8\pi\epsilon_{0}r_{p}^{3}} - \frac{e^{2}}{8\pi\epsilon_{0} r_{p}}
  \]
  Accordingly, we get new Hamiltonians
  \[
    H = H_{0} + H' = \qty(\frac{p^{2}}{2m} - \frac{e^{2}}{4\pi\epsilon_{0} r}) + \Theta(r_{p} - r)\qty(\frac{e^{2}}{4\pi\epsilon_{0} r}
    - \frac{e^{2}}{4\pi\epsilon_{0} r_{p}})
  \]
  and
  \[
    H = H_{0} + H' = \qty(\frac{p^{2}}{2m} - \frac{e^{2}}{4\pi\epsilon_{0} r}) + \Theta(r_{p} - r)\qty(\frac{e^{2}}{4\pi\epsilon_{0} r}
    - \frac{e^{2}r^{2}}{8\pi\epsilon_{0} r_{p}^{3}} - \frac{e^{2}}{8\pi\epsilon_{0}r_{p}})
  \]
  The first-order correction to the energies can be computed directly:
  \[
    E_{n}^{1} = \bra{\psi_{n}^{0}} H' \ket{\psi_{n}^{0}}
    = \bra{\psi_{n}^{0}} \Theta(r_{p} - r)\qty(\frac{e^{2}}{4\pi\epsilon_{0} r} - \frac{e^{2}}{8\pi\epsilon_{0} r_{p}}) \ket{\psi_{n}^{0}}
  \]
  \[
    = \frac{e^{2}}{4\pi\epsilon_{0}}\bra{n\ell m} \frac{\Theta(r_{p} - r)}{r} \ket{n\ell m}
    - \frac{e^{2}}{8\pi\epsilon_{0} r_{p}}\bra{n\ell m} \Theta(r_{p} - r) \ket{n\ell m}
  \]
  and
  \[
    E_{n}^{1} = \bra{\psi_{n}^{0}} H' \ket{\psi_{n}^{0}}
    = \bra{\psi_{n}^{0}} \Theta(r_{p} - r)\qty(\frac{e^{2}}{4\pi\epsilon_{0} r} - \frac{e^{2}r^{2}}{8\pi\epsilon_{0} r_{p}^{3}}
    - \frac{e^{2}}{8\pi\epsilon_{0} r_{p}}) \ket{\psi_{n}^{0}}
  \]
  \[
    = \frac{e^{2}}{4\pi\epsilon_{0}}\bra{n\ell m} \frac{\Theta(r_{p} - r)}{r} \ket{n\ell m}
    - \frac{e^{2}}{8\pi\epsilon_{0}r_{p}^{3}}\bra{n\ell m} r^{2}\Theta(r_{p} - r) \ket{n\ell m}
    - \frac{e^{2}}{8\pi\epsilon_{0} r_{p}}\bra{n\ell m} \Theta(r_{p} - r) \ket{n\ell m}
  \]
  To evaluate these, we need to compute several integrals; the angular variables are trivially 1, but the radial integrals are
  \[
    \bra{n\ell m} \frac{\Theta(r_{p} - r)}{r} \ket{n\ell m} = \int_{0}^{r_{p}}r|R_{n\ell}(r)|^{2}dr
  \]
  \[
    \bra{n\ell m} r^{2}\Theta(r_{p} - r) \ket{n\ell m} = \int_{0}^{r_{p}}r^{3}|R_{n\ell}(r)|^{2}dr
  \]
  \[
    \bra{n\ell m}\Theta(r_{p} - r) \ket{n\ell m} = \int_{0}^{r_{p}}r^{2}|R_{n\ell}(r)|^{2}dr
  \]
  which, since we're going to first-order anyway, can be estimated by exploiting orthogonality with the approximation
  \[
    \int_{0}^{r_{p}}r^{2}|R_{n\ell}(r)|^{2}dr \approx \int_{0}^{\infty}r^{2}|R_{n\ell}(r)|^{2}dr = 1
  \]
  since $R_{n\ell}(r) \sim e^{-r}$ as $r \to \infty$.
  This allows us to prove an inductive formula via integration by parts: forms a base case, and
  \[
    \int_{0}^{r_{p}}r^{3}|R_{n\ell}(r)|^{2}
  \]
  \[
    \int_{0}^{r_{p}}r^{n}|R_{n\ell}(r)|^{2}dr = r_{p}^{n-2} - \int_{0}^{r_{p}}
  \]

\end{proof}

\begin{plm}
  For a two-electron configuration $2p3p$, find the states in $^{2S+1}L_{J}$ notation.
\end{plm}

\begin{proof}
  One electron has $n = 2$, $\ell = 2$, and the other $n = 3$, $\ell = 2$.
  The possible total spin states of the system are the integers from $\frac{1}{2} + \frac{1}{2}$ to $\frac{1}{2} - \frac{1}{2}$, i.e. 1 and 0.
  The possible $L$ are the integers between $4$ and $0$, i.e. 0, 1, 2, 3, 4.
  If the spin is zero, the possible $J$ are 0, 1, 2, 3, 4, and if the spin is one, the possible $J$ are 0, 1, 2, 3, 4, 5.
  We now list all those states: for spin zero, all of $J$ must come from $L$, so
  \[
    {}^{1}S_{0}, {}^{1}P_{1}, {}^{1}D_{2}, {}^{1}F_{3}, {}^{1}G_{4},
  \]
  and for spin-1, each $L$ can have total spin $L + 1$, $L$, or $|L - 1|$, so
  \[
    {}^{3}S_{0}, {}^{3}S_{1}, {}^{3}P_{0}, {}^{3}P_{1}, {}^{3}P_{2}, {}^{3}D_{1}, {}^{3}D_{2}, {}^{3}D_{3}, {}^{3}F_{2}, {}^{3}F_{3} {}^{3}F_{4},
    {}^{3}G_{3}, {}^{3}G_{4}, {}^{3}G_{5}
  \]
  exhausts the possible states.
\end{proof}

\begin{plm}
  Consider the Zeeman splitting of the $^{2}P_{3/2}$ state.
  What are the spacings of the splitting?
  Give an estimate for a magnetic field of \SI{0.1}{T}.
\end{plm}

\begin{proof}
  To first order, the Zeeman effect corrections are of the form
  \[
    \mu_{B}g_{J}B_{ext}m_{j}
  \]
  where $g_{J} = 1 + \frac{j(j+1) - \ell(\ell + 1) + s(s + 1)}{2j(j + 1)}$.
  This is added onto the fine-structure-corrected energy
  \[
    E_{nj} = -\frac{\SI{-13.6}{eV}}{n^{2}}\qty[1 + \frac{\alpha^{2}}{n^{2}}\qty(\frac{n}{j + \frac{1}{2}} - \frac{3}{4})];
  \]
  evaluating for our state,
  \[
    E_{n(3/2)} = -\frac{\SI{-13.6}{eV}}{n^{2}}\qty[1 + \frac{\alpha^{2}}{n^{2}}\frac{2n - 3}{4}];
  \]
  and
  \[
    g_{J} = 1 + \frac{(3/2)(5/2) - (1)(2) + (1/2)(3/2)}{3(5/2)} = \frac{4}{3}.
  \]
  The splitting is then
  \[
    E_{n} = -\frac{\SI{-13.6}{eV}}{n^{2}}\qty[1 + \frac{\alpha^{2}}{n^{2}}\frac{2n - 3}{4}] + \frac{4\mu_{B}}{3}B_{ext}m_{j}.
  \]
  One has $-j \leq m_{j} \leq j$ in increments of $1/2$, so $m_{j} = -3/2, -1, -1/2, 0, 1/2, 1, 3/2$, yielding seven-fold splitting with spacing
  \[
    \frac{2\mu_{B}}{3}B_{ext}
  \]
  For the given field, the spacing is approximately
  \[
    \frac{2(\SI{9.27e-24}{J/T})}{3}(\SI{0.1}{T}) = \SI{6.18e-25}{J} = \SI{3.86e-6}{eV}
  \]
\end{proof}

\end{document}
