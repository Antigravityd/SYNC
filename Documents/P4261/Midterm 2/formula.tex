\documentclass[10pt]{article}

\usepackage[letterpaper]{geometry}
\usepackage{amsmath}
\usepackage{amssymb}
\usepackage{siunitx}
\usepackage{graphicx}


\begin{document}
\begin{tiny}


\section*{Crystal Structure}
A \textbf{lattice} is a vector space but with scalars in a ring instead of a field.
In our case, the ring is $\mathbb{Z}$. A \textbf{unit cell} is any subset of a lattice which may be tiled to recover the whole lattice.
A \textbf{primitive unit cell} is a unit cell that contains exactly one lattice point,
where lattice points are counted according to the angles the incoming and outgoing lattice vectors make.
The set of $\mathbb{R}^n$ points closest to a given lattice point is its \textbf{Wigner-Seitz cell},
constructed by drawing lines from one point to all its neighbors and forming a polygon from the perpendicular bisectors halfway along these lines.
Bases for lattices are constructed by choosing a reference point in the unit cell and labeling every lattice point in the unit cell as a linear combination of lattice vectors over $\mathbb{R}$.
The \textbf{primitive lattice vectors}, the basis for the primitive unit cell, for the elementary lattices considered in this class appear below.
\begin{center}
  \begin{tabular}{ |c|c|c|c| }
    \hline
    Lattice Type & $a_{1}$ & $a_{2}$ & $a_{3}$ \\
    \hline
    \hline
    simple cubic & [1,0,0] & [0,1,0] & [0,0,1] \\
    \hline
    bcc & [1,0,0] & [0,1,0] & $[\frac{1}{2},\frac{1}{2},\frac{1}{2}]$ \\
    \hline
    fcc & $[\frac{1}{2},\frac{1}{2},0]$ & $[\frac{1}{2},0,\frac{1}{2}]$ & $[0,\frac{1}{2},\frac{1}{2}]$ \\
    \hline
  \end{tabular}
\end{center}
The \textbf{reciprocal lattice points} $\vec{G}$ to a direct lattice $\vec{R}$ are defined by
$e^{i\vec{G}\cdot\vec{R}}$.
The primitive lattice vectors of the reciprocal lattice are defined by $\vec{a}_{i}\cdot\vec{b}_{j}=2\pi\delta_{ij}$
and may be constructed as
$\vec{b}_{i}=\frac{2\pi \vec{a}_{i+1}\times\vec{a}_{i+2}}{\vec{a}_{1}\cdot(\vec{a}_{2}\times\vec{a}_{3})}$
where $a_{i}$ are the primitive lattice vectors of the direct lattice and the addition in the subscripts is modulo 3.
One may consider the reciprocal lattice to be Fourier-conjugate to the direct lattice by writing the lattice as a delta function
of the lattice points by $\rho(r)=\sum_{n}\delta(r-an)$ in one dimension,
easily generalizable to more by identification of $r$ with $\vec{k}$ and $an$ with $\vec{G}$.
The Fourier convention for this class is
$\mathcal{F}(f(x))(k)=\int_{\mathbb{R}^{n}}f(x)e^{ik\cdot x}dx$;
the inverse transform is correspondingly
$\mathcal{F}^{-1}(g(k))(x)=\frac{1}{(2\pi)^{n}}\int_{\mathbb{R}^{n}}g(k)e^{-ik\cdot x}$.
A \textbf{lattice plane} is a plane containing 3 non-collinear points of a lattice.
A \textbf{family of lattice planes} is an infinite set of equally-spaced parallel lattice planes which together contain
every lattice point.
Families of lattice planes bijectively correspond to directions of reciprocal lattice vectors,
so that the separation between planes is $d=2\pi/|\vec{G}_{min}|$ where $\vec{G}_{min}$ is the shortest reciprocal lattice vector
in that direction.
The \textbf{Miller indices} are simply an expression of the reciprocal lattice points inside any unit cell in terms of the reciprocal
space vectors conjugate to edge vectors chosen for the unit cell in direct space;
they can be negative, in which case they are conventionally denoted with an overline.
To construct them, one takes the definition of the reciprocal lattice vectors from above and extends it to non-primitive-lattice
vectors.
The Miller indices do not necessarily represent reciprocal lattice vectors unless the unit cell is primitive.
A \textbf{Brillouin zone} is a primitive unit cell of the reciprocal lattice.
The $n$th Brillouin zone may be constructed by taking the perpendicular bisector halfway between the origin and each of the
reciprocal lattice vectors in the unit cell.
The $n$th Brillouin zone is the region where it takes at least $n-1$ crossings of perpendicular bisectors to reach the origin,
where crossing at intersections of bisectors is disallowed.

\section*{Scattering}
Fermi's golden rule states the probability of transitioning from a state $|\vec{k}\rangle$ to $|\vec{k'}\rangle$ per unit time
under a small perturbation of the potential $V$ is
$\Gamma(\vec{k'},\vec{k})=\frac{2\pi}{\hbar}|\langle \vec{k'}|V|\vec{k} \rangle|^{2}\delta(E_{k'}-E_{k})$ where $E_{k'},E_{k}$
are the energies of the associated states and $|k\rangle,|k'\rangle$ are the free-particle wave functions of that which is scattering.
The matrix term may be seen, by assuming everything to be periodic and integrating over the unit cell, to be zero except for those
vectors for which the \textbf{Laue condition} $\vec{k'}-\vec{k}=\vec{G}$ holds.
Equivalent to the Laue condition is the \textbf{Bragg condition}, that $n\lambda = 2d\sin\theta$ where $\lambda$ is the wavelength
of incoming particles, $d$ is the spacing between
\end{tiny}
\end{document}
%%% Local Variables:
%%% mode: latex
%%% TeX-master: t
%%% End:
