\documentclass{article}

\usepackage[letterpaper]{geometry}
\usepackage{siunitx}
\usepackage{amsmath}
\usepackage{amssymb}

\title{4261 HW 2}
\author{Duncan Wilkie}
\date{9 February 2022}

\begin{document}

\maketitle

\section*{1a}
If an electron has a scattering time $\tau$, then its probability of scattering in a time interval of width $dt$ is $\frac{dt}{\tau}$; conversely, its probability of not scattering is $(1-\frac{dt}{\tau})$. While it isn't being scattered, it accrues momentum according to the applied Lorentz force. Presuming in accordance with Drude theory that scattering zeroes out an electron's momentum , if an electron has momentum $\vec{p}(t)$ at time $t$, its momentum at $t+dt$ is, on average,
\[\vec{p}(t+dt)=0\frac{dt}{\tau}+(\vec{p}(t)+\vec{F}dt)\left( 1-\frac{dt}{\tau} \right)\]
Dropping quadratic terms in $dt$,
\[\frac{\vec{p}(t+dt)-\vec{p}(t)}{dt}=\vec{F}-\frac{\vec{p}(t)}{\tau}\Leftrightarrow \frac{d\vec{p}}{dt}=\vec{F}-\frac{\vec{p}}{\tau}\]
Presuming the current in the system is steady-state and there is no or negligible magnetic force, we may write out the Lorentz force term to obtain
\[\vec{p}=\tau e\vec{E}\Leftrightarrow \vec{v}=\frac{e\tau}{m}\vec{E}\]
Current is charge per unit time, so
\[\vec{j}=ne\vec{v}=\frac{ne^2\tau}{m}\vec{E}\]
Ohm's law is $j=\sigma\vec{E}$, so
\[\sigma=\frac{ne^2\tau}{m}\]

\section*{1b}
We continue from the derivation above, but we don't zero out the magnetic field. We may also choose a coordinate system so that $\vec{B}$ lies along the $z$-axis. We then have
\[m\vec{v}=\tau e\left( \vec{E}+\vec{v}\times \vec{B}\right) =\tau e\left( \vec{E}+(v_yB\hat{x}-v_xB\hat{y}) \right)\Leftrightarrow \vec{E}=\frac{m\vec{v}}{\tau e}+v_xB\hat{y}-v_yB\hat{x}\]
Using $\vec{j}=ne\vec{v}$, this is
\[\vec{E}=\frac{m\vec{j}}{n\tau e^2}+\frac{B}{ne}j_x\hat{y}-\frac{B}{ne}j_y\hat{x}\]
Considering this to be of the form
\[\vec{E}=\mathbf{\varrho}\vec{j}\Leftrightarrow
  \begin{pmatrix}
    E_x \\
    E_y \\
    E_z
  \end{pmatrix}
  =
  \begin{pmatrix}
    \rho_{xx} & \rho_{xy} & \rho_{xz} \\
    \rho_{yx} & \rho_{yy} & \rho_{yz} \\
    \rho_{zx} & \rho_{zy} & \rho_{zz}
  \end{pmatrix}
  \begin{pmatrix}
    j_{x}\\
    j_{y} \\
    j_{z}\\
  \end{pmatrix}
\]
we can see, by equating entries in the multiplication to the component of the desired equation,
\[\mathbf{\varrho}=
  \begin{pmatrix}
    \frac{m}{n\tau e^{2}} & -\frac{B}{ne} & 0 \\
    \frac{B}{ne} & \frac{m}{n\tau e^{2}} & 0 \\
    0 & 0 & \frac{m}{n\tau e^{2}}
  \end{pmatrix}
\]
The inverse of this matrix, the conductivity, is by row-reduction
\[\sigma=\varrho^{-1}=
  \begin{pmatrix}
    \frac{mn\tau e^2}{\tau^2e^2B^2+m^2} & \frac{n\tau^2e^3B}{\tau^2e^2B^2+m^2} & 0\\
    -\frac{n\tau^2e^3B}{\tau^2e^2B^2+m^2} & \frac{mn\tau e^2}{\tau^2e^2B^2+m^2} &0 \\
    0 & 0 & \frac{n\tau e^{2}}{m}
  \end{pmatrix}
\]
This was done via the Python script below, to avoid a pointlessly long hand-computation (do let me know if this is unacceptable):
\begin{verbatim}
Python 3.9.10 (main, Feb  9 2022, 19:11:53)
[GCC 11.2.1 20211127] on linux
Type "help", "copyright", "credits" or "license" for more information.
>>> import sympy as s
>>> m, n, t, e, B = s.symbols("m n t e B")
>>> r = s.Matrix([[m/(n*t*e**2), -B/(n*e), 0], \
                  [B/(n*e), m/(n*t*e**2), 0], \
                  [0, 0, m/(n*t*e**2)]])
>>> r.inv()
Matrix([
[e**2*m*n*t/(B**2*e**2*t**2 + m**2),     B*e**3*n*t**2/(B**2*e**2*t**2 + m**2), 0],
[-B*e**3*n*t**2/(B**2*e**2*t**2 + m**2), e**2*m*n*t/(B**2*e**2*t**2 + m**2),    0],
[0,                                      0,                            e**2*n*t/m]])
\end{verbatim}

\section*{1c}
The Hall coefficient is
\[R_{H}=\frac{\rho_{xy}}{|\vec{B}|}=-\frac{1}{ne}\]
Presuming there is one charge carrier per atom and using the molar mass of sodium, there are $(\SI{6.02e23}{atoms/mol})/(\SI{23}{g/mol})=\SI{2.62e22}{atoms/g}$, and the density of sodium is around a gram per cubic centimeter, so  $n\approx \SI{2.62e22}{carriers/cm^{3}}$. We then may calculate
\[R_{H}=-\frac{1}{(\SI{2.62e22}{cm^{-3}})(\SI{1.6e-19}{C})}=\SI{-2e-4}{cm^{3}/C}=\SI{-2e-10}{m^{3}/C}\]
The Hall voltage is
\[V_{H}=R_{H}\frac{IB}{d}=(\SI{-2e-10}{m^{3}/C})\frac{(\SI{1}{A})(\SI{1}{T})}{\SI{5e-3}{m}}=\SI{-4.02e-8}{V}\]
This is much lower than the precision most multimeters are capable of. However, an op-amp could be used with inputs of the Hall voltage and a very stable ground; the output would be the difference magnified by a factor on the order of $10^{3}$. This amplification may of course be chained as desired, being cognizant of the error propagation which may occur.

\section*{1d}
Sometimes, the sign of the charge carrier in the metal is measured to be positive rather than negative as predicted/assumed by Drude theory, and the $\frac{3k_{B}}{2}$ heat capacity per-particle derived from the kinetic theory approximation is extremely inconsistent with experiment.

\section*{1e}
Following similar lines to part (b), we presume $\tau\to\infty$, that $\vec{E}$ lies in the $x$-$y$ plane, and $\vec{j}=\vec{k}e^{i\omega t}$.
This yields, since the fields, forces, and currents aren't steady-state this time,
\[\frac{d\vec{p}}{dt}=e\left( \vec{E}+\vec{v}\times E \right)-\frac{\vec{p}}{\tau}\Leftrightarrow \frac{mi\omega\vec{j}}{ne}=e\left( \vec{E}+\frac{1}{ne}\vec{j}\times \vec{B} \right)-\frac{m\vec{j}}{ne\tau}\]
\[\Leftrightarrow \vec{E}=\frac{mi\omega\vec{j}}{ne^2}+\frac{B}{ne}j_{y}\vec{x}- \frac{B}{ne}j_{x}\vec{y}\]
This is an equation identical in form to that of part (b); equating components in the previous two equations yields
\[\varrho=
  \begin{pmatrix}
    \frac{mi\omega}{ne^{2}} & \frac{-B}{ne} & 0 \\
    \frac{B}{ne} & \frac{mi\omega}{ne^{2}} & 0 \\
    0 & 0 & \frac{mi\omega}{ne^{2}}
  \end{pmatrix}
\]
Inverting this by the same method,
\[\varsigma(\omega)=
  \begin{pmatrix}
    \frac{imn\omega e^{2}}{e^{2}B^{2}-m^{2}\omega^{2}} & \frac{ne^{3}B}{e^{2}B^{2}-m^{2}\omega^{2}} & 0 \\
    -\frac{n e^{3}B}{e^{2}B^{2}-m^{2}\omega^{2}} & \frac{imn\omega e^{2}}{e^{2}B^{2}-m^{2}\omega^{2}} & 0 \\
    0 & 0 & \frac{imn\omega e^{2}}{e^{2}B^{2}-m^{2}\omega^{2}}
  \end{pmatrix}
\]
These terms all diverge at $\omega=\frac{eB}{m}$, meaning the metal is a perfect conductor at that frequency. For an imperfect metal, this is the frequency at which the metal conducts best. Measuring the position of a peak of conductivity as a function of temperature of a metal subjected to a magnetic field would give an estimate of the electron mass in terms of the magnetic field's magnitude and the known electron charge as $m=\frac{eB}{\omega_{max}}$

\section*{2a}
The Fermi energy is roughly the value of the highest-energy occupied state of the system (technically, this value's average with the lowest unoccupied state just above it). The Fermi temperature is the temperature that corresponds to this energy value, and the Fermi surface is the boundary in wave-vector-space of the ball of electrons at absolute zero.

\section*{2b}
The total number of particles at $T=0$ is, in terms of the Heaviside step function $\Theta$,
\[N=2\frac{V}{(2\pi)^{3}}\int\Theta(E_{F}-\epsilon(\vec{k}))d\vec{k}=2\frac{V}{(2\pi)^{3}}\int_{k=0}^{k_{F}}d\vec{k}\]
\
\[\Leftrightarrow N=2\frac{V}{(2\pi)^{3}}\left( \frac{4}{3}\pi k_{F}^{3} \right)\Leftrightarrow k_{F}=\left( 3\pi^{2}\frac{N}{V} \right)^{1/3}\]
This implies the Fermi energy is
\[E_{F}=\frac{\hbar^{2}k_{F}^{2}}{2m}=\frac{\hbar^{2}(3\pi^{2}\frac{N}{V})^{2/3}}{2m}\]
Rearranging for $N$,
\[N=\frac{V}{3\pi^{2}}\left( \frac{2mE_{F}}{\hbar^{2}} \right)^{3/2}\]
Differentiating with respect to $E_{F}$,
\[\frac{dN}{dE_{F}}=\frac{3}{2}\frac{V}{3\pi^{2}}\left( \frac{2mE_{F}}{\hbar^{2}} \right)^{1/2}\left( \frac{2m}{\hbar^{2}} \right)=\frac{3}{2}\left( \frac{V}{3\pi^{2}}\left( \frac{2m}{\hbar^{2}} \right)^{3/2} \right)E_{F}^{{3/2 - 1}}=\frac{3}{2}\frac{N}{E_{F}}\]

\section*{2c}
The value for $n=\frac{N}{V}$ for sodium was found above to be $\SI{2.62e22}{carriers/cm^{3}}$, so the Fermi energy is
\[E_{F}=\frac{\hbar^{2}\left( 3\pi^{2}n \right)^{{2/3}}}{2m}=\frac{(\SI{1.05e-34}{J\cdot s})^{2}\left( 3\pi^{2}(\SI{2.62e28}{carriers/m^{3}}) \right)^{2/3}}{2(\SI{9.11e-31}{kg})}\]\[=\SI{5.11e-19}{J}=\SI{3.19}{eV}\]

\section*{2d}
This is identical to part (b), but the prefactor and integral are different since they depend on the dimension of $k$. Simply replacing them with their two-dimensional analogs yields
\[N=2\frac{A}{(2\pi)^{2}}\left( \pi k_{F}^{2} \right)\Leftrightarrow k_{F}=2\pi\sqrt{\frac{N}{2\pi A}}\]
Plugging this in to the equation for the Fermi energy,
\[E_{F}=\frac{\hbar^{2}k_{F}^{2}}{2m}=\frac{\hbar^{2}(4\pi^{2}N/2\pi A)}{2m}=\frac{\pi\hbar^{2}N}{mA}\]
Solving for $N$ and differentiating,
\[N=\frac{mAE_{F}}{\pi\hbar^{2}}\Rightarrow \frac{dN}{dE_{F}}=\frac{mA}{\pi\hbar^{2}}\]

\section*{3a}
The Fermi energy is
\[E_{F}=\frac{\hbar^{2}k_{F}^{2}}{2m}\]
Letting this be the kinetic energy of the electron at the Fermi radius, we may find the momentum by
\[E_{F}=\frac{p_{F}^{2}}{2m}\Rightarrow p_{F}=\hbar k_{F}\]
Dividing by the mass of the electron and plugging in the expression for the Fermi radius, we obtain the velocity
\[v_{F}=\frac{\hbar}{m}\left( 3\pi^{2}n \right)^{1/3}\]

\section*{3b}
By Ohm's law $\vec{j}=\sigma\vec{E}$ if $\vec{j}=ne\vec{v}_{d}$ (the drift velocity is that due to the electric field) we then have
\[ne\vec{v}_{d}=\sigma\vec{E}\Rightarrow |\vec{v}_{d}|=\left| \frac{\sigma \vec{E}}{ne}\right|\]
Using the Drude theory equations in steady-state,
\[\frac{d\vec{p}}{dt}=F-\frac{\vec{p}}{\tau}\Rightarrow e\vec{E}=\frac{\vec{p}}{\tau}=\frac{mv_{d}}{\tau}\Rightarrow \vec{j}=\frac{ne^{2}\tau\vec{E}}{m}\]
implying the conductivity is
\[\sigma=\frac{ne^{2}\tau}{m}\]
If the electrons are all moving with velocity on average $v_{F} \gg v_{d}$, then $\tau = \frac{\lambda}{v_{F}}$, yielding the equation
\[\sigma = \frac{ne^{2}\lambda}{mv_{F}}\]

\section*{3c}
The drift velocity is
\[v_{d}=\left| \frac{(\SI{5.9e7}{\ohm^{-1}m^{-1}})(\SI{1}{V/m})}{(\SI{8.45e28}{m^{-3}})(\SI{1.6e-19}{C})} \right|=\SI{4.36}{mm/s}\]
The Fermi velocity is
\[v_{F}=\frac{\hbar}{m}(3\pi^{2}n)^{1/3}=\frac{\SI{1.05e-34}{J\cdot s}}{\SI{9.11e-31}{kg}}(3\pi^{2}(\SI{8.45e28}{m^{-3}}))^{1/3}=\SI{1565}{km/s}\]
This is a difference on the order of $10^{9}$. Very large.

The mean free path of an electron in copper is approximately
\[\lambda=\frac{mv_{F}\sigma}{ne^{2}}=\frac{(\SI{9.11e-31}{kg})(\SI{1.565e6}{m/s})(\SI{5.9e7}{\ohm^{-1}m^{-1}})}{(\SI{8.45e28}{m^{-3}})(\SI{1.6e-19}{C})^{2}}=\SI{3.89e-8}{m}=\SI{38.9}{nm}\]
The spacing between copper atoms, as given by its lattice constant, is around $\SI{3.5}{\angstrom}$, so on average electrons travel about 100 atoms' distance before scattering.

\section*{4a}
At a temperature $T$, the electrons have energy $k_{B}T$. If this energy is increased, it will be by roughly a factor of $k_{B}T$, referring to the plot of $n_{F}(E/E_{F})$ in the book. Using the density of states to weight this energy value by the probability any given state will be close to the range in question, the expected total energy of the system is
\[E\approx(k_{B}T)^{2}D(E_{F})\]
Differentiating,
\[c=2k_{B}TD(E_{F})\]
This resembles the Fermi prediction in that it is linear in temperature; little else.

\section*{4b}
The Hamiltonian of a free electron gas in the presence of an external magnetic field is, including only the magnetic effect of spin alignment,
\[\mathcal{H}=\frac{\vec{p}^{2}}{2m}+g\mu_{B}\vec{B}\cdot\vec{\sigma}\]
where $g=2$ is the electron g-factor, $\mu_{B}=\frac{e\hbar}{2m_{e}}$ is the Bohr magneton, and $\sigma$ is the electron spin. If an electron is parallel to the magnetic field, its energy is therefore
\[E_{p}=\frac{\hbar\vec{k}^{2}}{2m}+\mu_{B}\vec{B}\]
and if it is antiparallel its energy is
\[E_{a}=\frac{\hbar\vec{k}^{2}}{2m}-\mu_{B}\vec{B}\]
The total energy is the sum of the total of each of these over its respective spin direction.
The magnetization is dipole moment per unit volume, and so is
\[M=-\frac{1}{V}\frac{dE}{dB}=-(\textrm{\# spin-up}-\textrm{\# spin-down})\frac{\mu_{B}}{V}\]
Near the Fermi energy in the absence of a magnetic field, the number of spin-up sates per unit volume is $g(E_{F})/2$, and the same for the spin-down states. When the magnetic field is applied, the spin-up states are $\mu_{B}B$ more energy-costly than they were previously, and the spin-down states are $\mu_{B}B$ less costly. This implies there will be $\mu_{B}Bg(E_{F})/2$ fewer electrons in spin-up states and the same amount more in spin-down states. The new moment is therefore
\[M=g(E_{F})\mu_{B}^{2}B\]
It follows that the magnetic susceptibility is
\[\chi=\frac{dM}{dH}=\mu_{0}\frac{dM}{dB}=\mu_{0}\mu_{B}^{2}g(E_{F})\]

\section*{4c}
The magnetic susceptibility of a classical electron gas wouldn't take into account the spins of the electrons, since point particles don't have angular momenta; it would only encompass the Lorentz force's effects.
Classically, the heat capacity would be nothing more than that of a monatomic gas, and so would be $3k_{B}$ per atom. This is quite different from the prediction above; at around \SI{100}{K} it's a hundredfold difference.

\section*{4d}
The linear term originates from the behavior of outer electrons of the atom whenever the material is at low enough temperature that most of the electrons are in their ground state. That's why it dominates at low temperature.
The cubic term originates from Debye theory, treating large oscillations (like those exhibited at high temperature) between molecules as quantized sound waves.

The coefficient of the linear term of temperature obtained from the Fermi electronic heat capacity is (ignoring the $T^{3}$ term as approximately zero)
\[\gamma=\frac{\pi^{2}}{3}\frac{3Nk_{B}}{2T_{F}}\Rightarrow E_{F}=k_{B}T_{F}=\frac{\pi^{2}}{3}\frac{3Nk_{B}^{2}}{2\gamma}=\frac{\pi^{2}}{3}\frac{3(\SI{6.02e23}{})(\SI{1.38e-23}{J/K})^{2}}{2(\SI{2.6e-3}{J/mol\cdot K^{4}})}=\SI{2.18e-19}{J}\]\[=\SI{1.36}{eV}\]
\end{document}
%%% Local Variables:
%%% mode: latex
%%% TeX-master: t
%%% End:
