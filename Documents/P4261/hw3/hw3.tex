\documentclass{article}

\usepackage[letterpaper]{geometry}
\usepackage{amsmath}
\usepackage{amssymb}
\usepackage{siunitx}
\usepackage{graphicx}

\title{4261 HW 3}
\author{Duncan Wilkie}
\date{?}

\begin{document}

\maketitle

\section*{1a}
Ionic bonds occur between atoms with high electronegativity differences,
particularly between atoms from the extreme left and extreme right of the periodic table,
i.e. pairs from groups 1 and 7 and from groups 2 and 6;
they are characterized by the electrons from one atom leaving their parent atom and becoming more strongly localized around another,
forming a positively charged donor ion and a negatively charged recipient.
These tend to be insulating, hard \& brittle, and soluble in polar solvents like water.
Covalent bonds are characterized by the electron not being localized to one or the other atom but ``shared'' between atoms,
with a significant probability of the bonding electron being found in the vicinity of either atom.
Such bonds are common between atoms with lower electronegativity differences, tend to result in hard, high-melting-temperature insulators.
Metallic bonds are characterized by de-localization of the bonding electron beyond the two bonding atoms,
allowing electrons to freely move throughout the material.
Commonly formed from atoms in the transition metals and groups 1 and 2, these materials tend to be ductile with relatively low
melting temperatures and conduct heat and electricity well.
Hydrogen bonds are formed in water and organic molecules as a result of the bare proton in a hydrogen nucleus whose electron is occupied by
a covalent bond having a Coulomb interaction with the electrons on nearby atoms. This is a relatively weak interaction,
though stronger than van der Waals.
The van der Waals interaction results from momentary alignment of the fluctuating dipole moments of electrons between molecules or atoms.
This is a very weak interaction, resulting in low melting points and soft, electrically-insulating materials, but among atoms with
very uninteresting electronic structures (mostly group 8) or very large molecules (e.g. long-chain lipids) it can be the dominant
interaction.

\section*{1b}
If an atom is thought of as instantaneously classical, as two point charges separated by some vector $\vec{d}$,
the atom will have a dipole moment $\vec{p}=e\vec{d}$ and instantaneously radiate a dipole field
\[\vec{E}=\frac{\vec{p}}{4\pi\epsilon_{0}r^{3}}\]
A second atom will have a dipole moment induced by this electric field, oriented in the opposite direction of the original dipole.
Oppositely-oriented dipoles attract; this is the van der Waals force.
The potential between the two dipoles will consequently be, presuming the second atom to linearly susceptible,
\[U=\frac{-|\vec{p}|\chi\vec{E}}{4\pi\epsilon_{0}r^{3}}=-\frac{|\vec{p}|^{2}\chi}{(4\pi\epsilon_{0}r^{3})^{2}}\]
The negative gradient of this potential, the van der Waals force, will therefore be proportional to $1/r^{7}$.

\section*{1c}
The energy change from the formation of NaCl is
\[\Delta E_{A+B\rightarrow AB}=(\textrm{ionization energy})_{A}-(\textrm{electron affinity})_{B}-(\textrm{cohesive energy of }AB)\]
The cohesive energy is mostly Coulomb, so we may estimate it as
\[
  (\textrm{cohesive energy of }AB)
  =\frac{1}{4\pi\epsilon_{0}}\frac{e^{2}}{r}
  =\frac{1}{4\pi(\SI{8.85e-12}{F/m})}\frac{(\SI{1.6e-19}{C})^{2}}{(\SI{0.236e-9}{m})}
  =\SI{9.75e-19}{J}=\SI{6.10}{eV}
\]
The energy change is therefore
\[
  \Delta E_{A+B\rightarrow AB}=\SI{5.14}{eV}-\SI{3.62}{eV}-\SI{6.10}{eV}=\SI{-4.58}{eV}
\]
The sign being negative corresponds to energy being released.
This more than the experimental value by $\SI{0.32}{eV}$.
The direction of the discrepancy is consistent with it being largely due to using the ionization energy
(i.e. energy required to lift the Na valence electron to infinity) and the electron affinity
(i.e. energy gained by letting an electron fall towards Cl from infinity) rather than the analogous values for the finite displacement
of $\SI{0.236}{nm}$.

\section*{2a}
The gradient in $\phi_{n}$ will vanish at the minimum, so
\[
  \nabla E=\nabla\left[\frac{\left( \sum_{n}\phi_{n}^{*}\langle n | \right)
      H\left(\sum_{m}\phi_{m}|m\rangle \right)}
    {\sum_{n}\phi_{n}\phi_{n}^{*}}\right]
  =\nabla\left[ \frac{\sum_{m,n}\phi_{n}^{*}\phi_{m}\langle n|H|m \rangle}{\sum_{n}\phi_{n}\phi_{n}^{*}} \right]
\]
\[
  =\nabla\left[ \frac{\sum_{m,n}\phi_{n}^{*}\phi_{m}\mathcal{H}_{n,m}}{\sum_{n}\phi_{n}\phi_{n}^{*}} \right]
\]
Each component will have the form
\[
  0=\frac{\partial E}{\partial \phi_{i}}|i\rangle
  \Rightarrow 0 = \frac{\sum_{m}\mathcal{H}_{n,m}\phi_{m}}{\sum_{n}|\phi_{n}|^{2}}
  -\left(  \frac{\sum_{n,m}\phi_{n}^{*}\mathcal{H}_{n,m}\phi_{m}}{\sum_{n}|\phi_{n}|^{2}}\right)\frac{\phi_{n}}{\sum_{n}|\phi_{n}|^{2}}
\]
Multiplying by the denominator sum,
\[\Rightarrow \sum_{m}\mathcal{H}_{n,m}\phi_{m}-E\phi_{n}=0\]
where we have re-substituted the original supposition of the form of $E$.
This is exactly the statement $\mathcal{H}\vec{\phi}=E\vec{\phi}$.

\section*{2b}
The two expressions for $V_{cross}$ and $t$ are equivalent because the atoms are indistinguishable, so one ought to be able to relabel
one's description of the system with the indices swapped without changing the physics.
The effective Hamiltonian may be written
\[\mathcal{H}=\langle n|H|m  \rangle=\langle n|K+V_{1}+V_{2}|m \rangle=
  \begin{pmatrix}
    \langle 1|K+V_{1}+V_{2}|1 \rangle & \langle 1 | K+V_{1}+V_{2}|2 \rangle \\
    \langle  2|K+V_{1}+V_{2}|1\rangle & \langle 2 | K+V_{1}+V_{2}|2 \rangle
  \end{pmatrix}
\]
Using $\langle 2|V_{2}|1  \rangle=\langle 1|V_{2}|2 \rangle^{*}$ because observables are Hermitian,
\[
  =
  \begin{pmatrix}
    \langle 1|\epsilon|1 \rangle + V_{cross} & \langle 1|\epsilon|2  \rangle -t \\
    \langle 2|\epsilon|1 \rangle - t^{*} & \langle 2|\epsilon|2 \rangle + V_{cross}
  \end{pmatrix}
\]
By orthonormality and linearity,
\[
  =
  \begin{pmatrix}
    \epsilon+V_{cross} & -t \\
    -t^{*} & \epsilon+V_{cross}
  \end{pmatrix}
\]
This has eigenvalue equation
\[
  \det(\mathcal{H}-\lambda I) = 0
  \Leftrightarrow
  \begin{vmatrix}
    \epsilon+V_{cross}-\lambda & -t \\
    -t^{*} & \epsilon+V_{cross}-\lambda
  \end{vmatrix}=0
  \Leftrightarrow (\epsilon+V_{cross}-\lambda)^{2}-tt^{*}=0
\]
\[
  \Leftrightarrow \lambda^{2}-(2V_{cross}+2\epsilon)\lambda + V_{cross}^{2}+2\epsilon V_{cross}+\epsilon^{2}-|t|^{2}=0
\]
\[
  \Leftrightarrow \lambda =\frac{2V_{cross}+2\epsilon\pm\sqrt{(2V_{cross}+2\epsilon)^{2}
      -4(V_{cross}^{2}+\epsilon^{2}+2\epsilon V_{cross}-|t|^{2})}}{2}
  =V_{cross}+\epsilon\pm|t|^{2}
\]
as desired.
The $V_{cross}$ term is due to the Coulomb force on one electron from the opposite nucleus, and for large distances this is approximately
the same magnitude but the opposite sign as the Coulomb potential due to the nuclear interaction.
Therefore, if we include the nuclear interaction in the energy eigenstates it will approximately cancel the cross term.
Equivalently, by Gauss's law the fact that the atom is electrically neutral implies the field goes to zero, at least for large distances.
This means the Coulomb force of one atom on the other's electron is zero, i.e. $V_{cross}=0$.
If this approximation were to hold for arbitrary inter-atomic separation, there would be a difference in energy between the two
eigenstates as the separation goes to zero, which would imply the lowest energy and therefore typical state of the system is
with zero separation.

\end{document}
%%% Local Variables:
%%% mode: latex
%%% TeX-master: t
%%% End:
