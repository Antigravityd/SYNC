\documentclass{article}

\usepackage[letterpaper]{geometry}
\usepackage{siunitx}
\usepackage{amsmath}
\usepackage{amssymb}
\usepackage{graphicx}

\title{4261 HW 4}
\author{Duncan Wilkie}
\date{?}

\begin{document}

\maketitle

\section*{1a}
The second term's exponent is 6 to correspond with the distance dependence of the van der Waals interaction, which also has $O(1/r^6)$
behavior as $x\to\infty$.
\section*{1b}
The minimum of the L-J potential occurs at
\[
  V'(x)=-48\epsilon\frac{\sigma^{12}}{x^{13}}+24\epsilon\frac{\sigma^{6}}{x^{7}}=0
  \Leftrightarrow x^{6}=2{\sigma^{6}}
  \Rightarrow x_{0}=\sigma \sqrt[6]{2}
\]
By the definition of Taylor series,
\[
  \kappa=V^{(2)}(x_{0})=624\epsilon\frac{\sigma^{12}}{x_{0}^{14}}-168\epsilon\frac{\sigma^{6}}{x_{0}^{8}}
  =\left(\frac{624}{2^{7/3}}-\frac{168}{2^{4/3}}\right)\frac{\epsilon}{\sigma^{2}}=57.15\frac{\epsilon}{\sigma^{2}}
\]
\[
  \kappa_{3}=V^{(3)}(x_{0})=-8736\epsilon\frac{\sigma^{12}}{x_{0}^{15}}+1344\epsilon\frac{\sigma^{6}}{x_{0}^{9}}
  =\left( \frac{1344}{2^{3/2}}-\frac{8736}{2^{5/2}}\right)\frac{\epsilon}{\sigma^{3}}=-1069\frac{\epsilon}{\sigma^{3}}
\]

\section*{2a}
The expansion is derived by
\[
  e^{-\beta V(x)}=e^{-\beta \left[ \frac{\kappa}{2}(x-x_{0})^{2}-\frac{\kappa_{3}}{6}(x-x_{0})^{3}+... \right]}
  =e^{-\beta\frac{\kappa}{2}(x-x_{0})^{2}}e^{-\beta\left[ -\frac{\kappa_{3}}{6}(x-x_{0})^{3}+... \right]}
\]
\[
  =e^{-\beta\frac{\kappa}{2}(x-x_{0})^{2}}
  \left[ 1+\beta(\frac{\kappa_{3}}{6}(x-x_{0})^{3}-...)+\frac{\beta}{2}(\frac{\kappa_{3}}{6}(x-x_{0})^{3}-...)^{2} +...\right]
\]
\[
  =e^{-\beta\frac{\kappa}{2}(x-x_{0})^{2}}\left[ 1+\frac{\beta\kappa_{3}}{6}(x-x_{0})^{3}+O(x^{4}) \right]
\]
The limits of integration may be taken to be infinite since the multiplier term will dominate in the limits and take the
contributions to the integral outside the correct interval to zero.

\section*{2b}
Substituting the expansion into the expectation expression,
\[
  \langle x \rangle_{\beta}=\frac{\int_{\mathbb{R}}xe^{-\frac{\beta\kappa}{2}(x-x_{0})^{2}}
    +x\frac{\beta\kappa_{3}}{6}(x-x_{0})^{3}e^{-\frac{\beta\kappa}{2}(x-x_{0})^{2}}dx}
  {\int_{\mathbb{R}}e^{-\frac{\beta\kappa}{2}(x-x_{0})^{2}}+\frac{\beta\kappa_{3}}{6}(x-x_{0})^{3}e^{-\frac{\beta\kappa}{2}(x-x_{0})^{2}}dx}
\]
In the numerator, the first term is odd and the second is even.
In the denominator, the first term is even and the second is odd.
Since the integration is over a symmetric interval, we may write
\[=\frac{\frac{\beta\kappa_{3}}{6}\int_{0}^{\infty}x(x-x_{0})^{3}e^{-\frac{\beta\kappa}{2}(x-x_{0})^{2}}dx}
  {\int_{0}^{\infty}e^{-\frac{\beta\kappa}{2}(x-x_{0})^{2}}dx}\]
The Gaussian integral in the denominator evaluates to $\sqrt{\frac{\pi}{2\beta\kappa}}$.

\end{document}
%%% Local Variables:
%%% mode: latex
%%% TeX-master: t
%%% End:
