\documentclass{article}

\usepackage[letterpaper]{geometry}
\usepackage{siunitx}
\usepackage{amsmath}
\usepackage{amssymb}
\usepackage{graphicx}

\title{4261 HW 4}
\author{Duncan Wilkie}
\date{?}

\begin{document}

\maketitle

\section*{1a}
The second term's exponent is 6 to correspond with the distance dependence of the van der Waals interaction, which also has $O(1/r^6)$
behavior as $x\to\infty$.
\section*{1b}
The minimum of the L-J potential occurs at
\[
  V'(x)=-48\epsilon\frac{\sigma^{12}}{x^{13}}+24\epsilon\frac{\sigma^{6}}{x^{7}}=0
  \Leftrightarrow x^{6}=2{\sigma^{6}}
  \Rightarrow x_{0}=\sigma \sqrt[6]{2}
\]
By the definition of Taylor series,
\[
  \kappa=V^{(2)}(x_{0})=624\epsilon\frac{\sigma^{12}}{x_{0}^{14}}-168\epsilon\frac{\sigma^{6}}{x_{0}^{8}}
  =\left(\frac{624}{2^{7/3}}-\frac{168}{2^{4/3}}\right)\frac{\epsilon}{\sigma^{2}}=57.15\frac{\epsilon}{\sigma^{2}}
\]
\[
  \kappa_{3}=V^{(3)}(x_{0})=-8736\epsilon\frac{\sigma^{12}}{x_{0}^{15}}+1344\epsilon\frac{\sigma^{6}}{x_{0}^{9}}
  =\left( \frac{1344}{2^{3/2}}-\frac{8736}{2^{5/2}}\right)\frac{\epsilon}{\sigma^{3}}=-1069\frac{\epsilon}{\sigma^{3}}
\]

\section*{2a}
The expansion is derived by
\[
  e^{-\beta V(x)}=e^{-\beta \left[ \frac{\kappa}{2}(x-x_{0})^{2}-\frac{\kappa_{3}}{6}(x-x_{0})^{3}+... \right]}
  =e^{-\beta\frac{\kappa}{2}(x-x_{0})^{2}}e^{-\beta\left[ -\frac{\kappa_{3}}{6}(x-x_{0})^{3}+... \right]}
\]
\[
  =e^{-\beta\frac{\kappa}{2}(x-x_{0})^{2}}
  \left[ 1+\beta(\frac{\kappa_{3}}{6}(x-x_{0})^{3}-...)+\frac{\beta}{2}(\frac{\kappa_{3}}{6}(x-x_{0})^{3}-...)^{2} +...\right]
\]
\[
  =e^{-\beta\frac{\kappa}{2}(x-x_{0})^{2}}\left[ 1+\frac{\beta\kappa_{3}}{6}(x-x_{0})^{3}+O(x^{4}) \right]
\]
The limits of integration may be taken to be infinite since the multiplier term will dominate in the limits and take the
contributions to the integral outside the correct interval to zero.

\section*{2b}
Substituting the expansion into the expectation expression,
\[
  \langle x \rangle_{\beta}=\frac{\int_{\mathbb{R}}xe^{-\frac{\beta\kappa}{2}(x-x_{0})^{2}}
    +x\frac{\beta\kappa_{3}}{6}(x-x_{0})^{3}e^{-\frac{\beta\kappa}{2}(x-x_{0})^{2}}dx}
  {\int_{\mathbb{R}}e^{-\frac{\beta\kappa}{2}(x-x_{0})^{2}}+\frac{\beta\kappa_{3}}{6}(x-x_{0})^{3}e^{-\frac{\beta\kappa}{2}(x-x_{0})^{2}}dx}
\]
Through even/odd arguments,
\[
  =\frac{\frac{\beta\kappa_{3}}{6}\int_{\mathbb{R}}x(x-x_{0})^{3}e^{-\frac{\beta\kappa}{2}(x-x_{0})^{2}}dx}
  {\int_{0}^{\infty}e^{-\frac{\beta\kappa}{2}(x-x_{0})^{2}}dx}
  =\frac{\beta\kappa_{3}}{6}\sqrt{\frac{\beta\kappa}{2\pi}}\int_{\mathbb{R}}(x+x_{0})x^{3}e^{-\beta\kappa x^{2}/2}dx
\]
\[
  =\frac{\beta\kappa_{3}}{6}\sqrt{\frac{\beta\kappa}{2\pi}}\int_{\mathbb{R}}x^{4}e^{-\beta\kappa x^{2}/2}dx
  =\frac{\beta\kappa_{3}}{6}\sqrt{\frac{\beta\kappa}{2\pi}}\frac{3\sqrt{2\pi}}{(\beta\kappa)^{5/2}}
\]
\[
  =\frac{\kappa_{3}}{2\beta\kappa^{2}}
\]
where the Gaussian-type integrals are computed via table.
Plugging this into the equation for the thermal expansion given, we do indeed obtain
\[
  \alpha=\frac{1}{x_{0}}\frac{d\langle x \rangle_{\beta}}{dT}=\frac{1}{x_{0}}\frac{d}{dT}\left( \frac{\kappa_{3}k_{B}T}{2\kappa^{2}} \right)
  =\frac{1}{x_{0}}\frac{k_{B}k_{3}}{2\kappa^{2}}
\]

\section*{2c}
This expansion is valid for low temperature, when $x\approx x_{0}$ because energy fluctuations due to temperature don't perturb
the interatomic spacing too far from its minimum value.

\section*{2d}
This model doesn't take into account quantum effects, since the interatomic vibrational energy is actually quantized into phonons.

\section*{3a}
From the model in the text, the predicted speed of sound is
\[
  v=\sqrt{\frac{\kappa a^{2}}{m}}
\]
where $\kappa$ is half the coefficient of the quadratic term of the interatomic potential Taylor expansion about its minimum,
$a$ is the equilibrium spacing between atoms (the minimum of the interatomic potential), and $m$ is the mass of the atoms.
In the case of nitrogen with a Lennard-Jones potential,
\[
  v=\sqrt{\frac{[57.15(\SI{10}{meV})/(\SI{0.34}{nm})^{2}][\SI{0.34}{nm}]^{2}2^{1/3}}{(\SI{39.9}{amu})}}
  =\SI{1300}{m/s}
\]

\section*{3b}
Simply computing,
\[
  \alpha=\frac{1}{\sigma\sqrt[6]{2}}\frac{k_{B}\kappa_{3}}{2\kappa^{2}}
  =\frac{1}{(\SI{0.34}{nm})\sqrt[6]{2}}\frac{(\SI{1.38e-23}{J/K})[-1069(\SI{10}{meV})/(\SI{0.34}{nm})^{3}]}
  {2(57.15\SI{10}{meV}/(\SI{0.34}{nm})^{2})^{2}}
  =\SI{1.26e-3}{K^{-1}}
\]
\end{document}
%%% Local Variables:
%%% mode: latex
%%% TeX-master: t
%%% End:
