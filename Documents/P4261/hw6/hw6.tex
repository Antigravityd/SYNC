\documentclass{article}

\usepackage[letterpaper]{geometry}
\usepackage{amsmath}
\usepackage{amssymb}
\usepackage{siunitx}
\usepackage{graphicx}

\title{4261 HW 6}
\author{Duncan Wilkie}
\date{?}

\begin{document}

\maketitle

\section*{1a}
We have an effective Schr\"odinger equation, following the cited problem,
\[
  E\phi_{n}=\epsilon\phi_{n}-t(\phi_{n+1}+\phi_{n-1})
\]
Applying the ansatz $\phi_{n}=e^{ikna}/\sqrt{N}$, where $N$ is some normalization constant, we get
\[
  Ee^{ikna}=\epsilon e^{ikna}-te^{ik(n+1)a}-te^{ik(n-1)a}
  \Leftrightarrow E=\epsilon-t(e^{ika}+e^{-ika})
  =\epsilon-2t\cos(ka)
\]
A plot with random values for the constants appears below.
\[
  \includegraphics[scale=.8]{plot1.png}
\]
If there are $N$ sites, and we have periodic boundary conditions $k=k+2\pi/a$, the total system $k$ is quantized as $k=2\pi m/(Na)$,
i.e. there are $N$ values of $k$ corresponding bijectively to $N$ eigenstates of the system.
Approximating $\cos x= 1-x^{2}/2$ near zero,
\[E(k)\approx \epsilon-2t+tk^{2}a^{2}\]
Comparing this to the free electron dispersion,
\[
  \frac{\hbar^{2} k^{2}}{2m^{*}}=ta^{2}k^{2}
  \Leftrightarrow m^{*}=\frac{\hbar^{2}}{2ta^{2}}
\]
The density of states is by definition
\[
  g(E)=\frac{dN}{dE}=\frac{dN}{dk}\frac{dk}{dE}
\]
The first term is constant, since the number of states per unit $k$ is constant; it is equal to $Na/2\pi$.
The second is, from the dispersion relation, $1/2ta\sin(ka)$, so in total we have
\[
  g(E)=\frac{N}{4\pi t\sin(ka)}
\]
Monovalent atoms imply half-filled bands, in which case the Fermi surface (the boundary between filled and unfilled states) occurs halfway
through the first Brillouin zone at $k=\pi/2a$ and $k=-\pi/2a$, which corresponds to a density of states $N/4\pi t$.
Heat capacity in terms of the density of states is, following section 4.2,
\[
  C_{V}=\tilde{\gamma}k_{B}^{2}g(E_{F})TV
  =\tilde{\gamma}k_{B}^{2}NTV/4\pi t
\]
If the atoms are divalent, then the band is fully filled, which means there's no way the energy of the system can change as the temperature
changes, i.e. $C_{V}=0$.
Similarly, the spin susceptibility is also zero, since there are no available states for the electrons to move into and align with the
applied magnetic field, i.e. $M(H)=0$.

\section*{1b}
Our unit cell is now expanded to include two adjacent atoms, so we have a system of effective Schr\"odinger equations of the same
type as above:
\[
  E\phi_{n,A}=\epsilon_{A} \phi_{n,A}-t(\phi_{n,B}+\phi_{n-1,B})
\]
\[
  E\phi_{n,B}=\epsilon_{B}\phi_{n,B}-t(\phi_{n+1,A}+\phi_{n,A})
\]
where $\phi_{n,A}$ ($\phi_{n,B}$) denotes the contribution to the $n$th site wave function from atom $A$ ($B$).
Using ans\"atze
\[
  \phi_{n,A}=Ae^{ikna}
\]
\[
  \phi_{n,B}=Be^{ikna}
\]
we have
\[
  AEe^{ikna}=\epsilon_{A} Ae^{ikna}-tB(e^{ikna}+e^{ik(n-1)a})
  \Leftrightarrow E=\epsilon_{A}-t\frac{B}{A}(1+e^{-ika})
\]
\[
  BEe^{ikna}=\epsilon_{B} Be^{ikna}-tA(e^{ik(n+1)a}+e^{ikna})
  \Leftrightarrow E=\epsilon_{B}-t\frac{A}{B}(e^{ika}+1)
\]
This system may be written in matrix form as
\[
  E
  \begin{pmatrix}
    A \\
    B
  \end{pmatrix}
  =
  \begin{pmatrix}
    \epsilon_{A} & -t(1+e^{-ika}) \\
    -t(1+e^{ika}) & \epsilon_{B}
  \end{pmatrix}
  \begin{pmatrix}
    A \\
    B
  \end{pmatrix}
\]
This is just the expression that $E$ is an eigenvalue of the matrix; we may solve for its value by computing
\[
  0=
  \begin{vmatrix}
    \epsilon_{A}-E & -t(1+e^{-ika}) \\
    -t(1+e^{ika}) & \epsilon_{B}-E
  \end{vmatrix}
  \Leftrightarrow 0=(\epsilon_{A}-E)(\epsilon_{B}-E)-t^{2}(1+e^{-ika})(1+e^{ika})
\]
\[
  \Leftrightarrow 0= \epsilon_{A}\epsilon_{B}-(\epsilon_{A}+\epsilon_{B})E+E^{2}-t^{2}(1+e^{ika}+e^{-ika}+1)
\]
\[
  \Leftrightarrow 0=\epsilon_{A}\epsilon_{B}-t^{2}(2+2\cos(ka))-(\epsilon_{A}+\epsilon_{B})E+E^{2}
\]
\[
  \Leftrightarrow E=\frac{\epsilon_{A}+\epsilon_{B}\pm\sqrt{(\epsilon_{A}+\epsilon_{B})^{2}-4(\epsilon_{A}\epsilon_{B}-t^{2}[2+2\cos(ka)])}}
  {2}
\]
Plotting this for random values of the constants, we obtain in the reduced zone scheme
\[
  \includegraphics[scale=.8]{plot2.png}
\]
and in the extended zone scheme
\[
  \includegraphics[scale=.8]{plot3.png}
\]
In the limit as $t\to 0$, $E\to \epsilon_{A}+\epsilon_{B}$ for the positive branch
and $E\to0$ for the negative branch.
Near the minimum, we can expand near $k=0$ $\cos(ka)\approx 1-k^{2}a^{2}/2$ to get
\[
  E\approx(\epsilon_{A}+\epsilon_{B})/2-\frac{1}{2}\sqrt{(\epsilon_{A}-\epsilon_{B})^{2}+4t^{2}[4-k^{2}a^{2}]}
\]
Expanding again using $\sqrt{a+x}\approx\sqrt{a}+\frac{x}{2\sqrt{a}}$,
\[
  E\approx (\epsilon_{A}+\epsilon_{B})/2-\frac{\sqrt{(\epsilon_{A}-\epsilon_{B})^{2}+16t^{2}}}{2}+\frac{1}{2}
  \frac{4t^{2}a^{2}k^{2}}{2\sqrt{(\epsilon_{A}-\epsilon_{B})^{2}+16t^{2}}}
  =\textrm{const}+\frac{t^{2}a^{2}k^{2}}{\sqrt{(\epsilon_{A}-\epsilon_{B})^{2}+16t^{2}}}
\]
Setting the quadratic part of this equation equal to the free electron dispersion,
\[
  \frac{\hbar^{2}k^{2}}{2m^{*}}=\frac{t^{2}a^{2}k^{2}}{\sqrt{(\epsilon_{A}-\epsilon_{B})^{2}+16t^{2}}}
  \Leftrightarrow m^{*}=\frac{\hbar^{2}\sqrt{(\epsilon_{A}-\epsilon_{B})^{2}+16t^{2}}}{2a^{2}t^{2}}
\]
If each atom is monovalent, there are two electrons per unit cell, filling the band and resulting in an insulator.
If $\epsilon_{A}=\epsilon_{B}$, we recover the previous problem's monatomic tight-binding chain solution, and the material becomes a metal.

\section*{2}
From the periodic boundary conditions, we have in any dimension
\[
  e^{ik(x+L)}=e^{ikx}\Leftrightarrow e^{ikL}=1 \Leftrightarrow kL=2\pi n\Leftrightarrow k=2\pi n/L
\]
i.e. each eigenstate corresponds to a volume $(2\pi/L)^{3}$ in $k$-space;
the first Brillouin zone ranges from $-\pi/a$ to $\pi/a$ in each direction and so has volume $(2\pi/a)^3$.
Dividing by the $k$-volume per eigenstate, there are $(L/a)^{3}$ eigenstates in the first Brillouin zone.
But this is exactly the expression for the number of primitive unit cells in the sample, which is the desired result.

\section*{3}
Since the potential is weak, the motion of the electrons is approximately free, and so a plane-wave state is a good approximation
to the solution.
Degenerate second-order perturbation theory implies that wave functions near the zone boundary may be decomposed
\[
  \Psi = A|k\rangle+B|k+G\rangle = Ae^{ikx}+Be^{i(k+G)x}
\]
We have an effective Schr\"odinger equation
\[
  \begin{pmatrix}
    \epsilon_{0}(k) & V_{G}^{*} \\
    V_{G} & \epsilon_{0}(k+G)
  \end{pmatrix}
  \begin{pmatrix}
    A \\
    B
  \end{pmatrix}
  =E
  \begin{pmatrix}
    A \\
    B
  \end{pmatrix}
\]
which is an eigenvalue problem equivalent to
\[
  0=
  \begin{vmatrix}
    \epsilon_{0}(k)-E & V_{G}^{*} \\
    V_{G} & \epsilon_{0}(k+G)-E
  \end{vmatrix}
  =\epsilon_{0}(k)\epsilon_{0}(k+G)-[\epsilon_{0}(k)+\epsilon_{0}(k+G)]E+E^{2}-|V_{G}|^{2}
\]
\[
  \Leftrightarrow E=\frac{\epsilon_{0}(k)+\epsilon_{0}(k+G)\pm
    \sqrt{[\epsilon_{0}(k)+\epsilon_{0}(k+G)]^{2}-4(\epsilon_{0}(k)\epsilon_{0}(k+G)-|V_{G}|^{2})}}{2}
\]
\[
  =\frac{1}{2}\left( \epsilon_{0}(k)+\epsilon_{0}(k+G)\pm
    \sqrt{[\epsilon_{0}(k)-\epsilon_{0}(k+G)]^{2}+4|V_{G}|^{2}} \right)
\]
Exactly on the Brillouin zone boundary, $\epsilon_{0}(k)=\epsilon_{0}(k+G)$, so
\[E=\epsilon\pm|V_{G}|\]
Applying the dispersion of the free electron,
\[
  E=\frac{\hbar^{2}k^{2}}{2m}+V_{0}\pm|V_{G}|
\]
The states are separated by $2|V_{G}|$ because the potential $V_{G}$ causes the scattered state to interfere with the incident state,
producing two states with different energies.
With a rather exaggerated band gap, the dispersion in the reduced zone scheme is sketched as
\[
  \includegraphics[scale=.8]{plot4.png}
\]
and in the extended zone scheme as
\[
  \includegraphics[scale=.8]{plot5.png}
\]
% TODO: copy the rest from book

\end{document}
%%% Local Variables:
%%% mode: latex
%%% TeX-master: t
%%% End:
