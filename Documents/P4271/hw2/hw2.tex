\documentclass{article}

\usepackage[letterpaper]{geometry}
\usepackage{tgpagella}
\usepackage{amsmath}
\usepackage{amssymb}
\usepackage{amsthm}
\usepackage{tikz}
\usepackage{minted}
\usepackage{physics}
\usepackage{siunitx}
\usepackage{mhchem}

\sisetup{detect-all}
\newtheorem{plm}{Problem}
\renewcommand*{\proofname}{Solution}

\title{Physics 4271 HW 2}
\author{Duncan Wilkie}
\date{10 February 2023}

\begin{document}

\maketitle

\begin{plm}
  Calculate the minimum energy required to be over the Coulomb barrier for:
  \begin{enumerate}
  \item $\ce{p + p}$,
  \item $\ce{p + ^{12}C}$,
  \item $\ce{^{4}He + ^{208}Pb}$.
  \end{enumerate}
\end{plm}

\begin{proof}
  The Coulomb potential is
  \[
    V(r) = \frac{1}{4\pi\epsilon_{0}}\frac{Z_{1}Z_{2}e^{2}}{r}.
  \]
  Whenever the reactants are close enough to make ``physical'' contact, the strong force kicks in;
  the Coulomb potential at this radius is roughly the peak of the Coulomb barrier.
  The radius of nuclei can be estimated as $r = \SI{1.2}{fm} \cdot A^{1/3}$.

  This gives us sufficient information to compute:
  \[
    r_{p + p} = 2r_{p} = 2(\SI{1.2}{fm})(1)^{1/3} = \SI{2.4}{fm}
    \Rightarrow E_{p + p} = \frac{1}{4\pi\epsilon_{0}}\frac{Z_{1}Z_{2}e^{2}}{r_{p + p}}
    = \frac{1}{4\pi(\SI{8.85e-12}{F/m})}\frac{1 \cdot 1 \cdot (\SI{1.6e-19}{C})^{2}}{\SI{2.4}{fm}}
  \]
  \[
    = \SI{9.6e-14}{J} = \SI{559}{keV}
  \]
  \[
    r_{\ce{p + {}^{12}C}} = r_{p} + r_{\ce{^{12}C}} = \SI{1.2}{fm} + \SI{1.2}{fm} \cdot (12)^{1/3} = \SI{3.95}{fm}
    \Rightarrow E_{\ce{p + ^{12}C}} = \frac{1}{4\pi\epsilon_{0}}\frac{Ze^{2}}{r_{\ce{p + ^{12}C}}}
  \]
  \[
    = \frac{1}{4\pi(\SI{8.85e-12}{F/m})}\frac{1 \cdot 6 \cdot (\SI{1.6e-19}{C})^{2}}{\SI{3.95}{fm}}
    = \SI{2.19}{MeV}
  \]
  \[
    r_{\ce{^{4}He + ^{208}Pb}} = r_{\ce{^{4}He}} + r_{\ce{^{208}Pb}} = \SI{1.2}{fm} \cdot 4^{1/3} + \SI{1.2}{fm} \cdot 208^{1/3}
    = \SI{9.01}{fm}
  \]
  \[
    \Rightarrow E_{\ce{^{4}He + ^{208}Pb}} = \frac{1}{4\pi\epsilon_{0}}\frac{Z_{1}Z_{2}e^{2}}{r}
    = \frac{1}{4\pi(\SI{8.85e-12}{F/m})}\frac{4 \cdot 208 \cdot (\SI{1.6e-19}{C})^{2}}{\SI{9.01}{fm}}
    = \SI{133}{GeV}
  \]
\end{proof}

\begin{plm}
  The cross section for charged-particle reactions is proportional to the probability of tunneling through the Coulomb barrier
  given by the Gamow factor, which has a convenient approximation:
  \[
    e^{-2\pi\eta} = e^{-2\pi Z_{1}Z_{2}e^{2}/\hbar\nu} = e^{-31.287Z_{1}Z_{2}\sqrt{\mu/E}}
  \]
  where $\mu$ is the reduced mass in \si{amu} and $E$ is the center-of-mass energy in \si{keV}.
  For the 3 cases you considered above, calculate the Gamow factor for an energy that is one-quarter the barrier energy you found in problem 1.
\end{plm}

\begin{proof}
  Plug-and-chug:
  \[
    \mu_{\ce{p + p}} = \frac{m_{p}m_{p}}{m_{p} + m_{p}} = \frac{m_{p}}{2} = \SI{0.5}{amu}
    \Rightarrow G_{\ce{p + p}} = \exp\qty(-31.287 \cdot 1 \cdot 1 \sqrt{\frac{\SI{0.5}{amu}}{\SI{559}{keV} / 4}})
    = \SI{0.94}{}
  \]
  \[
    \mu_{\ce{p + ^{12}C}} = \frac{m_{p}m_{\ce{^{12}C}}}{m_{p} + m_{\ce{^{12}C}}} = \frac{\SI{1}{amu}\cdot\SI{12}{amu}}{\SI{1}{amu}+\SI{12}{amu}}
    = \SI{0.92}{amu}
    \Rightarrow G_{\ce{p + ^{12}C}} = \exp\qty(-31.287 \cdot 1 \cdot 6 \sqrt{\frac{\SI{0.92}{amu}}{\SI{2.91}{MeV} / 4}})
  \]
  \[
    = \SI{0.81}{}
  \]
  \[
    \mu_{\ce{^{4}He + ^{208}Pb}} = frac{m_{\ce{^{4}He}}m_{\ce{^{208}{Pb}}}}{m_{\ce{^{4}He}} + m_{\ce{^{208}{Pb}}}}
    = \frac{\SI{4}{amu} \cdot \SI{208}{amu}}{\SI{4}{amu} + \SI{208}{amu}} = \SI{3.92}{amu}
  \]
  \[
    \Rightarrow G_{\ce{^{4}He + ^{208}Pb}} = \exp\qty(-31.287 \cdot 4 \cdot 208 \sqrt{\frac{\SI{3.92}{amu}}{\SI{133}{GeV} / 4}})
    = \SI{0.75}{}
  \]
\end{proof}

\begin{plm}
  A \SI{2}{MeV} beam of protons bombards a \ce{^{16}O} target and the differential cross section is measured to be \SI{0.094}{b/sr}
  at a lab angle of $167^{\circ}$.
  \begin{enumerate}
  \item What is the expected cross-section if you assume Rutherford scattering?
  \item What is the calculated Mott cross-section?
  \item How do your answers to (a) and (b) differ from the measured cross section and why might they be different?
  \end{enumerate}
\end{plm}

\begin{proof}
  Using the nonrelativistic Rutherford scattering formula (\SI{2}{MeV} is pretty slow),
  \[
    \qty(\dv{\sigma}{\Omega})_{\text{Rutherford}} = \frac{(zZe^{2})^{2}}{(4\pi\epsilon_{0})^{2} \cdot (4E_{kin})^{2} \sin^{4}\frac{\theta}{2}}
    = \frac{(1 \cdot 8 (\SI{1.6e-19}{C})^{2})^{2}}{(4\pi \cdot \SI{8.85e-12}{F/m})^{2} \cdot (4 \cdot \SI{2}{MeV})^{2}\sin^{4}\frac{167}{2}}
    = \SI{0.0212}{b/sr}.
  \]

  The Mott cross-section can be computed by
  \[
    \qty(\dv{\sigma}{\Omega})_{\text{Mott}} = \qty(\dv{\sigma}{\Omega})_{\text{Rutherford}} \cdot \cos^{2}\frac{\theta}{2}
    = (\SI{0.0212}{b/sr}) \cdot \cos^{2}\frac{167}{2} = \SI{2.72e-4}{b/sr}.
  \]
  The Rutherford cross-section is within an order of magnitude, whereas the Mott cross-section is pretty far off.
  The large discrepancy in the Mott cross-section is expected, since it's designed to account for spin effects in relativistic fermions,
  and this is a nonrelativistic boson scattering.
  The Ruterford cross-section doesn't take into account any spin effects, so that may be the source of its discrepancy.
\end{proof}

\begin{plm}
  Assume that $^{197}Au$ is made from a solid, uniform sphere of nuclear material with a radius of $R = \SI{1.2}{fm} \cdot A^{1/3}$.
  Calculate the form factor $F(q)$.
\end{plm}

\begin{proof}
  The charge distribution is
  \[
    \rho(r) =
    \begin{cases}
      \frac{Ze}{4\pi R^{3}/3} & r \leq R \\
      0 & r > R
    \end{cases}
  \]
  The nonzero density part can be computed to be $\frac{87 \cdot \SI{1.6e-19}{C}}{4\pi (\SI{1.2}{fm} \cdot 197^{1/3})/3} = \SI{4.76e-4}{C/m^{3}}$.
  Conditionnal on the Born approximation and presuming minimal nuclear recoil, the Fourier transform of this is the form factor:
  \[
    F(q^{2}) = (2\pi)^{3} \int_{\mathbb{R}^{3}}
  \]

\end{proof}

\begin{plm}
  Show that the mean-square charge radius of a uniformly charged sphere is $\expval{r^{2}} = 3R^{2}/5$.
\end{plm}

\begin{plm}
  A nuclear charge distribution more realistic than the uniformly charged distribution is the Fermi distribution,
  $\rho(r) = \frac{\rho_{0}}{1 + \exp\qty[(r - c)/a]}$.
  Find the value of $a$ if $t = \SI{2.3}{fm}$
\end{plm}

\begin{plm}[Bonus]
  Evaluate $\expval{r^{2}}$ for the Fermi distribution in Problem 6.
\end{plm}

\end{document}
