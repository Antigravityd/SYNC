\documentclass{article}

\usepackage[letterpaper]{geometry}
\usepackage{tgpagella}
\usepackage{amsmath}
\usepackage{amssymb}
\usepackage{amsthm}
\usepackage{tikz}
\usepackage{minted}
\usepackage{physics}
\usepackage{siunitx}
\usepackage{mhchem}

\sisetup{detect-all}
\newtheorem{plm}{Problem}
\renewcommand*{\proofname}{Solution}

\title{Physics 4271 HW 2}
\author{Duncan Wilkie}
\date{10 February 2023}

\begin{document}

\maketitle

\begin{plm}
  Calculate the minimum energy required to be over the Coulomb barrier for:
  \begin{enumerate}
  \item $\ce{p + p}$,
  \item $\ce{p + ^{12}C}$,
  \item $\ce{^{4}He + ^{208}Pb}$.
  \end{enumerate}
\end{plm}

\begin{plm}
  The cross section for charged-particle reactions is proportional to the probability of tunneling through the Coulomb barrier
  given by the Gamow factor, which haas a convenient approximation:
  \[
    e^{-2\pi\eta} = e^{-2\pi  ZZZ_{1}Z_{2}e^{2}/\hbar\nu} = e^{-31.287Z_{1}Z_{2}\sqrt{\mu/E}}
  \]
  where $\mu$ is the reduced mass in \si{amu} and $E$ is the center-of-mass energy in \si{keV}.
  For the 3 cases you considered above, calculate the Gamow factor for an energy that is one-quarter the barrier energy you found in problem 1.,
\end{plm}

\begin{plm}
  A \SI{2}{MeV} beam of protons bombards a \ce{^{16}O} target and the differential cross section is measured to be \SI{0.094}{b/sr}
  at a lab angle of $167^{\circ}$.
  \begin{enumerate}
  \item What is the expected cross section if you assume Rutherford scattering?
  \item What is the calculated Mott cross section?
  \item How do your answers to (a) and (b) differ from the measured cross section and why might they be different?
  \end{enumerate}
\end{plm}

\begin{plm}
  Assume that $^{197}Au$ is made from a solid, uniform sphere of nuclear material with a radius of $R = \SI{1.2}{fm} \cdot A^{1/3}$.
  Calculate the form factor $F(q)$.
\end{plm}

\begin{plm}
  Show that the mean-square charge radius of a uniformly charged sphere is $\expval{r^{2}} = 3R^{2}/5$.
\end{plm}

\begin{plm}
  A nuclear charge distribution more realistic than the uniformly charged distribution is the Fermi distribution,
  $\rho(r) = \frac{\rho_{0}}{1 + \exp\qty[(r - c)/a]}$.
  Find the value of $a$ if $t = \SI{2.3}{fm}$
\end{plm}

\begin{plm}[Bonus]
  Evaluate $\expval{r^{2}}$ for the Fermi distribution in Problem 6.
\end{plm}

\end{document}
