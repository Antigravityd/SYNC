\documentclass{article}

\usepackage[letterpaper]{geometry}
\usepackage{tgpagella}
\usepackage{amsmath}
\usepackage{amssymb}
\usepackage{amsthm}
\usepackage{tikz}
\usepackage{minted}
\usepackage{physics}
\usepackage{siunitx}

\sisetup{detect-all}
\newtheorem{plm}{Problem}
\renewcommand*{\proofname}{Solution}

\title{4721 HW 6}
\author{Duncan Wilkie}
\date{6 April 2023}

\begin{document}

\maketitle

\begin{plm}
  Common forms assumed for the momentum distributions of valence quarks in the proton are:
  \[
    F_{u} = xu(x) = a(1 - x)^{3}, \; F_{d}(x) = xd(x) = b(1 - x)^{3}.
  \]
  If the valence quarks account for half of the proton's momentum---i.e.
  \[
    \int_{0}^{1}xu(x)dx + \int_{0}^{1}xd(x)dx = \frac{1}{2},
  \]
  find the values of $a$ and $b$.
  Hint: the $u$ quarks carry approximately twice as much momentum as the $d$ quarks in the proton.
\end{plm}

\begin{proof}
  Some calculus:
  \[
    \int_{0}^{1}(1 - x)^{3}dx = \int_{1}^{0}u^{3} \cdot -du = \frac{u^{4}}{4}\eval_{u=0}^{u=1} = \frac{(1 - x)^{4}}{4}\eval_{x=1}^{x=0}
    = \frac{1}{4}
  \]
  Accordingly,
  \[
    \int_{0}^{1}xu(x)dx + \int_{0}^{1}xd(x)dx = \frac{1}{2}
    \Leftrightarrow a\int_{0}^{1}(1 - x)^{3}dx + b\int_{0}^{1}(1 - x)^{3}dx = \frac{1}{2}
    \Leftrightarrow \frac{a}{4} + \frac{b}{4} = \frac{1}{2}
    \Leftrightarrow a + b = 2.
  \]
  There are two valence up quarks, and one valence down quark, so one would expect the total momentum in the up quarks to be double
  that of the down quark---accordingly,
  \[
    2b + b = 2 \Leftrightarrow b = \frac{2}{3} \Rightarrow a = \frac{4}{3}.
  \]
\end{proof}

\begin{plm}
  What is the color wavefunction for mesons, in analogy to that for baryons of
  \[
    y_{baryon} = y_{space}y_{spin}(rgb + gbr + brg - rbg - bgr - grb)?
  \]
  Explain your answer.
\end{plm}

\begin{proof}
  This baryon color wavefunction comes from noticing that baryons are made up of three quarks, all of different color charge
  (so as to produce a color-neutral baryon), and then requiring the resulting wavefunction to be antisymmetric under particle interchange,
  so as to produce a fermionic composite particle.
  By contrast, mesons are bosonic, and so the color wavefunction must stay the same under particle interchange:
  \[
    y_{meson} = y_{space}y_{spin}\qty(r\bar{r} + g\bar{g} + b\bar{b}).
  \]
\end{proof}

\begin{plm}
  The diagram below shows the internal gluon interactions in a proton.
  Complete the diagram by labelling the color of the quarks and gluons.
\end{plm}

\begin{proof}
  The governing principle is that the net color of the proton is always white.

\end{proof}

\begin{plm}
  Which of the following processes are allowed?
  If not allowed, state why.
  If allowed, say whether the process is strong, weak, or electromagnetic.
  \begin{enumerate}
  \item $\nu_{e} + p \to e^{-} + \pi^{+} + p$
  \item $e^{+} + e^{-} \to \mu^{+} + \mu^{-}$
  \item $\Sigma^{-} \to n + \pi^{-}$
  \item $\bar{\nu}_{e} + p \to e^{-} + n$
  \item $e^{-} + p \to \nu_{e} + \pi^{0}$
  \end{enumerate}
\end{plm}

\begin{proof} \;
  \begin{enumerate}
  \item Allowed; weak.
  \item Allowed; electromagnetic.
  \item Allowed; strong.
  \item Disallowed; charge changes.
  \item Disallowed; baryon number changes.
  \end{enumerate}
\end{proof}

\begin{plm}[Double Points]
  The differential cross section for $e^{+} + e^{-} \to \mu^{+} + \mu^{-}$ is given by
  \[
    \dv{\sigma}{\Omega} = \frac{\alpha^{2}}{4s}(\hbar c)^{2}(1 + \cos^{2}\theta)
  \]
  in a collider experiment where $s = 4E_{e}$ and $E_{e}$ is the electron/positron energy.
  \begin{enumerate}
  \item Integrate over the solid angle to obtain an expression for the total cross section.
  \item If you use an electron beam energy of \SI{4}{GeV}, what rate of production of $\mu^{+}\mu^{-}$ would you expect at a luminosity of
    \SI{e33}{Hz/cm^{2}}?
  \item Calculate the ratio of the hadronic production cross section to that for $\mu^{+}\mu^{-}$ at $E_{e} = \SI{500}{GeV}$.
    If you use an electron beam energy of \SI{500}{GeV}, what must the luminosity be to measure the hadronic cross section within 24 hours
    with 10\% statistical uncertainty?
  \end{enumerate}
\end{plm}

\begin{proof} \;
  \begin{enumerate}
  \item We have a total cross-section
    \[
      \int \dv{\sigma}{\Omega} d\Omega = \int_{0}^{2\pi}\int_{0}^{\pi}\frac{\alpha^{2}}{4s}(\hbar c)^{2}\qty(1 + \cos^{2}\theta)\sin\theta
      d\theta d\phi
      = \frac{2\pi(\alpha\hbar c)^{2}}{4s}\int_{0}^{\pi}\sin\theta + \sin\theta\cos^{2}\theta d\theta
    \]
    \[
      = \frac{\pi(\alpha\hbar c)^{2}}{2s}\qty(2 - \int_{1}^{-1}u^{2}du)
      = \frac{\pi(\alpha\hbar c)^{2}}{2s}\qty(2 + \frac{u^{3}}{3}\eval_{-1}^{1})
      = \frac{\pi(\alpha\hbar c)^{2}}{2s}\qty(2 + \frac{2}{3})
      = \frac{4\pi(\alpha\hbar c)^{2}}{3s}
    \]
  \item If the electron beam energy is \SI{4}{GeV}, $s = 4 \cdot E_{e} = \SI{16}{GeV}$, and at the given luminosity,
    the expected production rate is
    \[
      L \cdot \sigma = \SI{e33}{Hz/cm^{2}} \cdot \frac{4\pi(\frac{1}{137} \cdot \SI{3.16e-24}{J \cdot cm})^{2}}{3 \cdot \SI{16}{GeV}}
      = \SI{8.71e-10}{Hz}
    \]
  \item % TODO
  \end{enumerate}
\end{proof}

\begin{plm}[Double Points]
  In an $e^{+}e^{-}$ collider experiment, a resonance $R$ is observed at $E_{cm} = \SI{10}{GeV}$ in both the $\mu^{+}\mu^{-}$
  and hadronic final states.
  The integrated cross sections are
  \[
    \int \sigma_{\mu \mu}(E)dE = \SI{10}{nb \cdot GeV}
  \]
  and
  \[
    \int \sigma_{h}(E)dE = \SI{300}{nb \cdot GeV}.
  \]
  Use a Breit-Wigner form for the resonance production to deduce the partial widths $\Gamma_{\mu\mu}$ and $\Gamma_{h}$ in \si{MeV}
  for the decays $R \to \mu^{+}\mu^{-}$ and $R \to \text{hadrons}$.
  Assume the integral
  \[
    \int_{resonance}\frac{dE}{(E - Mc^{2}) + \Gamma^{2}/4} \approx \frac{2\pi}{\Gamma}.
  \]
\end{plm}

\begin{proof}

\end{proof}

\begin{plm}
  Find the threshold kinetic energy for each of the following reactions, assuming the first particle to be incident on the second particle
  at rest:
  \begin{enumerate}
  \item $K^{-} + p \to \Xi^{-} + K^{+}$
  \item $\bar{p} + p \to \Upsilon$
  \item $\pi^{-} + p \to \omega + n$
  \end{enumerate}
\end{plm}

\end{document}
