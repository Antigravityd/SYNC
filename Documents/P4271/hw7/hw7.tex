\documentclass{article}

\usepackage[letterpaper]{geometry}
\usepackage{tgpagella}
\usepackage{amsmath}
\usepackage{amssymb}
\usepackage{amsthm}
\usepackage{tikz}
\usepackage{minted}
\usepackage{physics}
\usepackage{siunitx}
\usepackage{mhchem}

\sisetup{detect-all}
\newtheorem{plm}{Problem}
\renewcommand*{\proofname}{Solution}

\title{4271 HW 7}
\author{Duncan Wilkie}
\date{28 April 2023}

\begin{document}

\maketitle

\begin{plm}
  Suppose you have a KamLAND-like experiment where a detector \SI{200}{m} from a reactor detects $\bar{\nu}_e$ at 90\% of the flux expected
  with no oscillations.
  Assuming a 2-component model with maximal mixing ($\theta = 45^{\circ}$) and an average neutrino energy of \SI{3}{MeV},
  what is the squared mass difference of the $\bar{\nu}_{e}$ and its oscillating partner?
\end{plm}

\begin{proof}
  The two-component oscillation formula states
  \[
    P(\bar{\nu}_{e} \to \bar{\nu}') = \sin^{2}2\theta\sin^{2}\qty(1.27 \cdot \frac{\Delta m_{21}^{2}L}{E})
    \Leftrightarrow \Delta m_{21}^{2} = \frac{E}{1.27L}\sin^{-1}\qty(\frac{\sqrt{P}}{\sin 2\theta})
  \]
  \[
    = \frac{\SI{0.003}{GeV}}{1.27 \cdot \SI{0.2}{km}} \cdot \sin^{-1}\qty(\frac{\sqrt{0.1}}{\sin(2 \cdot 45^{\circ})})
    = \SI{0.218}{eV^{2}}
  \]
\end{proof}

\begin{plm}
  A neutron star is nearly pure neutron matter with a density near that of nuclear matter (\SI{0.17}{nucleons/fm^3}).
  \begin{enumerate}
  \item Calculate the radius of a $\SI{1.0}{M_\odot}$ neutron star and compare to that of the sun.
  \item A neutron star is formed from a supernova explosion when the core of the star (mostly iron and nickel) collapses.
    Assuming the neutron star is made by converting all of the protons in nickel into neutrons, calculate the number of neutrinos
    produced by the supernova explosion leaving behind a one-solar-mass neutron star.
  \item The Super-K solar neutrino detector contains \SI{500000}{t} pure water.
    Using your answer from the previous part, calculate the maximum distance at which Super-K can observe supernovae.
    Assume the cross-section for neutrino interactions is \SI{e-44}{cm^2}.
  \end{enumerate}
\end{plm}

\begin{proof}
  \;
  \begin{enumerate}
  \item A solar mass is
    \[
      M_{\odot} = \frac{\SI{2e30}{kg}}{\SI{1.6e-27}{kg}} = \SI{1.25e57}{neutron\;masses}.
    \]
    Accordingly, presuming the given density of a neutron star,
    \[
      \frac{M_{\odot}}{\rho} = V = \frac{4}{3}\pi R^{3}
      \Rightarrow R = \sqrt[3]{\frac{3M_{\odot}}{4\pi\rho}}
      = \sqrt[3]{\frac{3 \cdot \SI{1.25e57}{neutrons}}{4 \pi \cdot \SI{0.17}{neutrons/fm^{3}}}}
      = \SI{1.21e19}{fm} = \SI{12.1}{km}
    \]
  \item The number of neutrons necessary to make a solar mass was computed above.
    Using this, there are
    \[
      \frac{\SI{1.25e57}{neutrons}}{\SI{56}{nucleons/nickel\; nucleus}}
      = \SI{2.23e55}{nickel\; nuclei}
    \]
    that need to be converted for an all-nickel star to become an all-neutron star.
    For each of these nuclei, there are 28 protons that must be converted to neutrons, and for each conversion, a neutrino,
    so there are \SI{6.25e56}{neutrinos} produced over the entire conversion process.
  \item Presuming isotropic emissions of these neutrinos, they will be evenly spread over a spherical surface
    of radius equal to the distance to the supernova.
    This allows us to relate the luminosity at the detector over the entire supernova event to the distance, via
    \[
      L = \Phi N_{b} \Rightarrow L_{int} = FN_{b} = \frac{\SI{6.25e56}{neutrinos}}{{4}\pi R^{2}}N_{b}
    \]
    where $F$ is the neutrino fluence at Earth and $N_{b}$ is the number of target water particles.
    In order to detect a supernova, one needs at least one neutrino interaction (presuming, unrealistically, that there is no background
    so as to get an upper bound).
    Accordingly, by $N = \sigma L_{int}$, we can obtain an upper bound on the detectable distance of supernovae as
    \[
      R = \sqrt{\frac{\sigma \cdot \SI{6.25e56}{neutrinos}}{4\pi}N_{b}}
      = \sqrt{\frac{(\SI{e-44}{cm^{2}}) \cdot \SI{6.25e56}{neutrinos}}{4\pi}}
    \]
    \[
      \cdot \sqrt{\SI{50000}{t} \cdot \SI{5.46e29}{amu/t} \cdot \SI{18}{water\;molecules/amu}\cdot\SI{10}{electrons/water\;molecule}}
    \]
    \[
      = \SI{1.56e24}{cm} = \SI{0.51}{Mpc}
    \]
  \end{enumerate}
\end{proof}

\begin{plm}
  Calculate the energy of the Coulomb barrier for:
  \begin{enumerate}
  \item $\ce{d + t}$,
  \item $\ce{p + ^{12}C}$,
  \item $\ce{^{4}He + ^{12}C}$,
  \item $\ce{^{28}Si + ^{28}Si}$.
  \end{enumerate}
\end{plm}

\begin{proof} \;
  \begin{enumerate}
  \item The strong force kicks in at a distance of $R_{d} + R_{t} \approx \SI{1.2}{fm} \cdot 2^{1/3} + \SI{1.2}{fm} \cdot 3^{1/3}
    = \SI{3.24}{fm}$.
    Accordingly, the Coulomb barrier energy is
    \[
      E = \frac{ke^{2}}{R_{d} + R_{t}} = \frac{\SI{9e9}{N \cdot m^{2}/ C^{2}} \cdot (\SI{1.6e-19}{C})^{2}}{\SI{3.24e-15}{m}}
      = \SI{7.11e-14}{J} = \SI{444}{keV}
    \]
  \item As above, $R_{p} + R_{^{12}C} \approx \SI{1.2}{fm} \cdot 1^{1/3} + \SI{1.2}{fm} \cdot 12^{1/3} = \SI{3.95}{fm}$;
    \[
      E = \frac{6ke^{2}}{R_{p} + R_{^{12}C}} = \frac{\SI{9e9}{N \cdot m^{2}/C^{2}} \cdot 6(\SI{1.6e-19}{C})^{2}}{\SI{3.95e-15}{m}}
      = \SI{3.5e-13}{J} = \SI{2.19}{MeV}
    \]
  \item $R_{^{4}He} + R_{^{12}C} \approx \SI{1.2}{fm} \cdot 4^{1/3} + \SI{1.2}{fm} \cdot 12^{1/3} = \SI{4.65}{fm}$
    \[
      E = \frac{12ke^{2}}{R_{^{4}He} + R_{^{12}C}} = \frac{12 \cdot \SI{9e9}{N \cdot m^{2}/C^{2}} \cdot (\SI{1.6e-19}{C})^2}{\SI{4.65e-15}{m}}
      = \SI{5.95e-13}{J} = \SI{3.72}{MeV}
    \]
  \item $R_{^{28}Si} + R_{^{28}Si} \approx 2 \cdot \SI{1.2}{fm} \cdot 28^{1/3} = \SI{7.29}{fm}$
    \[
      E = \frac{196ke^{2}}{R_{^{28}Si} + R_{^{28}Si}} = \frac{196 \cdot \SI{9e9}{N\cdot m^{2}/C^{2}}\cdot(\SI{1.6e-19}{C})^2}{\SI{7.29e-15}{m}}
      = \SI{6.19e-12}{J} = \SI{38.7}{MeV}
    \]
  \end{enumerate}
\end{proof}

\begin{plm}
  Four hypothetical narrow s-wave resonances occur at low energies in the $\ce{^{20}Ne}(p, \gamma)\ce{^{21}Na}$ reaction
  at $E_{r}= 10, 40, 60,$ and \SI{120}{keV}.
  The resonance strengths are \SI{7.24e-33}{eV}, \SI{3.81e-15}{eV}, \SI{1.08e-9}{eV}, and \SI{3.27e-4}{eV}, respectively.
  \textbf{Without} calculating reaction rates, which resonances do you expect to dominate the total reaction rates at $T = \SI{0.03}{GK}$?
  At $T = \SI{0.09}{GK}$?
\end{plm}

\begin{proof}
  We can compute the Gamow peak for these two temperatures as
  \[
    E_{0} = 0.122\qty[10^{2} \cdot 11^{2} \frac{20 \cdot 21}{20 + 21} \cdot 0.03^{2}]^{1/3}
    = \SI{539}{keV};
  \]
  \[
    E_{0} = 0.122\qty[10^{2} \cdot 11^{2} \frac{20 \cdot 21}{20 + 21} \cdot 0.09^{2}]^{1/3}
    =  \SI{1.12}{MeV}.
  \]
  This is way off the end of the scale for the resonances, but this would lead one to conclude that the $\SI{120}{keV}$ resonance
  will dominate in both cases, with the lower-energy resonances being rarer.
\end{proof}

\begin{plm}
  For the hypothetical resonances given in Problem 4, calculate the reaction rates numerically for $T = \SI{0.03}{GK}$
  and $T = \SI{0.09}{GK}$.
  Is the Gamow peak concept valid in this case?
\end{plm}

\begin{proof}
  Scheme:
  \inputminted{scheme}{calc.scm}
  Since the Gamow peak applies to non-resonant reactions, one wouldn't expect it to apply to a resonant reaction;
  indeed, this indicates a \textit{larger} reaction rate for the \textit{lower} temperature---opposite the prediction above.
\end{proof}

\end{document}
