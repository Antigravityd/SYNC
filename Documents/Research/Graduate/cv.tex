%%%%%%%%%%%%%%%%%%%%%%%%%%%%%%%%%%%%%%%%%
% Classicthesis-Styled CV
% LaTeX Template
% Version 1.0 (22/2/13)
%
% This template has been downloaded from:
% http://www.LaTeXTemplates.com
%
% Original author:
% Alessandro Plasmati
%
% License:
% CC BY-NC-SA 3.0 (http://creativecommons.org/licenses/by-nc-sa/3.0/)
%
%%%%%%%%%%%%%%%%%%%%%%%%%%%%%%%%%%%%%%%%%

% ----------------------------------------------------------------------------------------
%	PACKAGES AND OTHER DOCUMENT CONFIGURATIONS
% ----------------------------------------------------------------------------------------

\documentclass{scrartcl}

\usepackage{tgpagella}

\reversemarginpar % Move the margin to the left of the page 

\newcommand{\MarginText}[1]{\marginpar{\raggedleft\itshape\small#1}} % New command defining the margin text style

\usepackage[nochapters]{classicthesis} % Use the classicthesis style for the style of the document
\usepackage[LabelsAligned]{currvita} % Use the currvita style for the layout of the document

\renewcommand{\cvheadingfont}{\LARGE\color{Maroon}} % Font color of your name at the top

\usepackage{hyperref} % Required for adding links	and customizing them
\hypersetup{colorlinks, breaklinks, urlcolor=Maroon, linkcolor=Maroon} % Set link colors

\newlength{\datebox}\settowidth{\datebox}{Spring 2011} % Set the width of the date box in each block

\newcommand{\NewEntry}[3]{\noindent\hangindent=2em\hangafter=0 \parbox{\datebox}{\small \textit{#1}}\hspace{1.5em} #2 #3 % Define a command for each new block - change spacing and font sizes here: #1 is the left margin, #2 is the italic date field and #3 is the position/employer/location field
  \vspace{0.5em}} % Add some white space after each new entry

\newcommand{\Description}[1]{\hangindent=2em\hangafter=0\noindent\raggedright\footnotesize{#1}\par\normalsize\vspace{1em}} % Define a command for descriptions of each entry - change spacing and font sizes here

% ----------------------------------------------------------------------------------------

\begin{document}

\thispagestyle{empty} % Stop the page count at the bottom of the first page

% ----------------------------------------------------------------------------------------
%	NAME AND CONTACT INFORMATION SECTION
% ----------------------------------------------------------------------------------------

\begin{cv}{\spacedallcaps{Duncan Wilkie}}\vspace{1.5em} % Your name

  \noindent\spacedlowsmallcaps{Personal Information}\vspace{0.5em} % Personal information heading

  \NewEntry{}{\textit{Born Arkansas,}}{3 July 2002} % Birthplace and date

  \NewEntry{email}{\href{mailto:dwilk14@lsu.edu}{dwilk14@lsu.edu}} % Email address

  % \NewEntry{website}{\href{http://www.johnsmith.com}{http://www.johnsmith.com}} % Personal website

  % \NewEntry{phone}{(H) +1 (479) 427 1795\ \ $\cdotp$\ \ (M) +1 (000) 111 1112} % Phone number(s)

  \vspace{1em} % Extra white space between the personal information section and goal

  \noindent\spacedlowsmallcaps{}\vspace{1em} % Goal heading, could be used for a quotation or short profile instead

  \Description{"The aim of theory really is, to a great extent,
    that of systematically organizing past experience in
    such a way that the next generation, our students and their students and so on,
    will be able to absorb the essential aspects in as painless a way as possible..."
    --- Michael Attiyah}\vspace{2em} % Goal text

  \noindent\spacedlowsmallcaps{Education}\vspace{1em}

  \NewEntry{2019---}{Bachelor of Science in Physics \ \ $\cdotp$ \ \ 3.87 GPA}

  \Description{\MarginText{Louisiana State University}
    As it's my primary discipline, I've taken a generalist's undergraduate curriculum,
    with classes from instrumentation electronics to solid state theory to gravitational wave astronomy.}

  \NewEntry{2019---}{Bachelor of Science in Math \ \ $\cdotp$ \ \ 3.70 GPA}

  \Description{\MarginText{Louisiana State University}
    Having attempted (at my grades' expense) to exhaust the undergraduate curriculum in pure mathematics as fast as possible,
    I've been taking graduate classes since sophomore year, with an eye towards the differential, (pseudo-)Riemannian,
    and algebraic geometry essential for advanced theoretical physics.}
  \vspace{1em}
  % ----------------------------------------------------------------------------------------
  %	WORK EXPERIENCE
  % ----------------------------------------------------------------------------------------

  \noindent\spacedlowsmallcaps{Research Projects}\vspace{1em}

  \NewEntry{Oct 2022---}{Mission Control Software for Lunar Payloads}

  \Description{\MarginText{Atlantis Industries}
    I was hired as a senior software engineer by Dr. Chancellor's startup, formed to commercialize technologies the lab and I
    had developed in an academic setting, initially funded through a US Space Command phase-II small business research grant.
    A flagship project of the company is Tiger Eye, a small, light, and low-power radiation detector developed for spaceflight applications.
    In partnership with Intuitive Machines, this detector is to be flown on a 5-year cislunar orbital mission,
    gathering data for space weather modeling.
    Transplanting much of the interface work done in summer '21, I've begun writing MacOS software for interacting with the payload,
    fetching and displaying its data whenever available. \\
    Advisor: Jeffery \textsc{Chancellor}\ \ $\cdotp$\  \ \href{mailto:jeff@spartanphysics.com}{jeff@spartanphysics.com}}


  \NewEntry{Nov 2021---}{A PHITS Python Porcelain}

  \Description{\MarginText{SpaRTAN Physics}
    Many modern computational physics programs present Python interfaces.
    I noticed at our group meetings that graduate students were struggling with the Particle and Heavy Ion Transport code System's
    card-based interface, so I offered to write a new, object-oriented one in Python that writes the analogous input file for an object tree,
    runs the transport code on it, and parses the output file to return for the user's further computational needs.
    This is tantamount to required tooling if one wants to run machine learning computations; for instance,
    a graduate project for automated design of spacecraft shielding requires kludgy, ad-hoc workarounds that slow the project tremendously.
    So far, the ``forward-mode'' part is completely written but not completely tested, while the output parsing
    can't be called complete in any sense, despite extensive labor. \\ % TODO: github
    Advisor: Jeffery \textsc{Chancellor}\ \ $\cdotp$\  \ \href{mailto:jeff@spartanphysics.com}{jeff@spartanphysics.com}}

  \NewEntry{Nov 2021---}{An Emacs Major Mode and Invocation Script for PHITS}

  \Description{\MarginText{SpaRTAN Physics}
    Learning to use PHITS for the above project, I wanted better editor support and a POSIX-compliant shell interface, so I wrote it.
    The former involves theme-aware syntax highlighting, automatic indentation and alignment, and keystroke execution
    and viewing of calculation results.
    The latter lets one select which parallelization scheme to use and handles otherwise tedious quirks of the MPI binary.}
  % ------------------------------------------------

  \NewEntry{Summer 2021}{Spaceflight Radiation Detector Development}

  \Description{\MarginText{SpaRTAN Physics}
    I was the primary software developer of an iOS-based interface for ADVACAM's MiniPIX detectors,
    doing 100\% of the Swift app development and much of the embedded C implementation of Apple's proprietary iAP2 protocol.
    The work was intended to fly on SpaceX's Inspiration 4 mission, and we hope it'll become useful for medical physics researchers.
    A preprint is available, and it has been submitted to NIM-A. \\ % TODO: preprint
    Advisor: Jeffery \textsc{Chancellor}\ \ $\cdotp$\  \ \href{mailto:jeff@spartanphysics.com}{jeff@spartanphysics.com}}

  % ------------------------------------------------

  \NewEntry{AY 2020}{Independent Reading in Quantum Information}

  \Description{\MarginText{}
    COVID left few traditional research opportunities, so I self-directed some quantum information reading.
    I began with the Chuang-Nielsen text, and later started to pick up papers in categorical quantum mechanics when I learned of its
    namesake mathematical field via classes touching on algebraic topology.
    Specifically, that of Abramsky and Coecke outlining the field and Selinger and Valiron's entailing
    application of linear logic's lambda calculi to quantum computation piqued my interest, as I had been playing with Lisps,
    which are a nearly direct implementation of Church's classical construction as a programming language.}

  % -----------------------------------------------


  \NewEntry{Summer 2020}{Data Science Intern}

  \Description{\MarginText{J. B. Hunt Transport Services}
    With the pandemic parking me back home and making academic development difficult,
    I found a remote position in which I could learn computer science skills likely to be useful in physics projects.
    I trained a machine learning model that estimates the repair time of tractors from basic data (e.g. mileage, repair location)
    based on Yandex's CatBoost that, to my knowledge, remains deployed.
    My work was twofold: port the massive, old IBM DB2 query over to Azure MySQL, and to produce a better model of the resulting data.
    The hybrid categorical-continuous nature of the data led to best performance by a CatBoost model among dozens tested within a
    hyperparameter optimization framework, beating the existing SPSS model by a difference in mean absolute percent error of 10\%.}

  % -----------------------------------------------

  \NewEntry{AY 2019}{Selected Readings in Functional Analysis}

  \Description{\MarginText{}
    I was directed to read about some of the basics of Lesbegue integration, function spaces, semigroup theory, divergent series,
    and asymptotic analysis.
    Specifically, Hardy's "Divergent Series," Yosida's "Operational Calculus: A Theory of Hyperfunctions,"
    and Estrada's "A Distributional Approach to Asymptotics" were major points of focus,
    alongside smaller passages from other works and non-published, one-on-one instruction.
    This work was to result in a poster on weak ODE solutions, but plans were called off due to COVID-induced cancellation
    of the poster session (and in-person meetings). \\
    Advisor: Frank \textsc{Neubrander}\ \ $\cdotp$\  \ \href{mailto:neubrand@math.lsu.edu}{neubrand@math.lsu.edu}}

  % ------------------------------------------------

  \vspace{1em} % Extra space between major sections

  % ----------------------------------------------------------------------------------------
  %	PUBLICATIONS
  % ----------------------------------------------------------------------------------------

  \spacedlowsmallcaps{Presentations}\vspace{1em}

  \NewEntry{October 2021}{Embedded Development for Spaceflight Radiation Detectors}

  \Description{\MarginText{LaSPACE Council Meeting}
    Showcase of the summer 2021 research described above at a statewide poster session for undergraduate
    and graduate students funded through NASA EPSCoR. \\
    Authors: Duncan Wilkie, Jacob Miller, Jared Taylor, Jeffery Chancellor}

  \vspace{1em} % Extra space between major sections

  \spacedlowsmallcaps{Computer Experience}\vspace{1em}
  \Description{
    \begin{itemize}
    \item Extensive
      \begin{itemize}
      \item Python and its EDSLs
      \item GNU/Linux, in theory and practice
      \item GNU Emacs and Emacs Lisp
      \item Embedded C (Texas Instruments' ARM CPUs)
      \item Swift and SwiftUI
      \item Scheme Lisp
      \item Parser combinators
      \item Java
      \item Numerical C++
      \item Gnuplot
      \item LaTeX and TikZ
      \item SageMath
      \end{itemize}
    \item Moderate
      \begin{itemize}
      \item Haskell
      \item Fortran 77 and 90
      \item Parser generators and EBNF
      \item GNU Octave/MATLAB
      \item Z80 and ARM Assembly
      \item GDB
      \item Profiling and optimization
      \end{itemize}
    \item Cursory
      \begin{itemize}
      \item Clash, SystemVerilog, and electronic design automation
      \item Compiler architecture and intermediate representations
      \item RISC-V
      \item CERN ROOT
      \item Rust
      \item Lua
      \item R
      \end{itemize}
  \end{itemize}}

  \spacedlowsmallcaps{Teaching}\vspace{1em}

  \Description{AY 2018 \ \ $\cdotp$ \ \ High/elementary school math and ACT prep tutoring job.}
  \Description{Aug 2019--- \ \ $\cdotp$ \ \ Informal homework help for other physics majors.}
  \Description{Fall 2020 \ \ $\cdotp$ \ \ Engineering physics recitation leader.}
  \Description{Apr 2022--- \ \ $\cdotp$ \ \ Presentations on relativity and differential geometry, discussions, and homework help in a math
  Discord server.}
  % ----------------------------------------------------------------------------------------
  %	OTHER INFORMATION
  % ----------------------------------------------------------------------------------------

  \spacedlowsmallcaps{Awards}\vspace{1em}

  \Description{2019- \ \ $\cdotp$ \ \ LSU Ann and Clarence P. Cazalot Jr College of Science Honors Scholar}

  \vspace{-0.5em}

  \Description{AY 2021\ \ $\cdotp$ \ \ Louisiana Space Grant Consortium Undergraduate Research Assistantship}

  \vspace{-0.5em} % Negative vertical space to counteract the vertical space between every \Description command

  \Description{2022\ \ $\cdotp$ \ \ LSU Goldwater Fellowship Nominee}

  % ------------------------------------------------


\end{cv}

\end{document}
