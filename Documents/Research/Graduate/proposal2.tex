\documentclass[12pt]{article}

\usepackage[letterpaper]{geometry}
\usepackage{tgpagella}
\usepackage{amsmath}
\usepackage{amssymb}
\usepackage{amsthm}
\usepackage{tikz}
\usepackage{minted}
\usepackage{physics}
\usepackage{siunitx}

\sisetup{detect-all}
\newtheorem{plm}{Problem}
\renewcommand*{\proofname}{Solution}
\bibliographystyle{IEEEtran}

\begin{document}

\section*{Personal Statement}
I've wanted to be a physicist since I was 5.
PBS programs like \textit{NOVA} and \textit{The Elegant Universe} directed my interest to the largest and smallest frontiers of human knowledge,
and, correspondingly, to the subject advancing in both directions simultaneously.
As I oriented more and more of my time towards pursuit of this goal, I found it harder and harder to specialize,
my interests if anything broadening as I wrapped up high school and moved into college.
I landed initially in mathematical research, as I had completed about 2 years of a math degree before my freshman year.
COVID cut this off, and I quickly had to recalibrate my plans; through a data science internship with a large freight company,
I got a crash-course in modern methods of data analysis from elementary statistics to machine learning.
Returning to school that fall, the pandemic still raged, and, presuming that professors would prefer to not do in-person research,
I decided to ramp up the difficulty of my courses, registering for a graduate course in topology alongside upper-division analysis and algebra.
Throughout this semester, I continued to find interesting problems external to my classes, browsing monographs in the QA section of the university library.
In the spring, a poor grade in the topology course was a good indication I was riding near my limits, and, as the pandemic waned,
I sparked up conversations with a quantum information researcher, continuing a concerning trend of project-hopping.
I was determined to stick with it this time, and planned out in detail the topics I would study under this new advisor.

All those plans went down the drain in short order.
The associate department chair interrupted my electronics lab final, asking if anyone knew how to write iOS apps and wanted to work on a space mission.
I, having wanted to do everything from astrophysics to aerospace engineering as a kid, volunteered, despite only fairly general programming experience.
He referred me to Dr. Chancellor, who summarily introduced me to the TRISH-funded project
to develop an iPhone interface to ADVACAM's Timepix3-based MiniPIX pixel detectors to fly on SpaceX's Inspiration 4 crewed mission.

I spent the next couple weeks frenetically trying to find summer housing while learning as much as possible about the detectors and developing iOS apps.
While two undergraduate salaries were budgeted for the latter task, no other volunteers could be found, for whatever reason.
I was solely responsible for the actual interface---the only other person doing direct design work
on the 5-person team was an electrical engineering undergrad.

Five people, only two of them doing direct design work, is an \textit{absurdly} small team for detector development.
And we had three months to do it.
Fortunately for me, the initial hang-up was on the electrical engineering side of things.
We were required to have a \textit{wired} interface between the detector and the device;
Apple seems to guard the means to accomplish that closer than 3-letter agencies do national secrets.
Development of the interface, however, could freely proceed in parallel, and over the two months it took for those strings to get pulled in the right places,
I put my nose to the SwiftUI grindstone and had almost fully developed the interface.
Operating on a testing database, it was capable of displaying extremely detailed information about received dose over time in different formats,
visualizing incoming frames, controlling the detector state, and importing calibrations files for different detectors.

It was early July when we finally obtained necessary technical documents to begin designing the embedded interface.
At roughly the same time, we were notified that the expected date of delivery was at the start, not the middle, of August;
with a $1/3$ cut to the remaining time, I was recast into an embedded development role.
In addition to my first Swift work, the entire project now was going to depended on my only C program.

Waiting for some specialized hardware, our electrical engineer selected a microcontroller and designed the physical layer.
I dove into the massive technical specification, trying to identify the path of least resistance to piping bytes from the detector into my app process.
When I finally began hitting keys in an IDE, another week had passed.
The rest of the month, we worked 12+ hour days, desperately trying to implement two convoluted protocol stacks with no further help than a
couple silicon vendor libraries for the link- and physical-layer protocols.

We had many setbacks induced by poor code architecture and infantile understanding of the embedded debugging process,
but at the end of it all, we had something that half-worked.
The app could talk to the detector!
Insufficient time remained to decode the detector's messages, feed information to my UI, and validate its correctness,
but we could save the binary to the file system for post-mission analysis.
So, we shipped a bare-bones system for detector control to Vandenberg.

The
\newpage
\section*{Project Narrative: Software for Estimating the Space Radiation Spectrum from Pixel Detector Track Structure}

According to the current understanding of space radiation and its biological effects,
far too few astronauts have exhibited serious exposure symptoms \cite{dummies}.
This is likely attributable to inadequate animal analogs (and otherwise suboptimal methodologies to boot), % TODO: cite this stuff
poor understanding of precise mechanisms of cellular damage, and coarse estimates of the so-called general cosmic radiation (GCR) spectrum,
the aggregate term for the energy and particle distributions of dangerous radiation in the space operational environment.
Were the prior two defects corrected, it would not improve outcomes without progress in the latter;
precision in the biological \textit{effects} of an exposure is anterior to precision in the exposure itself.

A substantial obstacle to better understanding the GCR spectrum is the inability to fly a fleet of detectors up there to check it out---we have detectors,
sure, but the semiconductor pixel detectors meeting the light-weight, low-power, low-maintenance requirements for spaceflight are generally capable
neither in hardware or software of presenting researchers with estimates of
\bibliography{proposal}
\end{document}
