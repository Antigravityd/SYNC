\documentclass{article}

\usepackage[letterpaper]{geometry}
\usepackage{tgpagella}
\usepackage{amsmath}
\usepackage{amssymb}
\usepackage{amsthm}
\usepackage{tikz}
\usepackage{minted}
\usepackage{physics}
\usepackage{siunitx}

\usepackage{tikz-cd}

\sisetup{detect-all}
\newtheorem{thm}{Theorem}
\newtheorem*{define}{Definition}

\title{A Based Derivative for Arbitrary Topological Spaces}
\author{Duncan Wilkie}
\date{24 October 2022}

\begin{document}

\maketitle

\section{Abstract}
There are many contexts throughout mathematics where a notion of a derivative is used.
We present a universal characterization of the derivative that depends only on a notion of continuity,
prove analogs of the usual derivative theorems in the general formalism, and recover myriad examples of the construction in mathematical practice.
We also explore the question of the integral cognate to this derivative through the fundamental theorem of calculus.

\section{Motivation}
Ordinarily, a function $f: \mathbb{R} \to \mathbb{R}$ is called \textit{differentiable at} $c$ if the two-sided limit
\[
  L = \lim_{h \to 0}\frac{f(c + h) - f(c)}{h}
\]
exists.
The number $L$ is then called the \textit{derivative} of $f$ at $c$, denoted $f'(c)$.

This work is motivated by alternative definition that has been used for pedagogical purposes but, to our knowledge, has never appeared in the literature:
a function $f: \mathbb{R} \to \mathbb{R}$ is called differentiable at $c$ if there exists a function $u_{c}(x)$ continuous at $c$ such that, for all $x$,
\[
  f(x) = f(c) + u_{c}(x)[x - c]
\]
The derivative of $f$ at $c$ is then defined to be $f'(c) = u_{c}(c)$.

We now prove two illuminating theorems about this new derivative that provide schemas for similar results to come.

\begin{thm}
  If the function $u_{c}(x)$ exists in the sense of the second definition, it is unique.
\end{thm}

\begin{proof}
  Suppose there were $u_{c}(x)$ and $u_{c}'(x)$ satisfying the conditions of the second definition.
  Then
  \[
    f(x) = f(c) + u_{c}(x)[x - c] = f(c) + u_{c}'(x)[x - c]
  \]
  The last equality is algebraically equivalent to $u_{c}(x) = u_{c}'(x)$.
\end{proof}

\begin{thm}
  The above two notions of the derivative agree, i.e. if a function is differentiable somewhere in one sense, it is differentiable there in the other,
  and the values of the derivatives are equal.
\end{thm}

\begin{proof}
  Consider the substitution into the limit definition $h = x - c$, yielding
  \[
    L = \lim_{x \to c}\frac{f(x) - f(c)}{x - c}.
  \]
  The function $u_{c}(x) = \frac{f(x) - f(c)}{x - c}$ is continuous at $c$ iff $L$ exists and equals $u_{c}(c)$ by the limit definition of continuity
  and satisfies the form of $u_{c}(x)$ in the definition.
\end{proof}

Intuitively, the function $u_{c}(x)$ is the slope of the secant line between $f(x)$ and $f(c)$.
This is quite similar to the geometric derivative that Leibniz knew. % TODO: plot for intuition.
However, a critical feature of this definition is that it's almost purely \textit{topological}: the key is the existence of a continuous map.

The remainder of the paper defines this derivative in a (demonstrably maximally) general context, derives some basic properties,
and investigates applications.

\section{The Based Derivative}
Our task is now to inspect the above notion and strip away anything field-theoretic until only topology remains.
One might say that a better view of $u_{c}(x)$ is as \textit{scaling} a change in the domain to a corresponding change in the codomain under the function;
left-multiplying by $u_{c}(x)$ can be viewed as a \textit{function} that maps $x - c$ to $f(x) - f(c)$.
The notion of ``change'' and ``distance'' provided by subtraction can also be lifted into an abstract topological space.
The expression $x - c$ represents an \textit{element} of $\mathbb{R}$; it also represents a \textit{continuous function} $\mathbb{R} \to \mathbb{R}$.
Such a function has a nice property that it's a homeomorphism; we want our definition to work for codomains nonhomeomorphic to domains,
so we'll relax this slightly to an embedding.
% TODO: additional properties if naturality isn't enough for this to play nice with the topology


Summarizing,

\begin{define}
  A map $f: X\to Y$ between topological spaces is called \textbf{based $g$-differentiable} at $c$ relative to a continuous map $g: X \to Y$
  containing $f(c)$ in its image if there exists a function $u_{c}: X \to (Y \to Y)$, continuous at $c$ in its first argument,
  such that for all $x$ the diagram
  % TODO: is this too weak? with $f(c) \in h(X)$?
  \begin{center}
    \begin{tikzcd}
      X \arrow[r, "g"] \arrow[rd, "f"]  & Y \arrow[d, "u_c(x)"]\\
        & Y
    \end{tikzcd}
  \end{center}
  commutes.
  The corresponding map $u_{c}(c): X \to Y$ is called the \textbf{based $g$-derivative} of $f$ at $c$.
\end{define}

Intuitively, function $u_c(x)$ maps the point $g(x)$, which is \textit{estimated value} of $f(x)$ according to the template $g$, to the actual point $f(x)$.
There is some freedom in the choice of topology on the set of continuous functions between spaces; a common choice is the compact-open topology.

\begin{thm}
  If a based derivative exists, it is unique.
\end{thm}

\begin{proof}
  Suppose $f: X \to Y$ admits two based derivatives $u_{c}, u_{c}'$ at $c$.
  The diagram says that for all continuous $g$ that $f = u_{c}(x) \circ g = u_{c}'(x) \circ g$ wherever the expression makes sense;
  since $g$ can be non-injective, the equality ought to be interpreted as between functions $g(X) \to f(g(X))$,
  in which case uniqueness of $u_{c}(x)$ (i.e. the mapping from a point to the transformation between estimates) is immediate.

  If $u_{c}(x)$ is continuous at $c$, then $u_{c}(c)$ exists and is uniquely determined by (the unique) $u_{c}(c)$.
\end{proof}


\begin{define}
  The $id_{X}$-derivative on endomorphisms $f: X \to X$ in the compact-open topology is simply the \textbf{based derivative} of $f$.
\end{define}


\begin{thm}
  The ordinary derivative and the based derivative on real functions are equivalent.
\end{thm}

\begin{proof}
  Continuity of the function $u_{c}(x) = f(c) + \frac{f(x) - f(c)}{x - c}(x - c)$ at $c$ is equivalent to the existence of
  $\lim_{x \to c}\frac{f(x) - f(c)}{x - c}$ by basic limit linearity theorems and equivalence of topological and limit-point continuity on $\mathbb{R}$.
\end{proof}

\begin{thm}
  The compact-open topology is the unique topology on functions $\mathbb{R} \to \mathbb{R}$ that agrees with
\end{thm}
\end{document}
