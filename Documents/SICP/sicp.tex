\documentclass{article}

\usepackage[letterpaper]{geometry}
\usepackage{tgpagella}
\usepackage{amsmath}
\usepackage{amssymb}
\usepackage{amsthm}
\usepackage{tikz}
\usepackage[outputdir=cache]{minted}
\usepackage{physics}

% \sisetup{detect-all}
\newtheorem{plm}{Problem}

\title{SICP Exercises}
\author{Duncan Wilkie}
\date{3 July 2023}

\begin{document}

\maketitle

\section{Chapter 1}

\begin{plm}
  Below is a sequence of expressions.
  What is the result printed by the interpreter in response to each expression?
  Assume that the sequence is to be evaluated in the order in which it is presented.
  \begin{verb}

    10

    (+ 5 3 4)

    (- 9 1)

    (/ 6 2)

    (+ (* 2 4) (- 4 6))

    (define a 3)

    (define b (+ a 1))

    (+ a b (* a b))

    (= a b)

    (if (and (> b a) (< b (* a b)))
             b
             a)

    (cond ((= a 4) 6)

          ((= b 4) (+ 6 7 a))

          (else 25))

    (+ 2 (if (> b a) b a))

    (* (cond ((> a b) a)

             ((< a b) b)

             (else -1))

       (+ a 1))
  \end{verb}
\end{plm}

\begin{proof} \;
  \begin{itemize}
  \item 10
  \item 12
  \item 8
  \item 3
  \item 6
  \item 3
  \item 4
  \item 19
  \item \#f
  \item 4
  \item 16
  \item 6
  \item 16
  \end{itemize}
\end{proof}

\end{document}
