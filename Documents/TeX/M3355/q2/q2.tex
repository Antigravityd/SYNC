\documentclass{article}
\usepackage[letterpaper]{geometry}
\title{3355 Quiz 2}
\author{Duncan Wilkie}
\date{25 September 2021}

\begin{document}

\maketitle

\section{}
Let $K$ be the event that you are killed in a shark attack while on vacation in Florida and let $A$ be the event that you are attacked by a shark while on vacation in Florida.
Then the probability that you are killed in a shark attack, given that you are attacked by a shark, while on vacation in Florida is
\[P(K|A) = \frac{P(KA)}{P(A)}=\frac{1/(85\times 10^6)}{3/(10\times 10^6)}=\frac{2}{51}\]
\section{}
$P(B|A)$ is the probability the second ball is red, given that the first ball was red.
The first ball being red reduces the number of red outcomes in the sample space, so that this is equivalent to the probability of drawing a red ball from an urn with 3 red balls and 5 blue balls.
The total number of balls is 8, so this probability is
\[P(B|A)=\frac{\textrm{number of red balls}}{\textrm{total number of balls}}=\frac{3}{8}\]
$P(B|A^c)$ is the probability the second ball is red, given that the first ball was not red, which in this case implies it was blue.
Similarly, this reduces the number of blue outcomes in the sample space, so that this is equivalent to the probability of drawing a red ball from an urn with 4 red balls and 4 blue balls.
The total number of balls is again 8, so this probability is
\[P(B|A^c)=\frac{\textrm{number of red balls}}{\textrm{total number of balls}}=\frac{4}{8}=\frac{1}{2}\]
The total probability that the second ball is red is therefore
\[P(B)=P(B|A)P(A)+P(B|A^c)P(A^c)=\left( \frac{3}{8} \right)\left( \frac{4}{9} \right)+\left( \frac{1}{2} \right)\left(1- \frac{4}{9} \right)=\frac{4}{9}\]
where we have found the probability the first ball is red by \[P(A)=\frac{\textrm{number of red balls}}{\textrm{total number of balls}}=\frac{4}{4+5}=\frac{4}{9}\]
and of course used that $P(A^c)=1-P(A)$.
\end{document}
%%% Local Variables:
%%% mode: latex
%%% TeX-master: t
%%% End:
