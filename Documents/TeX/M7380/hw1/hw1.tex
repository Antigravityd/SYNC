\documentclass{article}
\usepackage[letterpaper]{geometry}
\usepackage{amsmath}
\usepackage{amsfonts}

\title{7380 HW 1}
\author{Duncan Wilkie}
\date{12 September 2021}

\begin{document}

\maketitle

\section{}
Presuming $\epsilon$ is everywhere nonzero, we may rewrite (2) as $\frac{i}{\omega\epsilon}\left(\nabla\times H\right) = E$.
Taking the curl of both sides and substituting the right side of (1) in for $\nabla\times E$ yields
\[\nabla\times\left(\frac{i}{\omega\epsilon}\left(\nabla\times H\right)\right) = i\omega\mu H\]
Since $H$ is always perpendicular to the $(x_1,x_2)$ plane, it only has a component in the $x_3$ direction, and so its curl is computed as
$\left(\frac{\partial H}{\partial x_2}, -\frac{\partial H}{\partial x_1}, 0\right)$ using the abusive notation $H=|H|$.
Moving the constants to the other side, the subsequent curl is
\[\left(-\frac{1}{\epsilon}\frac{\partial^2H}{\partial x_3\partial x_1} - \frac{1}{\epsilon^2}\frac{\partial\epsilon}{\partial x_3}, \frac{1}{\epsilon^2}\frac{\partial \epsilon}{\partial x_3}\frac{\partial H}{\partial x_2}-\frac{1}{\epsilon}\frac{\partial^2 H}{\partial x_3\partial x_2}, \frac{1}{\epsilon^2}\left(\frac{\partial\epsilon}{\partial x_1}\frac{\partial H}{\partial x_1}+\frac{\partial \epsilon}{\partial x_2}\frac{\partial H}{\partial x_2}\right)+\frac{1}{\epsilon}\left(\frac{\partial^2 H}{\partial x_1^2}+\frac{\partial^2 H}{\partial x_2^2}\right)\right)\]
Since neither $H$ nor $\epsilon$ depend on $x_3$, all the terms containing those partials are zero. This expression then reduces to \[\left(0, 0, \frac{1}{\epsilon^2}\left(\frac{\partial\epsilon}{\partial x_1}\frac{\partial H}{\partial x_1}+\frac{\partial \epsilon}{\partial x_2}\frac{\partial H}{\partial x_2}\right)+\frac{1}{\epsilon}\left(\frac{\partial^2 H}{\partial x_1^2}+\frac{\partial^2 H}{\partial x_2^2}\right)\right)\]
On the right hand side, we similarly have a nonzero component only in the $x_3$ direction since $H$ is perpendicular to the $(x_1, x_2)$ plane. Thus, we obtain a single scalar equation. Applying the same abusive notation to the right side, this equation is
\[\frac{1}{\epsilon^2}\left(\frac{\partial\epsilon}{\partial x_1}\frac{\partial H}{\partial x_1}+\frac{\partial \epsilon}{\partial x_2}\frac{\partial H}{\partial x_2}\right)+\frac{1}{\epsilon}\left(\frac{\partial^2 H}{\partial x_1^2}+\frac{\partial^2 H}{\partial x_2^2}\right) = \omega^2\mu H\]
This is clearly a scalar second-order PDE for H, as desired.

\section{}
For the forward implication, let $\nabla \times F = A$. We then write
\[\nabla \times A = J \Leftrightarrow (\nabla\times A) \cdot \Phi = J\cdot \Phi\]
There is a vector identity for the cross product
\[\nabla\cdot (A\times B) = (\nabla \times A)\cdot B -A \cdot (\nabla \times B)\]
\[\Leftrightarrow (\nabla\times A) \cdot B = \nabla \cdot (A\times B) + A\cdot (\nabla \times B)\]
which we may apply to the previous expression to obtain
\[\nabla \cdot(A\times \Phi)+A\cdot (\nabla \times \Phi) = J\cdot \Phi\]
Expanding the expression for $A$,
\[\Leftrightarrow \nabla \cdot ((\nabla \times F) \times \Phi) + (\nabla \times F)\cdot (\nabla \times \Phi) = J\cdot \Phi\]
Integrating both sides over $\mathbb{R}^3$, set $B = (\nabla \times F)\times \Phi$ and use the linearity of the integral to obtain
\[\int \nabla \cdot B dV+\int (\nabla\times F)\cdot (\nabla \times \Phi)dV = \int J\cdot \Phi dV\]
Since $\Phi$ is compactly supported, there exists some bounded region $W\subseteq \mathbb{R}^3$ whose interior encloses the support of $\Phi$.
We may integrate over the interior of $W$ instead, since $\Phi$ being zero elsewhere implies the integrand and associated contributions to the integral are zero elsewhere.
The first summand on the left side is by the divergence theorem equal to
$\int_{\partial W}n\cdot B dS$, which, since we have established that the integrand is zero outside the support of $\Phi$, must be zero.
Therefore, we obtain the desired result, that
\[\int (\nabla\times F)\cdot(\nabla\times\Phi)dV = \int J\cdot \Phi dV\]
For the reverse direction, we may apply much of the same logic in reverse: we may add 0 to the left hand side in the form of $\int_{\partial W}n\cdot ((\nabla \times F)\times \Phi)=\int_W\nabla \cdot ((\nabla \times F)\times \Phi) dV$;
merging the integrals using linearity yields an integrand equal to the right hand side of the cross product identity used above. Applying it yields \[\int (\nabla\times\nabla \times F) \cdot \Phi = \int J \cdot \Phi \Leftrightarrow \int (\nabla\times\nabla\times F)\cdot \Phi - J\cdot \Phi = 0\]
\[\Leftrightarrow \int (\nabla\times\nabla\times F - J)\cdot \Phi = 0\]
using the linearity of the dot product. The problem then reduces, setting $C=\nabla\times\nabla\times F -J$, to showing that if $\forall\Phi$, $I=\int C\cdot \Phi = 0$, then $C=0$. We prove the contrapositive: let $C\neq 0$, then there exists some $\Phi$ such that $I\neq 0$. Since $F$ is smooth, all its derivatives are smooth, and so its curl and its curl's curl are both smooth. Since $J$ is also smooth, this implies $C$ is smooth and therefore continuous. Let $\Phi = C$. Then $C\cdot \Phi = C\cdot C = |C|^2$, which is also continuous since the Euclidean norm and squaring both are. Further, for a point $x$ such that $C(x)\neq 0$, $|C(x)|^2\neq 0$. Since $|C|^2$ is continuous and nonzero at a point, there is a region of positive measure where it is nonzero. Since the function is also non-negative everywhere, this implies its integral is nonzero.

\section{}
There is a vector calculus theorem that for $W\in\mathbb{R}^3$
\[\int_{\partial W} A\times ndS = -\int_W\nabla\times AdV\]
or, by anticommutativity of the cross product,
\[\int_{\partial W} n\times AdS  = \int_W\nabla\times AdV\]
Over each of $W_1$ and $W_2$, we construct a preliminary estimate of the surface integral, using $m$ for a general normal vector to a surface (i.e. sometimes meaning different things in a single equation corresponding to the surface being integrated over), as
\[\int_{\partial W_1}m\times F_1dS+\int_{\partial W_2}m\times F_2dS\]
This over-counts the regions $\partial W_1\cap\Gamma$ and $\partial W_2\cap\Gamma$, so we subtract them out to obtain
\[\int_{\partial W}m\times FdS=\int_{\partial W_1}m\times F_1dS+\int_{\partial W_2}m\times F_2dS - \int_{\partial W_1\cap\Gamma}m\times F_1dS -\int_{\partial W_2\cap\Gamma}m\times F_2dS\]
We may apply the theorem from above and use that $F_1=F_2=F$ on $W\setminus\Gamma$ to get
\[\int_{\partial W}m\times FdS=\int_{W_1\cup W_2}\nabla\times FdV - \int_{\partial W_1\cap\Gamma}m\times F_1dS -\int_{\partial W_2\cap\Gamma}m\times F_2dS\]
The rightmost two terms may be rewritten in terms of $n$, since in the first integral $m=n$ and in the second $m=-n$, to yield
\[\int_{\partial W}m\times FdS=\int_{W_1\cup W_2}\nabla\times FdV+\int_{\partial W_2\cap\Gamma}n\times F_2dS-\int_{\partial W_1\cap\Gamma}n\times F_1dS\]
By the linearity of the integral and cross product, and by definition of $[F]$, this is exactly the equation desired:
\[\int_{\partial W}m\times FdS=\int_{W_1\cup W_2}\nabla\times FdV+\int_{W\cap\Gamma}n\times[F]dS\]

\section{}
We may plug $v(k\cdot x\pm \omega t)$ into the PDE, and use the chain rule to obtain
\[\frac{\partial^2}{\partial t^2}v(k\cdot x\pm \omega t) = c^2\nabla^2v(k\cdot x\pm \omega t)\]
\[\Leftrightarrow \omega^2 v''(k\cdot x\pm \omega t)=c^2(k_1^2v''(k\cdot x\pm \omega t)+k_2^2v''(k\cdot x\pm \omega t)+k_3^2v''(k\cdot x\pm \omega t)) \]
\[\Leftrightarrow \omega^2=c^2|k|^2\Leftrightarrow |\omega| = c|k|\]
At points where $v''=0$, and therefore the above argument is made invalid, the PDE is satisfied trivially while for other points the argument holds. The above expression constrains all solutions of the PDE with the given form, and given that it holds, any twice-differentiable $v$ will satisfy the PDE, since the above argument was entirely equivalences.

In sum, all solutions of $\frac{\partial^2}{\partial t}u=c^2\nabla^2 u$ of the form $u(x,t)=v(k\cdot t\pm \omega t)$ are given by $v(k\cdot t\pm c|k|t)$ for all $v$ such that $v''$ exists everywhere.

\section{}
Let $\tau = t-t_0$. Then $\tilde{E}(x,t)=E(x,\tau)$, $\tilde{H}(x,t)=H(x,\tau)$ and subsequently
\[\nabla \times \tilde{E}(x,t) = -\frac{\partial}{\partial t}(\mu*\tilde{H}(x,t)) \Leftrightarrow \nabla\times E(x,\tau)=-\left( \frac{\partial}{\partial\tau}(\mu*H(x,\tau))\right)\frac{\partial\tau}{\partial t}\]
\[\Leftrightarrow \nabla \times E(x,\tau)=-\frac{\partial}{\partial\tau}(\mu*H(x,\tau))\]
This is identical to the assumption that $E$ and $H$ satisfy this equation in $t$.
An identical argument holds for the second equation of the system.
\end{document}
%%% Local Variables:
%%% mode: latex
%%% TeX-master: t
%%% End:
