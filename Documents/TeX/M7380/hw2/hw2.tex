\documentclass{article}

\usepackage{amsfonts}
\usepackage{amsmath}
\usepackage[letterpaper]{geometry}

\title{7380 HW 2}
\author{Duncan Wilkie}
\date{29 October 2021}

\begin{document}

\maketitle

\section*{5.5}
Suppose there existed a continuous $f$ that satisfied the $\delta$-function property, that
\[(f,\phi)=\int_\Omega f(x)\phi(x)dx=\phi(0)\] for all $\phi\in D(\Omega)$. Consider two such test functions $\phi_1,\phi_2$ with the same value at 0. Then
\[\int_\Omega f(x)\phi_1(x)dx-\int_\Omega f(x)\phi_2(x)dx = \phi_1(0)-\phi_2(0)\Rightarrow \int_\Omega f(x)(\phi_1(x)-\phi_2(x))dx = 0\]
We consider the slight modification of the fundamental lemma of the calculus of variations result proven on the previous homework where instead of requiring that the varying factor do so over all $C^\infty(\Omega)$, it does so over $\{\phi\in C^\infty(\Omega): \phi(0)=0\}$, and implying that $f(x)=0$ for all $x\neq 0 $, instead of for all $x$. The proof is much the same: if $f(x)\neq 0$, then there exists a point $x_0$ where it is nonzero. By continuity, there also exists an interval $[a,b]\ni x_0 $ where it is is supported. There also exists a closed interval not containing zero where it is supported: suppose $0\in[a,b]$; if $b\neq 0$ then $[b/2, b]$ is such an interval, otherwise $[a, a/2]$ is. Choose $\phi$ to be
\section*{5.11}
For $\sin(nx)$,
\[\sin(nx)\to 0\Leftrightarrow\int_\mathbb{R}\sin(nx)\phi(x)dx\to \int_\mathbb{R}0\cdot\phi(x)dx=0\]
Much like on the previous homework, since all $\phi$ are compactly supported (i.e. region of being nonzero is bounded) there exists some $M$ for each of them such that $\phi(v)=\phi(-v)=0$ for all $v\geq M$. The integral may the be rewritten and integration by parts applied as
\[\int_{-M}^M\sin(nx)\phi(x)dx=\frac{\cos(nx)}{n}\phi(x)\bigg|_{-M}^M-\int_{-M}^M\frac{\cos(nx)}{n}\phi'(x)dx\]
\[=-\int_{-M}^M\frac{\cos(nx)}{n}\phi'(x)dx\]
We now take the limit as $n\to\infty$. Since $|\cos(nx)|\leq 1$, $|\frac{\cos(nx)}{n}\phi'(x)| < \phi'(x)$, we may apply the dominated convergence theorem to see above integral is equal to
\[\int_{-M}^M\phi'(x)dx=\phi(x)\bigg|_{-M}^M=0\]
which proves the limit. On the other hand,
\end{document}
%%% Local Variables:
%%% mode: latex
%%% TeX-master: t
%%% End:
