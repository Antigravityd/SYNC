\documentclass{article}

\usepackage[letterpaper]{geometry}
\usepackage{amsmath}
\usepackage{amssymb}

\title{7380 HW Corrections}
\author{Duncan Wilkie}
\date{4 December 2021}

\begin{document}

\maketitle

\section*{5.20}
My confusion with this problem came down to the interpretation of ``with a piecewise continuous derivative,'' which I took to mean that a derivative is defined at every point of the domain of $f$, in concert with the usual analytical meaning of existence of a derivative. Under that definition, $f$ is vacuously piecewise continuous. I presume they mean instead ``has a piecewise continuous derivative on the set where $f$ is continuous,'' and proceed accordingly.

By definition,
\[\left\langle\frac{\partial f}{\partial x}, \phi\right\rangle=-\left\langle f, \frac{\partial \phi}{\partial x} \right\rangle=-\int_\mathbb{R} f\frac{\partial \phi}{\partial x}dx\]
Call the set on which $f$ is continuous $\Omega$ and the points of discontinuity $J=\{x_0,...,x_n\}$; the integral may be split over this partition:
\[\left\langle\frac{\partial f}{\partial x},\phi \right\rangle=-\int_\Omega f\frac{\partial \phi}{\partial x}dx - \int_Jf \frac{\partial \phi}{\partial x}dx\]
The first term is, since $\Omega$ is an open subset of $\mathbb{R}$ on which $f$ is $C^1$, exactly $\langle f', \phi \rangle$ where $f'$ is the classical derivative. Since $J$ is a finite set,
\end{document}
%%% Local Variables:
%%% mode: latex
%%% TeX-master: t
%%% End:
