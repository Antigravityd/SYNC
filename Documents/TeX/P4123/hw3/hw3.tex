\documentclass{article}
\usepackage[letterpaper]{geometry}

\title{4123 HW 3}
\author{Duncan Wilkie}
\date{19 October 2021}

\begin{document}

\maketitle

\section{}
Call the position of the natural length of the spring at $t=0$ $x_0$.
The Lagrangian is then $L=\frac{1}{2}m\dot{x}^2-\frac{1}{2}k(x-vt-x_0)^2$. The resulting Euler-Lagrange equation is
\[\frac{\partial L}{\partial x}=\frac{d}{dt}\frac{\partial L}{\partial \dot{x}}\Leftrightarrow -k(x-vt-x_0)=m\ddot{x} \Leftrightarrow m\ddot{x}+kx =kvt+kx_0\]
We may solve the homogeneous equation via the ansatz $x=e^{mt}$ as \[m=\pm i\sqrt{\frac{k}{m}}\Rightarrow x=c_1e^{ti\sqrt{\frac{k}{m}}}-c_2e^{-ti\sqrt{\frac{k}{m}}}=C_1\sin\left(t\sqrt{\frac{k}{m}}\right)+C_2\cos\left(t\sqrt{\frac{k}{m}}\right)\]\[=A\sin\left(t\sqrt{\frac{k}{m}}+\delta\right)\]
A particular solution to the non-homogeneous case is by inspection $x=vt+x_0$, so the full solution is
\[x(t)=x_0+vt+A\sin\left(t\sqrt{\frac{k}{m}}+\delta\right)\]
The generalized momentum is\[\frac{\partial L}{\partial \dot{x}}=m\dot{x}=m\left[  v+A\sqrt{\frac{k}{m}}\cos\left( t\sqrt{\frac{k}{m}}+\delta \right)\right]\]
If $x=vt$, $A=0$ and $x_0=0$. Therefore the generalized momentum is $mv$, which is a constant and therefore conserved.

\section{}
The contrapositive of this was proven in class, but I'll do it again.
If the Lagrangian is symmetric under time translations, it does not depend on time. Then we may apply the Euler-Lagrange first integral to each of the Euler-Lagrange equations for the system obtain
\[\frac{\partial L}{\partial \dot{q_i}}\dot{q}_i-L = \textrm{const}\]
The formal definition of the Hamiltonian is $H=\sum_i\frac{\partial L}{\partial\dot{q}_i}\dot{q}_i- L $. $L$ is a constant with respect to time, and the first-integral equation therefore implies that each term of the sum is constant as well. Thus, the Hamiltonian is not time-dependent.

\section{}
The Lagrangian for such a system is
\[L=\frac{1}{2}mv^2+mgy\]
This does not depend on $x$, $z$, or $t$, and so the quantities $p_x$, $p_z$, and $E$ are conserved.

\section{}
Applying the Euler-Lagrange first integral,
\[f-y'\frac{\partial f}{\partial y'}=c_1\Leftrightarrow y\sqrt{1+y'^2}-\frac{yy'^2}{\sqrt{1+y'^2}}=c_1\Leftrightarrow y+yy'^2-yy'^2=c_1\sqrt{1+y'^2}\]
\[\Leftrightarrow y^2=c_1^2(1+y'^2)\Leftrightarrow 1+y'^2= \frac{y^2}{c_1^2}\]
From the normal Euler-Lagrange equations,
\[\frac{\partial f}{\partial y}=\frac{d}{dx}\frac{\partial f}{\partial y'}\Leftrightarrow \sqrt{1+y'^2}=\frac{d}{dx}\left[ \frac{yy'}{\sqrt{1+y'^2}} \right]\]
\[\Leftrightarrow \sqrt{1+y'^2}=\frac{y'^2}{\sqrt{1+y'^2}}+yy''\left(\frac{1}{\sqrt{1+y'^2}}- \frac{y'^2}{(1+y'^2)^{3/2}} \right)\]
\[\Leftrightarrow (1+y'^2)^2=y'^2+y'^4+yy''+yy'^2y''-yy'^2y''\]
\[\Leftrightarrow y'^4+2y'^2+1=y'^2+y'^4+yy''\]\[\Leftrightarrow y'^2-yy''+1=0\]
Substituting the first-integral expression for $1+y'^2$,
\[\frac{y^2}{c_1^2}=yy''\Leftrightarrow y''-c^2y=0\]
Via the ansatz $y=e^{mx}$, $m^2-c^2=0\Rightarrow y=c_1e^{cx}+c_2e^{-cx}$ is the general solution to the homogeneous equation and $y=c_1e^{cx}+c_2e^{-cx}+1$ is the particular solution.

\end{document}
%%% Local Variables:
%%% mode: latex
%%% TeX-master: t
%%% End:
