\documentclass{article}

\title{4123 HW 4}
\author{Duncan Wilkie}
\date{9 November 2021}
\begin{document}

\maketitle

\section*{1a}
The Lagrangian for this system is
\[L=\frac{1}{2}m(\dot{x}^2+\dot{y}^2)+mgy\]
The modified Euler-Lagrange equations are
\[\frac{\partial L}{\partial x}+\lambda\frac{\partial C}{\partial x}=\frac{d}{dt}\frac{\partial L}{\partial \dot{x}}\Leftrightarrow 2\lambda x=m\ddot{x}\]
and
\[\frac{\partial L}{\partial y}+\lambda\frac{\partial C}{\partial y}=\frac{d}{dt}\frac{\partial L}{\partial\dot{y}}\Leftrightarrow mg+2\lambda y = m\ddot{y}\]
Adding these together and solving for $\lambda$, we obtain
\[mg+2\lambda(x+y)=m(\ddot{x}+\ddot{y})\Leftrightarrow \lambda = \frac{m}{2}\frac{\ddot{x}+\ddot{y}-g}{x+y}\]

\section*{1b}
Differentiating the constraint twice,
\[C=0\Rightarrow x\dot{x}+y\dot{y}=0\Rightarrow x\ddot{x}+\dot{x}^2+y\ddot{y}+\dot{y}^2=0\]
Solving for $\ddot{x}$ and $\ddot{y}$,
\[\ddot{x}=-\frac{y\ddot{y}+\dot{x}^2+\dot{y}^2}{x}\]
\[\ddot{y}=-\frac{x\ddot{x}+\dot{x}^2+\dot{y}^2}{y}\]
Adding,
\[\ddot{x}+\ddot{y}=-\left( \frac{y^2\ddot{y}+y\dot{x}^2+y\dot{y}^2+x^2\ddot{x}+x\dot{x}^2+x\dot{y}^2}{xy} \right)\]\[=\frac{-1}{xy}\left( x^2\ddot{x}+y^2\ddot{y}+(x+y)(\dot{x}^2+\dot{y}^2) \right)\]
Plugging this in to the equation for $\lambda$,
\[\lambda = -\frac{m}{2}\left(  \frac{x^2\ddot{x}+y^2\ddot{y}}{x^2y+y^2x}+\frac{\dot{x}^2+\dot{y}^2}{xy} +\frac{g}{x+y}\right)\]
%\[=-\frac{m}{2}\left( \frac{x^2\ddot{x}+y^2\ddot{y}+(x+y)(\dot{x}^2+\dot{y}^2)+gxy}{x^2y+y^2x}\right)\]
Multiplying by $1=\frac{l^2}{l^2}$ and noting that $l^2=x^2+y^2=(x+y)^2-2xy$,
%\[=-\frac{m}{2l^2}\left[ \left( l^2(x^2\ddot{x}+y^2\ddot{y})+\frac{(x+y)^2(\dot{x}^2+\dot{y}^2)}{xy}-2\dot{x}^2-\right)\]
\[=\]

\section*{2a}
Since there is no ``real'' potential, there is no force on $m_2$ and it must move with constant velocity. 

\section*{2b}
The resulting Lagrangian in the CM frame (which is approximately the $m_2$ frame for the same reason) is
\[L=\frac{1}{2}m_2\dot{r}^2-\frac{l^2}{2m_2 r^2}\]
The resulting Euler-Lagrange equation of motion is
\[\frac{\partial L}{\partial r}=\frac{d}{dt}\frac{\partial L}{\partial \dot{r}}\Leftrightarrow \frac{l^2}{m_2 r^3}=m_2\ddot{r}\Leftrightarrow \frac{l^2}{m_2^2}=r^3\ddot{r}\]
There is a centrifugal force term present.
This is a second-order autonomous equation with no first-derivative dependence, and so can be solved as
\[t(r)+c_1=\pm\int\frac{dr}{\sqrt{2\int\frac{l^2}{m_2^2 r^3}dr+c_2}}=\pm{\frac{m_2}{l}}\int\frac{dr}{\sqrt{c_2-\frac{1}{r^2}}}=\pm{\frac{m_2}{l}}\frac{r\sqrt{c_2-\frac{1}{r^2}}}{c_2}\]
\[\Rightarrow \sqrt{r^2c_2-{1}}=\pm\left(c_2t{\frac{l}{m_2}}+c_1c_2\right)\]
\[\Leftrightarrow r^2={1+\left( t{\frac{l}{m_2}}+c \right)^2}\]
This is a hyperbola, translated in the $t$-axis by $c$. This is not consistent with a constant-velocity trajectory, but since hyperbolas have oblique asymptotes it becomes approximately a constant-velocity trajectory. Therefore, the deviance we see is a breakdown of the approximation of the CM frame by the $m_1$ frame.

\section*{3}
The ordinary central-force radial Lagrangian is
\[L=\frac{1}{2}\mu\dot{r}^2-\frac{l^2}{2\mu r^2}-V(r)\]
The corresponding Euler-Lagrange equation is
\[\frac{\partial L}{\partial r}=\frac{d}{dt}\frac{\partial L}{\partial \dot{r}}\Leftrightarrow -\frac{\partial V}{\partial r}+\frac{l^2}{\mu r^3}=\mu\ddot{r}\Leftrightarrow F(r)+\frac{l^2}{\mu r^3}=\mu\ddot{r}\]
Taking $r$ to be a function of $\theta$ alone, $\frac{d^2s}{d\theta^2}=\frac{\partial s}{\partial r}r''+\frac{\partial^2s}{\partial\theta\partial r}r'=\frac{-r''}{r^2}$ (since $s$ is not directly dependent on $\theta$). The target equation may be written in terms of $r$ as
\[\frac{-r''}{r^2}+\frac{1}{r}=\frac{-\mu r^2}{l^2}F(r)\Leftrightarrow \frac{r''-r}{r^2}\frac{l^2}{\mu r^2}=F(r)
\]\[\Leftrightarrow \frac{l^2r''}{\mu r^4}-\frac{l^2}{\mu r^3}=F(r)\]
Noting that by the chain rule $\ddot{r}=r'\ddot{\theta}+r''\dot{\theta}^2=r''\dot{\theta}^2=r''\left(\frac{l}{\mu r^2}\right)^2$ (constant angular momentum $\Rightarrow$ no angular acceleration), $\frac{l^2r''}{\mu r^4}=\mu \ddot{r}$. Therefore, the target equation is
\[\mu\ddot{r}-\frac{l^2}{\mu r^3}=F(r)\]
which is clearly equivalent to the the Euler-Lagrange equation found above. 

\section*{4}
We have $s=\frac{1}{k\theta^2}$, so $\frac{d^2s}{d\theta^2}=\frac{6}{k\theta^4}$. The above equation is then
\[\frac{6}{k\theta^4}+\frac{1}{k\theta^2}=\frac{-\mu k^2\theta^4}{l^2}F(r)\]
\[\Leftrightarrow F(r)=\frac{-l^2}{\mu k^2\theta^4}\left(\frac{6}{k\theta^4}+\frac{1}{k\theta^2}\right)\]
\end{document}
%%% Local Variables:
%%% mode: latex
%%% TeX-master: t
%%% End:
