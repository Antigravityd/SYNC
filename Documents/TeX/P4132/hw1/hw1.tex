\documentclass{article}

\usepackage[letterpaper]{geometry}
\usepackage{amsmath}
\usepackage{amssymb}

\begin{document}

\section*{7.2a}
The voltage drop across a capacitor is $ V = \frac{q}{C}$, and the voltage drop across the resistor is $V=IR=\frac{dq}{dt}R$. Tracking the voltage drops around the loop,
\[\frac{q}{C}+\frac{dq}{dt}R=0\Leftrightarrow -\frac{dt}{RC}=\frac{dq}{q}\Leftrightarrow \ln q=-\frac{t}{RC}+c_1\Leftrightarrow q(t)=c_2e^{-{t}/{RC}}\]
Applying the boundary condition $q(0)=CV_0$,
\[q(t)=CV_0e^{-t/RC}\]
The current is
\[I(t)=\frac{dq}{dt}=\frac{V_0}{R}e^{-t/RC}\]

\section*{7.2b}
The initial energy in the capacitor is $E=\frac{1}{2}CV_0^2$. Integrating, \[P=I^2R=\frac{V_0^2}{R}e^{-2t/RC}\]
\[\Rightarrow E=\int_0^\infty\frac{V_0^2}{R}e^{-2t/RC}dt=\frac{V_0^2}{R}\left( -\frac{RC}{2}e^{-2t/RC}\bigg|_0^\infty \right)=\frac{1}{2}CV_0^2\]

\section*{7.2c}
Similarly to above,
\[V_0-\frac{q}{C}-\frac{dq}{dt}R=0\]
Using non-homogeneous constant coefficients,
\[q(t)=c_1e^{-t/RC}+CV_0\]
Applying the boundary condition that $q(0)=0$,
\[q(t)=CV_0(1-e^{-t/RC})\]
Differentiating,
\[I(t)=\frac{V_0}{R}e^{-t/RC}\]

\section*{7.2d}
The total energy output of the battery is
\[E=\int_0^\infty V_0I(t)dt=\int_0^\infty \frac{V_0^2}{R}e^{-t/RC}dt=\frac{V_0^2}{R}\left( -RCe^{-t/RC}\bigg|_0^\infty \right)=C{V_0^2}\]
The energy dissipated by the resistor is found by
\[P=I^2R=\frac{V_0^2}{R}e^{-2t/RC}\Rightarrow E=\int_0^\infty \frac{V_0^2}{R}e^{-2t/RC}dt=\frac{1}{2}CV_0^2\]
The energy left on the capacitor at the end is the difference between the two:
\[E_o=E_R+E_C\Rightarrow E_C=CV_0^2-\frac{1}{2}CV_0^2=\frac{1}{2}CV_0^2\]
This half the total work done by the battery, and confirms what is obvious; as $t\to\infty$, the current becomes negligible, so the voltage across the capacitor is $V_0$, so the energy stored in it must be the above.

\section*{7.3a}


\end{document}

%%% Local Variables:
%%% mode: latex
%%% TeX-master: t
%%% End:
