\chapter*{How to Read This Book}
I write this book to motivate.
As such, there's plenty of material in each chapter that's logically unnecessary for its subject, and is there for your philosophical satiety.
It's nevertheless my belief that this material is \textit{essential} for an integrated understanding---omit my rambles at your peril.

The text begins with a look at numbers and their history from a mature mathematical perspective.
This leads to the core philosophical insight of an \textbf{action}: a set whose elements somehow ``do something'' to another set.
In the next part, the idea of an action is made more precise in the context of studying symmetries---following intuitive geometric examples,
the abstract definition of a \textbf{group} is bootstrapped from laws constraining reasonable interpretations of what ``symmetry'' means.
The third part deals with \textbf{rings}, which are groups that act on themselves rather than another set.
Equivalently, they describe symmetries that are themselves symmetric; despite seeming like a special requirement,
examples in application and other mathematics abound (indeed, arithmetic can be seen as the study of particularly nice rings).
In the fourth part, we expound on the observation that in ring theory we often found ourselves almost-but-not-quite repeating our study of groups.
The core commonalities are extracted into the notion of a \textbf{category},
which provides an even better notion of an action with the idea of a \textbf{functor},
and deciding how to tell when two functors are truly different or nothing more than different notation for the same idea (e.g. left/right actions)
leads to the \textbf{natural transformation}.
We also find we can reinterpret the process of going from groups to rings---which seemed to yield richer results---in the new language.
This is explored in the final part, where the process is iterated again, forming \textbf{modules}: ``rings of rings,''
or ``symmetries of symmetries acting on symmetries.''
We apply our newfound techniques to great effect here.
This is the end of the road, though; if you have something that plays nice in the same way rings and modules do
(anything with an \textbf{Abelian category} structure), then you can always recast it as some kind of module
(even though you'd best stay in category-land).

I advise going through this in order.
A principal distinction of this book is its development of category-theoretic concepts without requiring of its reader
graduate-level familiarity with two or three massive subject areas.
The downside of this, however, is comparative over-dependence on the subject areas it does develop.
Standalone treatments of category theory (my current favorite is \cite{Reihl}) are able to cite dozens of applications
of their not-immediately-natural concepts to the most distant areas of mathematics, overwhelming the reader's disquiet with their sheer utility.
This is rhetorically effective for the initiate; for the newcomer to formal mathematics, though (welcome!),
it's necessary to build those concepts out from scratch, which is what the initial sections do.
For some hard evidence, here's a section-level dependency graph of the results save basic set theory and logic: % TODO

The situation with examples and problems, by contrast, is more nuanced.
I mentally divide these into several categories, based on what I've seen in other pedagogical material; I distinguish them for your convenience.
Some examples are throwaway, serving only to illustrate a concept for those finding a mere definition wanting.
These I tend to skip, but they're so absolutely helpful when they're necessary that I include them.
Do whatever suits you; they're indicated with a <symbol here>. % TODO
Others are primary mental models for their respective concept: intuitively mapping \textit{any} situation onto the specific case
works so well that it's permissible to use the specific as a mental image for the general, and is probably the most practical way to do it.
These (indicated by a <symbol here>) I wouldn't omit---I've certainly be guilty of doing it myself, but will appeal to fog of the moment % TODO
(most texts don't classify like this, and it's hard to tell what's what without foreknowledge of the subject).
Still others are critical counterexamples.
These demonstrate why and where different mental models for a slippery concept fail, and provide important whetstones against which to throw
intuitions, conjectures, and the like.
These are absolutely crucial to good understanding: any test worth its ink will ask you to produce one or two,
and there are those who have built careers on careful construction of convoluted objects that explode prior understanding\footnote
{
  For example, it was thought for a good century in topology that continuous structure and smooth structure always agree---that is,
  homeomorphic $\Leftrightarrow$ diffeomorphic. In fact, not only is this not true, it's not true even for \textit{flat, Euclidean space}!
  Weirder still, it happens to be true in all dimensions except 4, in which there are uncountably infinitely many incompatible structures!
  This leads to the mathematical limerick $1, 1, 1, \mathfrak{c}, 1, 1, \ldots$
  (which has a Jeapordy answer: ``How many distinct manifold structures are there on $\mathbb{R}^{n}$, for each positive $n$?'' )
}.
These have a <symbol here>. % TODO

Problems in general are the \textit{only} way to truly understand things.
Nevertheless, man's finite time makes it likely I'm the only one doing everything, and I classify them for you to prioritize to taste.
There are two broad categories: important and trivial problems.
These terms are in the sense of likelihood you'll encounter concepts they introduce beyond this book; trivial problems can be quite difficult,
but are meant to cultivate skill, and nothing more, whereas important problems ask you to develop the fundamental properties
of a construction of genuine mathematical interest.
Obviously, the important problems are good candidates for deference---you should at least read them, for nominal familiarity with what they define.
In addition, there are different problem difficulties, and different problem targets.
The former is fairly obvious: there are easy warm-up problems that merely help embed a definition in your brain (no mark),
reasonable problems that involve a nontrivial step or two ($*$), difficult problems with several steps that you need to find yourself ($**$),
and open research problems meant as fodder for your daydreams ($***$).
Targets indicate what is expected for a solution.
These range, in rough order of freedom in your response, from a right answer, a proof, an algorithm, a question,
evidence of critical thinking, to multiple of any of the previous.
This might correlate with difficulty in practice, independent though it is in theory.
For some exercises, my solution requires nonconstructive machinery; this is indicated in the same way as for results in the text.
Try to do better than me, but don't worry to much if you can't!

While I've done every problem to ensure they're appropriate, I don't share those solutions, because they will immediately go on libgen,
and I feel worse off for having used such pirated solutions on occasion myself.
See the appendix for how to determine if a solution is right, even without a key.
